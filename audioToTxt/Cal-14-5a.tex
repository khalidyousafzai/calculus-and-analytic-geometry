\جزوحصہ{سوالات 14.4}
مسئلہ گرین کی تصدیق۔

سوال 1 تا 4 میں میدان 
\عددی{F = M i + N j} کے لیے مساوات (11) اور (12) کی قیمتیں تلاش کر کے مسئلہ گریں کی تصدیق کریں۔ دونوں صورتوں میں تکمل کا دائرہ کار قرص  
\عددی{R: x^2 + y^2 \le a^2} اور اس کی تحدید ی دائرہ 
\عددی{
C: r = (a cos t)I + (a sin t)j, 0 \le t \le 2\pi
} لیں۔
\(
F = -yi + xj
\)
\(
F = yi
\)
\(
F = 2xi - 3yj
\)
\(
F = -x^2yi + xy^2j
\)
خلاف گهڑی دائری بہاؤ اور باہر کی طرف بہاؤ۔ سوال 5 تا 10 میں میدان $F$ اور منحنی $C$ کے لیے مسئلہ گرین استعمال کرتے ہوئے خلاف گھڑی دائری بہاؤ اور باہر کی طرف بہاؤ تلاش کریں۔
\(
F = (x - y)i + (y - x)j 
\)
\(
F = (x^2 + 4y)i + (x + y^2)j
\)
\(
F = (y^2 - x^2)i + (x^2 + y^2)j
\)
\(
F = (x + y)i - (x^2 + y^2)j
\)
\(
F = (x + e^x \sin y)i + (x + e^x \cos y)j
\)
\(
F = (\tan^(-1) y/x)i + \ln(x^2 + y^2 )
\)
\حصہ{سطحی رقبہ اور سطحی تکملات}

ہم مستوی کے کسی خطہ پر تفاعل کا تکمل لینا جانتے ہیں لیکن ایسی صورت میں کیا ہوگا جب تفاعل ایک فوسی سطح پر پایا جاتا ہو ايسى صورت میں اس کے تکمل کی قیمت کیسے حاصل کی جائے گی ان تکملات جنہیں سطحی تکملات کہتے ہیں کی قیمت تلاش کرنے کی خاطر اس سطح کے نیچے محددی سطح کے خطہ پر تمکمل کو دہرا تکمل لکھا جاتا ہے شکل (14.39) حصہ (14.7) اور (14.8) میں ہم دیکھیں گے کہ سطحی تکملات کی مدد سے مسئلہ گرین کو تین اباد میں عمومیت دی جا سکتی ہے سطحی تکمل کی تعریف شکل (14.40) میں سطح \عددی(S) اور اس کے نیچے مستوی میں اس کا سایہ خطہ \عددی(R) دکھایا گیا ہے سطح کی تعریفی مساوات 
\عددی(
f(x, y, c) = c
) ہے۔ ہموار سطح عددی{\nabla f} استمراری ہے اور عددی{ S} پر کہیں بھی صفر نہیں ہے۔ ہم اس کے رقبہ کی تعریف اور قیمت عددی{ R} پر دہرا تمکمل کی صورت میں کر سکتے ہیں۔ پہلی قدم میں خطہ عددی{ R} کی خانہ بندی چھوٹے چھوٹے مستطیلوں  عددی{\delta A_k} میں یوں کی جاتی ہے جیسے ہم عددی{ R} پر تمکمل کی تعریف پیش کرنا چاہتے ہوں۔ یہ عددی{ S} کے رقبہ کی تعریف پیش کرنے کا کا پہلا قدم ہے۔ ہر ایک عددی{\delta A} کے بالکل اوپر کچھ بلندی پر سطح عددی{\delta \sigma_k} پایا جاتا ہے جس کو ہم مماثی سطح کے چھوٹے حصہ عددی{\delta P_k} سے تخمین دے سکتے ہیں اس۔ کی بالکل ٹھیک وضاحت کچھ یوں ہے۔ رقبہ عددی{\delta A_k} کے پچھلے کونے عددی{C_k} کے بالکل اوپر نقطہ عددی{
T_k (X_k , Y_k , Z_k)
} پر سطح کے مماس کا عددی{\delta P_k} ایک ٹکڑا ہے۔ اگر مماسی سطح عددی{ R} کا متوازی ہو تب عددی{\delta P_k} رقبہ عددی{\delta A_k} کا تماسل ہوگا بصورت دیگر یہ ایک مستطیل ہوگا جس کا رقبہ عددی{\delta A_k} کے رقبہ سے کچھ زیادہ ہوگا شکل \حوالہ{14.41} میں عددی{\delta \sigma_k} اور عددی{\delta P_k} کی تکبیر کردہ منظر پیش کیا گیا ہے جہاں عددی{T_k} پر ڈلوان سمتیہ عددی{
\nabla f(x_k , y_k , z_k)
} اور عددی{ R} کا عمودی اکائی سمتیہ عددی{ P} دکھائے گئے ہیں۔ اس شکل میں عددی{\nabla f} اور عددی{p} کے مابین زاویہ عددی{\gamma_k} بھی دکھایا گیا ہے۔ اس شکل میں دیگر سمتیات عددی{u_k} اور عددی{v_k} مماثی مستوی میں عددی{\delta P_k} کے کناروں پر پائے جاتے ہیں۔ یوں عددی{
u_k \times v_k
} اور عددی{\nabla f} دونوں مماسی مستوی کو عمودی ہیں۔ اعلی سمتی ہند سیہ سے ہم جانتے ہیں کہ کسی بھی مستوی جس کا عمرود عددی{p} ہو پر ایک ایسے مستطیل جس کو عددی{u_k} اور عددی{v_k} تعین کرتے ہوں کی تذلیل کا رقبہ عددی{
|(u_k \times v_k) \cdot p|
} ہوگا۔ یوں درج ذیل لکھا جا سکتا ہے۔
\begin{align}
|(u_k \times v_k) \cdot p| = \delta A_k
\end{align}
اب سمتی ضرب کی ایک خاصیت یہ ہے کہ عددی{
|u_k \times v_k|
} رقبہ عددی{\delta P_k} ہوگا۔ لہذا مساوات سے۔
\begin{align}
|u_k \times v_k| |p| |\cos (angle between u_k \times v_k and p)| = \delta A_k
\end{align}
یا 
\[
\delta P_k |\cos \gamma_k | = \delta A_k
\]
یا 
\(
\cos \gamma_k \ne 0
\)
کی صورت میں 
\[
\delta P_k = \frac{\delta A_k}{|\cos \gamma_k|}
\]
لکھا جا سکتا ہے۔ جب تک 
\(
\nabla f
\)
زمینی مستوی کا متوازی اور 
\(
\nabla f \cdot p \ne 0
\)
ہو 
\(
\cos \gamma_k \ne 0
\)
ہوگا۔

چونکہ سطحی ٹکڑے 
\(
\delta \sigma_k
\)
جو مل کر رقبہ 
\(S\)
دیتے ہیں کو 
\(\delta P_k\)
تخمینا ظاہر کرتا ہے لہذا مجموعہ 
\[
\sum \delta P_k = \sum \frac{\delta A_k}{|\cos \gamma_k|}
\]
سطحی رقبہ عددی{S} کا تخمین نظر اتا ہے۔ ہم یہ بھی دیکھ سکتے ہیں کہ خطہ عددی{R} کو مزید چھوٹے خانوں میں تقسیم کرنے سے یہ تخمین بہتر ہوگی۔ درحقیقت مساوات تین کے دائیں ہاتھ مجموعے دوہرے تکمل 
\[
\int \int_R \frac{1}{|\cos \gamma|} \diff A
\]
کے تخمینی مجموعے ہیں۔ اسی بنا پر جب بھی یہ تکمل موجود ہو ہم اسے عددی{S} کے رقبہ کی تعریف لیتے ہیں۔ 

\جزوحصہ{عملی کلیہ}
کسی بھی سطح 
\(
f (x, y, z) = c
\)
کے لیے 
\(
|\nabla f \cdot p| = |\nabla f| |p| |\cos \gamma|
\)
ہوگا لہذا 
\(
\frac{1}{|\cos \gamma |} = \frac{|\nabla f|}{|\nabla f \cdot p|}
\)
ہوگا اس کو مساوات \حوالہ{4} کے ساتھ ملا کر رقبہ کا عملی کلیہ حاصل ہوتا ہے۔ 

سطحی رقبہ کا کلیہ بند اور محدود مستوی میں خطہ \(R\) پر سطح 
\(
f (x, y, z) = c
\)
کا رقبہ 
\{
Surface area = \int \int_R \frac{|nebla f|}{|\nebla f \cdot p|} \diff A
\}
ہوگا۔ جہاں \(p\) خطہ \(R\) کا اکائی عمودی سمتیا ہے اور 
\(
\nebla f \cdot p \ne 0
\)
ہے۔ یوں \(\nebla f\) کے قدر تقسیم \(R\) کو \(\nabla f\) کا غیر  سمتی عمودی جزو کا \(R\) پر دوہرا تکمل رقبہ ہوگا۔ ہم نے \(\nabla f\) کو استمراری تصور کرتے ہوئے اور پورے \(R\) پر \(\nabla f \cdot \ne 0\) تصور کرتے ہوئے مساوات \حوالہ{5} حاصل کی اور جب بھی یہ تکبر موجود ہو اس کی قیمت کو سطح \(f (x, y, z) = c\) کہ اس حصہ کے رقبہ کی تعریف لی جاتی ہے جو \(R\) کے اوپر پایا جاتا ہو۔ 

\ابتدا{مثال} 
قطر مکافی مجسم 
\(
x^2 + y^2 - z = 0
\)
کے نچلے حصے کو مستوی 
\(
z = 4
\)
ایک سطح کاٹتا ہے۔ اس کا رقبہ تلاش کریں۔ 
حل 
ہم مشتری \(x y\) میں  سطح \(S\) اور اس کے نیچے خطہ \(R\) کا خاکہ بناتے ہیں۔ شکل \حوالہ{14.42} سطح \(S\) سطح 
\(
f (x, y, z) = x^2 + y^2 - z = 0
\)
کا حصہ ہے اور \(R\) مستوی \(x y\) میں قرض 
\(
x^2 + y^2 \le 4
\)
ہے۔ مستوی \(R\) کا اکائی عمودی سمتیا 
\(
p = k
\)
ہے۔ سطح پر کسی نقطہ \(x, y, z\) پر 
\{
f (x, y, z) = x^2 + y^2 - z
\nabla f = 2 x i + 2 y j - k
|\nabla f| = \sqrt{(2x)^2 + (2y)^2 + (-1)^2}
= \sqrt{4x^2 + 4y^2 + 1}
|\nabla f \cdot p| = |\nabla f \cdot k| = | -1 | = 1
\}
ہوگا۔ خطہ \(R\) میں 
\(
\dif A = \dif x \dif y
\)
ہوگا۔ لہذا درج ذیل ہوگا۔ 
\{
Surface area = \iint_R \frac{|\nabla f|}{|\nabla f \cdot p|} \dif A \\
& = \iint_{x^2 + y^2 \le 4} \sqrt{4x^2 + 4y^2 + 1} \dif x \dif y \\
& = \int^{2 \pi}_0 \int^2_0 \sqrt{4r^2 + 1} r \dif r \dif \theta \\
& = \int^{2 \pi}_0 \big[ \frac{1}{12} (4 r^2 + 1)^{\frac{3}{2}} \big]^2_0 \dif \theta \\
& = \int^{2 \pi}_0 \frac{1}{12} (17^{\frac{3}{2}} -1) \dif \theta = \frac{\pi}{6} (17 \sqrt{17} - 1)
\}
\انتہا{مثال} 
\ابتدا{مثال} 
نصف کرہ 
\(
x^2 + y^2 + z^2 =2, z \ge 0
\)
سے بیلن 
\(
x^2 + y^2 =1
\)
ایک ٹوپی کاٹتا ہے۔ اس ٹوپی کا رقبہ تلاش کریں۔ شکل \حوالہ{14.43} 

حل

ٹوپی \(S\) ہم سطح 
\(
f (x, y, z) = x^2 + y^2 + z^2 = 2
\)
کا ایک حصہ ہے۔ یہ ایک ایک مطابقت کے ساتھ مستوی 
\(
x y
\)
 میں قرص 
\(
R: x^2 + y^2 \le 1
\) 
پر تذلیل کرتا ہے۔ سمتیا \(p = k\)
\(
R
\)
کے مستوی کو عمودی ہے۔ سطح میں کسی بھی نقطے پر 
\{
f (x, y, z) = x^2 + y^2 + z^2 \\
& \nabla f = 2x i + 2y j + 2 z k \\
& |\nabla f| = 2 \sqrt{x^2 + y^2 + z^2} = 2 \sqrt{2} \\
& |\nabla f \cdot p| = |\nabla f \cdot k| = |2 z| = 2 z
\}
ہوگا لہذا درج ذیل ہوگا۔ 
\{
\iint_R \frac{|\nabla f|}{|\nabla f \cdot p|} \dif A = \iint_R \frac{2 \sqrt{2}}{2 z} \dif A = \sqrt{2} \iint_R \frac{\dif A}{z}
\} 
یہاں \(z\) کا کیا کرنا ہوگا چونکہ کرہ میں کسی بھی نقطے پر 
\(z\)
محدد کو 
\(z\)
ظاہر کرتا ہے لہذا اسے ہم \(x\) اور \(y\) کی صورت میں لکھ سکتے ہیں۔ 
\{
z = \sqrt{2 - x^2 - y^2}
\} 
\حوالہ{6} میں اسے پر کرتے ہیں۔
\{
Surface area = \sqrt{2} \iint_R \frac{\dif A}{z} = \sqrt{2} \iint_{x^2 + y^2 \le 1} \frac{\dif A}{\sqrt{2 - x^2 - y^2}} \\
& = \sqrt{2} \int^{2 \pi}_0 \int^1_0 \frac{r \dif r \dif \theta}{\sqrt{2 - r^2}} \\
& \sqrt{2} \int^{2 \pi}_0 \big[ -(2 - r^2)^{\frac{1}{2}} \big]^{r = 1}_{r = 0} \dif \theta \\
& = \sqrt{2} \int^{2 \pi}_0 (\sqrt{2} - 1) \dif \theta = 2 \pi (2 - \sqrt{2})
\}

