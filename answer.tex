 {\urduTechTermsfont {حصہ}} \protect ١١.\protect ٥\hskip 1em\relax {\urduTechTermsfont {صفحہ}} \protect ١٣١٧
\begin {description}\setlength {\parskip }{0pt} \setlength {\itemsep }{0pt plus 1pt}
\item [
\protect ١١.\protect ٢٢٧)
]
$x=3+t,\, y=-4+t,\,z=-1+t$
\item [
\protect ١١.\protect ٢٢٩)
]
$x=-2+5t,\,y=5t,\,z=3-5t$
\item [
\protect ١١.\protect ٢٣١)
]
$x=0,\,y=2t,\,z=t$
\item [
\protect ١١.\protect ٢٣٣)
]
$x=1,\,y=1,\,z=1+t$
\item [
\protect ١١.\protect ٢٣٥)
]
$x=t,\,y=-7+2t,\,z=2t$
\item [
\protect ١١.\protect ٢٣٧)
]
$x=t,\,y=0,\,z=0$
\item [
\protect ١١.\protect ٢٣٩)
]
$x=t,\,y=t,\,z=3/2t,\,0\le t\le 1$
 \begin {center} \begin {tikzpicture}[font=\small ,x={(-0.5cm,-0.5cm)},y={(1cm,0)},z={(0,1cm)}] \draw [-latex](0,0,0)--(1.5,0,0)node[left]{$x$}; \draw [-latex](0,0,0)--(0,1.5,0)node[right]{$y$}; \draw [-latex](0,0,0)--(0,0,1.5)node[left]{$z$}; \draw [->-=0.5](0,0,0)node[circ]{}node[left]{$(0,0,0)$}--(1,1,1.5)coordinate(kA)node[circ]{}node[right]{$(1,1,\tfrac {3}{2})$}; \draw [dashed](kA)--(1,1,0)--(1,0,0); \draw [dashed](1,1,0)--(0,1,0); \end {tikzpicture} \end {center} 
\item [
\protect ١١.\protect ٢٤١)
]
$x=1,\,y=1+t,\,z=0,\,-1\le t\le 0$
 \begin {center} \begin {tikzpicture}[font=\small ,x={(-0.5cm,-0.5cm)},y={(1cm,0)},z={(0,1cm)}] \draw [-latex](0,0,0)--(1.5,0,0)node[left]{$x$}; \draw [-latex](0,0,0)--(0,1.5,0)node[right]{$y$}; \draw [-latex](0,0,0)--(0,0,0.75)node[left]{$z$}; \draw [->-=0.5](1,0,0)node[circ]{}node[below,xshift={2.5ex}]{$(1,0,0)$}--(1,1,0)node[circ]{}node[right]{$(1,1,0)$}; \draw [dashed](1,1,0)--(0,1,0); \end {tikzpicture} \end {center} 
\item [
\protect ١١.\protect ٢٤٣)
]
$x=0,\,y=1-2t,\,z=1,\,0\le t\le 1$
 \begin {center} \begin {tikzpicture}[font=\small ,x={(-0.5cm,-0.5cm)},y={(1cm,0)},z={(0,1cm)}] \draw [-latex](0,0,0)--(0.5,0,0)node[left]{$x$}; \draw [-latex](0,-1.5,0)--(0,1.5,0)node[right]{$y$}; \draw [-latex](0,0,0)--(0,0,1.5)node[left]{$z$}; \draw [->-=0.25](0,1,1)node[circ]{}node[above]{$(0,1,1)$}--(0,-1,1)node[circ]{}node[above]{$(0,-1,1)$}; \end {tikzpicture} \end {center} 
\item [
\protect ١١.\protect ٢٤٥)
]
$x=2-2t,\,y=2t,\,z=2-2t,\,0\le t\le 1$
 \begin {center} \begin {tikzpicture}[font=\small ,x={(-0.5cm,-0.5cm)},y={(1cm,0)},z={(0,1cm)}] \draw [-latex](0,0,0)--(1.5,0,0)node[left]{$x$}; \draw [-latex](0,0,0)--(0,1.5,0)node[right]{$y$}; \draw [-latex](0,0,0)--(0,0,1.5)node[left]{$z$}; \draw [->-=0.5](1,0,1)node[circ]{}node[left]{$(2,0,2)$}--(0,1,0)node[circ]{}node[below]{$(0,2,0)$}; \draw [dashed](1,0,0)--(1,0,1)--(0,0,1); \end {tikzpicture} \end {center} 
\item [
\protect ١١.\protect ٢٤٧)
]
$3x-2y-z=-3$
\item [
\protect ١١.\protect ٢٤٩)
]
$7x-5y-4z=6$
\item [
\protect ١١.\protect ٢٥١)
]
$x+3y+4z=34$
\item [
\protect ١١.\protect ٢٥٣)
]
$(1,2,3),\,\, -20x+12y+z=7$
\item [
\protect ١١.\protect ٢٥٥)
]
$y+z=3$
\item [
\protect ١١.\protect ٢٥٧)
]
$x-y+z=0$
\item [
\protect ١١.\protect ٢٥٩)
]
$2\sqrt {30}$
\item [
\protect ١١.\protect ٢٦١)
]
$0$
\item [
\protect ١١.\protect ٢٦٣)
]
$\tfrac {9\sqrt {42}}{7}$
\item [
\protect ١١.\protect ٢٦٥)
]
$3$
\item [
\protect ١١.\protect ٢٦٧)
]
$19/5$
\item [
\protect ١١.\protect ٢٦٩)
]
$5/3$
\item [
\protect ١١.\protect ٢٧١)
]
$9/\sqrt {41}$
\item [
\protect ١١.\protect ٢٧٣)
]
$\pi /4$
\item [
\protect ١١.\protect ٢٧٥)
]
$1.76$\, ریڈیئن
\item [
\protect ١١.\protect ٢٧٧)
]
$0.82$\,ریڈیئن
\item [
\protect ١١.\protect ٢٧٩)
]
$(3/2,-3/2,1/2)$
\item [
\protect ١١.\protect ٢٨١)
]
$(1,1,0)$
\item [
\protect ١١.\protect ٢٨٣)
]
$x=1-t,\,y=1+t,\,z=-1$
\item [
\protect ١١.\protect ٢٨٥)
]
$x=4,\,y=3+6t,\,z=1+3t$
\item [
\protect ١١.\protect ٢٨٧)
]
 \عددی {L_1} اور \عددی {L_2} متقاطع ہیں؛ \عددی {L_2} اور \عددی {L_3} متوازی ہیں؛ \عددی {L_1} اور \عددی {L_3} غیر ہمسطحی ہیں۔ 
\item [
\protect ١١.\protect ٢٨٩)
]
$x=2+2t,\,y=-4-t,\,z=7+3t;\,x=-2-t,\, y=-2+t/2,\, z=1-3/2t$
\item [
\protect ١١.\protect ٢٩١)
]
$(0,-\tfrac {1}{2},-\tfrac {3}{2}),\, (-1,0,-3),\,(1,-1,0)$
\item [
\protect ١١.\protect ٢٩٥)
]
 بہت سارے مختلف جوابات ممکن ہیں۔ ان میں سے ایک جواب ہے: $x+y=3,\,2y+z=7$ 
\item [
\protect ١١.\protect ٢٩٧)
]
 ماسوائے ان سطحوں کے جو مبدا سے گزرتے ہوں یا جو محددی محور کے متوازی ہوں تمام سطحوں کو \عددی {x/a+y/b+z/c=1} سے ظاہر کیا جا سکتا ہے۔ 
\end {description}
 {\urduTechTermsfont {حصہ}} \protect ١١.\protect ٦\hskip 1em\relax {\urduTechTermsfont {صفحہ}} \protect ١٣٣٣
\begin {description}\setlength {\parskip }{0pt} \setlength {\itemsep }{0pt plus 1pt}
\item [
\protect ١١.\protect ٣٠١)
]
 شکل \حوالہ {شکل_سوال_سمتیہ_سطح_اور_مساوات_ملائیں_الف}
\item [
\protect ١١.\protect ٣٠٣)
]
 شکل \حوالہ {شکل_سوال_سمتیہ_سطح_اور_مساوات_ملائیں_پ}
\item [
\protect ١١.\protect ٣٠٥)
]
 شکل \حوالہ {شکل_سوال_سمتیہ_سطح_اور_مساوات_ملائیں_ٹ}
\item [
\protect ١١.\protect ٣٠٧)
]
 شکل \حوالہ {شکل_سوال_سمتیہ_سطح_اور_مساوات_ملائیں_ج}
\item [
\protect ١١.\protect ٣٠٩)
]
 شکل \حوالہ {شکل_سوال_سمتیہ_سطح_اور_مساوات_ملائیں_ح}
\item [
\protect ١١.\protect ٣١١)
]
 شکل \حوالہ {شکل_سوال_سمتیہ_سطح_اور_مساوات_ملائیں_د}
\item [
\protect ١١.\protect ٣١٣)
]
$x^2+y^2=4$
 \begin {center} \begin {tikzpicture}[font=\small ,declare function={fx(\r ,\t )=\r *cos(\t );fy(\r ,\t )=\r *sin(\t );}] \begin {axis}[view/h=135,axis equal,axis lines=center,xtick={\empty },ytick={\empty },ztick={\empty },xlabel={$x$},ylabel={$y$},zlabel={$z$},xlabel style={anchor=east},ylabel style={anchor=west},zlabel style={anchor=east}, axis x line=none,axis y line=none,axis z line=none] \addplot 3[domain y=0:360] ({fx(2,y)},{fy(2,y)},{3}); \addplot 3[domain y=0:360] ({fx(2,y)},{fy(2,y)},{0}); \addplot 3[domain y=0:360] ({fx(2,y)},{fy(2,y)},{-3}); \addplot 3[] coordinates {({fx(2,-30)},{fy(2,-30)},{-3})({fx(2,-30)},{fy(2,-30)},{3})}; \addplot 3[] coordinates {({fx(2,150)},{fy(2,150)},{-3})({fx(2,150)},{fy(2,150)},{3})}; \addplot 3[-latex] coordinates {(0,0,0)(6,0,0)}node[below]{$x$}; \addplot 3[-latex] coordinates {(0,0,0)(0,5,0)}node[below]{$y$}; \addplot 3[-latex] coordinates {(0,0,0)(0,0,6)}node[left]{$z$}; \end {axis} \end {tikzpicture} \end {center} 
\item [
\protect ١١.\protect ٣١٥)
]
$z=y^2-1$
 \begin {center} \begin {tikzpicture}[font=\small ,declare function={fz(\y )=(\y )^2-1;}] \begin {axis}[view/h=135,axis equal,axis lines=center,xtick={\empty },ytick={\empty },ztick={\empty },xlabel={$x$},ylabel={$y$},zlabel={$z$},xlabel style={anchor=east},ylabel style={anchor=west},zlabel style={anchor=east}, axis x line=none,axis y line=none,axis z line=none] \addplot 3[samples y=0,domain=-1:1]({0},{x},{fz(x)}); \addplot 3[samples y=0,domain=-1:1]({-1},{x},{fz(x)}); \addplot 3[samples y=0,domain=-1:1]({1},{x},{fz(x)}); \addplot 3[] coordinates {({-1},{-1},{fz(-1)})({1},{-1},{fz(-1)})}; \addplot 3[] coordinates {({-1},{1},{fz(1)})({1},{1},{fz(1)})}; \addplot 3[-latex] coordinates {(0,0,0)(2,0,0)}node[below]{$x$}; \addplot 3[-latex] coordinates {(0,0,0)(0,2,0)}node[below]{$y$}; \addplot 3[-latex] coordinates {(0,0,0)(0,0,1.5)}node[left]{$z$}; \end {axis} \end {tikzpicture} \end {center} 
\item [
\protect ١١.\protect ٣١٧)
]
 $x^2+4z^2=16$ \begin {center} \begin {tikzpicture}[font=\small ,declare function={fx(\r ,\t )=4*\r *cos(\t );fz(\r ,\t )=2*\r *sin(\t );}] \pgfmathsetmacro {\ta }{0} \pgfmathsetmacro {\tb }{90} \pgfmathsetmacro {\tc }{180} \pgfmathsetmacro {\td }{270} \pgfmathsetmacro {\ky }{12} \begin {axis}[view/h=135,axis equal,axis lines=center,xtick={\empty },ytick={\empty },ztick={\empty },xlabel={$x$},ylabel={$y$},zlabel={$z$},xlabel style={anchor=east},ylabel style={anchor=west},zlabel style={anchor=east}, axis x line=none,axis y line=none,axis z line=none] \addplot 3[domain y=0:360] ({fx(2,y)},-\ky ,{fz(2,y)}); \addplot 3[domain y=0:360] ({fx(2,y)},0,{fz(2,y)}); \addplot 3[domain y=0:360] ({fx(2,y)},\ky ,{fz(2,y)}); \addplot 3[] coordinates {({fx(2,\ta )},-\ky ,{fz(2,\ta )})({fx(2,\ta )},\ky ,{fz(2,\ta )})}; \addplot 3[] coordinates {({fx(2,\tb )},-\ky ,{fz(2,\tb )})({fx(2,\tb )},\ky ,{fz(2,\tb )})}; \addplot 3[] coordinates {({fx(2,\tc )},-\ky ,{fz(2,\tc )})({fx(2,\tc )},\ky ,{fz(2,\tc )})}; \addplot 3[] coordinates {({fx(2,\td )},-\ky ,{fz(2,\td )})({fx(2,\td )},\ky ,{fz(2,\td )})}; \addplot 3[-latex] coordinates {(0,0,0)(2.5,0,0)}node[below]{$x$}; \addplot 3[-latex] coordinates {(0,0,0)(0,\ky +0.5,0)}node[below]{$y$}; \addplot 3[-latex] coordinates {(0,0,0)(0,0,2.5)}node[left]{$z$}; \end {axis} \end {tikzpicture} \end {center} 
\item [
\protect ١١.\protect ٣١٩)
]
 $z^2-y^2=1$ \begin {center} \begin {tikzpicture}[font=\small ,declare function={fz(\y )=sqrt((\y )^2+1);}] \pgfmathsetmacro {\kx }{2} \pgfmathsetmacro {\ky }{1} \begin {axis}[view/h=135,axis equal,axis lines=center,xtick={\empty },ytick={\empty },ztick={\empty },xlabel={$x$},ylabel={$y$},zlabel={$z$},xlabel style={anchor=east},ylabel style={anchor=west},zlabel style={anchor=east}, axis x line=none,axis y line=none,axis z line=none] \addplot 3[samples y=0,domain=-\ky :\ky ] (-\kx ,x,{fz(x)}); \addplot 3[samples y=0,domain=-\ky :\ky ] (\kx ,x,{fz(x)}); \addplot 3[] coordinates{(-\kx ,-\ky ,{fz(-\ky )}) (\kx ,-\ky ,{fz(-\ky )})}; \addplot 3[] coordinates{(-\kx ,\ky ,{fz(-\ky )}) (\kx ,\ky ,{fz(-\ky )})}; \addplot 3[samples y=0,domain=-\ky :\ky ] (-\kx ,x,{-fz(x)}); \addplot 3[samples y=0,domain=-\ky :\ky ] (\kx ,x,{-fz(x)}); \addplot 3[] coordinates{(-\kx ,-\ky ,{-fz(-\ky )}) (\kx ,-\ky ,{-fz(-\ky )})}; \addplot 3[] coordinates{(-\kx ,\ky ,{-fz(-\ky )}) (\kx ,\ky ,{-fz(-\ky )})}; \addplot 3[-latex] coordinates {(0,0,0)(2.5,0,0)}node[below]{$x$}; \addplot 3[-latex] coordinates {(0,0,0)(0,\ky +0.5,0)}node[below]{$y$}; \addplot 3[-latex] coordinates {(0,0,0)(0,0,2.5)}node[left]{$z$}; \end {axis} \end {tikzpicture} \end {center} 
\item [
\protect ١١.\protect ٣٢١)
]
 $9x^2+y^2+z^2=9$ \begin {center} \begin {tikzpicture}[font=\small ,declare function={f(\y )=sqrt(9-(\y )^2);g(\x )=sqrt(9-9*(\x )^2);h(\x )=3*sqrt(1-(\x )^2);}] \begin {axis}[view/h=135,axis equal,axis lines=center,xtick={\empty },ytick={\empty },ztick={\empty },xlabel={$x$},ylabel={$y$},zlabel={$z$},xlabel style={anchor=east},ylabel style={anchor=west},zlabel style={anchor=east}, axis x line=none,axis y line=none,axis z line=none] \addplot 3[samples y=0,domain=-3:3] (0,x,{f(x)}); \addplot 3[samples y=0,domain=-3:3] (0,x,{-f(x)}); \addplot 3[samples y=0,domain=-1:1] (x,0,{g(x)}); \addplot 3[samples y=0,domain=-1:1] (x,0,{-g(x)}); \addplot 3[samples y=0,domain=-1:1] (x,{h(x)},0); \addplot 3[samples y=0,domain=-1:1] (x,{-h(x)},0); \addplot 3[-latex] coordinates {(0,0,0)(2.5,0,0)}node[below]{$x$}; \addplot 3[-latex] coordinates {(0,0,0)(0,2+0.5,0)}node[below]{$y$}; \addplot 3[-latex] coordinates {(0,0,0)(0,0,2.5)}node[left]{$z$}; \end {axis} \end {tikzpicture} \end {center} 
\item [
\protect ١١.\protect ٣٢٣)
]
 $4x^2+9y^2+4z^2=36$ \begin {center} \begin {tikzpicture}[font=\small ,declare function={f(\y )=1/2*sqrt(36-9*(\y )^2);g(\x )=sqrt(9-(\x )^2);h(\x )=1/3*sqrt(36-4*(\x )^2);}] \begin {axis}[view/h=135,axis equal,axis lines=center,xtick={\empty },ytick={\empty },ztick={\empty },xlabel={$x$},ylabel={$y$},zlabel={$z$},xlabel style={anchor=east},ylabel style={anchor=west},zlabel style={anchor=east}, axis x line=none,axis y line=none,axis z line=none] \addplot 3[samples y=0,domain=-2:2,smooth] (0,x,{f(x)}); \addplot 3[samples y=0,domain=-2:2,smooth] (0,x,{-f(x)}); \addplot 3[samples y=0,domain=-3:3,smooth] (x,0,{g(x)}); \addplot 3[samples y=0,domain=-3:3,smooth] (x,0,{-g(x)}); \addplot 3[samples y=0,domain=-3:3,smooth] (x,{h(x)},0); \addplot 3[samples y=0,domain=-3:3,smooth] (x,{-h(x)},0); \addplot 3[-latex] coordinates {(0,0,0)(4,0,0)}node[below]{$x$}; \addplot 3[-latex] coordinates {(0,0,0)(0,5,0)}node[below]{$y$}; \addplot 3[-latex] coordinates {(0,0,0)(0,0,3.5)}node[left]{$z$}; \end {axis} \end {tikzpicture} \end {center} 
\item [
\protect ١١.\protect ٣٢٥)
]
 $z=x^2+4y^2$ \begin {center} \begin {tikzpicture}[font=\small ,declare function={fx(\r ,\t )=2*\r *cos(\t );fy(\r ,\t )=\r *sin(\t );hz(\x ,\y )=(\x )^2+4*(\y )^2;}] \pgfmathsetmacro {\ra }{1} \pgfmathsetmacro {\rb }{4} \begin {axis}[view/h=135,axis equal,axis lines=center,xtick={\empty },ytick={\empty },ztick={\empty },xlabel={$x$},ylabel={$y$},zlabel={$z$},xlabel style={anchor=east},ylabel style={anchor=west},zlabel style={anchor=east}, axis x line=none,axis y line=none,axis z line=none] \addplot 3[domain=-2:2,domain y=0:360,smooth,variable=\r ,variable y=\t ] ({fx(\ra ,t)},{fy(\ra ,t)},4); \addplot 3[domain=-2:2,domain y=-1:1,smooth] (0,{y},{hz(0,y)}); \addplot 3[samples y=0,domain=-2:2,domain=-2:2,smooth] ({x},0,{hz(x,0)}); \addplot 3[-latex] coordinates {(0,0,0)(1,0,0)}node[below]{$x$}; \addplot 3[-latex] coordinates {(0,0,0)(0,1,0)}node[below]{$y$}; \addplot 3[-latex] coordinates {(0,0,0)(0,0,5.5)}node[left]{$z$}; \end {axis} \end {tikzpicture} \end {center} 
\item [
\protect ١١.\protect ٣٢٧)
]
 $z=8-x^2-y^2$ \begin {center} \begin {tikzpicture}[font=\small ,declare function={fx(\r ,\t )=sqrt(8)*\r *cos(\t );fy(\r ,\t )=sqrt(8)*\r *sin(\t );hz(\x ,\y )=8-(\x )^2-(\y )^2;}] \pgfmathsetmacro {\ra }{2*sqrt(2)} \begin {axis}[view/h=135,axis equal,axis lines=center,xtick={\empty },ytick={\empty },ztick={\empty },xlabel={$x$},ylabel={$y$},zlabel={$z$},xlabel style={anchor=east},ylabel style={anchor=west},zlabel style={anchor=east}, axis x line=none,axis y line=none,axis z line=none] \addplot 3[domain=-\ra :\ra ,domain y=0:360,smooth,variable=\r ,variable y=\t ] ({fx(1,t)},{fy(1,t)},0); \addplot 3[domain=-\ra :\ra ,domain y=-\ra :\ra ,smooth] (0,{y},{hz(0,y)}); \addplot 3[samples y=0,domain=-\ra :\ra ,domain=-\ra :\ra ,smooth] ({x},0,{hz(x,0)}); \addplot 3[-latex] coordinates {(0,0,0)(4,0,0)}node[below]{$x$}; \addplot 3[-latex] coordinates {(0,0,0)(0,4,0)}node[below]{$y$}; \addplot 3[-latex] coordinates {(0,0,0)(0,0,9)}node[left]{$z$}; \end {axis} \end {tikzpicture} \end {center} 
\item [
\protect ١١.\protect ٣٢٩)
]
 $x=4-4y^2-z^2$ \begin {center} \begin {tikzpicture}[font=\small ,declare function={fy(\r ,\t )=\r *cos(\t );fz(\r ,\t )=2*\r *sin(\t );hx(\y ,\z )=4-4*(\y )^2-(\z )^2;}] \pgfmathsetmacro {\ra }{2} \begin {axis}[view/h=135,axis equal,axis lines=center,xtick={\empty },ytick={\empty },ztick={\empty },xlabel={$x$},ylabel={$y$},zlabel={$z$},xlabel style={anchor=east},ylabel style={anchor=west},zlabel style={anchor=east}, axis x line=none,axis y line=none,axis z line=none] \addplot 3[domain=-\ra :\ra ,domain y=0:360,smooth,variable=\r ,variable y=\t ] (0,{fy(1,t)},{fz(1,t)}); \addplot 3[domain=-2:2,domain y=-2:2,smooth] ({hx(0,y)},0,y); \addplot 3[samples y=0,domain=-1:1,domain=-1:1,smooth] ({hx(x,0)},x,0); \addplot 3[-latex] coordinates {(0,0,0)(5,0,0)}node[below]{$x$}; \addplot 3[-latex] coordinates {(0,0,0)(0,1.5,0)}node[below]{$y$}; \addplot 3[-latex] coordinates {(0,0,0)(0,0,2.5)}node[left]{$z$}; \end {axis} \end {tikzpicture} \end {center} 
\item [
\protect ١١.\protect ٣٣١)
]
 $x^2+y^2=z^2$ \begin {center} \begin {tikzpicture}[font=\small ,declare function={fx(\r ,\t )=\r *cos(\t );fy(\r ,\t )=\r *sin(\t );}] \pgfmathsetmacro {\ra }{2} \begin {axis}[view/h=135,axis equal,axis lines=center,xtick={\empty },ytick={\empty },ztick={\empty },xlabel={$x$},ylabel={$y$},zlabel={$z$},xlabel style={anchor=east},ylabel style={anchor=west},zlabel style={anchor=east}, axis x line=none,axis y line=none,axis z line=none] \addplot 3[domain=-\ra :\ra ,domain y=0:360,smooth,variable=\r ,variable y=\t ] ({fx(1,t)},{fy(1,t)},1); \addplot 3[domain=-\ra :\ra ,domain y=0:360,smooth,variable=\r ,variable y=\t ] ({fx(1,t)},{fy(1,t)},-1); \addplot 3[domain=-1:1,domain y=-1:1,smooth] (0,y,y); \addplot 3[domain=-1:1,domain y=-1:1,smooth] (0,y,-y); \addplot 3[samples y=0,domain=-1:1,domain y=-1:1,smooth] (x,0,x); \addplot 3[samples y=0,domain=-1:1,domain y=-1:1,smooth] (x,0,-x); \addplot 3[-latex] coordinates {(0,0,0)(2,0,0)}node[below]{$x$}; \addplot 3[-latex] coordinates {(0,0,0)(0,2,0)}node[below]{$y$}; \addplot 3[-latex] coordinates {(0,0,0)(0,0,2)}node[left]{$z$}; \end {axis} \end {tikzpicture} \end {center} 
\item [
\protect ١١.\protect ٣٣٣)
]
 $4x^2+9z^2=9y^2$ \begin {center} \begin {tikzpicture}[font=\small ,declare function={fx(\r ,\t )=3/2*\r *cos(\t );fz(\r ,\t )=\r *sin(\t );}] \pgfmathsetmacro {\ra }{2} \begin {axis}[view/h=135,axis equal,axis lines=center,xtick={\empty },ytick={\empty },ztick={\empty },xlabel={$x$},ylabel={$y$},zlabel={$z$},xlabel style={anchor=east},ylabel style={anchor=west},zlabel style={anchor=east}, axis x line=none,axis y line=none,axis z line=none] \addplot 3[domain=-\ra :\ra ,domain y=0:360,smooth,variable=\r ,variable y=\t ] ({fx(1,t)},1,{fz(1,t)}); \addplot 3[domain=-\ra :\ra ,domain y=0:360,smooth,variable=\r ,variable y=\t ] ({fx(1,t)},-1,{fz(1,t)}); \addplot 3[domain=-1:1,domain y=-1:1,smooth] (0,y,y); \addplot 3[domain=-1:1,domain y=-1:1,smooth] (0,y,-y); \addplot 3[-latex] coordinates {(0,0,0)(2,0,0)}node[below]{$x$}; \addplot 3[-latex] coordinates {(0,0,0)(0,2,0)}node[below]{$y$}; \addplot 3[-latex] coordinates {(0,0,0)(0,0,2)}node[left]{$z$}; \end {axis} \end {tikzpicture} \end {center} 
\item [
\protect ١١.\protect ٣٣٥)
]
 $x^2+y^2-z^2=1$ \begin {center} \begin {tikzpicture}[font=\small ,declare function={fy(\x )=sqrt(2-(\x )^2);hz(\x ,\y )=sqrt((\x )^2+(\y )^2-1);}] \pgfmathsetmacro {\ra }{sqrt(2)} \begin {axis}[view/h=135,axis equal,axis lines=center,xtick={\empty },ytick={\empty },ztick={\empty },xlabel={$x$},ylabel={$y$},zlabel={$z$},xlabel style={anchor=east},ylabel style={anchor=west},zlabel style={anchor=east}, axis x line=none,axis y line=none,axis z line=none] \addplot 3[samples y=0,domain=-sqrt(2):sqrt(2)] (x,{fy(x)},1); \addplot 3[samples y=0,domain=-sqrt(2):sqrt(2)] (x,{-fy(x)},1); \addplot 3[samples y=0,domain=-sqrt(2):sqrt(2)] (x,{fy(x)},-1); \addplot 3[samples y=0,domain=-sqrt(2):sqrt(2)] (x,{-fy(x)},-1); \addplot 3[domain=1:\ra ,domain y=1:\ra ,smooth] (0,y,{hz(0,y)}); \addplot 3[domain=1:\ra ,domain y=1:\ra ,smooth] (0,y,{-hz(0,y)}); \addplot 3[domain=1:\ra ,domain y=1:\ra ,smooth] (0,-y,{hz(0,y)}); \addplot 3[domain=1:\ra ,domain y=1:\ra ,smooth] (0,-y,{-hz(0,y)}); \addplot 3[samples y=0,domain=1:\ra ,domain y=1:\ra ,smooth] (x,0,{hz(x,0)}); \addplot 3[samples y=0,domain=1:\ra ,domain y=1:\ra ,smooth] (x,0,{-hz(x,0)}); \addplot 3[samples y=0,domain=1:\ra ,domain y=1:\ra ,smooth] (-x,0,{hz(x,0)}); \addplot 3[samples y=0,domain=1:\ra ,domain y=1:\ra ,smooth] (-x,0,{-hz(x,0)}); \addplot 3[-latex] coordinates {(0,0,0)(2.5,0,0)}node[below]{$x$}; \addplot 3[-latex] coordinates {(0,0,0)(0,2.5,0)}node[below]{$y$}; \addplot 3[-latex] coordinates {(0,0,0)(0,0,2.5)}node[left]{$z$}; \end {axis} \end {tikzpicture} \end {center} 
\item [
\protect ١١.\protect ٣٣٧)
]
 $\tfrac {y^2}{4}+\tfrac {z^2}{9}-\tfrac {x^2}{4}=1$ \begin {center} \begin {tikzpicture}[font=\small ,declare function={fz(\z )=3*sqrt(2-1/4*(\z )^2);hz(\x ,\y )=3/2*sqrt(4+(\x )^2-(\y )^2);}] \pgfmathsetmacro {\ra }{sqrt(2)} \begin {axis}[view/h=135,axis equal,axis lines=center,xtick={\empty },ytick={\empty },ztick={\empty },xlabel={$x$},ylabel={$y$},zlabel={$z$},xlabel style={anchor=east},ylabel style={anchor=west},zlabel style={anchor=east}, axis x line=none,axis y line=none,axis z line=none] \addplot 3[samples y=0,domain=-sqrt(8):sqrt(8)] (2,x,{fz(x)}); \addplot 3[samples y=0,domain=-sqrt(8):sqrt(8)] (2,x,{-fz(x)}); \addplot 3[samples y=0,domain=-sqrt(8):sqrt(8)] (-2,x,{fz(x)}); \addplot 3[samples y=0,domain=-sqrt(8):sqrt(8)] (-2,x,{-fz(x)}); \addplot 3[samples y=0,domain=-2:2,domain y=-2:2]({x},{0},{hz(x,0)}); \addplot 3[samples y=0,domain=-2:2,domain y=-2:2]({x},{0},{-hz(x,0)}); \addplot 3[domain=-2:2,domain y=-2:2]({0},{y},{hz(0,y)}); \addplot 3[domain=-2:2,domain y=-2:2]({0},{y},{-hz(0,y)}); \addplot 3[-latex] coordinates {(0,0,0)(5.5,0,0)}node[below]{$x$}; \addplot 3[-latex] coordinates {(0,0,0)(0,6,0)}node[below]{$y$}; \addplot 3[-latex] coordinates {(0,0,0)(0,0,6)}node[left]{$z$}; \end {axis} \end {tikzpicture} \end {center} 
\item [
\protect ١١.\protect ٣٣٩)
]
 $z^2-x^2-y^2=1$ \begin {center} \begin {tikzpicture}[font=\small ,declare function={fz(\x ,\y )=sqrt(1+(\x )^2+(\y )^2);hy(\x )=sqrt(8-(\x )^2);}] \pgfmathsetmacro {\ra }{sqrt(2)} \begin {axis}[view/h=135,axis equal,axis lines=center,xtick={\empty },ytick={\empty },ztick={\empty },xlabel={$x$},ylabel={$y$},zlabel={$z$},xlabel style={anchor=east},ylabel style={anchor=west},zlabel style={anchor=east}, axis x line=none,axis y line=none,axis z line=none] \addplot 3[samples y=0,domain=-sqrt(8):sqrt(8)] (x,{hy(x)},3); \addplot 3[samples y=0,domain=-sqrt(8):sqrt(8)] (x,{-hy(x)},3); \addplot 3[samples y=0,domain=-sqrt(8):sqrt(8)] (x,{hy(x)},-3); \addplot 3[samples y=0,domain=-sqrt(8):sqrt(8)] (x,{-hy(x)},-3); \addplot 3[samples y=0,domain=-sqrt(8):sqrt(8)] (x,0,{fz(x,0)}); \addplot 3[samples y=0,domain=-sqrt(8):sqrt(8)] (x,0,{-fz(x,0)}); \addplot 3[domain=-sqrt(8):sqrt(8)] (0,y,{fz(0,y)}); \addplot 3[domain=-sqrt(8):sqrt(8)] (0,y,{-fz(0,y)}); \addplot 3[-latex] coordinates {(0,0,0)(5.5,0,0)}node[below]{$x$}; \addplot 3[-latex] coordinates {(0,0,0)(0,6,0)}node[below]{$y$}; \addplot 3[-latex] coordinates {(0,0,0)(0,0,6)}node[left]{$z$}; \end {axis} \end {tikzpicture} \end {center} 
\item [
\protect ١١.\protect ٣٤١)
]
 $x^2-y^2-\tfrac {z^2}{4}=1$ \begin {center} \begin {tikzpicture}[font=\small ,declare function={fx(\y ,\z )=sqrt(1+(\y )^2+1/4*(\z )^2);hz(\y )=2*sqrt(8-(\y )^2);}] \pgfmathsetmacro {\ra }{sqrt(8)} \begin {axis}[view/h=135,axis equal,axis lines=center,xtick={\empty },ytick={\empty },ztick={\empty },xlabel={$x$},ylabel={$y$},zlabel={$z$},xlabel style={anchor=east},ylabel style={anchor=west},zlabel style={anchor=east}, axis x line=none,axis y line=none,axis z line=none] \addplot 3[samples y=0,domain=-sqrt(8):sqrt(8)] (3,x,{hz(x)}); \addplot 3[samples y=0,domain=-sqrt(8):sqrt(8)] (3,x,{-hz(x)}); \addplot 3[samples y=0,domain=-sqrt(8):sqrt(8)] (-3,x,{hz(x)}); \addplot 3[samples y=0,domain=-sqrt(8):sqrt(8)] (-3,x,{-hz(x)}); \addplot 3[samples y=0,domain=-\ra :\ra ]({fx(x,0)},{x},{0}); \addplot 3[samples y=0,domain=-\ra :\ra ]({-fx(x,0)},{x},{0}); \addplot 3[domain y=-6:6]({fx(0,y)},{0},{y}); \addplot 3[domain y=-6:6]({-fx(0,y)},{0},{y}); \addplot 3[-latex] coordinates {(0,0,0)(5.5,0,0)}node[below]{$x$}; \addplot 3[-latex] coordinates {(0,0,0)(0,6,0)}node[below]{$y$}; \addplot 3[-latex] coordinates {(0,0,0)(0,0,6)}node[left]{$z$}; \end {axis} \end {tikzpicture} \end {center} 
\item [
\protect ١١.\protect ٣٤٣)
]
 $y^2-x^2=z$ \begin {center} \begin {tikzpicture}[font=\small ,declare function={fz(\x ,\y )=(\y )^2-(\x )^2;}] \pgfmathsetmacro {\ra }{1} \pgfmathsetmacro {\rb }{1} \begin {axis}[view/h=120,axis equal,axis lines=center,xtick={\empty },ytick={\empty },ztick={\empty },xlabel={$x$},ylabel={$y$},zlabel={$z$},xlabel style={anchor=east},ylabel style={anchor=west},zlabel style={anchor=east}, axis x line=none,axis y line=none,axis z line=none] \addplot 3[samples y=0,domain=-\ra :\ra ] (x,0,{fz(x,0)}); \addplot 3[samples y=0,domain=-\rb :\rb ] (x,1,{fz(x,1)}); \addplot 3[samples y=0,domain=-\rb :\rb ] (x,-1,{fz(x,-1)}); \addplot 3[domain y=-1:1] (0,y,{fz(0,y)}); \addplot 3[domain y=-1:1] (1,y,{fz(1,y)}); \addplot 3[domain y=-1:1] (-1,y,{fz(-1,y)}); \addplot 3[-latex] coordinates {(0,0,0)(2.5,0,0)}node[below]{$x$}; \addplot 3[-latex] coordinates {(0,0,0)(0,2,0)}node[below]{$y$}; \addplot 3[-latex] coordinates {(0,0,0)(0,0,1.5)}node[left]{$z$}; \end {axis} \end {tikzpicture} \end {center} 
\item [
\protect ١١.\protect ٣٤٥)
]
 $x^2+y^2+z^2=4$ \begin {center} \begin {tikzpicture}[font=\small ,declare function={fz(\x ,\y )=sqrt(4-(\x )^2-(\y )^2);fy(\x )=sqrt(4-(\x )^2);}] \pgfmathsetmacro {\ra }{2} \begin {axis}[view/h=120,axis equal,axis lines=center,xtick={\empty },ytick={\empty },ztick={\empty },xlabel={$x$},ylabel={$y$},zlabel={$z$},xlabel style={anchor=east},ylabel style={anchor=west},zlabel style={anchor=east}, axis x line=none,axis y line=none,axis z line=none] \addplot 3[samples y=0,domain=-\ra :\ra ] (x,0,{fz(x,0)}); \addplot 3[samples y=0,domain=-\ra :\ra ] (x,0,{-fz(x,0)}); \addplot 3[domain y=-\ra :\ra ] (0,y,{fz(0,y)}); \addplot 3[domain y=-\ra :\ra ] (0,y,{-fz(0,y)}); \addplot 3[samples y=0,domain=-\ra :\ra ] (x,{fy(x)},0); \addplot 3[samples y=0,domain=-\ra :\ra ] (x,{-fy(x)},0); \addplot 3[-latex] coordinates {(0,0,0)(3,0,0)}node[below]{$x$}; \addplot 3[-latex] coordinates {(0,0,0)(0,2.5,0)}node[below]{$y$}; \addplot 3[-latex] coordinates {(0,0,0)(0,0,2.5)}node[left]{$z$}; \end {axis} \end {tikzpicture} \end {center} 
\item [
\protect ١١.\protect ٣٤٧)
]
 $z=1+y^2-x^2$ \begin {center} \begin {tikzpicture}[font=\small ,declare function={fz(\x ,\y )=1+(\y )^2-(\x )^2;}] \pgfmathsetmacro {\ra }{1} \begin {axis}[view/h=120,axis equal,axis lines=center,xtick={\empty },ytick={\empty },ztick={\empty },xlabel={$x$},ylabel={$y$},zlabel={$z$},xlabel style={anchor=east},ylabel style={anchor=west},zlabel style={anchor=east}, axis x line=none,axis y line=none,axis z line=none] \addplot 3[samples y=0,domain=-\ra :\ra ] (x,1,{fz(x,1)}); \addplot 3[samples y=0,domain=-\ra :\ra ] (x,-1,{fz(x,-1)}); \addplot 3[domain y=-\ra :\ra ] (0,y,{fz(0,y)}); \addplot 3[domain y=-\ra :\ra ] (1,y,{fz(1,y)}); \addplot 3[domain y=-\ra :\ra ] (-1,y,{fz(-1,y)}); \addplot 3[-latex] coordinates {(0,0,0)(\ra +0.5,0,0)}node[below]{$x$}; \addplot 3[-latex] coordinates {(0,0,0)(0,\ra ,0)}node[below]{$y$}; \addplot 3[-latex] coordinates {(0,0,0)(0,0,1.5)}node[left]{$z$}; \end {axis} \end {tikzpicture} \end {center} 
\item [
\protect ١١.\protect ٣٤٩)
]
 $y=-x^2-z^2$ \begin {center} \begin {tikzpicture}[font=\small ,declare function={fy(\x ,\z )=-(\x )^2-(\z )^2;fz(\x )=sqrt(1-(\x )^2);}] \pgfmathsetmacro {\ra }{1} \begin {axis}[view/h=120,axis equal,axis lines=center,xtick={\empty },ytick={\empty },ztick={\empty },xlabel={$x$},ylabel={$y$},zlabel={$z$},xlabel style={anchor=east},ylabel style={anchor=west},zlabel style={anchor=east}, axis x line=none,axis y line=none,axis z line=none] \addplot 3[samples y=0,domain=-\ra :\ra ] (x,{fy(x,0)},0); \addplot 3[domain y=-\ra :\ra ] (0,{fy(0,y)},y); \addplot 3[samples y=0,domain=-\ra :\ra ] (x,-1,{fz(x)}); \addplot 3[samples y=0,domain=-\ra :\ra ] (x,-1,{-fz(x)}); \addplot 3[-latex] coordinates {(0,0,0)(\ra ,0,0)}node[below]{$x$}; \addplot 3[-latex] coordinates {(0,0,0)(0,0.5,0)}node[below]{$y$}; \addplot 3[-latex] coordinates {(0,0,0)(0,0,\ra )}node[left]{$z$}; \end {axis} \end {tikzpicture} \end {center} 
\item [
\protect ١١.\protect ٣٥١)
]
 $16x^2+4y^2=1$ \begin {center} \begin {tikzpicture}[font=\small ,declare function={fy(\x )=1/2*sqrt(1-16*(\x )^2);}] \pgfmathsetmacro {\ra }{1/4} \pgfmathsetmacro {\rb }{1/2} \begin {axis}[view/h=120,axis equal,axis lines=center,xtick={\empty },ytick={\empty },ztick={\empty },xlabel={$x$},ylabel={$y$},zlabel={$z$},xlabel style={anchor=east},ylabel style={anchor=west},zlabel style={anchor=east}, axis x line=none,axis y line=none,axis z line=none] \addplot 3[samples y=0,domain=-\ra :\ra ] (x,{fy(x)},1); \addplot 3[samples y=0,domain=-\ra :\ra ] (x,{-fy(x)},1); \addplot 3[samples y=0,domain=-\ra :\ra ] (x,{fy(x)},0); \addplot 3[samples y=0,domain=-\ra :\ra ] (x,{-fy(x)},0); \addplot 3[samples y=0,domain=-\ra :\ra ] (x,{fy(x)},-1); \addplot 3[samples y=0,domain=-\ra :\ra ] (x,{-fy(x)},-1); \addplot 3[]coordinates {(\ra ,0,-1)(\ra ,0,1)}; \addplot 3[]coordinates {(-\ra ,0,-1)(-\ra ,0,1)}; \addplot 3[]coordinates {(0,\rb ,-1)(0,\rb ,1)}; \addplot 3[]coordinates {(0,-\rb ,-1)(0,-\rb ,1)}; \addplot 3[-latex] coordinates {(0,0,0)(1.5,0,0)}node[below]{$x$}; \addplot 3[-latex] coordinates {(0,0,0)(0,1,0)}node[below]{$y$}; \addplot 3[-latex] coordinates {(0,0,0)(0,0,1.5)}node[left]{$z$}; \end {axis} \end {tikzpicture} \end {center} 
\item [
\protect ١١.\protect ٣٥٣)
]
 $x^2+y^2-z^2=4$ \begin {center} \begin {tikzpicture}[font=\small ,declare function={fz(\x ,\y )=sqrt((\x )^2+(\y )^2-4);fy(\x )=sqrt(9-\x ^2);}] \pgfmathsetmacro {\ra }{2} \pgfmathsetmacro {\rb }{3} \pgfmathsetmacro {\rc }{sqrt(5)} \begin {axis}[view/h=120,axis equal,axis lines=center,xtick={\empty },ytick={\empty },ztick={\empty },xlabel={$x$},ylabel={$y$},zlabel={$z$},xlabel style={anchor=east},ylabel style={anchor=west},zlabel style={anchor=east}, axis x line=none,axis y line=none,axis z line=none] \addplot 3[samples y=0,domain=\ra :\rb ] (x,0,{fz(x,0)}); \addplot 3[samples y=0,domain=\ra :\rb ] (x,0,{-fz(x,0)}); \addplot 3[samples y=0,domain=\ra :\rb ] (-x,0,{fz(x,0)}); \addplot 3[samples y=0,domain=\ra :\rb ] (-x,0,{-fz(x,0)}); \addplot 3[domain y=\ra :\rb ] (0,y,{fz(0,y)}); \addplot 3[domain y=\ra :\rb ] (0,y,{-fz(0,y)}); \addplot 3[domain y=\ra :\rb ] (0,-y,{fz(0,y)}); \addplot 3[domain y=\ra :\rb ] (0,-y,{-fz(0,y)}); \addplot 3[samples y=0,domain=-3:3](x,{fy(x)},\rc ); \addplot 3[samples y=0,domain=-3:3](x,{-fy(x)},\rc ); \addplot 3[samples y=0,domain=-3:3](x,{fy(x)},-\rc ); \addplot 3[samples y=0,domain=-3:3](x,{-fy(x)},-\rc ); \addplot 3[-latex] coordinates {(0,0,0)(2.5,0,0)}node[below]{$x$}; \addplot 3[-latex] coordinates {(0,0,0)(0,2.5,0)}node[below]{$y$}; \addplot 3[-latex] coordinates {(0,0,0)(0,0,2.5)}node[left]{$z$}; \end {axis} \end {tikzpicture} \end {center} 
\item [
\protect ١١.\protect ٣٥٥)
]
 $x^2+z^2=y$ \begin {center} \begin {tikzpicture}[font=\small ,declare function={fy(\x ,\z )=\x ^2+\z ^2;fz(\x )=sqrt(1-\x ^2);}] \pgfmathsetmacro {\ra }{1} \begin {axis}[view/h=120,axis equal,axis lines=center,xtick={\empty },ytick={\empty },ztick={\empty },xlabel={$x$},ylabel={$y$},zlabel={$z$},xlabel style={anchor=east},ylabel style={anchor=west},zlabel style={anchor=east}, axis x line=none,axis y line=none,axis z line=none] \addplot 3[samples y=0,domain=-\ra :\ra ] (x,{fy(x,0)},0); \addplot 3[domain y=-\ra :\ra ] (0,{fy(0,y)},y); \addplot 3[samples y=0,domain=-\ra :\ra ](x,1,{fz(x)}); \addplot 3[samples y=0,domain=-\ra :\ra ](x,1,{-fz(x)}); \addplot 3[-latex] coordinates {(0,0,0)(1,0,0)}node[below]{$x$}; \addplot 3[-latex] coordinates {(0,0,0)(0,2,0)}node[below]{$y$}; \addplot 3[-latex] coordinates {(0,0,0)(0,0,0.5)}node[left]{$z$}; \end {axis} \end {tikzpicture} \end {center} 
\item [
\protect ١١.\protect ٣٥٧)
]
 $x^2+z^2=1$ \begin {center} \begin {tikzpicture}[font=\small ,declare function={fz(\x )=sqrt(1-\x ^2);}] \pgfmathsetmacro {\ra }{1} \begin {axis}[view/h=120,axis equal,axis lines=center,xtick={\empty },ytick={\empty },ztick={\empty },xlabel={$x$},ylabel={$y$},zlabel={$z$},xlabel style={anchor=east},ylabel style={anchor=west},zlabel style={anchor=east}, axis x line=none,axis y line=none,axis z line=none] \addplot 3[samples y=0,domain=-\ra :\ra ] (x,1,{fz(x)}); \addplot 3[samples y=0,domain=-\ra :\ra ] (x,1,{-fz(x)}); \addplot 3[samples y=0,domain=-\ra :\ra ] (x,-1,{fz(x)}); \addplot 3[samples y=0,domain=-\ra :\ra ] (x,-1,{-fz(x)}); \addplot 3[]coordinates {(-1,-1,0)(-1,1,0)}; \addplot 3[]coordinates {(1,-1,0)(1,1,0)}; \addplot 3[]coordinates {(0,-1,1)(0,1,1)}; \addplot 3[]coordinates {(0,-1,-1)(0,1,-1)}; \addplot 3[-latex] coordinates {(0,0,0)(2,0,0)}node[below]{$x$}; \addplot 3[-latex] coordinates {(0,0,0)(0,2,0)}node[below]{$y$}; \addplot 3[-latex] coordinates {(0,0,0)(0,0,1.25)}node[left]{$z$}; \end {axis} \end {tikzpicture} \end {center} 
\item [
\protect ١١.\protect ٣٥٩)
]
 $16y^2+9z^2=4x^2$ \begin {center} \begin {tikzpicture}[font=\small ,declare function={fx(\y ,\z )=1/2*sqrt(16*\y ^2+9*\z ^2);fz(\y )=1/3*sqrt(9-16*\y ^2);}] \pgfmathsetmacro {\ra }{1} \pgfmathsetmacro {\rb }{3/4} \begin {axis}[view/h=120,axis equal,axis lines=center,xtick={\empty },ytick={\empty },ztick={\empty },xlabel={$x$},ylabel={$y$},zlabel={$z$},xlabel style={anchor=east},ylabel style={anchor=west},zlabel style={anchor=east}, axis x line=none,axis y line=none,axis z line=none] \addplot 3[domain y=-\ra :\ra ] ({fx(0,y)},0,y); \addplot 3[domain y=-\ra :\ra ] ({-fx(0,y)},0,y); \addplot 3[samples y=0,domain=-\rb :\rb ] ({fx(x,0)},x,0); \addplot 3[samples y=0,domain=-\rb :\rb ] ({-fx(x,0)},x,0); \addplot 3[samples y=0,domain=-\rb :\rb ] (1.5,x,{fz(x)}); \addplot 3[samples y=0,domain=-\rb :\rb ] (1.5,x,{-fz(x)}); \addplot 3[samples y=0,domain=-\rb :\rb ] (-1.5,x,{fz(x)}); \addplot 3[samples y=0,domain=-\rb :\rb ] (-1.5,x,{-fz(x)}); \addplot 3[-latex] coordinates {(0,0,0)(2,0,0)}node[below]{$x$}; \addplot 3[-latex] coordinates {(0,0,0)(0,2,0)}node[below]{$y$}; \addplot 3[-latex] coordinates {(0,0,0)(0,0,1.25)}node[left]{$z$}; \end {axis} \end {tikzpicture} \end {center} 
\item [
\protect ١١.\protect ٣٦١)
]
 $9x^2+4y^2+z^2=36$ \begin {center} \begin {tikzpicture}[font=\small ,declare function={fz(\x ,\y )=sqrt(36-9*(\x )^2-4*(\y )^2);fy(\x )=1/2*sqrt(36-9*\x ^2);}] \pgfmathsetmacro {\ra }{3} \pgfmathsetmacro {\rb }{2} \begin {axis}[view/h=120,axis equal,axis lines=center,xtick={\empty },ytick={\empty },ztick={\empty },xlabel={$x$},ylabel={$y$},zlabel={$z$},xlabel style={anchor=east},ylabel style={anchor=west},zlabel style={anchor=east}, axis x line=none,axis y line=none,axis z line=none] \addplot 3[domain y=-\ra :\ra ] (0,y,{fz(0,y)}); \addplot 3[domain y=-\ra :\ra ] (0,y,{-fz(0,y)}); \addplot 3[samples y=0,domain=-\rb :\rb ] (x,0,{fz(x,0)}); \addplot 3[samples y=0,domain=-\rb :\rb ] (x,0,{-fz(x,0)}); \addplot 3[samples y=0,domain=-\rb :\rb ](x,{fy(x)},0); \addplot 3[samples y=0,domain=-\rb :\rb ](x,{-fy(x)},0); \addplot 3[-latex] coordinates {(0,0,0)(6,0,0)}node[below]{$x$}; \addplot 3[-latex] coordinates {(0,0,0)(0,5,0)}node[below]{$y$}; \addplot 3[-latex] coordinates {(0,0,0)(0,0,8)}node[left]{$z$}; \end {axis} \end {tikzpicture} \end {center} 
\item [
\protect ١١.\protect ٣٦٣)
]
 $x^2+y^2-16z^2=16$ \begin {center} \pgfmathsetmacro {\ra }{4} \pgfmathsetmacro {\rb }{8} \pgfmathsetmacro {\rc }{64} \pgfmathsetmacro {\rd }{sqrt(3)} \begin {tikzpicture}[font=\small ,declare function={fz(\x ,\y )=1/4*sqrt(\x ^2+\y ^2-16);fy(\x )=sqrt(\rc -\x ^2);}] \begin {axis}[view/h=120,axis equal,axis lines=center,xtick={\empty },ytick={\empty },ztick={\empty },xlabel={$x$},ylabel={$y$},zlabel={$z$},xlabel style={anchor=east},ylabel style={anchor=west},zlabel style={anchor=east}, axis x line=none,axis y line=none,axis z line=none] \addplot 3[domain y=\ra :\rb ] (0,y,{fz(0,y)}); \addplot 3[domain y=\ra :\rb ] (0,y,{-fz(0,y)}); \addplot 3[domain y=\ra :\rb ] (0,-y,{fz(0,y)}); \addplot 3[domain y=\ra :\rb ] (0,-y,{-fz(0,y)}); \addplot 3[samples y=0,domain=\ra :\rb ] (x,0,{fz(x,0)}); \addplot 3[samples y=0,domain=\ra :\rb ] (x,0,{-fz(x,0)}); \addplot 3[samples y=0,domain=\ra :\rb ] (-x,0,{fz(x,0)}); \addplot 3[samples y=0,domain=\ra :\rb ] (-x,0,{-fz(x,0)}); \addplot 3[samples y=0,domain=-\rb :\rb ](x,{fy(x)},\rd ); \addplot 3[samples y=0,domain=-\rb :\rb ](x,{-fy(x)},\rd ); \addplot 3[samples y=0,domain=-\rb :\rb ](x,{fy(x)},-\rd ); \addplot 3[samples y=0,domain=-\rb :\rb ](x,{-fy(x)},-\rd ); \addplot 3[-latex] coordinates {(0,0,0)(\rb +4,0,0)}node[below]{$x$}; \addplot 3[-latex] coordinates {(0,0,0)(0,\rb +2,0)}node[below]{$y$}; \addplot 3[-latex] coordinates {(0,0,0)(0,0,8)}node[left]{$z$}; \end {axis} \end {tikzpicture} \end {center} 
\item [
\protect ١١.\protect ٣٦٥)
]
 $z=-x^2-y^2$ \begin {center} \pgfmathsetmacro {\ra }{1} \begin {tikzpicture}[font=\small ,declare function={fz(\x ,\y )=-\x ^2-\y ^2;fy(\x )=sqrt(1-\x ^2);}] \begin {axis}[view/h=135,axis equal,axis lines=center,xtick={\empty },ytick={\empty },ztick={\empty },xlabel={$x$},ylabel={$y$},zlabel={$z$},xlabel style={anchor=east},ylabel style={anchor=west},zlabel style={anchor=east}, axis x line=none,axis y line=none,axis z line=none] \addplot 3[domain y=-\ra :\ra ] (0,y,{fz(0,y)}); \addplot 3[samples y=0,domain=-\ra :\ra ] (x,0,{fz(x,0)}); \addplot 3[samples y=0,domain=-\ra :\ra ](x,{fy(x)},-1); \addplot 3[samples y=0,domain=-\ra :\ra ](x,{-fy(x)},-1); \addplot 3[-latex] coordinates {(0,0,0)(1.5,0,0)}node[below]{$x$}; \addplot 3[-latex] coordinates {(0,0,0)(0,1.5,0)}node[below]{$y$}; \addplot 3[-latex] coordinates {(0,0,0)(0,0,0.5)}node[left]{$z$}; \end {axis} \end {tikzpicture} \end {center} 
\item [
\protect ١١.\protect ٣٦٧)
]
 $x^2-4y^2=1$ \begin {center} \pgfmathsetmacro {\ra }{1} \pgfmathsetmacro {\rb }{1} \begin {tikzpicture}[font=\small ,declare function={fx(\y )=sqrt(1+4*\y ^2);}] \begin {axis}[view/h=135,axis equal,axis lines=center,xtick={\empty },ytick={\empty },ztick={\empty },xlabel={$x$},ylabel={$y$},zlabel={$z$},xlabel style={anchor=east},ylabel style={anchor=west},zlabel style={anchor=east}, axis x line=none,axis y line=none,axis z line=none] \addplot 3[domain y=-\ra :\ra ] ({fx(y)},y,\rb ); \addplot 3[domain y=-\ra :\ra ] ({-fx(y)},y,\rb ); \addplot 3[domain y=-\ra :\ra ] ({fx(y)},y,-\rb ); \addplot 3[domain y=-\ra :\ra ] ({-fx(y)},y,-\rb ); \addplot 3[]coordinates {({fx(-\ra )},{-\ra },{-\rb }) ({fx(-\ra )},{-\ra },{\rb })}; \addplot 3[]coordinates {({fx(\ra )},{\ra },{-\rb }) ({fx(\ra )},{\ra },{\rb })}; \addplot 3[]coordinates {({-fx(-\ra )},{-\ra },{-\rb }) ({-fx(-\ra )},{-\ra },{\rb })}; \addplot 3[]coordinates {({-fx(\ra )},{\ra },{-\rb }) ({-fx(\ra )},{\ra },{\rb })}; \addplot 3[]coordinates {({fx(0)},{0},{-\rb }) ({fx(0)},{0},{\rb })}; \addplot 3[]coordinates {({-fx(0)},{0},{-\rb }) ({-fx(0)},{0},{\rb })}; \addplot 3[-latex] coordinates {(0,0,0)(4,0,0)}node[below]{$x$}; \addplot 3[-latex] coordinates {(0,0,0)(0,3,0)}node[below]{$y$}; \addplot 3[-latex] coordinates {(0,0,0)(0,0,2)}node[left]{$z$}; \end {axis} \end {tikzpicture} \end {center} 
\item [
\protect ١١.\protect ٣٦٩)
]
 $4y^2+z^2-4x^2=4$ \begin {center} \begin {tikzpicture}[font=\small ,declare function={fx(\y ,\z )=1/2*sqrt(4*(\y )^2+(\z )^2-4);fz(\x ,\y )=2*sqrt(1+(\x )^2-(\y )^2);}] \pgfmathsetmacro {\ra }{3} \pgfmathsetmacro {\rb }{3/2-0.001} \pgfmathsetmacro {\rc }{sqrt(5)/2} \begin {axis}[view/h=135,axis equal,axis lines=center,xtick={\empty },ytick={\empty },ztick={\empty },xlabel={$x$},ylabel={$y$},zlabel={$z$},xlabel style={anchor=east},ylabel style={anchor=west},zlabel style={anchor=east}, axis x line=none,axis y line=none,axis z line=none] \addplot 3[domain y=2:\ra ] ({fx(0,y)},0,y); \addplot 3[domain y=2:\ra ] ({-fx(0,y)},0,y); \addplot 3[domain y=2:\ra ] ({fx(0,y)},0,-y); \addplot 3[domain y=2:\ra ] ({-fx(0,y)},0,-y); \addplot 3[samples y=0,domain=1:\rb ] ({fx(x,0)},x,0); \addplot 3[samples y=0,domain=1:\rb ] ({-fx(x,0)},x,0); \addplot 3[samples y=0,domain=1:\rb ] ({fx(x,0)},-x,0); \addplot 3[samples y=0,domain=1:\rb ] ({-fx(x,0)},-x,0); \addplot 3[domain y=-\rb :\rb ]({\rc },{y},{fz(\rc ,y)}); \addplot 3[domain y=-\rb :\rb ]({\rc },{y},{-fz(\rc ,y)}); \addplot 3[domain y=-\rb :\rb ]({-\rc },{y},{fz(-\rc ,y)}); \addplot 3[domain y=-\rb :\rb ]({-\rc },{y},{-fz(-\rc ,y)}); \addplot 3[-latex] coordinates {(0,0,0)(2.5,0,0)}node[below]{$x$}; \addplot 3[-latex] coordinates {(0,0,0)(0,4.5,0)}node[below]{$y$}; \addplot 3[-latex] coordinates {(0,0,0)(0,0,4.5)}node[left]{$z$}; \end {axis} \end {tikzpicture} \end {center} 
\item [
\protect ١١.\protect ٣٧١)
]
 $x^2+y^2=z$ \begin {center} \begin {tikzpicture}[font=\small ,declare function={fz(\x ,\y )=\x ^2+\y ^2;fy(\x )=sqrt(1-\x ^2);}] \pgfmathsetmacro {\ra }{1} \begin {axis}[view/h=135,axis equal,axis lines=center,xtick={\empty },ytick={\empty },ztick={\empty },xlabel={$x$},ylabel={$y$},zlabel={$z$},xlabel style={anchor=east},ylabel style={anchor=west},zlabel style={anchor=east}, axis x line=none,axis y line=none,axis z line=none] \addplot 3[samples y=0,domain=-\ra :\ra ] (x,0,{fz(x,0)}); \addplot 3[domain y=-\ra :\ra ] (0,y,{fz(0,y)}); \addplot 3[samples y=0,domain=-\ra :\ra ](x,{fy(x)},1); \addplot 3[samples y=0,domain=-\ra :\ra ](x,{-fy(x)},1); \addplot 3[-latex] coordinates {(0,0,0)(0.5,0,0)}node[below]{$x$}; \addplot 3[-latex] coordinates {(0,0,0)(0,0.5,0)}node[below]{$y$}; \addplot 3[-latex] coordinates {(0,0,0)(0,0,1.75)}node[left]{$z$}; \end {axis} \end {tikzpicture} \end {center} 
\item [
\protect ١١.\protect ٣٧٣)
]
 $yz=1$ \begin {center} \begin {tikzpicture}[font=\small ,declare function={fz(\y )=1/\y ;}] \pgfmathsetmacro {\ra }{0.3} \pgfmathsetmacro {\rb }{1/\ra } \pgfmathsetmacro {\rc }{1} \begin {axis}[view/h=135,axis equal,axis lines=center,xtick={\empty },ytick={\empty },ztick={\empty },xlabel={$x$},ylabel={$y$},zlabel={$z$},xlabel style={anchor=east},ylabel style={anchor=west},zlabel style={anchor=east}, axis x line=none,axis y line=none,axis z line=none] \addplot 3[domain y=\ra :\rb ] (\rc ,\y ,{fz(y)}); \addplot 3[domain y=\ra :\rb ] (-\rc ,\y ,{fz(y)}); \addplot 3[]coordinates {(\rc ,\ra ,{fz(\ra )})(-\rc ,\ra ,{fz(\ra )})}; \addplot 3[]coordinates {(\rc ,\rb ,{fz(\rb )})(-\rc ,\rb ,{fz(\rb )})}; \addplot 3[domain y=-\ra :-\rb ] (\rc ,\y ,{fz(y)}); \addplot 3[domain y=-\ra :-\rb ] (-\rc ,\y ,{fz(y)}); \addplot 3[]coordinates {(\rc ,-\ra ,{fz(-\ra )})(-\rc ,-\ra ,{-fz(\ra )})}; \addplot 3[]coordinates {(\rc ,-\rb ,{fz(-\rb )})(-\rc ,-\rb ,{-fz(\rb )})}; \addplot 3[-latex] coordinates {(0,0,0)(\rb +0.5,0,0)}node[below]{$x$}; \addplot 3[-latex] coordinates {(0,0,0)(0,\rb +0.5,0)}node[below]{$y$}; \addplot 3[-latex] coordinates {(0,0,0)(0,0,\rb +0.5)}node[left]{$z$}; \end {axis} \end {tikzpicture} \end {center} 
\item [
\protect ١١.\protect ٣٧٥)
]
 $9x^2+16y^2=4z^2$ \begin {center} \begin {tikzpicture}[font=\small ,declare function={fz(\x ,\y )=1/2*sqrt(9*\x ^2+16*\y ^2);fy(\x )=1/4*sqrt(16-9*\x ^2);}] \pgfmathsetmacro {\ra }{1} \pgfmathsetmacro {\rb }{4/3} \begin {axis}[view/h=135,axis equal,axis lines=center,xtick={\empty },ytick={\empty },ztick={\empty },xlabel={$x$},ylabel={$y$},zlabel={$z$},xlabel style={anchor=east},ylabel style={anchor=west},zlabel style={anchor=east}, axis x line=none,axis y line=none,axis z line=none] \addplot 3[domain y=-\ra :\ra ] (0,\y ,{fz(0,y)}); \addplot 3[domain y=-\ra :\ra ] (0,\y ,{-fz(0,y)}); \addplot 3[samples y=0,domain=-\rb :\rb ] (\x ,0,{fz(x,0)}); \addplot 3[samples y=0,domain=-\rb :\rb ] (\x ,0,{-fz(x,0)}); \addplot 3[samples y=0,domain=-\rb :\rb ] (\x ,{fy(x)},2); \addplot 3[samples y=0,domain=-\rb :\rb ] (\x ,{-fy(x)},2); \addplot 3[samples y=0,domain=-\rb :\rb ] (\x ,{fy(x)},-2); \addplot 3[samples y=0,domain=-\rb :\rb ] (\x ,{-fy(x)},-2); \addplot 3[-latex] coordinates {(0,0,0)(\ra +0.5,0,0)}node[below]{$x$}; \addplot 3[-latex] coordinates {(0,0,0)(0,\ra +0.5,0)}node[right]{$y$}; \addplot 3[-latex] coordinates {(0,0,0)(0,0,\ra +2.5)}node[left]{$z$}; \end {axis} \end {tikzpicture} \end {center} 
\item [
\protect ١١.\protect ٣٧٧)
]
 (ا) \عددی {\tfrac {2\pi (9-c^2)}{9}}، (ب) \عددی {8\pi }، (ج) \عددی {\tfrac {4\pi a b c}{3}} 
\item [
\protect ١١.\protect ٣٨١)
]
 راس \عددی {(0,y_1,cy_1^2/b^22)}،\\ ماسکہ \عددی {(0,y_1,cy_1^2/b^2-a^2/(4c))} 
\end {description}
 {\urduTechTermsfont {حصہ}} \protect ١١.\protect ٧\hskip 1em\relax {\urduTechTermsfont {صفحہ}} \protect ١٣٤٦
\begin {description}\setlength {\parskip }{0pt} \setlength {\itemsep }{0pt plus 1pt}
\item [
\protect ١١.\protect ٣٩٥)
]
 نلکی \عددی {(0,0,0)}، کروی \عددی {(0,0,0)} 
\item [
\protect ١١.\protect ٣٩٧)
]
 نلکی \عددی {(1,\pi /2,0)}، کروی \عددی {(1,\pi /2,\pi /2)} 
\item [
\protect ١١.\protect ٣٩٩)
]
 کارتیسی \عددی {(1,0,0)}، کروی \عددی {(1,\pi /2,0)} 
\item [
\protect ١١.\protect ٤٠١)
]
 کارتیسی \عددی {(0,1,1)}، کروی \عددی {(\sqrt {2},\pi /4,\pi /2)} 
\item [
\protect ١١.\protect ٤٠٣)
]
 کارتیسی \عددی {(0,-2\sqrt {2},0)}،نلکی \عددی {(2\sqrt {2},3\pi /2,0)} 
\item [
\protect ١١.\protect ٤٠٥)
]
 \عددی {x^2+y^2=0}، \عددی {\phi =0} یا \عددی {\phi =\pi } یعنی محور \عددی {z} 
\item [
\protect ١١.\protect ٤٠٧)
]
 \عددی {z=0}، \عددی {\theta =\pi /2}، مستوی \عددی {xy} 
\item [
\protect ١١.\protect ٤٠٩)
]
 \عددی {z=\rho }، \عددی {0\le \rho \le 1}؛ \عددی {\theta =\pi /4}، \عددی {0\le r\le \sqrt {2}}؛ ایک محدود ترخیم 
\item [
\protect ١١.\protect ٤١١)
]
 \عددی {x=0}، \عددی {\phi =\pi /2}، مستوی \عددی {yz} 
\item [
\protect ١١.\protect ٤١٣)
]
 \عددی {\rho ^2+z^2=4}، \عددی {r=2}؛ رداس \عددی {2} کا کرہ جس کا مرکز مبدا پر ہے۔ 
\item [
\protect ١١.\protect ٤١٥)
]
 \عددی {x^2+y^2+(z-5/2)^2=25/4}، \عددی {\rho ^2+z^2=5z}، رداس \عددی {5/2} کا کرہ جس کا مرکز \عددی {(0,0,5/2)} (کارتیسی) ہے۔ 
\item [
\protect ١١.\protect ٤١٧)
]
 \عددی {y=1}، \عددی {r\sin \theta \sin \phi =1}، مستوی \عددی {y=1} 
\item [
\protect ١١.\protect ٤١٩)
]
 \عددی {z=\sqrt {2}}، سطح \عددی {z=\sqrt {2}} 
\item [
\protect ١١.\protect ٤٢١)
]
 \عددی {\rho ^2+z^2=2z}، \عددی {z\le 1}؛ \عددی {r=2\cos \theta }، \عددی {\pi /4\le \theta \le \pi /2}؛ نچلا نصف کرہ جس کا رداس \عددی {1} اور مرکز \عددی {(0,0,1)} (کارتیسی) ہے۔ 
\item [
\protect ١١.\protect ٤٢٣)
]
 \عددی {x^2+y^2+z^2=9}، \عددی {-3/2\le z\le 3/2}؛ \عددی {\rho ^2+z^2=9}، \عددی {-3/2\le z\le 3/2}، رداس \عددی {3} کے کرہ کا وہ حصہ جو سطح \عددی {z=-3/2} اور سطح \عددی {z=3/2} کے بیچ ہے۔ کرہ کا مرکز مبدا پر ہے۔ 
\item [
\protect ١١.\protect ٤٢٥)
]
 \عددی {z=4-4(x^2+y^2)}، \عددی {0\le z\le 4}؛ \عددی {r\cos \theta =4-4 r^2\sin ^2\theta }، \عددی {0\le \theta \le \pi /2}، قطع مکافی سطح \عددی {z=4-4(x^2+y^2)} کا بالائی حصہ جس کو مستوی \عددی {xy} کاٹتا ہے۔ 
\item [
\protect ١١.\protect ٤٢٧)
]
 \عددی {z=-\sqrt {x^2+y^2}}، \عددی {-1\le z\le 0}؛ \عددی {z=-\rho }، \عددی {0\le \rho \le 1}، ترخیم جس کا راس مبدا پر ہے، اس کا قاعدہ ، مستوی \عددی {z=-1} میں دائرہ \عددی {x^2+y^2=1} ہے، 
\item [
\protect ١١.\protect ٤٢٩)
]
 \عددی {z+x^2-y^2=0} یا \عددی {z=y^2-x^2}، \عددی {\cos \theta +r\sin ^2\theta \cos 2\phi =0}، قطع زائد قطع مکافی سطح 
\item [
\protect ١١.\protect ٤٣١)
]
 \عددی {(2,3,1)} 
\item [
\protect ١١.\protect ٤٣٣)
]
 مستوی \عددی {\rho \phi } میں دائرہ \عددی {\rho =-2\sin \phi } کا پیدا کردہ ، محور \عددی {z} کا متوازی قائمہ دائری بیلن ۔ \begin {center} \begin {tikzpicture}[declare function={fr(\t )=-2*sin(\t );}] \pgfmathsetmacro {\ta }{20} \pgfmathsetmacro {\tb }{290} \begin {axis}[axis equal,axis lines=center,view/h=120, axis x line=none, axis y line=none, axis z line=none] \addplot 3[smooth,data cs=polar,samples y=0,domain=0:360](x,{fr(x)},1); \addplot 3[smooth,data cs=polar,samples y=0,domain=0:360](x,{fr(x)},0); \addplot 3[smooth,data cs=polar,samples y=0,domain=0:360](x,{fr(x)},-1); \addplot 3[data cs=polar]coordinates{(\ta ,{fr(\ta )},-1)(\ta ,{fr(\ta )},1)}; \addplot 3[data cs=polar]coordinates{(\tb ,{fr(\tb )},-1)(\tb ,{fr(\tb )},1)}; \addplot 3[-latex]coordinates {(0,0,0)(3,0,0)}node[left]{$x$}; \addplot 3[-latex]coordinates {(0,0,0)(0,1,0)}node[right]{$y$}; \addplot 3[-latex]coordinates {(0,0,0)(0,0,2)}node[right]{$z$}; \end {axis} \end {tikzpicture} \end {center} 
\item [
\protect ١١.\protect ٤٣٥)
]
 محور \عددی {z} کے متوازی لکیروں کی پیدا کردہ نلکی جس کو مستوی \عددی {\rho \phi } میں قلب نما \عددی {\rho =1-\cos \phi } پیدا کرتا ہے۔ \begin {center} \begin {tikzpicture}[declare function={fr(\t )=1-cos(\t );}] \pgfmathsetmacro {\ta }{135} \pgfmathsetmacro {\tb }{250} \begin {axis}[axis equal,axis lines=center,view/h=110,axis x line=none, axis y line=none, axis z line=none] \addplot 3[smooth,data cs=polar,samples y=0,domain=0:360](x,{fr(x)},1); \addplot 3[smooth,data cs=polar,samples y=0,domain=0:360](x,{fr(x)},0); \addplot 3[smooth,data cs=polar,samples y=0,domain=0:360](x,{fr(x)},-1); \addplot 3[data cs=polar]coordinates{(\ta ,{fr(\ta )},-1)(\ta ,{fr(\ta )},1)}; \addplot 3[data cs=polar]coordinates{(\tb ,{fr(\tb )},-1)(\tb ,{fr(\tb )},1)}; \addplot 3[-latex]coordinates {(0,0,0)(2.5,0,0)}node[left]{$x$}; \addplot 3[-latex]coordinates {(0,0,0)(0,2,0)}node[right]{$y$}; \addplot 3[-latex]coordinates {(0,0,0)(0,0,2.5)}node[left]{$z$}; \end {axis} \end {tikzpicture} \end {center} 
\item [
\protect ١١.\protect ٤٣٧)
]
 سطح طواف قلب نما جو محور \عددی {y} کے لحاض سے تشاکلی ہے۔ مبدا پر کنگرہ نیچے رخ ہے۔ \begin {center} \begin {tikzpicture}[declare function={fx(\t ,\p )=(1-cos(\t ))*sin(\t )*cos(\p );fy(\t ,\p )=(1-cos(\t ))*sin(\t )*sin(\p );fz(\t ,\p )=(1-cos(\t ))*cos(\t );}] \pgfmathsetmacro {\pa }{45} \pgfmathsetmacro {\pb }{90} \pgfmathsetmacro {\pc }{135} \pgfmathsetmacro {\ta }{85} \begin {axis}[axis equal,axis lines=center,view/h=110,axis x line=none, axis y line=none, axis z line=none] \addplot 3[smooth,domain y=0:360]({fx(\ta ,y)},{fy(\ta ,y)},{fz(\ta ,y)}); \addplot 3[smooth,domain=0:180,samples y=0]({fx(x,\pa )},{fy(x,\pa )},{fz(x,\pa )}); \addplot 3[smooth,domain=0:180,samples y=0]({fx(x,\pb )},{fy(x,\pb )},{fz(x,\pb )}); \addplot 3[smooth,domain=0:180,samples y=0]({fx(x,\pc )},{fy(x,\pc )},{fz(x,\pc )}); \addplot 3[smooth,domain=0:180,samples y=0]({fx(x,-\pa )},{fy(x,-\pa )},{fz(x,-\pa )}); \addplot 3[smooth,domain=0:180,samples y=0]({fx(x,-\pb )},{fy(x,-\pb )},{fz(x,-\pb )}); \addplot 3[smooth,domain=0:180,samples y=0]({fx(x,-\pc )},{fy(x,-\pc )},{fz(x,-\pc )}); \addplot 3[-latex]coordinates {(0,0,0)(3.5,0,0)}node[left]{$x$}; \addplot 3[-latex]coordinates {(0,0,0)(0,1.5,0)}node[right]{$y$}; \addplot 3[-latex]coordinates {(0,0,0)(0,0,1)}node[left]{$z$}; \end {axis} \end {tikzpicture} \end {center} 
\item [
\protect ١١.\protect ٤٣٩)
]
 (ب) \عددی {\theta =\pi /2} 
\item [
\protect ١١.\protect ٤٤٣)
]
 سطح کی مساوات \عددی {\rho =f(z)} ہمیں بتاتی ہے کہ نقطہ \عددی {(\rho ,\phi ,z)=(f(z),\phi ,z)} تمام \عددی {\phi } کے لئے سطح پر واقع ہو گا۔ بالخصوص جس بھی \عددی {(f(z),\phi ,z)} اس سطح پر پایا جاتا ہو اس وقت \عددی {(f(z),\phi +\pi ,z) } بھی اس سطح پر پایا جائے گا لہٰذا محور \عددی {z} کے لحاض سے یہ سطح تشاکلی ہے۔ 
\end {description}
 {\urduTechTermsfont {حصہ}} \protect ١٢.\protect ١\hskip 1em\relax {\urduTechTermsfont {صفحہ}} \protect ١٣٦٣
\begin {description}\setlength {\parskip }{0pt} \setlength {\itemsep }{0pt plus 1pt}
\item [
\protect ١٢.\protect ١)
]
 $y=x^2-2x,\, \kvec {v}=\ai +2\aj ,\, \kvec {a}=2\aj $ 
\item [
\protect ١٢.\protect ٣)
]
 $y=\tfrac {2}{9}x^2,\,\kvec {v}=3\ai +4\aj ,\,\kvec {a}=3\ai +8\aj $ 
\item [
\protect ١٢.\protect ٥)
]
 $t=\tfrac {\pi }{4}:\,\kvec {v}=\tfrac {\sqrt {2}}{2}\ai -\tfrac {\sqrt {2}}{2}\aj ,\,\kvec {a}=-\tfrac {\sqrt {2}}{2}\ai -\tfrac {\sqrt {2}}{2}\aj ;\, t=\tfrac {\pi }{2}:\,\kvec {v}=-\aj ,\,\kvec {a}=-\ai $ \begin {center} \begin {tikzpicture}[font=\small ] \pgfmathsetmacro {\r }{1.25} \draw [-latex](-1.25*\r ,0)--(1.75*\r ,0)node[right]{$x$}; \draw [-latex](0,-1.25*\r )--(0,1.5*\r )node[right]{$y$}; \draw (0,\r )node[left,yshift=1ex]{$1$}; \draw (0,0)node[below left]{$O$} circle (\r ); \draw [thick,latex-](0,0)--++(45:\r )node[pos=0.6,shift={(135:0.2)},yshift=1ex]{$\kvec {a}(\tfrac {\pi }{4})$}; \draw [thick,-latex](45:\r )-++(-45:\r )node[pos=0.5,above right]{$\kvec {v}(\tfrac {\pi }{4})$}; \draw [thick,-latex](\r ,0)--++(0,-\r )node[pos=0.5,right]{$\kvec {v}(\tfrac {\pi }{2})$}; \draw [thick,-latex](\r ,0)--(0,0)node[pos=0.5,below]{$\kvec {a}(\tfrac {\pi }{1})$}; \end {tikzpicture} \end {center} 
\item [
\protect ١٢.\protect ٧)
]
 $t=\pi :\, \kvec {v}=2\ai ,\,\kvec {a}=-\aj ; \, t=\tfrac {3\pi }{2}:\,\kvec {v}=\ai -\aj ,\, \kvec {a}=-\ai $ \begin {center} \begin {tikzpicture}[scale=0.75,declare function={fx(\x )=2*pi/360*\x -sin(\x );fy(\x )=1-cos(\x );}] \draw [-latex](0,0)--(7,0)node[right]{$x$}; \draw [-latex](0,0)--(0,2.25)node[right]{$y$}; \draw (0.1,1)--++(-0.2,0)node[left]{$1$}; \draw (0.1,2)--++(-0.2,0)node[left]{$2$}; \draw (pi,0.1)--++(0,-0.2)node[below]{$\pi $}; \draw (2*pi,0.1)--++(0,-0.2)node[below]{$2\pi $}; \draw [domain=0:360]plot ({fx(\x )},{fy(\x )}); \draw [-latex]({fx(180)},{fy(180)})node[circ]{}node[pin={[pin edge=-]135:{$t=\pi $}}]{}--++(2,0)node[pos=0.5,above]{$\kvec {v}(\pi )$}; \draw [-latex]({fx(180)},{fy(180)})--++(0,-1)node[pos=0.5,left]{$\kvec {a}(\pi )$}; \draw [-latex]({fx(270)},{fy(270)})node[circ]{}node[pin={[pin edge=-]45:{$t=\tfrac {3\pi }{2}$}}]{}--++(1,-1)node[pos=0.5,above right]{$\kvec {v}(\tfrac {3\pi }{2})$}; \draw [-latex]({fx(270)},{fy(270)})--++(-1,0)node[pos=0.5,below]{$\kvec {a}(\tfrac {3\pi }{2})$}; \end {tikzpicture} \end {center} 
\item [
\protect ١٢.\protect ٩)
]
 $\kvec {v}=\ai +2t\aj +2\ak ;\, \kvec {a}=2\aj ;\, \text {رفتار} 3; \text {رخ} \tfrac {1}{3}\ai +\tfrac {2}{3}\aj +\tfrac {2}{3}\ak ;\kvec {v(1)}=3(\tfrac {1}{3}\ai +\tfrac {2}{3}\aj +\tfrac {2}{3}\ak )$ 
\item [
\protect ١٢.\protect ١١)
]
 $\kvec {v}=(-2\sin t)\ai +(3\cos t)\aj +4\ak ;\, \kvec {a}=(-2\cos t)\ai -(3\sin t)\aj ;\text {رفتار}2\sqrt {5};$\\ $ \text {رخ}\tfrac {-1}{\sqrt {5}}\ai +\tfrac {2}{\sqrt {5}}\ak ;\kvec {v}(\pi /2)=2\sqrt {5}[-\tfrac {1}{\sqrt {5}}\ai +\tfrac {2}{\sqrt {5}}\ak ]$ 
\item [
\protect ١٢.\protect ١٣)
]
 $\kvec {v}=(\tfrac {2}{t+1})\ai +2t\aj +t\ak ;\kvec {a}=\tfrac {-2}{(t+1)^2}\ai +2\aj +\ak ; \text {رفتار} \sqrt {6};\text {رخ} \tfrac {1}{\sqrt {6}}\ai +\tfrac {2}{\sqrt {6}}\aj +\tfrac {1}{\sqrt {6}}\ak ;\, \kvec {v}(1)=\sqrt {6}(\tfrac {1}{\sqrt {6}}\ai +\tfrac {2}{\sqrt {6}}\aj +\tfrac {1}{\sqrt {6}}\ak )$ 
\item [
\protect ١٢.\protect ١٥)
]
 $\tfrac {\pi }{2}$ 
\item [
\protect ١٢.\protect ١٧)
]
 $\tfrac {\pi }{2}$ 
\item [
\protect ١٢.\protect ١٩)
]
 $t=0,\, \pi ,\, 2\pi $ 
\item [
\protect ١٢.\protect ٢١)
]
 $\tfrac {1}{4}\ai +7\aj +\tfrac {3}{2}\ak $ 
\item [
\protect ١٢.\protect ٢٣)
]
 $\tfrac {\pi +2\sqrt {2}}{2}\aj +2\ak $ 
\item [
\protect ١٢.\protect ٢٥)
]
 $(\ln 4)\ai +(\ln 4)\aj +(\ln 2)\ak $ 
\item [
\protect ١٢.\protect ٢٧)
]
 $\kvec {r}(t)=(\tfrac {-t^2}{2}+1)\ai +(\tfrac {-t^2}{2}+2)\aj +(\tfrac {-t^2}{2}+3)\ak $ 
\item [
\protect ١٢.\protect ٢٩)
]
 $\kvec {r}(t)=((t+1)^{3/2}-1)\ai +(-e^{-t}+1)\aj +(\ln (t+1)+1)\ak $ 
\item [
\protect ١٢.\protect ٣١)
]
 $\kvec {r}(t)=8t\ai +8t\aj +(-16t^2+100)\ak $ 
\item [
\protect ١٢.\protect ٣٣)
]
 $x=t,\, y=-1,\, z=1+t$ 
\item [
\protect ١٢.\protect ٣٥)
]
 $x=at,\, y=a,\, z=2\pi b+bt$ 
\item [
\protect ١٢.\protect ٣٧)
]
 ا) (1) مستقل رفتار \عددی {1}؛ (2) جی ہاں (3) گھڑی کے مخالف رخ (4) جی ہاں\\ ب) (1) مستقل رفتار \عددی {2} (2) جی ہاں (3) گھڑی کے مخالف رخ (4) جی ہاں\\ ج) (1) مستقل رفتار \عددی {1} (2) جی ہاں (3) گھڑی کے مخالف رخ (4) یہ \عددی {(1,0)} کی بجائے \عددی {(0,-1)} سے ابتدا کرتا ہے\\ د) (1) مستقل رفتار \عددی {1} (2) جی ہاں (3) گھڑی کے رخ (4) جی ہاں\\ ہ) (1) متغیر رفتار (2) نہیں (3) گھڑی کے مخالف رخ (4) جی ہاں 
\item [
\protect ١٢.\protect ٣٩)
]
 $\kvec {r}(t)=(\tfrac {3}{2}t^2+\tfrac {6}{\sqrt {11}}t+1)\ai -(\tfrac {1}{2}t^2+\tfrac {2}{\sqrt {11}}t-2)\aj +(\tfrac {1}{2}t^2+\tfrac {2}{\sqrt {11}}t+3)\ak =(\tfrac {1}{2}t^2+\tfrac {2t}{\sqrt {11}})(3\ai -\aj +\ak )+(\ai +2\aj +3\ak )$ 
\item [
\protect ١٢.\protect ٤١)
]
 $\kvec {v}(t)=2\sqrt {5}\ai +\sqrt {5}\aj $ 
\item [
\protect ١٢.\protect ٤٣)
]
 زیادہ سے زیادہ \عددی {\abs {\kvec {v}}=3}، کم سے کم \عددی {\abs {\kvec {v}}=2}، زیادہ سے زیادہ \عددی {\abs {\kvec {a}}=3}، کم سے کم \عددی {\abs {\kvec {a}}=2} 
\end {description}
 {\urduTechTermsfont {حصہ}} \protect ١٢.\protect ٢\hskip 1em\relax {\urduTechTermsfont {صفحہ}} \protect ١٣٧٩
\begin {description}\setlength {\parskip }{0pt} \setlength {\itemsep }{0pt plus 1pt}
\item [
\protect ١٢.\protect ٦٤)
]
 $\SI {50}{\second }$ 
\item [
\protect ١٢.\protect ٦٦)
]
 (ا) \عددی {\SI {72.2}{\second }}، \عددی {\SI {25510}{\meter }}؛ (ب) \عددی {\SI {4020}{\meter }}؛ (ج) \عددی {\SI {6378}{\meter }} 
\item [
\protect ١٢.\protect ٦٨)
]
 $t\approx \SI {2.1257}{\second },\quad x\approx \SI {20.14}{\meter }$ 
\item [
\protect ١٢.\protect ٧٠)
]
 $v_0=\SI {9.9}{\meter \per \second },\,\alpha =18.4^{\circ },\alpha =71.6^{\circ }$ 
\item [
\protect ١٢.\protect ٧٢)
]
 $\SI {174}{\kilo \meter \per \hour }$ 
\item [
\protect ١٢.\protect ٧٤)
]
 گیند درخت کے پتوں کو چھوتا ہوا اسے پار کر پائے گا۔ 
\item [
\protect ١٢.\protect ٧٦)
]
 \عددی {\SI {24.87}{\meter \per \second }} 
\item [
\protect ١٢.\protect ٨٠)
]
 \عددی {\SI {141}{\percent }} 
\item [
\protect ١٢.\protect ٨٤)
]
 \عددی {1.789} سیکنڈ، \عددی {\SI {19.92}{\meter }} 
\item [
\protect ١٢.\protect ٨٨)
]
 $\kvec {v}(t)=-gt\ak +\kvec {v}_0,\, \kvec {r}(t)=-\tfrac {1}{2}gt^2\ak +\kvec {v}_0t$ 
\end {description}
 {\urduTechTermsfont {حصہ}} \protect ١٢.\protect ٣\hskip 1em\relax {\urduTechTermsfont {صفحہ}} \protect ١٣٨٧
\begin {description}\setlength {\parskip }{0pt} \setlength {\itemsep }{0pt plus 1pt}
\item [
\protect ١٢.\protect ٩١)
]
 $\kvec {T}=(-\tfrac {2}{3}\sin t)\ai +(\tfrac {2}{3}\cos t)\aj +\tfrac {\sqrt {5}}{3}\ak ,\quad 3\pi $ 
\item [
\protect ١٢.\protect ٩٣)
]
 $\kvec {T}=\tfrac {1}{\sqrt {1+t}}\ai +\tfrac {\sqrt {t}}{\sqrt {1+t}}\ak ,\quad \tfrac {52}{3}$ 
\item [
\protect ١٢.\protect ٩٥)
]
 $\kvec {T}=-\cos t\aj +\sin t\ak ,\quad \tfrac {3}{2}$ 
\item [
\protect ١٢.\protect ٩٧)
]
 $\kvec {T}=(\tfrac {\cos t-t\sin t}{t+1})\ai +(\tfrac {\sin t+t\cos t}{t+1})\aj +(\tfrac {\sqrt {2}t^{1/2}}{t+1})\ak ,\quad \tfrac {\pi ^2}{2}+\pi $ 
\item [
\protect ١٢.\protect ٩٩)
]
 $(0,5,24\pi )$ 
\item [
\protect ١٢.\protect ١٠١)
]
 $s(t)=5t,\quad L=\tfrac {5\pi }{2}$ 
\item [
\protect ١٢.\protect ١٠٣)
]
 $s(t)=\sqrt {3}e^t-\sqrt {3},\quad L=\tfrac {3\sqrt {3}}{4}$ 
\item [
\protect ١٢.\protect ١٠٥)
]
 $\sqrt {2}+\ln (1+\sqrt {2})$ 
\item [
\protect ١٢.\protect ١٠٧)
]
 ا) نلکی \عددی {x^2+y^2=1} اور مستوی \عددی {x+z=1}\\ د) \عددی {L=\int _0^{2\pi }\sqrt {1+\sin ^2t}\dif t}\\ ہ) \عددی {L\approx 7.64} 
\end {description}
 {\urduTechTermsfont {حصہ}} \protect ١٢.\protect ٤\hskip 1em\relax {\urduTechTermsfont {صفحہ}} \protect ١٤٠٣
\begin {description}\setlength {\parskip }{0pt} \setlength {\itemsep }{0pt plus 1pt}
\item [
\protect ١٢.\protect ١٠٩)
]
 $\kvec {T}=(\cos t)\ai -(\sin t)\aj ,\, \kvec {N}=(-\sin t)\ai -(\cos t)\aj ,\,\kappa =\cos t$ 
\item [
\protect ١٢.\protect ١١١)
]
 $\kvec {T}=\tfrac {1}{\sqrt {1+t^2}}\ai -\tfrac {t}{\sqrt {1+t^2}}\aj ,\,\kvec {N}=\tfrac {-t}{\sqrt {1+t^2}}\ai -\tfrac {1}{\sqrt {1+t^2}}\aj ,\,\kappa =\tfrac {1}{2(\sqrt {1+t^2})^3}$ 
\item [
\protect ١٢.\protect ١١٣)
]
 $\kvec {a}=\tfrac {2t}{\sqrt {1+t^2}}\kvec {T}+\tfrac {2}{\sqrt {1+t^2}}\kvec {N}$ 
\item [
\protect ١٢.\protect ١١٥)
]
 (ب) \عددی {\cos x} 
\item [
\protect ١٢.\protect ١١٧)
]
 (ب) \عددی {\kvec {N}=\tfrac {-2e^{2t}}{\sqrt {1+4e^{4t}}}\ai +\tfrac {1}{\sqrt {1+4e^{4t}}}\aj } ، (ج) \عددی {\kvec {N}=-\tfrac {1}{2}(\sqrt {4-t^2}\ai +t\aj )} 
\item [
\protect ١٢.\protect ١١٩)
]
 $\kvec {T}=\tfrac {3\cos t}{5}\ai -\tfrac {3\sin t}{5}\aj +\tfrac {4}{5}\ak ,\,\kvec {N}=(-\sin t)\ai -(\cos t)\aj ,\,\kvec {B}=(\tfrac {4}{5}\cos t)\ai -(\tfrac {4}{5}\sin t)\aj -\tfrac {3}{5}\ak ,\,\kappa =\tfrac {3}{25},\,\tau =-\tfrac {4}{25}$ 
\item [
\protect ١٢.\protect ١٢١)
]
 $\kvec {T}=(\tfrac {\cos t-\sin t}{\sqrt {2}})\ai +(\tfrac {\cos t+\sin t}{\sqrt {2}})\aj ,\,\kvec {N}=(\tfrac {-\cos t-\sin t}{\sqrt {2}})\ai +(\tfrac {-\sin t+\cos t}{\sqrt {2}})\aj ,\,\kvec {B}=\ak ,\,\kappa =\tfrac {1}{e^t\sqrt {2}},\, \tau =0$ 
\item [
\protect ١٢.\protect ١٢٣)
]
 $\kvec {T}=\tfrac {t}{\sqrt {t^2+1}}\ai +\tfrac {1}{\sqrt {t^2+1}}\aj ,\,\kvec {N}=\tfrac {\ai }{\sqrt {t^2+1}}-\tfrac {t\aj }{\sqrt {t^2+1}},\, \kvec {B}=-\ak ,\,\kappa =\tfrac {1}{t(t^2+1)^{3/2}},\,\tau =0$ 
\item [
\protect ١٢.\protect ١٢٥)
]
 $\kvec {T}=(\sech \tfrac {t}{a})\ai +(\tanh \tfrac {t}{a})\aj ,\,\kvec {N}=(-\tanh \tfrac {t}{a})\ai +(\sech \tfrac {t}{a})\aj ,\,\kvec {B}=\ak ,\,\kappa =\tfrac {1}{a}\sech ^2\tfrac {t}{a},\,\tau =0$ 
\item [
\protect ١٢.\protect ١٢٧)
]
 $\kvec {a}=\abs {a}\kvec {N}$ 
\item [
\protect ١٢.\protect ١٢٩)
]
 (ا) \عددی {\kvec {a}(1)=\tfrac {4}{3}\kvec {T}+\tfrac {2\sqrt {5}}{3}\kvec {N}} 
\item [
\protect ١٢.\protect ١٣١)
]
 $\kvec {a}(0)=2\kvec {N}$ 
\item [
\protect ١٢.\protect ١٣٣)
]
 $\kvec {r}(\tfrac {\pi }{4})=\tfrac {\sqrt {2}}{2}\ai +\tfrac {\sqrt {2}}{2}\aj -\ak ,\,\kvec {T}(\tfrac {\pi }{4})=-\tfrac {\sqrt {2}}{2}\ai +\tfrac {\sqrt {2}}{2}\aj ,\,\kvec {N}(\tfrac {\pi }{4})=-\tfrac {\sqrt {2}}{2}\ai -\tfrac {\sqrt {2}}{2}\aj ,\,\kvec {B}(\tfrac {\pi }{4})=\ak ;$\\ مستوی دائرہ انحنا \عددی {z=-1} ہے؛ عمودی مستوی \عددی {-x+y=0} ہے؛ سمت کار مستوی \عددی {x+y=\sqrt {2}} ہے۔ 
\item [
\protect ١٢.\protect ١٣٥)
]
 جی ہاں۔اگر گاڑی مڑتی سڑک \عددی { (\kappa \ne 0)} پر چل رہی ہو تب \عددی {a_N=\kappa \abs {\kvec {v}}^2\ne 0} اور \عددی {\kvec {a}\ne \kvec {0}} ہو گا۔ 
\item [
\protect ١٢.\protect ١٣٩)
]
 $\abs {\kvec {F}}=\kappa (m(\tfrac {\dif s}{\dif t})^2)$ 
\item [
\protect ١٢.\protect ١٤٣)
]
 $\tfrac {1}{2b}$ 
\item [
\protect ١٢.\protect ١٤٧)
]
 (ا) \عددی {b-a}، (ب) \عددی {\pi } 
\item [
\protect ١٢.\protect ١٥٣)
]
 $\kappa (x)=\tfrac {2}{(1+4x^2)^{3/2}}$ 
\item [
\protect ١٢.\protect ١٥٥)
]
 $\kappa (x)=\tfrac {\abs {\sin x}}{(1+\cos ^2x)^{3/2}}$ 
\item [
\protect ١٢.\protect ١٦٥)
]
 \عددی {\kvec {v}} کے اجزاء:\عددی {-1.8701}، \عددی {0.7089}، \عددی {1.0000}؛ \عددی {\kvec {a}} کے اجزاء:\عددی {-1.6960}، \عددی {-2.0307}، \عددی {0}؛ رفتار \عددی {2.2361}؛ \عددی {\kvec {T}} کے اجزاء: \عددی {-0.8364}، \عددی {0.3170}، \عددی {0.4472}؛ \عددی {\kvec {N}} کے اجزاء: \عددی {-0.4143}، \عددی {-0.8998}، \عددی {-0.1369}؛ \عددی {\kvec {B}} کے اجزاء:\عددی {0.3590}، \عددی {-0.2998}، \عددی {0.8839}؛ انحنا \عددی {0.5060}؛ مروڑ \عددی {0.2813}؛ اسراع کا مماسی جزو:\عددی {0.7746}؛ اسراع کا عمودی جزو \عددی {2.5298}؛ 
\item [
\protect ١٢.\protect ١٦٧)
]
 \عددی {\kvec {v}} کے اجزاء:\عددی {2.0000}، \عددی {0}، \عددی {0.1629}؛ \عددی {\kvec {a}} کے اجزاء:\عددی {0}، \عددی {-1.0000}، \عددی {0.0086}؛ رفتار \عددی {2.0066}؛ \عددی {\kvec {T}} کے اجزاء: \عددی {0.9967}، \عددی {0}، \عددی {0.0812}؛ \عددی {\kvec {N}} کے اجزاء: \عددی {-0.0007}، \عددی {-1.0000}، \عددی {0.0086}؛ \عددی {\kvec {B}} کے اجزاء:\عددی {0.0812}، \عددی {-0.0086}، \عددی {-0.9967}؛ انحنا \عددی {0.2484}؛ مروڑ \عددی {-0.0411}؛ اسراع کا مماسی جزو:\عددی {0.0007}؛ اسراع کا عمودی جزو \عددی {1.0000}؛ 
\end {description}
 {\urduTechTermsfont {حصہ}} \protect ١٢.\protect ٥\hskip 1em\relax {\urduTechTermsfont {صفحہ}} \protect ١٤١٩
\begin {description}\setlength {\parskip }{0pt} \setlength {\itemsep }{0pt plus 1pt}
\item [
\protect ١٢.\protect ١٦٩)
]
 $T=\SI {93.2}{\minute }$ 
\item [
\protect ١٢.\protect ١٧١)
]
 $a=\SI {6763}{\kilo \meter }$ 
\item [
\protect ١٢.\protect ١٧٣)
]
 $T=\SI {1655}{\minute }$ 
\item [
\protect ١٢.\protect ١٧٥)
]
 $a=\SI {20430}{\kilo \meter }$ 
\item [
\protect ١٢.\protect ١٧٧)
]
 $\abs {v}=1.9966\times 10^7r^{-1/2}\,\si {\meter \per \second }$ 
\item [
\protect ١٢.\protect ١٧٩)
]
 دائرہ: \عددی {v_0=\sqrt {\tfrac {GM}{r_0}}}؛ ترخیم: \عددی {\sqrt {\tfrac {GM}{r_0}}<v_0<\sqrt {\tfrac {2GM}{r_0}}}؛ قطع مکافی : \عددی {v_0=\sqrt {\tfrac {2GM}{r_0}}}؛ قطع زائد: \عددی {v_0>\sqrt {\tfrac {2GM}{r_0}}} 
\item [
\protect ١٢.\protect ١٨٣)
]
 (ا) \عددی {x(t)=2+(3-4\cos (\pi t))\cos (\pi t)}\\ \عددی {y(t)=(3-4\cos (\pi t))\sin (\pi t)} 
\item [
\protect ١٢.\protect ١٨٥)
]
 (ج) $\kvec {v}=\dot {r}\kvec {u}_r+r\dot {\theta }\kvec {u}_{\theta }+\dot {z}\ak $\\ $\kvec {a}=(\ddot {r}-r\dot {\theta }^2)\kvec {u}_r+(r\ddot {\theta }+2\dot {r}\dot {\theta })\kvec {u}_{\theta }+\ddot {z}\ak $ 
\item [
\protect ١٢.\protect ١٨٧)
]
 (ا) $\kvec {u}_r=\sin \theta \cos \phi \ai +\sin \theta \sin \phi \aj +\cos \theta \ak $\\ $\kvec {u}_{\theta }=\cos \theta \cos \phi \ai +\cos \theta \sin \phi \aj -\sin \theta \ak $\\ $\kvec {u}_{\phi }=-\sin \phi \ai +\cos \phi \aj $ 
\end {description}
 {\urduTechTermsfont {حصہ}} \protect ١٣.\protect ١\hskip 1em\relax {\urduTechTermsfont {صفحہ}} \protect ١٤٣٤
\begin {description}\setlength {\parskip }{0pt} \setlength {\itemsep }{0pt plus 1pt}
\item [
\protect ١٣.\protect ١)
]
 (ا) مستوی \عددی {xy} میں تمام نقاط، (ب) تمام حقیقی، (ج) خطوط \عددی {y-x=c}، (د) کوئی سرحدی نقطہ نہیں ہے (ہ) کھلا اور بند دونوں، (و) غیر محدود 
\item [
\protect ١٣.\protect ٣)
]
 (ا) مستوی \عددی {xy} میں تمام نقاط، (ب) \عددی {z\ge 0}، (ج) \عددی {f(x,y)=0} کے لئے مرکز عین مبدا پر ہے؛ \عددی {f(x,y)\ne 0} کے لئے وہ ترخیم جس کا مرکز \عددی {(0,0)} جبکہ محور اکبر اور محور اصغر بالترتیب محور \عددی {x} اور محور \عددی {y} پر ہوں، (د) کوئی سرحدی نقطہ نہیں ہے، (ہ) کھلا اور بند دونوں، (و) غیر محدود 
\item [
\protect ١٣.\protect ٥)
]
 (ا) مستوی \عددی {xy} میں تمام نقاط، (ب) تمام حقیقی، (ج) \عددی {f(x,y)=0} کے لئے محور \عددی {x} اور محور \عددی {y} جبکہ \عددی {f(x,y)\ne 0} کے لئے قطع زائد جس کے متقارب محور \عددی { x} اور محور \عددی {y} ہیں ، (د) کوئی سرحدی نقطہ نہیں ہے، (ہ) کھلا اور بند دونوں، (و) غیر محدود 
\item [
\protect ١٣.\protect ٧)
]
 (ا) وہ تمام \عددی {(x,y)} جو \عددی {x^2+y^2<16} کو مطمئن کرتے ہوں، (ب) \عددی {z\ge \tfrac {1}{4}}، (ج) وہ دائرے جن کے مرکز مبدا پر ہوں اور جن کے رداس \عددی {r<4} ہوں، (د) دائرہ \عددی {x^2+y^2=16} سرحد ہے، (ہ) کھلا، (و) محدود 
\item [
\protect ١٣.\protect ٩)
]
 (ا) \عددی {(x,y)\ne (0,0)}، (ب) تمام حقیقی، (ج) وہ دائرے جن کے مراکز مبدا پر ہوں اور جن کے رداس \عددی {r>0} ہوں، (د) واحد نقطہ \عددی {(0,0)} سرحدی نقطہ ہے، (ہ) کھلا، (و) غیر محدود 
\item [
\protect ١٣.\protect ١١)
]
 (ا) وہ تمام \عددی {(x,y)} جو \عددی {-1\le y-x\le 1} کو مطمئن کرتے ہوں، (ب) \عددی {-\tfrac {\pi }{2}\le z\le \tfrac {\pi }{2}}، (ج) \عددی {y-x=c} طرز کے خطوط جہاں \عددی {-1\le c\le 1} ہے، (د) دو سیدھے خطوط \عددی {y=1+x} اور \عددی {y=-1+x} سرحد ہیں، (ہ) بند، (و) غیر محدود 
\item [
\protect ١٣.\protect ١٣)
]
 شکل \حوالہ {شکل_سوال_کثیرالمتغیر_الف} جو تفاعل \عددی {z=\cos x\cos ye^{-\sqrt {x^2+y^2}/4}} ہے۔ 
\item [
\protect ١٣.\protect ١٤)
]
 شکل \حوالہ {شکل_سوال_کثیرالمتغیر_ب} جو تفاعل \عددی {z=-\tfrac {xy^2}{x^2+y^2}} ہے۔ 
\item [
\protect ١٣.\protect ١٥)
]
 شکل \حوالہ {شکل_سوال_کثیرالمتغیر_ج} جو تفاعل \عددی {z=\tfrac {1}{4x^2+y^2}} ہے۔ 
\item [
\protect ١٣.\protect ١٦)
]
 شکل \حوالہ {شکل_سوال_کثیرالمتغیر_د} جو تفاعل \عددی {z=e^{-y}\cos x} ہے۔ 
\item [
\protect ١٣.\protect ١٧)
]
 شکل \حوالہ {شکل_سوال_کثیرالمتغیر_ہ} جو تفاعل \عددی {z=\tfrac {xy(x^2-y^2)}{x^2+y^2}} ہے۔ 
\item [
\protect ١٣.\protect ١٨)
]
 شکل \حوالہ {شکل_سوال_کثیرالمتغیر_و} جو تفاعل \عددی {z=y^2-y^4-x^2} ہے۔ 
\item [
\protect ١٣.\protect ١٩)
]
 \begin {minipage}{0.45\textwidth } \centering \begin {tikzpicture}[declare function={f(\x ,\y )=(\y )^2;}] \begin {axis}[view/h=135,small,axis lines=center,domain=-1:1,domain y=-1:1,colormap={kgray}{gray(0.2cm)=(0.6);gray(1cm)=(0.9);},xtick={\empty },ytick={\empty },ztick={\empty },enlargelimits=true,xlabel={$x$},ylabel={$y$},zlabel={$z$},xlabel style={anchor=north east},ylabel style={anchor=north west},zlabel style={anchor=south}] \addplot 3[mesh,samples=25]{f(x,y)}; \end {axis} \end {tikzpicture} \end {minipage}\hfill \begin {minipage}{0.45\textwidth } \centering \begin {tikzpicture}[declare function={f(\x ,\y )=(\y )^2;}] \begin {axis}[view/h=135,small,axis lines=center,domain=-1:1,domain y=-1:1,colormap={kgray}{gray(0.2cm)=(0.6);gray(1cm)=(0.9);},xtick={\empty },ytick={\empty },ztick={\empty },enlargelimits=true,hide z axis,,xlabel={$x$},ylabel={$y$},zlabel={$z$},xlabel style={anchor=north east},ylabel style={anchor=north west},zlabel style={anchor=south},colormap={kdark}{gray(0.2cm)=(0.2);gray(1cm)=(0.2);}] \addplot 3[, contour gnuplot={output point meta=rawz,number=4,labels=true,},samples=41,z filter/.code=\def \pgfmathresult {0},]{f(x,y)}; \end {axis} \end {tikzpicture} \end {minipage} 
\item [
\protect ١٣.\protect ٢١)
]
 \begin {minipage}{0.45\textwidth } \centering \begin {tikzpicture}[declare function={f(\x ,\y )=(\x )^2+(\y )^2;}] \begin {axis}[view/h=135,small,axis lines=center,domain=-1:1,domain y=-1:1,colormap={kgray}{gray(0.2cm)=(0.6);gray(1cm)=(0.9);},xtick={\empty },ytick={\empty },ztick={\empty },enlargelimits=true,xlabel={$x$},ylabel={$y$},zlabel={$z$},xlabel style={anchor=north east},ylabel style={anchor=north west},zlabel style={anchor=south}] \addplot 3[mesh,samples=25]{f(x,y)}; \end {axis} \end {tikzpicture} \end {minipage}\hfill \begin {minipage}{0.45\textwidth } \centering \begin {tikzpicture}[declare function={f(\x ,\y )=(\x )^2+(\y )^2;}] \begin {axis}[view/h=135,small,axis lines=center,domain=-1:1,domain y=-1:1,colormap={kgray}{gray(0.2cm)=(0.6);gray(1cm)=(0.9);},xtick={\empty },ytick={\empty },ztick={\empty },enlargelimits=true,hide z axis,xlabel={$x$},ylabel={$y$},zlabel={$z$},xlabel style={anchor=north east},ylabel style={anchor=north west},zlabel style={anchor=south},colormap={kdark}{gray(0.2cm)=(0.2);gray(1cm)=(0.2);}] \addplot 3[, contour gnuplot={output point meta=rawz,number=4,labels=true,},samples=41,z filter/.code=\def \pgfmathresult {0},]{f(x,y)}; \end {axis} \end {tikzpicture} \end {minipage} 
\item [
\protect ١٣.\protect ٢٣)
]
 \begin {minipage}{0.45\textwidth } \centering \begin {tikzpicture}[declare function={f(\x ,\y )=-(\x )^2-(\y )^2;}] \begin {axis}[view/h=135,small,axis lines=center,domain=-1:1,domain y=-1:1,colormap={kgray}{gray(0.2cm)=(0.6);gray(1cm)=(0.9);},xtick={\empty },ytick={\empty },ztick={\empty },enlargelimits=true,xlabel={$x$},ylabel={$y$},zlabel={$z$},xlabel style={anchor=north east},ylabel style={anchor=north west},zlabel style={anchor=south}] \addplot 3[mesh,samples=25]{f(x,y)}; \end {axis} \end {tikzpicture} \end {minipage}\hfill \begin {minipage}{0.45\textwidth } \centering \begin {tikzpicture}[declare function={f(\x ,\y )=-(\x )^2-(\y )^2;}] \begin {axis}[view/h=135,width=6cm,axis lines=center,domain=-1:1,domain y=-1:1,colormap={kgray}{gray(0.2cm)=(0.6);gray(1cm)=(0.9);},xtick={\empty },ytick={\empty },ztick={\empty },enlargelimits=true,hide z axis,xlabel={$x$},ylabel={$y$},zlabel={$z$},xlabel style={anchor=north east},ylabel style={anchor=north west},zlabel style={anchor=south},colormap={kdark}{gray(0.2cm)=(0.2);gray(1cm)=(0.2);}] \addplot 3[, contour gnuplot={output point meta=rawz,number=4,labels=true,},samples=41,z filter/.code=\def \pgfmathresult {0},]{f(x,y)}; \end {axis} \end {tikzpicture} \end {minipage} 
\item [
\protect ١٣.\protect ٢٥)
]
 \begin {minipage}{0.45\textwidth } \centering \begin {tikzpicture}[declare function={f(\x ,\y )=4*(\x )^2+(\y )^2;}] \begin {axis}[view/h=135,small,axis lines=center,domain=-1:1,domain y=-1:1,colormap={kgray}{gray(0.2cm)=(0.6);gray(1cm)=(0.9);},xtick={\empty },ytick={\empty },ztick={\empty },enlargelimits=true,xlabel={$x$},ylabel={$y$},zlabel={$z$},xlabel style={anchor=north east},ylabel style={anchor=north west},zlabel style={anchor=south}] \addplot 3[mesh,samples=25]{f(x,y)}; \end {axis} \end {tikzpicture} \end {minipage}\hfill \begin {minipage}{0.45\textwidth } \centering \begin {tikzpicture}[declare function={f(\x ,\y )=4*(\x )^2+(\y )^2;}] \begin {axis}[view/h=135,small,axis lines=center,domain=-1:1,domain y=-1:1,colormap={kgray}{gray(0.2cm)=(0.6);gray(1cm)=(0.9);},xtick={\empty },ytick={\empty },ztick={\empty },enlargelimits=true,hide z axis,xlabel={$x$},ylabel={$y$},zlabel={$z$},xlabel style={anchor=north east},ylabel style={anchor=north west},zlabel style={anchor=south},colormap={kdark}{gray(0.2cm)=(0.2);gray(1cm)=(0.2);}] \addplot 3[, contour gnuplot={output point meta=rawz,number=4,labels=true,},samples=41,z filter/.code=\def \pgfmathresult {0},]{f(x,y)}; \end {axis} \end {tikzpicture} \end {minipage} 
\item [
\protect ١٣.\protect ٢٧)
]
 \begin {minipage}{0.45\textwidth } \centering \begin {tikzpicture}[declare function={f(\x ,\y )=1-abs(\y );}] \begin {axis}[view/h=135,small,axis lines=center,domain=-1:1,domain y=-1:1,colormap={kgray}{gray(0.2cm)=(0.6);gray(1cm)=(0.9);},xtick={\empty },ytick={\empty },ztick={\empty },enlargelimits=true,xlabel={$x$},ylabel={$y$},zlabel={$z$},xlabel style={anchor=north east},ylabel style={anchor=north west},zlabel style={anchor=south}] \addplot 3[mesh,samples=25]{f(x,y)}; \end {axis} \end {tikzpicture} \end {minipage}\hfill \begin {minipage}{0.45\textwidth } \centering \begin {tikzpicture}[declare function={f(\x ,\y )=1-abs(\y );}] \begin {axis}[view/h=135,small,axis lines=center,domain=-1:1,domain y=-1:1,colormap={kgray}{gray(0.2cm)=(0.6);gray(1cm)=(0.9);},xtick={\empty },ytick={\empty },ztick={\empty },enlargelimits=true,hide z axis,xlabel={$x$},ylabel={$y$},zlabel={$z$},xlabel style={anchor=north east},ylabel style={anchor=north west},zlabel style={anchor=south},colormap={kdark}{gray(0.2cm)=(0.2);gray(1cm)=(0.2);}] \addplot 3[, contour gnuplot={output point meta=rawz,number=4,labels=true,},samples=41,z filter/.code=\def \pgfmathresult {0},]{f(x,y)}; \end {axis} \end {tikzpicture} \end {minipage} 
\item [
\protect ١٣.\protect ٢٩)
]
 \begin {center} \begin {tikzpicture}[declare function={fx(\x ,\y )=sin(\x )*cos(\y );fy(\x ,\y )=sin(\x )*sin(\y );fz(\x ,\y )=cos(\x );}] \begin {axis}[view/h=135,small,axis lines=center,domain=0:180,domain y=0:360,colormap={kgray}{gray(0.2cm)=(0.6);gray(1cm)=(0.6);},xtick={\empty },ytick={\empty },ztick={\empty },enlargelimits=true,xlabel={$x$},ylabel={$y$},zlabel={$z$},xlabel style={anchor=north east},ylabel style={anchor=north west},zlabel style={anchor=south}] \addplot 3[mesh,samples=15]({fx(x,y)},{fy(x,y)},{fz(x,y)}); \end {axis} \end {tikzpicture} \end {center} $f(x,y,z)=x^2+y^2+z^2=1$ 
\item [
\protect ١٣.\protect ٣١)
]
 \begin {center} \begin {tikzpicture}[declare function={f(\x ,\y )=1-\x ;}] \begin {axis}[view/h=135,small,axis lines=center,domain=-1:1,domain y=-1:1,colormap={kgray}{gray(0.2cm)=(0.6);gray(1cm)=(0.6);},xtick={\empty },ytick={\empty },ztick={\empty },enlargelimits=true,xlabel={$x$},ylabel={$y$},zlabel={$z$},xlabel style={anchor=north east},ylabel style={anchor=north west},zlabel style={anchor=south}] \addplot 3[mesh,samples=10]{f(x,y)}; \end {axis} \end {tikzpicture} \end {center} $f(x,y,z)=x+z=1$ 
\item [
\protect ١٣.\protect ٣٣)
]
 \begin {center} \begin {tikzpicture}[declare function={fx(\x ,\y )=cos(\x );fy(\x ,\y )=sin(\x );fz(\x ,\y )=\y ;}] \begin {axis}[view/h=135,small,axis lines=center,domain=0:360,domain y=-1:1,colormap={kgray}{gray(0.2cm)=(0.6);gray(1cm)=(0.6);},xtick={\empty },ytick={\empty },ztick={\empty },enlargelimits=true,xlabel={$x$},ylabel={$y$},zlabel={$z$},xlabel style={anchor=north east},ylabel style={anchor=north west},zlabel style={anchor=south}] \addplot 3[smooth,mesh,samples=25,samples y=5]({fx(x,y)},{fy(x,y)},{fz(x,y)}); \end {axis} \end {tikzpicture} \end {center} $f(x,y,z)=x^2+y^2=1$ 
\item [
\protect ١٣.\protect ٣٥)
]
 \begin {center} \begin {tikzpicture}[declare function={f(\x ,\y )=1+(\x )^2+(\y )^2;}] \begin {axis}[view/h=135,small,axis lines=center,domain=-1:1,domain y=-1:1,colormap={kgray}{gray(0.2cm)=(0.6);gray(1cm)=(0.6);},xtick={\empty },ytick={\empty },ztick={\empty },enlargelimits=true,xlabel={$x$},ylabel={$y$},zlabel={$z$},xlabel style={anchor=north east},ylabel style={anchor=north west},zlabel style={anchor=south}] \addplot 3[mesh,samples=15]{f(x,y)}; \end {axis} \end {tikzpicture} \end {center} $f(x,y,z)=z-x^2-y^2=1$\\ یعنی $z=x^2+y^2+1$ 
\item [
\protect ١٣.\protect ٣٧)
]
 $x^2+y^2=10$ 
\item [
\protect ١٣.\protect ٣٩)
]
 $\tan ^{-1}y-\tan ^{-1}x=2\tan ^{-1}\sqrt {2}$ 
\item [
\protect ١٣.\protect ٤١)
]
 $\sqrt {x-y}-\ln z=2$ 
\item [
\protect ١٣.\protect ٤٣)
]
 $\tfrac {x+y}{z}=\ln 2$ 
\item [
\protect ١٣.\protect ٤٥)
]
 جی ہاں، \عددی {2000} 
\item [
\protect ١٣.\protect ٤٧)
]
 $\SI {63}{\kilo \meter }$ 
\end {description}
 {\urduTechTermsfont {حصہ}} \protect ١٣.\protect ٢\hskip 1em\relax {\urduTechTermsfont {صفحہ}} \protect ١٤٤٥
\begin {description}\setlength {\parskip }{0pt} \setlength {\itemsep }{0pt plus 1pt}
\item [
\protect ١٣.\protect ٦١)
]
 $\tfrac {5}{2}$ 
\item [
\protect ١٣.\protect ٦٣)
]
 $2\sqrt {6}$ 
\item [
\protect ١٣.\protect ٦٥)
]
 $1$ 
\item [
\protect ١٣.\protect ٦٧)
]
 $\tfrac {1}{2}$ 
\item [
\protect ١٣.\protect ٦٩)
]
 $1$ 
\item [
\protect ١٣.\protect ٧١)
]
 $0$ 
\item [
\protect ١٣.\protect ٧٣)
]
 $0$ 
\item [
\protect ١٣.\protect ٧٥)
]
 $-1$ 
\item [
\protect ١٣.\protect ٧٧)
]
 $2$ 
\item [
\protect ١٣.\protect ٧٩)
]
 $\tfrac {1}{4}$ 
\item [
\protect ١٣.\protect ٨١)
]
 $\tfrac {19}{12}$ 
\item [
\protect ١٣.\protect ٨٣)
]
 $2$ 
\item [
\protect ١٣.\protect ٨٥)
]
 $3$ 
\item [
\protect ١٣.\protect ٨٧)
]
 (ا) تمام \عددی {(x,y)}، (ب) ماسوائے \عددی {(0,0)} تمام \عددی {(x,y)} 
\item [
\protect ١٣.\protect ٨٩)
]
 (ا) تمام \عددی {(x,y)} ماسوائے جہاں \عددی {x=0} یا \عددی {y=0} ہو، (ب) تمام \عددی {(x,y)} 
\item [
\protect ١٣.\protect ٩١)
]
 (ا) تمام \عددی {(x,y,z)}، (ب) نلکی \عددی {x^2+y^2=1} کی اندرون کے علاوہ تمام \عددی {(x,y,z)} 
\item [
\protect ١٣.\protect ٩٣)
]
 (ا) وہ تمام \عددی {(x,y,z)} جہاں \عددی {z\ne 0} ہو، (ب) وہ تمام \عددی {(x,y,z)} جہاں \عددی {x^2+y^2\ne 1} ہو۔ 
\item [
\protect ١٣.\protect ٩٥)
]
 راہ \عددی {y=x,x>0} اور \عددی {y=x,x<0} لیں۔ 
\item [
\protect ١٣.\protect ٩٧)
]
 راہ \عددی {y=kx^2} لیں جہاں \عددی {k} ایک مستقل ہو۔ 
\item [
\protect ١٣.\protect ٩٩)
]
 راہ \عددی {y=mx} لیں جہاں \عددی {m} ایک مستقل \عددی {m\ne -1} ہو۔ 
\item [
\protect ١٣.\protect ١٠١)
]
 راہ \عددی {y=kx^2} لیں جہاں \عددی {k} ایک مستقل \عددی {k\ne 0} ہو۔ 
\item [
\protect ١٣.\protect ١٠٣)
]
 نہیں 
\item [
\protect ١٣.\protect ١٠٥)
]
 حد \عددی {1} ہے۔ 
\item [
\protect ١٣.\protect ١٠٧)
]
 حد \عددی {0} ہے۔ 
\item [
\protect ١٣.\protect ١٠٩)
]
 (ا) \عددی {\left . f(x,y)\right \vert _{y=mx}=\sin 2\theta } جہاں\\ \عددی {\tan \theta =m} ہے۔ 
\item [
\protect ١٣.\protect ١١١)
]
 $0$ 
\item [
\protect ١٣.\protect ١١٣)
]
 غیر موجود ہے۔ 
\item [
\protect ١٣.\protect ١١٥)
]
 $\tfrac {\pi }{2}$ 
\item [
\protect ١٣.\protect ١١٧)
]
 $f(0,0)=\ln 3$ 
\item [
\protect ١٣.\protect ١٢١)
]
 $\delta =0.1$ 
\item [
\protect ١٣.\protect ١٢٣)
]
 $\delta =0.005$ 
\item [
\protect ١٣.\protect ١٢٥)
]
 $\delta =\sqrt {0.015}$ 
\item [
\protect ١٣.\protect ١٢٧)
]
 $\delta =0.005$ 
\end {description}
 {\urduTechTermsfont {حصہ}} \protect ١٣.\protect ٣\hskip 1em\relax {\urduTechTermsfont {صفحہ}} \protect ١٤٦٣
\begin {description}\setlength {\parskip }{0pt} \setlength {\itemsep }{0pt plus 1pt}
\item [
\protect ١٣.\protect ١٣١)
]
 $\tfrac {\partial f}{\partial x}=4x,\quad \tfrac {\partial f}{\partial y}=-3$ 
\item [
\protect ١٣.\protect ١٣٣)
]
 $\tfrac {\partial f}{\partial x}=2x(y+2),\quad \tfrac {\partial f}{\partial y}=x^2-1$ 
\item [
\protect ١٣.\protect ١٣٥)
]
 $\tfrac {\partial f}{\partial x}=2y(xy-1),\quad \tfrac {\partial f}{\partial y}=2x(xy-1)$ 
\item [
\protect ١٣.\protect ١٣٧)
]
 $\tfrac {\partial f}{\partial x}=\tfrac {x}{\sqrt {x^2+y^2}},\quad \tfrac {\partial f}{\partial y}=\tfrac {y}{\sqrt {x^2+y^2}}$ 
\item [
\protect ١٣.\protect ١٣٩)
]
 $\tfrac {\partial f}{\partial x}=\tfrac {-1}{(x+y)^2},\quad \tfrac {\partial f}{\partial y}=\tfrac {-1}{(x+y)^2}$ 
\item [
\protect ١٣.\protect ١٤١)
]
 $\tfrac {\partial f}{\partial x}=\tfrac {-y^2-1}{(xy-1)^2},\quad \tfrac {\partial f}{\partial y}=\tfrac {-x^2-1}{(xy-1)^2}$ 
\item [
\protect ١٣.\protect ١٤٣)
]
 $\tfrac {\partial f}{\partial x}=e^{x+y+1},\quad \tfrac {\partial f}{\partial y}=e^{x+y+1}$ 
\item [
\protect ١٣.\protect ١٤٥)
]
 $\tfrac {\partial f}{\partial x}=\tfrac {1}{x+y},\quad \tfrac {\partial f}{\partial y}=\tfrac {1}{x+y}$ 
\item [
\protect ١٣.\protect ١٤٧)
]
 $\tfrac {\partial f}{\partial x}=2\sin (x-3y)\cos (x-3y),$\\ $\tfrac {\partial f}{\partial y}=-6\sin (x-3y)\cos (x-3y)$ 
\item [
\protect ١٣.\protect ١٤٩)
]
 $\tfrac {\partial f}{\partial x}=yx^{y-1},\quad \tfrac {\partial f}{\partial y}=x^y\ln x$ 
\item [
\protect ١٣.\protect ١٥١)
]
 $\tfrac {\partial f}{\partial x}=-g(x),\quad \tfrac {\partial f}{\partial y}=g(y)$ 
\item [
\protect ١٣.\protect ١٥٣)
]
 $f_x=y^2,\quad f_y=2xy,\quad f_z=-4z$ 
\item [
\protect ١٣.\protect ١٥٥)
]
 $f_x=1,f_y=-y(y^2+z^2)^{-1/2},$\\ $f_z=-z(y^2+z^2)^{-1/2}$ 
\item [
\protect ١٣.\protect ١٥٧)
]
 $f_x=\tfrac {yz}{\sqrt {1-x^2y^2z^2}}, f_y=\tfrac {xz}{\sqrt {1-x^2y^2z^2}},$\\ $f_z=\tfrac {xy}{\sqrt {1-x^2y^2z^2}}$ 
\item [
\protect ١٣.\protect ١٥٩)
]
 $f_x=\tfrac {1}{x+2y+3z},f_y=\tfrac {2}{x+2y+3z},$\\ $f_z=\tfrac {3}{x+2y+3z}$ 
\item [
\protect ١٣.\protect ١٦١)
]
 $f_x=-2xe^{-(x^2+y^2+z^2)},$\\ $f_y=-2ye^{-(x^2+y^2+z^2)},$\\ $f_z=-2ze^{-(x^2+y^2+z^2)}$ 
\item [
\protect ١٣.\protect ١٦٣)
]
 $f_x=\sech ^2(x+2y+3z),$\\ $f_y=2\sech ^2(x+2y+3z),$\\ $f_z=3\sech ^2(x+2y+3z)$ 
\item [
\protect ١٣.\protect ١٦٥)
]
 $\tfrac {\partial f}{\partial t}=-2\pi \sin (2\pi t-\alpha ),$\\ $\tfrac {\partial f}{\partial \alpha }=\sin (2\pi t-\alpha )$ 
\item [
\protect ١٣.\protect ١٦٧)
]
 $\tfrac {\partial h}{\partial \rho }=\sin \theta \cos \phi ,$\\ $\tfrac {\partial h}{\partial \theta }=\rho \cos \theta \cos \phi ,$\\ $\frac {\partial h}{\partial \phi }=-\rho \sin \theta \sin \phi $ 
\item [
\protect ١٣.\protect ١٦٩)
]
 $W_P(P,H,\delta ,v,g)=H,$\\ $W_H(P,H,\delta ,v,g)=P+\tfrac {\delta v^2}{2g},$\\ $W_{\delta }(P,H,\delta ,v,g)=\tfrac {Hv^2}{2g},$\\ $W_v(P,H,\delta ,v,g)=\tfrac {H\delta v}{g},$\\ $W_g(P,H,\delta ,v,g)=-\tfrac {H\delta v^2}{2g^2}$ 
\item [
\protect ١٣.\protect ١٧١)
]
 $\tfrac {\partial f}{\partial x}=1+y, \tfrac {\partial f}{\partial y}=1+x,\frac {\partial ^{\,2} f}{\partial x^2}=0,$\\ $\frac {\partial ^{\,2} f}{\partial y^2}=0,\frac {\partial ^{\,2} f}{\partial y\partial x}=\frac {\partial ^{\,2} f}{\partial x\partial y}=1$ 
\item [
\protect ١٣.\protect ١٧٣)
]
 $\tfrac {\partial g}{\partial x}=2xy+y\cos x,$\\ $\tfrac {\partial g}{\partial y}=x^2-\sin y+\sin x,$\\ $\frac {\partial ^{\,2} g}{\partial x^2}=2y-y\sin x,\frac {\partial ^{\,2} g}{\partial y^2}=-\cos y,$\\ $\frac {\partial ^{\,2} g}{\partial y\partial x}=\frac {\partial ^{\,2} f}{\partial x\partial y}=2x+\cos x$ 
\item [
\protect ١٣.\protect ١٧٥)
]
 $\tfrac {\partial r}{\partial x}=\tfrac {1}{x+y},\tfrac {\partial r}{\partial y}=\tfrac {1}{x+y},\frac {\partial ^{\,2} r}{\partial x^2}=\tfrac {-1}{(x+y)^2}$\\ $\frac {\partial ^{\,2} r}{\partial y^2}=\tfrac {-1}{(x+y)^2},\frac {\partial ^{\,2} r}{\partial y\partial x}=\frac {\partial ^{\,2} f}{\partial x\partial y}=\tfrac {-1}{(x+y)^2}$ 
\item [
\protect ١٣.\protect ١٧٧)
]
 $\tfrac {\partial w}{\partial x}=\tfrac {2}{2x+3y},\tfrac {\partial w}{\partial y}=\tfrac {3}{2x+3y},$\\ $\frac {\partial ^{\,2} w}{\partial y\partial x}=\frac {\partial ^{\,2} f}{\partial x\partial y}=\tfrac {-6}{(2x+3y)^2}$ 
\item [
\protect ١٣.\protect ١٧٩)
]
 $\tfrac {\partial w}{\partial x}=y^2+2xy^3+3x^2y^4,$\\ $\tfrac {\partial w}{\partial y}=2xy+3x^2y^2+4x^3y^3,$\\ $\frac {\partial ^{\,2} w}{\partial y\partial x}=2y+6xy^2+12x^2y^3,$\\ $\frac {\partial ^{\,2} f}{\partial x\partial y}=2y+6xy^2+12x^2y^3$ 
\item [
\protect ١٣.\protect ١٨١)
]
 (ا) پہلے \عددی {x}، (ب) پہلے \عددی {y}، (ج) پہلے \عددی {x}، (د) پہلے \عددی {x}، (ہ) پہلے \عددی {y}، (ہ) پہلے \عددی {y} 
\item [
\protect ١٣.\protect ١٨٣)
]
 $f_x(1,2)=-13, f_y(1,2)=-2$ 
\item [
\protect ١٣.\protect ١٨٥)
]
 $12$ 
\item [
\protect ١٣.\protect ١٨٧)
]
 $-2$ 
\item [
\protect ١٣.\protect ١٨٩)
]
 $\tfrac {\partial A}{\partial a}=\tfrac {a}{bc\sin A},\tfrac {\partial A}{\partial b}=\tfrac {c\cos A-b}{bc\sin A}$ 
\item [
\protect ١٣.\protect ١٩١)
]
 $v_x=\tfrac {\ln v}{(\ln u)(\ln v)-1}$ 
\end {description}
 {\urduTechTermsfont {حصہ}} \protect ١٣.\protect ٤\hskip 1em\relax {\urduTechTermsfont {صفحہ}} \protect ١٤٨١
\begin {description}\setlength {\parskip }{0pt} \setlength {\itemsep }{0pt plus 1pt}
\item [
\protect ١٣.\protect ٢٠٦)
]
 (ا) \عددی {L(x,y)=1}،\\ (ب) \عددی {L(x,y)=2x+2y-1} 
\item [
\protect ١٣.\protect ٢٠٨)
]
 (ا) \عددی {L(x,y)=3x-4y+5}،\\ (ب) \عددی {L(x,y)=3x-4y+5} 
\item [
\protect ١٣.\protect ٢١٠)
]
 (ا) \عددی {L(x,y)=1+x}،\\ (ب) \عددی {L(x,y)=-y+\tfrac {\pi }{2}} 
\item [
\protect ١٣.\protect ٢١٢)
]
 \عددی {L(x,y)=7+x-6y;0.06} 
\item [
\protect ١٣.\protect ٢١٤)
]
 \عددی {L(x,y)=x+y+1;0.08} 
\item [
\protect ١٣.\protect ٢١٦)
]
 \عددی {L(x,y)=1+x;0.0222} 
\item [
\protect ١٣.\protect ٢١٨)
]
 چھوٹے ضلع پر زیادہ توجہ دیں۔یہ زیادہ بڑا جزوی تفرق دے گا۔ 
\item [
\protect ١٣.\protect ٢٢٠)
]
 خلل کی زیادہ سے زیادہ مقدار (اندازاً) \عددی {\le 0.31} ہو گی۔ 
\item [
\protect ١٣.\protect ٢٢٢)
]
 زیادہ سے زیادہ فی صد خلل \عددی {\mp \SI {4.83}{\percent }} ہو گا۔ 
\item [
\protect ١٣.\protect ٢٢٤)
]
 \عددی {\abs {x-1}\le 0.014} اور \عددی {\abs {y-1}\le 0.014} لیں۔ 
\item [
\protect ١٣.\protect ٢٢٦)
]
 $\approx \SI {0.1}{\percent }$ 
\item [
\protect ١٣.\protect ٢٢٨)
]
 (ا) \عددی {L(x,y,z)=2x+2y+2z-3}،\\ (ب) \عددی {L(x,y,z,)=y+z}، \\ (ج) \عددی {L(x,y,z)=0} 
\item [
\protect ١٣.\protect ٢٣٠)
]
 (ا) \عددی {L(x,y,z)=x}،\\ (ب) \عددی {L(x,y,z)=\tfrac {x}{\sqrt {2}}+\tfrac {y}{\sqrt {2}}}،\\ (ج) \عددی {L(x,y,z)=\tfrac {x}{3}+\tfrac {2y}{3}+\tfrac {2z}{3}} 
\item [
\protect ١٣.\protect ٢٣٢)
]
 (ا) \عددی {L(x,y,z)=2+x}،\\ (ب) \عددی {L(x,y,z)=x-y-z+\tfrac {\pi }{2}+1}،\\ (ج) \عددی {L(x,y,z)=x-y-z+\tfrac {\pi }{2}+1} 
\item [
\protect ١٣.\protect ٢٣٤)
]
 $L(x,y,z)=2x-6y-2z+6,\, 0.0024$ 
\item [
\protect ١٣.\protect ٢٣٦)
]
 $L(x,y,z)=x+y-z-1,\,0.00135$ 
\item [
\protect ١٣.\protect ٢٣٨)
]
 (ا) $S_0(\tfrac {1}{100}\dif p+\dif x-5\dif w-30\dif h)$\\ (ب) قد میں تبدیلی کو زیادہ حساس ہے۔ 
\item [
\protect ١٣.\protect ٢٤٠)
]
 \عددی {d} میں تبدیلی کو \عددی {f} زیادہ حساس ہے۔ 
\item [
\protect ١٣.\protect ٢٤٤)
]
 ممکنہ خلل کی مقدار \عددی {\le 4.8} ہو گی۔ 
\item [
\protect ١٣.\protect ٢٤٦)
]
 جی ہاں 
\end {description}
 {\urduTechTermsfont {حصہ}} \protect ١٣.\protect ٥\hskip 1em\relax {\urduTechTermsfont {صفحہ}} \protect ١٤٩٣
\begin {description}\setlength {\parskip }{0pt} \setlength {\itemsep }{0pt plus 1pt}
\item [
\protect ١٣.\protect ٢٤٨)
]
 $\tfrac {\dif w}{\dif t}=0,\, \tfrac {\dif w}{\dif t}(\pi )=0$ 
\item [
\protect ١٣.\protect ٢٥٠)
]
 $\tfrac {\dif w}{\dif t}=1,\, \tfrac {\dif w}{\dif t}(3)=1$ 
\item [
\protect ١٣.\protect ٢٥٢)
]
 $\tfrac {\dif w}{\dif t}=4t\tan ^{-1}t+1,\, \tfrac {\dif w}{\dif t}(1)=\pi +1$ 
\item [
\protect ١٣.\protect ٢٥٤)
]
 (ا) $\tfrac {\partial z}{\partial r}=4\cos \theta \ln (r\sin \theta )+4\cos \theta ,$\\ $\tfrac {\partial z}{\partial \theta }=-4r\sin \theta \ln (r\sin \theta )+\tfrac {4r\cos ^2\theta }{\sin \theta }$\\ (ب) $\tfrac {\partial z}{\partial r}=\sqrt {2}(\ln 2+2),$\\ $\tfrac {\partial z}{\partial \theta }=-2\sqrt {2}\ln 2+4\sqrt {2}$ 
\item [
\protect ١٣.\protect ٢٥٦)
]
 (ا) $\tfrac {\partial w}{\partial u}=2u+4uv,$\\ $\tfrac {\partial w}{\partial v}=-2v+2u^2$\\ (ب) $\tfrac {\partial w}{\partial u}=3,\,\tfrac {\partial w}{\partial v}=-\tfrac {3}{2}$ 
\item [
\protect ١٣.\protect ٢٥٨)
]
 (ا) $\tfrac {\partial u}{\partial x}=0,\,\tfrac {\partial u}{\partial y}=\tfrac {z}{(z-y)^2}$\\ $\tfrac {\partial u}{\partial z}=\tfrac {-y}{(z-y)^2}$\\ (ب) $\tfrac {\partial u}{\partial x}=0,\tfrac {\partial u}{\partial y}=1,\tfrac {\partial u}{\partial z}=-2$ 
\item [
\protect ١٣.\protect ٢٦٠)
]
 $\tfrac {\partial z}{\partial t}=\tfrac {\partial z}{\partial x}\tfrac {\dif x}{\dif t}+\tfrac {\partial z}{\partial y}\tfrac {\dif y}{\dif t}$ 
\item [
\protect ١٣.\protect ٢٦٢)
]
 $\tfrac {\partial w}{\partial u}=\tfrac {\partial w}{\partial x}\tfrac {\partial x}{\partial u}+\tfrac {\partial w}{\partial y}\tfrac {\partial y}{\partial u}+\tfrac {\partial w}{\partial z}\tfrac {\partial z}{\partial u}$\\ $\tfrac {\partial w}{\partial v}=\tfrac {\partial w}{\partial x}\tfrac {\partial x}{\partial v}+\tfrac {\partial w}{\partial y}\tfrac {\partial y}{\partial v}+\tfrac {\partial w}{\partial z}\tfrac {\partial z}{\partial v}$ 
\item [
\protect ١٣.\protect ٢٦٤)
]
 $\tfrac {\partial w}{\partial u}=\tfrac {\partial w}{\partial x}\tfrac {\partial x}{\partial u}+\tfrac {\partial w}{\partial y}\tfrac {\partial y}{\partial u}$\\ $\tfrac {\partial w}{\partial v}=\tfrac {\partial w}{\partial x}\tfrac {\partial x}{\partial v}+\tfrac {\partial w}{\partial y}\tfrac {\partial y}{\partial v}$ 
\item [
\protect ١٣.\protect ٢٦٦)
]
 $\tfrac {\partial z}{\partial t}=\tfrac {\partial z}{\partial x}\tfrac {\partial x}{\partial t}+\tfrac {\partial z}{\partial y}\tfrac {\partial y}{\partial t}$\\ $\tfrac {\partial z}{\partial s}=\tfrac {\partial z}{\partial x}\tfrac {\partial x}{\partial s}+\tfrac {\partial z}{\partial y}\tfrac {\partial y}{\partial s}$ 
\item [
\protect ١٣.\protect ٢٦٨)
]
 $\tfrac {\partial w}{\partial s}=\tfrac {\dif w}{\dif u}\tfrac {\partial u}{\partial s},\,\tfrac {\partial w}{\partial t}=\tfrac {\dif w}{\dif u}\tfrac {\partial u}{\partial t}$ 
\item [
\protect ١٣.\protect ٢٧٠)
]
 چونکہ \عددی {\tfrac {\dif y}{\dif r}=0} ہے لہٰذا \\ $\tfrac {\partial w}{\partial r}=\tfrac {\partial w}{\partial x}\tfrac {\dif x}{\dif r}+\tfrac {\partial w}{\partial y}\tfrac {\dif y}{\dif r}=\tfrac {\partial w}{\partial x}\tfrac {\dif x}{\dif r}$\\ چونکہ \عددی {\tfrac {\dif x}{\dif s}=0} ہے لہٰذا \\ $\tfrac {\partial w}{\partial s}=\tfrac {\partial w}{\partial x}\tfrac {\dif x}{\dif s}+\tfrac {\partial w}{\partial y}\tfrac {\dif y}{\dif s}=\tfrac {\partial w}{\partial y}\tfrac {\dif y}{\dif s}$ 
\item [
\protect ١٣.\protect ٢٧٢)
]
 $\tfrac {4}{3}$ 
\item [
\protect ١٣.\protect ٢٧٤)
]
 $-\tfrac {4}{5}$ 
\item [
\protect ١٣.\protect ٢٧٦)
]
 $\tfrac {\partial z}{\partial x}=\tfrac {1}{4},\,\tfrac {\partial z}{\partial y}=-\tfrac {3}{4}$ 
\item [
\protect ١٣.\protect ٢٧٨)
]
 $\tfrac {\partial z}{\partial x}=-1,\,\tfrac {\partial z}{\partial y}=-1$ 
\item [
\protect ١٣.\protect ٢٨٠)
]
 $12$ 
\item [
\protect ١٣.\protect ٢٨٢)
]
 $-7$ 
\item [
\protect ١٣.\protect ٢٨٤)
]
 $\tfrac {\partial z}{\partial u}=2,\,\tfrac {\partial z}{\partial v}=1$ 
\item [
\protect ١٣.\protect ٢٨٦)
]
 $\SI {-0.00005}{\ampere \per \second }$ 
\item [
\protect ١٣.\protect ٢٩٢)
]
 \عددی {(\cos 1,\sin 1,1)} اور \عددی {(\cos (-2),\sin (-2),-2)} 
\item [
\protect ١٣.\protect ٢٩٤)
]
 (ا) \عددی {(-\tfrac {\sqrt {2}}{2},\tfrac {\sqrt {2}}{2})} اور \عددی {(\tfrac {\sqrt {2}}{2},-\tfrac {\sqrt {2}}{2})} پر زیادہ سے زیادہ،\\ \عددی {(\tfrac {\sqrt {2}}{2},\tfrac {\sqrt {2}}{2})} اور \عددی {(-\tfrac {\sqrt {2}}{2},-\tfrac {\sqrt {2}}{2})} کم سے کم۔\\ (ب) زیادہ =\عددی {6}، کم =\عددی {2} 
\item [
\protect ١٣.\protect ٢٩٦)
]
 $2x\sqrt {x^8+x^3}+\int _0^{x^2}\tfrac {3x^2}{2\sqrt {t^4+x^3}}\dif t$ 
\end {description}
 {\urduTechTermsfont {حصہ}} \protect ١٣.\protect ٦\hskip 1em\relax {\urduTechTermsfont {صفحہ}} \protect ١٥٠٤
\begin {description}\setlength {\parskip }{0pt} \setlength {\itemsep }{0pt plus 1pt}
\item [
\protect ١٣.\protect ٢٩٨)
]
 (ا) \عددی {0}، (ب) \عددی {1+2z}، (ج) \عددی {1+2z} 
\item [
\protect ١٣.\protect ٣٠٠)
]
 (ا) $\tfrac {\partial U}{\partial P}+\tfrac {\partial U}{\partial T}(\tfrac {H}{nR})$ (ب) $\tfrac {\partial U}{\partial P}(\tfrac {nR}{H})+\tfrac {\partial U}{\partial T}$ 
\item [
\protect ١٣.\protect ٣٠٢)
]
 (ا) \عددی {5}، (ب) \عددی {5} 
\item [
\protect ١٣.\protect ٣٠٤)
]
 $\tfrac {x}{\sqrt {x^2+y^2}}$ 
\end {description}
 {\urduTechTermsfont {حصہ}} \protect ١٣.\protect ٧\hskip 1em\relax {\urduTechTermsfont {صفحہ}} \protect ١٥٢٠
\begin {description}\setlength {\parskip }{0pt} \setlength {\itemsep }{0pt plus 1pt}
\item [
\protect ١٣.\protect ٣١٠)
]
 \begin {minipage}{0.45\textwidth } \centering \begin {tikzpicture}[font=\scriptsize ,declare function={f(\x )=\x -1;}] \begin {axis}[axis equal,small,axis lines=middle,xtick={1,2},ytick={1,2},xlabel={$x$},ylabel={$y$},ymax=3,xlabel style={anchor=west},ylabel style={anchor=east},xmax=5] \addplot [domain=-0.25:3]{f(x)}node[right]{$y-x=-1$}; \addplot [-latex,thick] coordinates{(2,1)(1,2)}node[above,xshift=2ex]{$\nabla f=-\ai +\aj $}; \addplot []coordinates{(2,1)}node[circ]{}node[right]{$(2,1)$}; \end {axis} \end {tikzpicture} \end {minipage} 
\item [
\protect ١٣.\protect ٣١٢)
]
 \begin {minipage}{0.45\textwidth } \centering \begin {tikzpicture}[font=\scriptsize ,declare function={f(\x )=(\x )^2-1;}] \begin {axis}[axis equal,small,axis lines=middle,xtick={1},ytick={-1},xlabel={$x$},ylabel={$y$},xlabel style={anchor=west},ylabel style={anchor=east},xmax=2,ymax=1.5] \addplot [domain=-1.25:1.25]{f(x)}node[above,xshift=2ex]{$y-x^2=-1$}; \addplot [-latex,thick] coordinates{(-1,0)(1,1)}node[above,xshift=2ex]{$\nabla f=2\ai +\aj $}; \addplot []coordinates{(-1,0)}node[circ]{}node[below]{$(-1,0)$}; \end {axis} \end {tikzpicture} \end {minipage} 
\item [
\protect ١٣.\protect ٣١٤)
]
 $\nabla f=3\ai +2\aj -4\ak $ 
\item [
\protect ١٣.\protect ٣١٦)
]
 $\nabla f=-\frac {26}{27}\ai +\frac {23}{54}\aj -\frac {23}{54}\ak $ 
\item [
\protect ١٣.\protect ٣١٨)
]
 $-4$ 
\item [
\protect ١٣.\protect ٣٢٠)
]
 $\frac {31}{13}$ 
\item [
\protect ١٣.\protect ٣٢٢)
]
 $3$ 
\item [
\protect ١٣.\protect ٣٢٤)
]
 $2$ 
\item [
\protect ١٣.\protect ٣٢٦)
]
 $\uu =-\tfrac {1}{\sqrt {2}}\ai +\tfrac {1}{\sqrt {2}}\aj ,\,(D_{\uu }f)_{N_0}=\sqrt {2};$\\ $-\uu =\tfrac {1}{\sqrt {2}}\ai -\tfrac {1}{\sqrt {2}}\aj ,$\\ $(D_{-\uu }f)_{N_0}=-\sqrt {2}$ 
\item [
\protect ١٣.\protect ٣٢٨)
]
 $\uu =\tfrac {1}{3\sqrt {3}}\ai -\tfrac {5}{3\sqrt {3}}\aj -\tfrac {1}{3\sqrt {3}}\ak ,$\\ $(D_{\uu }f)_{N_0}=3\sqrt {3};$\\ $-\uu =-\tfrac {1}{3\sqrt {3}}\ai +\tfrac {5}{3\sqrt {3}}\aj +\tfrac {1}{3\sqrt {3}}\ak ,$\\ $(D_{-\uu }f)_{N_0}=-3\sqrt {3};$ 
\item [
\protect ١٣.\protect ٣٣٠)
]
 $\uu =\tfrac {1}{\sqrt {3}}(\ai +\aj +\ak ),\,(D_{\uu }f)_{N_0}=2\sqrt {3};$\\ $-\uu =-\tfrac {1}{\sqrt {3}}(\ai +\aj +\ak ),$\\ $(D_{-\uu }f)_{N_0}=-2\sqrt {3};$ 
\item [
\protect ١٣.\protect ٣٣٢)
]
 $\dif f=\tfrac {9}{910}\approx 0.01$ 
\item [
\protect ١٣.\protect ٣٣٤)
]
 $\dif g=0$ 
\item [
\protect ١٣.\protect ٣٣٦)
]
 مماس \عددی {x+y+z=3}، عمودی خط\\ \عددی {x=1+2t,\,y=1+2t,\,z=1+2t} 
\item [
\protect ١٣.\protect ٣٣٨)
]
 مماس \عددی {2x-z-2=0}، عمودی خط\\ \عددی {x=2-4t,\,y=0,\,z=2+2t} 
\item [
\protect ١٣.\protect ٣٤٠)
]
 مماس \عددی {2x+2y+z-4=0}، عمودی خط\\ \عددی {x=2t,\,y=1+2t,\,z=2+t} 
\item [
\protect ١٣.\protect ٣٤٢)
]
 مماس \عددی {x+y+z-1=0}، عمودی خط\\ \عددی {x=t,\,y=1+t,\,z=t} 
\item [
\protect ١٣.\protect ٣٤٤)
]
 $2x-z-2=0$ 
\item [
\protect ١٣.\protect ٣٤٦)
]
 $x-y+2z-1=0$ 
\item [
\protect ١٣.\protect ٣٤٨)
]
 \begin {minipage}{0.45\textwidth } \centering \begin {tikzpicture}[font=\scriptsize ,scale=0.25,declare function={f(\x )=2*sqrt(2)-\x ;}] \draw [-latex](-3,0)--(5,0)node[right]{$x$}; \draw [-latex](0,-2.5)--(0,4)node[right]{$y$}; \draw [thick,-latex]({sqrt(2)},{sqrt(2)})node[circ]{}node[right]{$(2\sqrt {2},2\sqrt {2})$}--++({2*sqrt(2)},{2*sqrt(2)})node[above]{$\nabla f=2\sqrt {2}\ai +2\sqrt {2}\aj $}; \draw (0,0) circle (2); \draw (-135:2)node[left]{$x^2+y^2=4$}; \draw [domain=-1:4]plot (\x ,{f(\x )})node[below,xshift=2ex]{$y=-x+2\sqrt {2}$}; \end {tikzpicture} \end {minipage} 
\item [
\protect ١٣.\protect ٣٥٠)
]
 \begin {minipage}{0.45\textwidth } \centering \begin {tikzpicture}[font=\scriptsize ,declare function={f(\x )=-4/\x ;g(\x )=\x -4;}] \begin {axis}[axis equal,small,axis lines=middle,xlabel={$x$},ylabel={$y$},xlabel style={anchor=west},ylabel style={anchor=east},xtick={\empty },ytick={\empty }] \addplot [domain=-4:-1]{f(x)}node[pos=0.5,left]{$xy=-4$}; \addplot [domain=1:4]{f(x)}; \addplot [domain=-1:5]{g(x)}node[above,xshift=-1ex]{$y=x-4$}; \addplot [-latex,thick]coordinates {(2,-2)(0,0)}node[pos=0,circ]{}node[pos=0,right]{$(2,-2)$}node[below left]{$\nabla f=-2\ai +2\aj $}; \end {axis} \end {tikzpicture} \end {minipage} 
\item [
\protect ١٣.\protect ٣٥٢)
]
 $x=1,y=1+2t,z=1-2t$ 
\item [
\protect ١٣.\protect ٣٥٤)
]
 $x=1-2t,y=1,z=\tfrac {1}{2}+2t$ 
\item [
\protect ١٣.\protect ٣٥٦)
]
 $x=1+90t,y=1-90t,z=3$ 
\item [
\protect ١٣.\protect ٣٥٨)
]
 $\uu =\tfrac {7}{\sqrt {53}}\ai -\tfrac {2}{\sqrt {53}}\aj ,$\\ $-\uu =-\tfrac {7}{\sqrt {53}}\ai +\tfrac {2}{\sqrt {53}}\aj $ 
\item [
\protect ١٣.\protect ٣٦٠)
]
 نہیں، زیادہ سے زیادہ شرح تبدیلی \عددی {\sqrt {185}<14} ہے۔ 
\item [
\protect ١٣.\protect ٣٦٢)
]
 $-\tfrac {7}{\sqrt {5}}$ 
\item [
\protect ١٣.\protect ٣٦٤)
]
 (ا) {\scriptsize {$\tfrac {\sqrt {3}}{2}\sin \sqrt {3}-\tfrac {1}{2}\cos \sqrt {3}\approx \SI {0.935}{\celsius \per \meter }$}}\\ (ب) {\scriptsize {$\sqrt {3}\sin \sqrt {3}-\cos \sqrt {3}\approx \SI {1.87}{\celsius \per \second }$}} 
\item [
\protect ١٣.\protect ٣٦٦)
]
 \عددی {-\tfrac {\pi }{4}} پر \عددی {-\tfrac {\pi }{2\sqrt {2}}}؛ \عددی {0} پر \عددی {0}؛ \عددی {\tfrac {\pi }{4}} پر \عددی {\tfrac {\pi }{2\sqrt {2}}} 
\end {description}
 {\urduTechTermsfont {حصہ}} \protect ١٣.\protect ٨\hskip 1em\relax {\urduTechTermsfont {صفحہ}} \protect ١٥٣٤
\begin {description}\setlength {\parskip }{0pt} \setlength {\itemsep }{0pt plus 1pt}
\item [
\protect ١٣.\protect ٣٧٥)
]
 $f(-3,3)=-5$ مقامی کم سے کم قیمت نقطہ 
\item [
\protect ١٣.\protect ٣٧٧)
]
 $f(\tfrac {2}{3},\tfrac {4}{3})=0$ مقامی زیادہ سے زیادہ قیمت نقطہ 
\item [
\protect ١٣.\protect ٣٧٩)
]
 $f(-2,1)$ نقطہ زین 
\item [
\protect ١٣.\protect ٣٨١)
]
 $f(\tfrac {6}{5},\tfrac {69}{25})$ نقطہ زین 
\item [
\protect ١٣.\protect ٣٨٣)
]
 $f(2,1)$ نقطہ زین 
\item [
\protect ١٣.\protect ٣٨٥)
]
 $f(2,-1)=-6$ مقامی کم سے کم قیمت نقطہ 
\item [
\protect ١٣.\protect ٣٨٧)
]
 $f(1,2)$ نقطہ زین 
\item [
\protect ١٣.\protect ٣٨٩)
]
 \عددی {f(0,0)} نقطہ زین 
\item [
\protect ١٣.\protect ٣٩١)
]
 \عددی {f(0,0)}نقطہ زین؛ \عددی {f(-\tfrac {2}{3},\tfrac {2}{3})=\tfrac {170}{27}} مقامی زیادہ سے زیادہ قیمت نقطہ 
\item [
\protect ١٣.\protect ٣٩٣)
]
 \عددی {f(0,0)=0} مقامی کم سے کم قیمت نقطہ؛ \عددی {f(1,-1)} نقطہ زین 
\item [
\protect ١٣.\protect ٣٩٥)
]
 \عددی {f(0,0)}نقطہ زین؛ \عددی {f(\tfrac {4}{9},\tfrac {4}{3})=-\tfrac {64}{81}} مقامی کم سے کم قیمت نقطہ 
\item [
\protect ١٣.\protect ٣٩٧)
]
 \عددی {f(0,0)}نقطہ زین؛ \عددی {f(0,2)=-12} مقامی کم سے کم قیمت نقطہ؛ \عددی {f(-2,0)=-4} مقامی زیادہ سے زیادہ قیمت نقطہ؛ \عددی {f(-2,2)} نقطہ زین 
\item [
\protect ١٣.\protect ٣٩٩)
]
 \عددی {f(0,0)}نقطہ زین؛ \عددی {f(1,1)=2}، \عددی {f(-1,-1)=2} مقامی زیادہ سے زیادہ قیمت نقطہ 
\item [
\protect ١٣.\protect ٤٠١)
]
 \عددی {f(0,0)=-1} مقامی زیادہ سے زیادہ قیمت نقطہ 
\item [
\protect ١٣.\protect ٤٠٣)
]
 \عددی {f(n\pi ,0)}نقطہ زین؛ ہر \عددی {n} پر \عددی {f(n\pi ,0)=0} 
\item [
\protect ١٣.\protect ٤٠٥)
]
 \عددی {(0,0)} پر مطلق زیادہ سے زیادہ قیمت \عددی {1} جبکہ \عددی {(1,2)} پر مطلق کم سے کم قیمت \عددی {-5} 
\item [
\protect ١٣.\protect ٤٠٧)
]
 \عددی {(0,2)} پر مطلق زیادہ سے زیادہ قیمت \عددی {4} جبکہ \عددی {(0,0)} پر مطلق کم سے کم قیمت \عددی {0} 
\item [
\protect ١٣.\protect ٤٠٩)
]
 \عددی {(0,-3)} پر مطلق زیادہ سے زیادہ قیمت \عددی {11} جبکہ \عددی {(4,-2)} پر مطلق کم سے کم قیمت \عددی {-10} 
\item [
\protect ١٣.\protect ٤١١)
]
 \عددی {(2,0)} پر مطلق زیادہ سے زیادہ قیمت \عددی {4} جبکہ \عددی {(3,-\tfrac {\pi }{4})}، \عددی {(3,\tfrac {\pi }{4})}، \عددی {(1,-\tfrac {\pi }{4})} اور \عددی {(1,\tfrac {\pi }{4})} پر مطلق کم سے کم قیمت \عددی {(\tfrac {3\sqrt {2}}{2})} 
\item [
\protect ١٣.\protect ٤١٣)
]
 \عددی {a=-3,\, b=2} 
\item [
\protect ١٣.\protect ٤١٥)
]
 \عددی {(-\tfrac {1}{2},\tfrac {\sqrt {3}}{2})} اور \عددی {(-\tfrac {1}{2},-\tfrac {\sqrt {3}}{2})} پر گرم ترین \عددی { \SI {2.25}{\celsius }} جبکہ \عددی {(\tfrac {1}{2},0)} پر سرد ترین \عددی {\SI {-0.25}{\celsius }} 
\item [
\protect ١٣.\protect ٤١٧)
]
 (ا) \عددی {f(0,0)} نقطہ زین، (ب) \عددی {f(1,2)} مقامی کم سے کم قیمت نقطہ، (ج) \عددی {f(1,-2)} مقامی کم سے کم قیمت نقطہ، (د) \عددی {f(-1,-2)} نقطہ زین 
\item [
\protect ١٣.\protect ٤٢٣)
]
 $(\tfrac {1}{6},\tfrac {1}{3},\tfrac {355}{36})$ 
\item [
\protect ١٣.\protect ٤٢٧)
]
 (ا) نصف دائرہ : \عددی {t=\tfrac {\pi }{4}} پر مقامی زیادہ سے زیادہ قیمت \عددی {f=2\sqrt {2}} جبکہ \عددی {t=\pi } پر مقامی کم سے کم قیمت \عددی {f=-2}؛ چوتھائی دائرہ: \عددی {t=\tfrac {\pi }{4}} پر مقامی زیادہ سے زیادہ قیمت \عددی {f=2\sqrt {2}} جبکہ \عددی {t=0} اور \عددی {t=\tfrac {\pi }{2}} پر مقامی کم سے کم قیمت \عددی {f=2} (ب) نصف دائرہ: \عددی {t=\tfrac {\pi }{4}} پر مقامی زیادہ سے زیادہ قیمت \عددی {g=2} جبکہ \عددی {t=\tfrac {3\pi }{4}} پر مقامی کم سے کم قیمت \عددی {g=-2}؛ چوتھائی دائرہ: \عددی {t=\tfrac {\pi }{4}} پر مقامی زیادہ سے زیادہ قیمت \عددی {g=2} جبکہ \عددی {t=0} اور \عددی {t=\tfrac {\pi }{2}} پر مقامی کم سے کم قیمت \عددی {g=0}؛ (ج) نصف دائرہ: \عددی {t=0} اور \عددی {t=\pi } پر مقامی زیادہ سے زیادہ قیمت \عددی {h=8} جبکہ \عددی {t=\tfrac {\pi }{2}} پر کم سے کم قیمت \عددی {h=4}؛ چوتھائی دائرہ: \عددی {t=0} پر مقامی زیادہ سے زیادہ قیمت \عددی {h=8} جبکہ \عددی {t=\tfrac {\pi }{2}} پر مقامی کم سے کم قیمت \عددی {h=4} 
\item [
\protect ١٣.\protect ٤٢٩)
]
 (ا) \عددی {t=-\tfrac {1}{2}} پر کم سے کم قیمت \عددی {f=-\tfrac {1}{2}} جبکہ کوئی زیادہ سے زیادہ قیمت نہیں پائی جاتی ہے۔ (ب) \عددی {t=-1} اور \عددی {t=0} پر زیادہ سے زیادہ \عددی {f=0} جبکہ \عددی {t=-\tfrac {1}{2}} پر کم سے کم \عددی {f=-\tfrac {1}{2}} پائی جاتی ہے۔ (ج) \عددی {t=1} پر زیادہ سے زیادہ \عددی {f=4} جبکہ \عددی {t=0} پر کم سے کم قیمت \عددی {f=0} پائی جاتی ہے۔ 
\item [
\protect ١٣.\protect ٤٣١)
]
 $y=-\frac {20}{13}x+\frac {9}{13},\quad \left .y\right \vert _{x=4}=-\frac {71}{13}$ 
\item [
\protect ١٣.\protect ٤٣٣)
]
 $y=\frac {3}{2}x+\frac {1}{6},\quad \left .y\right \vert _{x=4}=\frac {37}{6}$ 
\item [
\protect ١٣.\protect ٤٣٥)
]
 $y=51.3x+3467$ 
\end {description}
 {\urduTechTermsfont {حصہ}} \protect ١٣.\protect ٩\hskip 1em\relax {\urduTechTermsfont {صفحہ}} \protect ١٥٥٤
\begin {description}\setlength {\parskip }{0pt} \setlength {\itemsep }{0pt plus 1pt}
\item [
\protect ١٣.\protect ٤٤٣)
]
 $(\mp \tfrac {1}{\sqrt {2}},\tfrac {1}{2})$,$(\mp \tfrac {1}{\sqrt {2}},-\tfrac {1}{2})$ 
\item [
\protect ١٣.\protect ٤٤٥)
]
 $39$ 
\item [
\protect ١٣.\protect ٤٤٧)
]
 $(3,\mp 3\sqrt {2})$ 
\item [
\protect ١٣.\protect ٤٤٩)
]
 (ا) \عددی {8}، (ب) \عددی {64} 
\item [
\protect ١٣.\protect ٤٥١)
]
 $r=\SI {2}{\centi \meter },\,h=\SI {4}{\centi \meter }$ 
\item [
\protect ١٣.\protect ٤٥٣)
]
 $l=4\sqrt {2},\, w=3\sqrt {2}$ 
\item [
\protect ١٣.\protect ٤٥٥)
]
 کم سے کم \عددی {f(0,0)=0}، زیادہ سے زیادہ \عددی {f(2,4)=20} 
\item [
\protect ١٣.\protect ٤٥٧)
]
 کم سے کم \عددی {0^{\circ }}، زیادہ سے زیادہ \عددی {125^{\circ }} 
\item [
\protect ١٣.\protect ٤٥٩)
]
 $(\tfrac {3}{2},2,\tfrac {5}{2})$ 
\item [
\protect ١٣.\protect ٤٦١)
]
 $1$ 
\item [
\protect ١٣.\protect ٤٦٣)
]
 $(0,0,2),\, (0,0,-2)$ 
\item [
\protect ١٣.\protect ٤٦٥)
]
 کم سے کم \عددی {f(-1,2,-5)=-30}، زیادہ سے زیادہ \عددی {f(1,-2,5)=30} 
\item [
\protect ١٣.\protect ٤٦٧)
]
 $3,3,3$ 
\item [
\protect ١٣.\protect ٤٦٩)
]
 \عددی {\tfrac {2}{\sqrt {3}}} ضرب \عددی {\tfrac {2}{\sqrt {3}}} ضرب \عددی {\tfrac {2}{\sqrt {3}}} اکائیاں 
\item [
\protect ١٣.\protect ٤٧١)
]
 $(\mp \tfrac {4}{3},-\tfrac {4}{3},-\tfrac {4}{3})$ 
\item [
\protect ١٣.\protect ٤٧٣)
]
 $U(8,14)=128$ 
\item [
\protect ١٣.\protect ٤٧٥)
]
 $f(\tfrac {2}{3},\tfrac {4}{3},-\tfrac {4}{3})=\tfrac {4}{3}$ 
\item [
\protect ١٣.\protect ٤٧٧)
]
 $(2,4,4)$ 
\item [
\protect ١٣.\protect ٤٧٩)
]
 \عددی {(\mp \sqrt {6},-\sqrt {3},1)} پر کم سے کم \عددی {1-6\sqrt {3}} اور \عددی {(\mp \sqrt {6},\sqrt {3},1)} پر زیادہ سے زیادہ \عددی {1+6\sqrt {3}} 
\item [
\protect ١٣.\protect ٤٨١)
]
 \عددی {(\mp \sqrt {2},\mp \sqrt {2},0)} پر کم سے کم \عددی {2} اور \عددی {(0,0,\mp 2)} پر زیادہ سے زیادہ \عددی {4} 
\end {description}
 {\urduTechTermsfont {حصہ}} \protect ١٣.\protect ١٠\hskip 1em\relax {\urduTechTermsfont {صفحہ}} \protect ١٥٦٤
\begin {description}\setlength {\parskip }{0pt} \setlength {\itemsep }{0pt plus 1pt}
\item [
\protect ١٣.\protect ٤٩٣)
]
 مربعی \عددی {x+xy}، کعبی \عددی {x+xy+\tfrac {1}{2}xy^2} 
\item [
\protect ١٣.\protect ٤٩٥)
]
 مربعی \عددی {xy}، کعبی \عددی {xy} 
\item [
\protect ١٣.\protect ٤٩٧)
]
 مربعی \عددی {y+\tfrac {1}{2}(2xy-y^2)}، کعبی\\ \عددی {y+\tfrac {1}{2}(2xy-y^2)+\tfrac {1}{6}(3x^2y-3xy^2+2y^3)} 
\item [
\protect ١٣.\protect ٤٩٩)
]
 مربعی \عددی {x^2+y^2}، کعبی \عددی {x^2+y^2} 
\item [
\protect ١٣.\protect ٥٠١)
]
 مربعی \عددی {1+(x+y)+(x+y)^2}، کعبی \\ \عددی {1+(x+y)+(x+y)^2+(x+y)^3} 
\item [
\protect ١٣.\protect ٥٠٣)
]
 مربعی \عددی {1-\tfrac {1}{2}x^2-\tfrac {1}{2}y^2}؛ \\ خلل \عددی {E(x,y)\le 0.00134} 
\end {description}
 {\urduTechTermsfont {حصہ}} \protect ١٤.\protect ١\hskip 1em\relax {\urduTechTermsfont {صفحہ}} \protect ١٥٨١
\begin {description}\setlength {\parskip }{0pt} \setlength {\itemsep }{0pt plus 1pt}
\item [
\protect ١٤.\protect ١)
]
 $16$ \begin {center} \begin {tikzpicture}[font=\scriptsize ] \pgfmathsetmacro {\a }{2.5} \pgfmathsetmacro {\b }{1.5} \draw [-latex](0,0)--(3,0)node[right]{$x$}; \draw [-latex](0,0)--(0,2)node[above]{$y$}; \draw [fill=lgray](0,0) rectangle++(\a ,\b )node[above right]{$(3,2)$}; \draw (0,0)node[below left]{$0$} (\a ,0)node[below]{$3$} (0,\b )node[left]{$2$}; \end {tikzpicture} \end {center} 
\item [
\protect ١٤.\protect ٣)
]
 $1$ \begin {center} \begin {tikzpicture}[font=\scriptsize ] \pgfmathsetmacro {\a }{2} \pgfmathsetmacro {\b }{1} \draw [-latex](-1.25,0)--(1.5,0)node[right]{$x$}; \draw [-latex](0,-1.25)--(0,0.5)node[above]{$y$}; \draw [fill=lgray](-1,0) rectangle++(\a ,-\b )node[right]{$(1,-1)$}; \draw (-1,0)node[below left]{$-1$} (-1,-\b )node[left]{$(-1,-1)$} (1,0)node[below right]{$1$}; \end {tikzpicture} \end {center} 
\item [
\protect ١٤.\protect ٥)
]
 $\frac {\pi ^2}{2}+2$ \begin {center} \begin {tikzpicture}[font=\scriptsize ] \pgfmathsetmacro {\a }{1} \pgfmathsetmacro {\b }{1} \draw [-latex](0,0)--(1.5,0)node[right]{$x$}; \draw [-latex](0,0)--(0,1.5)node[above]{$y$}; \draw [fill=lgray](0,0)--(\a ,\b )node[right]{$(\pi ,\pi )$}--(\a ,0)node[below]{$\pi $}--cycle; \draw (0,1)node[left]{$\pi $}--++(0.1,0) (0,0)node[below left]{$0$}; \end {tikzpicture} \end {center} 
\item [
\protect ١٤.\protect ٧)
]
 $8\ln 8-16+e$ \begin {center} \begin {tikzpicture}[font=\scriptsize ,declare function={f(\y )=ln(\y );}] \pgfmathsetmacro {\a }{ln(8)} \pgfmathsetmacro {\b }{ln(ln(8))} \begin {axis}[axis on top,small,axis lines=middle,xlabel={$x$},ylabel={$y$},xtick={\b },xticklabels={$\ln \ln 8$},ytick={1,\a },yticklabels={$1$,$\ln 8$},ymin=0,enlargelimits=true,xlabel style={at={(current axis.right of origin)},anchor=west},ylabel style={at={(current axis.above origin)},anchor=south},xmax=1.2] \addplot [name path=kf,domain=1:ln(8)]({f(x)},{x})node[pos=0.25,below right]{$x=\ln y$}node[right]{$(\ln \ln 8,\ln 8)$}; \addplot [draw=none,name path=ky]coordinates{(0,1)(0,\a )}; \addplot []coordinates{(0,\a )(\b ,\a )}; \addplot [lgray]fill between[of={kf and ky}]; \end {axis} \end {tikzpicture} \end {center} 
\item [
\protect ١٤.\protect ٩)
]
 $e-2$ \begin {center} \begin {tikzpicture}[font=\scriptsize ,declare function={f(\y )=(\y )^2;}] \pgfmathsetmacro {\a }{0} \pgfmathsetmacro {\b }{1} \begin {axis}[axis on top,small,axis lines=middle,xlabel={$x$},ylabel={$y$},xtick={1},xticklabels={$1$},ytick={1},yticklabels={$1$}, ymin=0,enlargelimits=true,xlabel style={at={(current axis.right of origin)},anchor=west},ylabel style={at={(current axis.above origin)},anchor=south},xmax=1.1] \addplot [name path=kf,domain=0:1]({f(x)},{x})node[pos=0.5,below right]{$x=y^2$}node[right]{$(1,1)$}; \addplot [draw=none,name path=ky]coordinates{(0,0)(0,1)}; \addplot []coordinates{(0,1)(1,1)}; \addplot [lgray]fill between[of={kf and ky}]; \end {axis} \end {tikzpicture} \end {center} 
\item [
\protect ١٤.\protect ١١)
]
 $\frac {3}{2}\ln 2$ 
\item [
\protect ١٤.\protect ١٣)
]
 $\frac {1}{6}$ 
\item [
\protect ١٤.\protect ١٥)
]
 $-\frac {1}{10}$ 
\item [
\protect ١٤.\protect ١٧)
]
 $8$ \begin {center} \begin {tikzpicture}[font=\scriptsize ,declare function={f(\y )=ln(\y );}] \pgfmathsetmacro {\a }{ln(8)} \pgfmathsetmacro {\b }{ln(ln(8))} \begin {axis}[clip=false,axis on top,small,axis lines=middle,xlabel={$p$},ylabel={$v$},xtick={-2,2},xticklabels={$-2$,$2$},ytick={-2}, yticklabels={$-2$},enlargelimits=true,xlabel style={at={(current axis.right of origin)},anchor=west},ylabel style={at={(current axis.above origin)},anchor=south}] \addplot [fill=lgray]coordinates{(0,0)(2,-2)(-2,-2)(0,0)}; \addplot []coordinates{(-2,-2)}node[left]{$(-2,-2)$}; \addplot []coordinates{(2,-2)}node[right]{$(2,-2)$}; \addplot []coordinates{(-1,-1)}node[above left]{$v=p$}; \addplot []coordinates{(1,-1)}node[above right]{$v=-p$}; \end {axis} \end {tikzpicture} \end {center} 
\item [
\protect ١٤.\protect ١٩)
]
 $2\pi $ \begin {center} \begin {tikzpicture}[font=\scriptsize ,declare function={f(\x )=1/(cos(deg(\x )));}] \pgfmathsetmacro {\a }{-pi/3} \pgfmathsetmacro {\b }{pi/3} \begin {axis}[clip=false,axis on top,small,axis lines=middle,xlabel={$t$},ylabel={$u$},xtick={\a ,\b },xticklabels={$-\tfrac {\pi }{3}$,$\tfrac {\pi }{3}$},ytick={1,2},yticklabels={$1$,$2$}, ymin=0,enlargelimits=true,xlabel style={at={(current axis.right of origin)},anchor=west},ylabel style={at={(current axis.above origin)},anchor=south},xmax=1.1] \addplot [name path=kf,domain=\a :\b ](x,{f(x)})node[pos=0.75,above left]{$u=\sec t$}node[above,pos=0]{$(-\tfrac {\pi }{3},2)$}node[above]{$(\tfrac {\pi }{3},2)$}; \addplot [draw=none,name path=kx]coordinates{(\a ,0)(\b ,0)}; \addplot [lgray]fill between[of={kf and kx}]; \addplot []coordinates{(\a ,0)(\a ,2)}; \addplot []coordinates{(\b ,0)(\b ,2)}; \end {axis} \end {tikzpicture} \end {center} 
\item [
\protect ١٤.\protect ٢١)
]
 $\int _2^4\int _0^{(4-y)/2}\dif x\dif y$ \begin {center} \begin {tikzpicture}[font=\scriptsize ,declare function={f(\x )=4-2*\x ;}] \pgfmathsetmacro {\a }{0} \pgfmathsetmacro {\b }{1} \begin {axis}[clip=false,axis on top,small,axis lines=middle,xlabel={$x$},ylabel={$y$},xtick={1},xticklabels={$1$},ytick={2,4},yticklabels={$2$,$4$}, ymin=0,enlargelimits=true,xlabel style={at={(current axis.right of origin)},anchor=west},ylabel style={at={(current axis.above origin)},anchor=south},xmax=1.1] \addplot [name path=kf,domain=\a :\b ](x,{f(x)})node[pos=0.5,above right]{$y=4-2x$}node[right]{$(1,2)$}; \addplot [name path=kx]coordinates{(0,2)(1,2)}; \addplot [lgray]fill between[of={kf and kx}]; \end {axis} \end {tikzpicture} \end {center} 
\item [
\protect ١٤.\protect ٢٣)
]
 $\int _0^1\int _{x^2}^{x}\dif y\dif x$ \begin {center} \begin {tikzpicture}[font=\scriptsize ,declare function={f(\x )=\x ^2;g(\x )=\x ;}] \pgfmathsetmacro {\a }{0} \pgfmathsetmacro {\b }{1} \begin {axis}[clip=false,axis on top,small,axis lines=middle,xlabel={$x$},ylabel={$y$},xtick={1},xticklabels={$1$},ytick={1},yticklabels={$1$}, ymin=0,enlargelimits=true,xlabel style={at={(current axis.right of origin)},anchor=west},ylabel style={at={(current axis.above origin)},anchor=south},xmax=1.1] \addplot [name path=kf,domain=\a :\b ](x,{f(x)})node[pos=0.5,below right]{$y=x^2$}node[above,right]{$(1,1)$}; \addplot [name path=kg,domain=\a :\b ](x,{g(x)})node[pos=0.5,above left]{$y=x$}; \addplot [lgray]fill between[of={kf and kg}]; \end {axis} \end {tikzpicture} \end {center} 
\item [
\protect ١٤.\protect ٢٥)
]
 $\int _1^e\int _{\ln y}^{1}\dif x\dif y$ \begin {center} \begin {tikzpicture}[font=\scriptsize ,declare function={f(\x )=e^(\x );}] \pgfmathsetmacro {\a }{0} \pgfmathsetmacro {\b }{1} \pgfmathsetmacro {\e }{2.718} \begin {axis}[clip=false,axis on top,small,axis lines=middle,xlabel={$x$},ylabel={$y$},xtick={1},xticklabels={$1$},ytick={1,\e },yticklabels={$1$,$e$}, ymin=0,enlargelimits=true,xlabel style={at={(current axis.right of origin)},anchor=west},ylabel style={at={(current axis.above origin)},anchor=south},xmax=1.1] \addplot [name path=kf,domain=\a :\b ](x,{f(x)})node[pos=0.5,above left]{$y=e^x$}node[above,right]{$(1,e)$}; \addplot [name path=kx]coordinates{(0,1)(1,1)}node[right]{$(1,1)$}; \addplot [lgray]fill between[of={kf and kx}]; \addplot []coordinates{(1,1)(1,\e )}; \end {axis} \end {tikzpicture} \end {center} 
\item [
\protect ١٤.\protect ٢٧)
]
 $\int _0^9\int _{0}^{(\sqrt {9-y})/2}16x\dif x\dif y$ \begin {center} \begin {tikzpicture}[font=\scriptsize ,declare function={f(\x )=9-4*(\x )^2;}] \pgfmathsetmacro {\a }{0} \pgfmathsetmacro {\b }{3/2} \pgfmathsetmacro {\e }{2.718} \begin {axis}[clip=false,axis on top,small,axis lines=middle,xlabel={$x$},ylabel={$y$},xtick={1.5},xticklabels={$\tfrac {3}{2}$},ytick={9},yticklabels={$9$}, ymin=0,enlargelimits=true,xlabel style={at={(current axis.right of origin)},anchor=west},ylabel style={at={(current axis.above origin)},anchor=south}] \addplot [name path=kf,domain=\a :\b ](x,{f(x)})node[pos=0.5,above right]{$y=9-4x^2$}; \addplot [name path=kx,draw=none]coordinates{(0,0)(1.5,0)}; \addplot [lgray]fill between[of={kf and kx}]; \end {axis} \end {tikzpicture} \end {center} 
\item [
\protect ١٤.\protect ٢٩)
]
 $\int _{-1}^{1}\int _{0}^{\sqrt {1-x^2}}3y\dif y\dif x$ \begin {center} \begin {tikzpicture}[font=\scriptsize ,declare function={fx(\x )=cos(\x );fy(\x )=sin(\x );}] \pgfmathsetmacro {\a }{0} \pgfmathsetmacro {\b }{180} \pgfmathsetmacro {\e }{2.718} \begin {axis}[axis equal,clip=false,axis on top,small,axis lines=middle,xlabel={$x$},ylabel={$y$},xtick={-1,1},xticklabels={$-1$,$1$},ytick={1},yticklabels={$1$}, ymin=0,ymax=1.25,enlargelimits=true,xlabel style={at={(current axis.right of origin)},anchor=west},ylabel style={at={(current axis.above origin)},anchor=south}] \addplot [name path=kf,domain=\a :\b ]({fx(x)},{fy(x)})node[pos=0.25,above right]{$x^2+y^2=1$}; \addplot [name path=kx]coordinates{(-1,0)(1,0)}; \addplot [lgray]fill between[of={kf and kx}]; \end {axis} \end {tikzpicture} \end {center} 
\item [
\protect ١٤.\protect ٣١)
]
 $2$ \begin {center} \begin {tikzpicture}[font=\scriptsize ,declare function={f(\x )=\x ;}] \pgfmathsetmacro {\a }{0} \pgfmathsetmacro {\b }{pi} \begin {axis}[clip=false,axis on top,small,axis lines=middle,xlabel={$x$},ylabel={$y$},xtick={\b },xticklabels={$\pi $},ytick={\b },yticklabels={$\pi $}, ymin=0,enlargelimits=true,xlabel style={at={(current axis.right of origin)},anchor=west},ylabel style={at={(current axis.above origin)},anchor=south},xmax=4] \addplot [name path=kf,domain=\a :\b ](x,{f(x)})node[pos=0.5,below right]{$y=x$}node[right]{$(\pi ,\pi )$}; \addplot [name path=kx]coordinates{(0,\b )(\b ,\b )}; \addplot [lgray]fill between[of={kf and kx}]; \addplot []coordinates{(0,\b )(\b ,\b )}; \end {axis} \end {tikzpicture} \end {center} 
\item [
\protect ١٤.\protect ٣٣)
]
 $\frac {e-2}{2}$ \begin {center} \begin {tikzpicture}[font=\scriptsize ,declare function={f(\x )=\x ;}] \pgfmathsetmacro {\a }{0} \pgfmathsetmacro {\b }{1} \begin {axis}[clip=false,axis on top,small,axis lines=middle,xlabel={$x$},ylabel={$y$},xtick={1},xticklabels={$1$},ytick={1},yticklabels={$1$}, ymin=0,enlargelimits=true,xlabel style={at={(current axis.right of origin)},anchor=west},ylabel style={at={(current axis.above origin)},anchor=south},xmax=1.25] \addplot [name path=kf,domain=\a :\b ](x,{f(x)})node[pos=0.5,above left]{$y=x$}node[above,right]{$(1,1)$}; \addplot [name path=kx]coordinates{(0,0)(1,0)}; \addplot [lgray]fill between[of={kf and kx}]; \addplot []coordinates{(1,0)(1,1)}; \end {axis} \end {tikzpicture} \end {center} 
\item [
\protect ١٤.\protect ٣٥)
]
 $2$ \begin {center} \begin {tikzpicture}[font=\scriptsize ,declare function={f(\x )=2*\x ;}] \pgfmathsetmacro {\a }{0} \pgfmathsetmacro {\b }{sqrt(ln(3))} \pgfmathsetmacro {\c }{2*\b } \begin {axis}[clip=false,axis on top,small,axis lines=middle,xlabel={$x$},ylabel={$y$},xtick={\b },xticklabels={$\sqrt {\ln 3}$},ytick={\c },yticklabels={$2\sqrt {\ln 3}$}, ymin=0,enlargelimits=true,xlabel style={at={(current axis.right of origin)},anchor=west},ylabel style={at={(current axis.above origin)},anchor=south},xmax=1.5] \addplot [name path=kf,domain=\a :\b ](x,{f(x)})node[pos=0.5,above left]{$y=2x$}node[above,right]{$(\sqrt {\ln 3},2\sqrt {\ln 3})$}; \addplot [name path=kx]coordinates{(0,0)(\b ,0)}; \addplot [lgray]fill between[of={kf and kx}]; \addplot []coordinates{(\b ,0)(\b ,\c )}; \end {axis} \end {tikzpicture} \end {center} 
\item [
\protect ١٤.\protect ٣٧)
]
 $\tfrac {1}{80\pi }$ \begin {center} \begin {tikzpicture}[font=\scriptsize ,declare function={f(\x )=\x ^4;}] \pgfmathsetmacro {\a }{0} \pgfmathsetmacro {\b }{0.5} \pgfmathsetmacro {\c }{0.0625} \begin {axis}[scaled y ticks=false,clip=false,axis on top,small,axis lines=middle,xlabel={$x$},ylabel={$y$},xtick={\b },xticklabels={$0.5$},ytick={\c },yticklabels={$0.0625$}, ymin=0,enlargelimits=true,xlabel style={at={(current axis.right of origin)},anchor=west},ylabel style={at={(current axis.above origin)},anchor=south},xmax=0.75] \addplot [name path=kf,domain=\a :\b ](x,{f(x)})node[pos=0.75,above left]{$y=x^4$}node[above,]{$(0.5,0.0625)$}; \addplot [name path=kx]coordinates{(0,0)(\b ,0)}; \addplot [lgray]fill between[of={kf and kx}]; \addplot []coordinates{(\b ,0)(\b ,\c )}; \end {axis} \end {tikzpicture} \end {center} 
\item [
\protect ١٤.\protect ٣٩)
]
 $-\tfrac {2}{3}$ 
\item [
\protect ١٤.\protect ٤١)
]
 $\tfrac {4}{3}$ 
\item [
\protect ١٤.\protect ٤٣)
]
 $\tfrac {625}{12}$ 
\item [
\protect ١٤.\protect ٤٥)
]
 $16$ 
\item [
\protect ١٤.\protect ٤٧)
]
 $20$ 
\item [
\protect ١٤.\protect ٤٩)
]
 $2(1+\ln 2)$ 
\item [
\protect ١٤.\protect ٥١)
]
 $1$ 
\item [
\protect ١٤.\protect ٥٣)
]
 $\pi ^2$ 
\item [
\protect ١٤.\protect ٥٥)
]
 $-\tfrac {1}{4}$ 
\item [
\protect ١٤.\protect ٥٧)
]
 $\tfrac {20\sqrt {3}}{9}$ 
\item [
\protect ١٤.\protect ٥٩)
]
 $\int _0^1\int _x^{2-x}(x^2+y^2)\dif y\dif x=\tfrac {4}{3}$ \begin {center} \begin {tikzpicture}[font=\scriptsize ,declare function={f(\x )=2-\x ;g(\x )=\x ;}] \pgfmathsetmacro {\a }{0} \pgfmathsetmacro {\b }{1} \begin {axis}[clip=false,axis on top,small,axis lines=middle,xlabel={$x$},ylabel={$y$},xtick={\b },xticklabels={$1$},ytick={1,2},yticklabels={$1$,$2$}, ymin=0,enlargelimits=true,xlabel style={at={(current axis.right of origin)},anchor=west},ylabel style={at={(current axis.above origin)},anchor=south}] \addplot [name path=kf,domain=\a :\b ](x,{f(x)})node[pos=0.5,sloped,above]{$y=2-x$}; \addplot [name path=kg,domain=\a :\b ](x,{g(x)})node[pos=0.5,sloped,below]{$y=x$}; \addplot [lgray]fill between[of={kf and kg}]; \end {axis} \end {tikzpicture} \end {center} 
\item [
\protect ١٤.\protect ٦٧)
]
 $0.603$ 
\item [
\protect ١٤.\protect ٦٩)
]
 $0.233$ 
\end {description}
 {\urduTechTermsfont {حصہ}} \protect ١٤.\protect ٢\hskip 1em\relax {\urduTechTermsfont {صفحہ}} \protect ١٥٩٥
\begin {description}\setlength {\parskip }{0pt} \setlength {\itemsep }{0pt plus 1pt}
\item [
\protect ١٤.\protect ٧١)
]
 $\int _0^2\int _0^{2-x}\dif y\dif x=2,\,\, \int _0^2\int _0^{2-y}\dif x\dif y=2$ \begin {center} \begin {tikzpicture}[font=\small ] \draw [-latex](0,0)--(3,0)node[right]{$x$}; \draw [-latex](0,0)--(0,2.5)node[left]{$y$}; \draw [fill=lgray](0,2)node[left]{$2$}--(2,0)node[pos=0.5,above right,font=\scriptsize ]{$y=2-x$}node[below]{$2$}--(0,0)--(0,2); \end {tikzpicture} \end {center} 
\item [
\protect ١٤.\protect ٧٣)
]
 $\int _{-2}^1\int _{y-2}^{-y^2}\dif x\dif y=\frac {9}{2}$ \begin {center} \begin {tikzpicture}[font=\small ,declare function={f(\y )=-\y ^2;}] \begin {axis}[axis on top,clip=false,small,axis lines=center,view/h=130,colormap={}{gray(0cm)=(0.6);gray(1cm)=(0.9);},enlargelimits=true,xlabel={$x$},ylabel={$y$},zlabel={$z$},xtick={-2,-4},ytick={-2,1},ztick={\empty },xlabel style={anchor=west},ylabel style={anchor=south}] \addplot [name path=kf,domain=-2:1]({f(x)},x)node[pos=0.25,below right]{$x=-y^2$}; \addplot [name path=kg]coordinates{(-1,1)(-4,-2)}node[pos=0,above]{$(-1,1)$}node[below]{$(-4,-2)$}node[pos=0.75,above left]{$y=x+2$}; \addplot [lgray]fill between[of={kf and kg}]; \end {axis} \end {tikzpicture} \end {center} 
\item [
\protect ١٤.\protect ٧٥)
]
 $\int _0^{\ln 2}\int _0^{e^x}\dif y\dif x=1$ \begin {center} \begin {tikzpicture}[font=\small ,declare function={f(\x )=e^(\x );}] \pgfmathsetmacro {\k }{ln(2)} \begin {axis}[axis on top,clip=false,small,axis lines=center,view/h=130,colormap={}{gray(0cm)=(0.6);gray(1cm)=(0.9);},enlargelimits=true,xlabel={$x$},ylabel={$y$},zlabel={$z$},xtick={\k },xticklabels={$\ln 2$},ytick={1,2},ztick={\empty },xlabel style={anchor=west},ylabel style={anchor=south},ymin=0] \addplot [name path=kf,domain=0:\k ](x,{f(x)})node[pos=0.5,above left]{$y=e^x$}node[above]{$(\ln 2,2)$}; \addplot [name path=kg]coordinates{(\k ,2)(\k ,0)}; \addplot [draw=none,name path=kx]coordinates{(0,0)(\k ,0)}; \addplot [lgray]fill between[of={kf and kx}]; \end {axis} \end {tikzpicture} \end {center} 
\item [
\protect ١٤.\protect ٧٧)
]
 $\int _0^1\int _{y^2}^{2y-y^2}\dif x\dif y=\frac {1}{3}$ \begin {center} \begin {tikzpicture}[font=\small ,declare function={f(\y )=\y ^2;g(\y )=2*\y -\y ^2;}] \pgfmathsetmacro {\k }{ln(2)} \begin {axis}[axis on top,clip=false,small,axis lines=center,view/h=130,colormap={}{gray(0cm)=(0.6);gray(1cm)=(0.9);},enlargelimits=true,xlabel={$x$},ylabel={$y$},zlabel={$z$},xtick={1},ytick={1},ztick={\empty },xlabel style={anchor=west},ylabel style={anchor=south},ymin=0] \addplot [name path=kf,domain=0:1]({f(x)},x)node[pos=0.5,above left]{$x=y^2$}node[above]{$(1,1)$}; \addplot [name path=kg,domain=0:1](g(x),x)node[pos=0.45,below right]{$x=2y-y^2$}; \addplot [lgray]fill between[of={kf and kg}]; \end {axis} \end {tikzpicture} \end {center} 
\item [
\protect ١٤.\protect ٧٩)
]
 $12$ \begin {center} \begin {tikzpicture}[font=\small ,declare function={f(\x )=sqrt(3*\x );g(\x )=1/2*\x ;}] \pgfmathsetmacro {\k }{ln(2)} \begin {axis}[axis on top,clip=false,small,axis lines=center,view/h=130,colormap={}{gray(0cm)=(0.6);gray(1cm)=(0.9);},enlargelimits=true,xlabel={$x$},ylabel={$y$},zlabel={$z$},xtick={12},ytick={6},ztick={\empty },xlabel style={anchor=west},ylabel style={anchor=south},ymin=0] \addplot [name path=kfa,domain=0:1](x,{f(x)}); \addplot [name path=kf,domain=1:12](x,{f(x)})node[pos=0.5,above left]{$y^2=3x$}node[above]{$(12,6)$}; \addplot [name path=kg,domain=0:12](x,{g(x)})node[pos=0.45,below right]{$y=\tfrac {x}{2}$}; \addplot [lgray]fill between[of={kfa and kg},soft clip={domain=0:1}]; \addplot [lgray]fill between[of={kf and kg},soft clip={domain=1:12}]; \end {axis} \end {tikzpicture} \end {center} 
\item [
\protect ١٤.\protect ٨١)
]
 $\sqrt {2}-1$ \begin {center} \begin {tikzpicture}[font=\small ,declare function={f(\x )=cos(deg(\x ));g(\x )=sin(deg(\x ));}] \pgfmathsetmacro {\k }{pi/4} \pgfmathsetmacro {\kk }{1/sqrt(2)} \begin {axis}[axis on top,clip=false,small,axis lines=center,view/h=130,colormap={}{gray(0cm)=(0.6);gray(1cm)=(0.9);},enlargelimits=true,xlabel={$x$},ylabel={$y$},zlabel={$z$},xtick={\k },xticklabels={$\tfrac {\pi }{4}$},ytick={\kk },yticklabels={$\tfrac {\sqrt {2}}{2}$},ztick={\empty },xlabel style={anchor=west},ylabel style={anchor=south},ymin=0] \addplot [name path=kf,domain=0:\k ](x,{f(x)})node[pos=0.5,above right]{$y=\cos x$}node[right]{$(\tfrac {\pi }{4},\tfrac {\sqrt {2}}{2})$}; \addplot [name path=kg,domain=0:\k ](x,{g(x)})node[pos=0.45,below right]{$y=\sin x$}; \addplot [lgray]fill between[of={kf and kg}]; \end {axis} \end {tikzpicture} \end {center} 
\item [
\protect ١٤.\protect ٨٣)
]
 $\frac {3}{2}$ \begin {center} \begin {tikzpicture}[font=\small ,declare function={f(\x )=1-\x ;g(\x )=-1/2*\x ;h(\x )=-2*\x ;}] \begin {axis}[axis on top,clip=false,small,axis lines=center,view/h=130,colormap={}{gray(0cm)=(0.6);gray(1cm)=(0.9);},enlargelimits=true,xlabel={$x$},ylabel={$y$},zlabel={$z$},xtick={-1,2},ytick={2},ztick={\empty },xlabel style={anchor=west},ylabel style={anchor=south}] \addplot [name path=kf,domain=-1:2](x,{f(x)})node[pos=0.5,above right]{$y=1-x$}node[pos=0,above]{$(-1,2)$}node[right]{$(2,-1)$}; \addplot [name path=kg,domain=0:2](x,{g(x)})node[pos=0.55,below left]{$y=-\tfrac {x}{2}$}; \addplot [name path=kh,domain=-1:0](x,{h(x)})node[pos=0.5,below left]{$y=-2x$}; \addplot [lgray]fill between[of={kf and kg},soft clip={domain=0:2}]; \addplot [lgray]fill between[of={kf and kh},soft clip={domain=-1:0}]; \end {axis} \end {tikzpicture} \end {center} 
\item [
\protect ١٤.\protect ٨٥)
]
 (ا) \عددی {0}، (ب) \عددی {\tfrac {4}{\pi ^2}} 
\item [
\protect ١٤.\protect ٨٧)
]
 $\tfrac {8}{3}$ 
\item [
\protect ١٤.\protect ٨٩)
]
 $\bar {x}=\tfrac {5}{14},\,\bar {y}=\tfrac {38}{35}$ 
\item [
\protect ١٤.\protect ٩١)
]
 $\bar {x}=\tfrac {64}{35},\,\bar {y}=\tfrac {5}{7}$ 
\item [
\protect ١٤.\protect ٩٣)
]
 $\bar {x}=0,\,\bar {y}=\tfrac {4}{3\pi }$ 
\item [
\protect ١٤.\protect ٩٥)
]
 $\bar {x}=\bar {y}=\tfrac {4a}{3\pi }$ 
\item [
\protect ١٤.\protect ٩٧)
]
 $\bar {x}=\tfrac {\pi }{2},\,\bar {y}=\tfrac {\pi }{8}$ 
\item [
\protect ١٤.\protect ٩٩)
]
 $\bar {x}=-1,\,\bar {y}=\tfrac {1}{4}$ 
\item [
\protect ١٤.\protect ١٠١)
]
 $I_x=\tfrac {64}{105},\,R_x=2\sqrt {\frac {2}{7}}$ 
\item [
\protect ١٤.\protect ١٠٣)
]
 $\bar {x}=\tfrac {3}{8},\,\bar {y}=\tfrac {17}{16}$ 
\item [
\protect ١٤.\protect ١٠٥)
]
 $\bar {x}=\tfrac {11}{3},\,\bar {y}=\tfrac {14}{27},\, I_y=432,\,R_y=4$ 
\item [
\protect ١٤.\protect ١٠٧)
]
 $\bar {x}=0,\,\bar {y}=\tfrac {13}{31},\, I_y=\tfrac {7}{5},\,R_y=\sqrt {\frac {21}{31}}$ 
\item [
\protect ١٤.\protect ١٠٩)
]
 $\bar {x}=0,\,\bar {y}=\tfrac {7}{10},\,I_x=\frac {9}{10},\, I_y=\frac {3}{10}$\\ $I_0=\frac {6}{5},\, R_x=\frac {3\sqrt {6}}{10},\,R_y=\frac {3\sqrt {2}}{10},\,R_0=\frac {3\sqrt {2}}{5}$ 
\item [
\protect ١٤.\protect ١١١)
]
 $\num {40000}(1-e^{-2})\ln (\tfrac {7}{2})\approx \num {43329}$ 
\item [
\protect ١٤.\protect ١١٣)
]
 \عددی {0<a\le \tfrac {5}{2}} 
\item [
\protect ١٤.\protect ١١٥)
]
 $(\bar {x},\bar {y})=(2/\pi ,0)$ 
\item [
\protect ١٤.\protect ١١٧)
]
 (ا) \عددی {\tfrac {3}{2}}، (ب) ایک جیسے ہیں۔ 
\item [
\protect ١٤.\protect ١٢٣)
]
 (ا) \عددی {(\tfrac {7}{5},\tfrac {31}{10})}، (ب) \عددی {(\tfrac {19}{7},\tfrac {18}{7})}، (ج) \عددی {(\tfrac {9}{2},\tfrac {19}{8})}، (د) \عددی {(\tfrac {11}{4},\tfrac {43}{16})} 
\item [
\protect ١٤.\protect ١٢٥)
]
 مشترک سرحد پر ہونے کے لئے \عددی {h=a\sqrt {2}}، مثلث کے اندر ہونے کے لئے \عددی {h>a\sqrt {2}} 
\end {description}
 {\urduTechTermsfont {حصہ}} \protect ١٤.\protect ٣\hskip 1em\relax {\urduTechTermsfont {صفحہ}} \protect ١٦٠٧
\begin {description}\setlength {\parskip }{0pt} \setlength {\itemsep }{0pt plus 1pt}
\item [
\protect ١٤.\protect ١٢٧)
]
 $\tfrac {\pi }{2}$ 
\item [
\protect ١٤.\protect ١٢٩)
]
 $\tfrac {\pi }{8}$ 
\item [
\protect ١٤.\protect ١٣١)
]
 $\pi a^2$ 
\item [
\protect ١٤.\protect ١٣٣)
]
 $36$ 
\item [
\protect ١٤.\protect ١٣٥)
]
 $(1-\ln 2)\pi $ 
\item [
\protect ١٤.\protect ١٣٧)
]
 $(2\ln 2-1)(\pi /2)$ 
\item [
\protect ١٤.\protect ١٣٩)
]
 $\tfrac {\pi }{2}+1$ 
\item [
\protect ١٤.\protect ١٤١)
]
 $\pi (\ln (4)-1)$ 
\item [
\protect ١٤.\protect ١٤٣)
]
 $2(\pi -1)$ 
\item [
\protect ١٤.\protect ١٤٥)
]
 $12\pi $ 
\item [
\protect ١٤.\protect ١٤٧)
]
 $\tfrac {3\pi }{8}+1$ 
\item [
\protect ١٤.\protect ١٤٩)
]
 $4$ 
\item [
\protect ١٤.\protect ١٥١)
]
 $6\sqrt {3}-2\pi $ 
\item [
\protect ١٤.\protect ١٥٣)
]
 $\bar {x}=\tfrac {5}{6},\,\bar {y}=0$ 
\item [
\protect ١٤.\protect ١٥٥)
]
 $\tfrac {2a}{3}$ 
\item [
\protect ١٤.\protect ١٥٧)
]
 $\tfrac {2a}{3}$ 
\item [
\protect ١٤.\protect ١٥٩)
]
 $2\pi $ 
\item [
\protect ١٤.\protect ١٦١)
]
 $\tfrac {4}{3}+\tfrac {5\pi }{8}$ 
\item [
\protect ١٤.\protect ١٦٣)
]
 (ا) \عددی {\sqrt {\pi /2}}، (ب) \عددی {1} 
\item [
\protect ١٤.\protect ١٦٥)
]
 $\pi \ln 4$\quad نہیں 
\item [
\protect ١٤.\protect ١٦٧)
]
 $\tfrac {1}{2}(a^2+2h^2)$ 
\end {description}
 {\urduTechTermsfont {حصہ}} \protect ١٤.\protect ٤\hskip 1em\relax {\urduTechTermsfont {صفحہ}} \protect ١٦١٩
\begin {description}\setlength {\parskip }{0pt} \setlength {\itemsep }{0pt plus 1pt}
\item [
\protect ١٤.\protect ١٧٣)
]
 $1$ 
\item [
\protect ١٤.\protect ١٧٥)
]
 $\int _{0}^{1}\int _{0}^{2-2x}\int _{0}^{3-3x-3y/2}\dif z\dif y\dif x,$\\ $\int _{0}^{2}\int _{0}^{1-y/2}\int _{0}^{3-3x-3y/2}\dif z\dif x\dif y,$\\ $\int _{0}^{1}\int _{0}^{3-3x}\int _{0}^{2-2x-2z/3}\dif y\dif z\dif x,$\\ $\int _{0}^{3}\int _{0}^{1-z/3}\int _{0}^{2-2x-2z/3}\dif y\dif x\dif z,$\\ $\int _{0}^{2}\int _{0}^{3-3y/2}\int _{0}^{1-y/2-z/3}\dif x\dif z\dif y,$\\ $\int _{0}^{3}\int _{0}^{2-2z/3}\int _{0}^{1-y/2-z/3}\dif x\dif y\dif z$\quad تمام تکملات کا جواب \عددی {1} ہے۔ 
\item [
\protect ١٤.\protect ١٧٧)
]
 $\int _{-2}^{2}\int _{-\sqrt {4-x^2}}^{\sqrt {4-x^2}}\int _{x^2+y^2}^{8-x^2-y^2}1 \dif z\dif y\dif x,$\\ $\int _{-2}^{2}\int _{-\sqrt {4-y^2}}^{\sqrt {4-y^2}}\int _{x^2+y^2}^{8-x^2-y^2} 1\dif z\dif x\dif y,$\\ $\int _{-2}^{2}\int _{4}^{8-y^2}\int _{-\sqrt {8-z-y^2}}^{\sqrt {8-z-y^2}}1\dif x\dif z\dif y+\int _{-2}^{2}\int _{y^2}^{4}\int _{-\sqrt {z-y^2}}^{\sqrt {z-y^2}}1\dif x\dif z\dif y,$\\ $\int _{4}^{8}\int _{-\sqrt {8-z}}^{\sqrt {8-z}}\int _{-\sqrt {8-z-y^2}}^{\sqrt {8-z-y^2}}1\dif x\dif y\dif z+\int _{0}^{4}\int _{-\sqrt {z}}^{\sqrt {z}}\int _{-\sqrt {z-y^2}}^{\sqrt {z-y^2}}1\dif x\dif y\dif z,$\\ $\int _{-2}^{2}\int _{4}^{8-x^2}\int _{-\sqrt {8-z-x^2}}^{\sqrt {8-z-x^2}}1\dif y\dif z\dif x+\int _{-2}^{2}\int _{x^2}^{4}\int _{-\sqrt {z-x^2}}^{\sqrt {z-x^2}}1\dif y\dif z\dif x,$\\ $\int _{4}^{8}\int _{-\sqrt {8-z}}^{\sqrt {8-z}}\int _{-\sqrt {8-z-x^2}}^{\sqrt {8-z-x^2}}1\dif y\dif x\dif z+\int _{0}^{4}\int _{-\sqrt {z}}^{\sqrt {z}}\int _{-\sqrt {z-x^2}}^{\sqrt {z-x^2}}1\dif y\dif x\dif zS$\quad تمام تکملات کا جواب \عددی {16\pi } ہے۔ 
\item [
\protect ١٤.\protect ١٧٩)
]
 $1$ 
\item [
\protect ١٤.\protect ١٨١)
]
 $1$ 
\item [
\protect ١٤.\protect ١٨٣)
]
 $\tfrac {\pi ^3}{2}(1-\cos 1)$ 
\item [
\protect ١٤.\protect ١٨٥)
]
 $18$ 
\item [
\protect ١٤.\protect ١٨٧)
]
 $\tfrac {7}{6}$ 
\item [
\protect ١٤.\protect ١٨٩)
]
 $0$ 
\item [
\protect ١٤.\protect ١٩١)
]
 $\tfrac {1}{2}-\tfrac {\pi }{8}$ 
\item [
\protect ١٤.\protect ١٩٣)
]
 \begin {enumerate}[a.] \item $\int _{-1}^1\int _0^{1-x^2}\int _{x^2}^{1-z}\dif y\dif z\dif x$ \item $\int _0^1\int _{-\sqrt {1-z}}^{\sqrt {1-z}}\int _{x^2}^{1-z}\dif y\dif x\dif z$ \item $\int _0^1\int _0^{1-z}\int _{-\sqrt {y}}^{\sqrt {y}}\dif y\dif z\dif x$ \item $\int _0^1\int _0^{1-y}\int _{-\sqrt {y}}^{\sqrt {y}}\dif x\dif z\dif y$ \item $\int _0^1\int _{-\sqrt {y}}^{\sqrt {y}}\int _{0}^{1-y}\dif z\dif x\dif y$ \end {enumerate} 
\item [
\protect ١٤.\protect ١٩٥)
]
 $\tfrac {2}{3}$ 
\item [
\protect ١٤.\protect ١٩٧)
]
 $\tfrac {20}{3}$ 
\item [
\protect ١٤.\protect ١٩٩)
]
 $1$ 
\item [
\protect ١٤.\protect ٢٠١)
]
 $\tfrac {16}{3}$ 
\item [
\protect ١٤.\protect ٢٠٣)
]
 $8\pi -\tfrac {32}{3}$ 
\item [
\protect ١٤.\protect ٢٠٥)
]
 $2$ 
\item [
\protect ١٤.\protect ٢٠٧)
]
 $4\pi $ 
\item [
\protect ١٤.\protect ٢٠٩)
]
 $\tfrac {31}{3}$ 
\item [
\protect ١٤.\protect ٢١١)
]
 $1$ 
\item [
\protect ١٤.\protect ٢١٣)
]
 $2\sin 4$ 
\item [
\protect ١٤.\protect ٢١٥)
]
 $4$ 
\item [
\protect ١٤.\protect ٢١٧)
]
 \عددی {a=3} یا \عددی {a=\tfrac {13}{3}} 
\end {description}
 {\urduTechTermsfont {حصہ}} \protect ١٤.\protect ٦\hskip 1em\relax {\urduTechTermsfont {صفحہ}} \protect ١٦٤٣
\begin {description}\setlength {\parskip }{0pt} \setlength {\itemsep }{0pt plus 1pt}
\item [
\protect ١٤.\protect ٢٥٣)
]
 \(\tfrac {4\pi (\sqrt {2}-1)}{3}\) 
\item [
\protect ١٤.\protect ٢٥٥)
]
 \(\tfrac {17\pi }{5}\) 
\item [
\protect ١٤.\protect ٢٥٧)
]
 \(\pi (6\sqrt {2}-8)\) 
\item [
\protect ١٤.\protect ٢٥٩)
]
 \(\tfrac {3\pi }{10}\) 
\item [
\protect ١٤.\protect ٢٦١)
]
 \(\tfrac {\pi }{3}\) 
\item [
\protect ١٤.\protect ٢٦٣)
]
 (ا) \(\int _{0}^{2\pi }\int _{0}^{1}\int _{0}^{\sqrt {4-\rho ^2}}\rho \dif z\dif \rho \dif \phi \)\\ (ب) \(\int _{0}^{2\pi }\int _{0}^{\sqrt {3}}\int _{0}^{1}\rho \dif \rho \dif z\dif \phi +\int _{0}^{2\pi }\int _{\sqrt {3}}^{2}\int _{0}^{\sqrt {4-z^2}}\rho \dif \rho \dif z\dif \phi \)\\ (ج) \(\int _{0}^{1}\int _{0}^{\sqrt {4-\rho ^2}}\int _{0}^{2\pi }\rho \dif \phi \dif z\dif \rho \) 
\item [
\protect ١٤.\protect ٢٦٥)
]
 \(\int _{-\pi /2}^{\pi /2}\int _{0}^{\cos \phi }\int _{0}^{3\rho ^2}F(\rho ,\phi ,z) \rho \dif z\dif \rho \dif \phi \) 
\item [
\protect ١٤.\protect ٢٦٧)
]
 \(\int _{0}^{\pi }\int _{0}^{2\sin \phi }\int _{0}^{4-\rho \sin \phi }F(\rho ,\phi ,z) \dif z\,\rho \dif \rho \dif \phi \) 
\item [
\protect ١٤.\protect ٢٦٩)
]
 \(\int _{-\pi /2}^{\pi /2}\int _{1}^{1+\cos \phi }\int _{0}^{4}F(\rho ,\phi ,z) \dif z\,\rho \dif \rho \dif \phi \) 
\item [
\protect ١٤.\protect ٢٧١)
]
 \(\int _{0}^{\pi /4}\int _{0}^{\sec \phi }\int _{0}^{2-\rho \sin \phi }F(\rho ,\phi ,z) \dif z\,\rho \dif \rho \dif \phi \) 
\item [
\protect ١٤.\protect ٢٧٣)
]
 \(\pi ^2 \) 
\item [
\protect ١٤.\protect ٢٧٥)
]
 \(\pi /3 \) 
\item [
\protect ١٤.\protect ٢٧٧)
]
 \(5\pi \) 
\item [
\protect ١٤.\protect ٢٧٩)
]
 \(2\pi \) 
\item [
\protect ١٤.\protect ٢٨١)
]
 \((\tfrac {8-5\sqrt {2}}{2})\pi \) 
\item [
\protect ١٤.\protect ٢٨٣)
]
 (ا) \(\int _{0}^{2\pi }\int _{0}^{\pi /6}\int _{0}^{2}r^2\sin \theta \dif r\dif \theta \dif \phi +\int _{0}^{2\pi }\int _{\pi /6}^{\pi /2}\int _{0}^{\csc \theta } r^2\sin \theta \dif r\dif \theta \dif \phi \)\\ (ب) \(\int _{0}^{2\pi }\int _{1}^{2}\int _{\pi /6}^{\sin ^{-1}(1/r)} r^2\sin \theta \dif \theta \dif r\dif \phi +\int _{0}^{2\pi }\int _{0}^{2}\int _{0}^{\pi /6}r^2\sin \theta \dif \theta \dif r\dif \phi \) 
\item [
\protect ١٤.\protect ٢٨٥)
]
 \(\int _{0}^{2\pi }\int _{0}^{\pi /2}\int _{\cos \theta }^{2} r^2\sin \theta \dif r\dif \theta \dif \phi =\tfrac {31\pi }{6}\) 
\item [
\protect ١٤.\protect ٢٨٧)
]
 \(\int _{0}^{2\pi }\int _{0}^{\pi }\int _{0}^{1-\cos \theta }r^2\sin \theta \dif r\dif \theta \dif \phi =\tfrac {8\pi }{3}\) 
\item [
\protect ١٤.\protect ٢٨٩)
]
 \(\int _{0}^{2\pi }\int _{\pi /4}^{\pi /2}\int _{0}^{2\cos \theta }r^2\sin \theta \dif r\dif \theta \dif \phi =\tfrac {\pi }{3}\) 
\item [
\protect ١٤.\protect ٢٩١)
]
 (ا) \(8\int _{0}^{\pi /2}\int _{0}^{\pi /2}\int _{0}^{2}r^2\sin \theta \dif r\dif \theta \dif \phi \)\\ (ب) \(8\int _{0}^{\pi /2}\int _{0}^{2}\int _{0}^{\sqrt {4-\rho ^2}}\rho \dif z\dif \rho \dif \phi \)\\ (ج) \(8\int _{0}^{2}\int _{0}^{\sqrt {4-x^2}}\int _{0}^{\sqrt {4-x^2-y^2}}\dif z\dif y\dif x\) 
\item [
\protect ١٤.\protect ٢٩٣)
]
 (ا) \(\int _{0}^{2\pi }\int _{0}^{\pi /3}\int _{\sec \theta }^{2}r^2\sin \theta \dif r\dif \theta \dif \phi \)\\ (ب) \(\int _{0}^{2\pi }\int _{0}^{\sqrt {3}}\int _{1}^{\sqrt {4-\rho ^2}}\rho \dif z\dif \rho \dif \phi \)\\ (ج) \(\int _{-\sqrt {3}}^{\sqrt {3}}\int _{-\sqrt {3-x^2}}^{\sqrt {3-x^2}}\int _{1}^{\sqrt {4-x^2-y^2}}\dif z\dif y\dif x\)\\ (د) \(\tfrac {5\pi }{3}\) 
\item [
\protect ١٤.\protect ٢٩٥)
]
 \(8\pi /3\) 
\item [
\protect ١٤.\protect ٢٩٧)
]
 \(9/4\) 
\item [
\protect ١٤.\protect ٢٩٩)
]
 \((3\pi -4)/18\) 
\item [
\protect ١٤.\protect ٣٠١)
]
 \(\tfrac {2\pi a^3}{3}\) 
\item [
\protect ١٤.\protect ٣٠٣)
]
 \(\tfrac {5\pi }{3}\) 
\item [
\protect ١٤.\protect ٣٠٥)
]
 \(\pi /2\) 
\item [
\protect ١٤.\protect ٣٠٧)
]
 \(\tfrac {4(2\sqrt {2}-1)\pi }{3}\) 
\item [
\protect ١٤.\protect ٣٠٩)
]
 \(16\pi \) 
\item [
\protect ١٤.\protect ٣١١)
]
 \(5\pi /2\) 
\item [
\protect ١٤.\protect ٣١٣)
]
 \(\tfrac {4\pi (8-3\sqrt {3})}{3}\) 
\item [
\protect ١٤.\protect ٣١٥)
]
 \(2/3\) 
\item [
\protect ١٤.\protect ٣١٧)
]
 \(3/4\) 
\item [
\protect ١٤.\protect ٣١٩)
]
 \(\bar {x}=\bar {y}=0,\,\bar {z}=3/8\) 
\item [
\protect ١٤.\protect ٣٢١)
]
 \((\bar {x},\bar {y},\bar {z})=(0,0,3/8)\) 
\item [
\protect ١٤.\protect ٣٢٣)
]
 \(\bar {x}=\bar {y}=0,\,\bar {z}=5/6\) 
\item [
\protect ١٤.\protect ٣٢٥)
]
 \(I_z=30\pi ,\, R_z=\sqrt {\tfrac {5}{2}}\) 
\item [
\protect ١٤.\protect ٣٢٧)
]
 \(I_x=\pi /4\) 
\item [
\protect ١٤.\protect ٣٢٩)
]
 \(\tfrac {a^4h\pi }{10}\) 
\item [
\protect ١٤.\protect ٣٣١)
]
 (ا) \((\bar {x},\bar {y},\bar {z})=(0,0,4/5),\, I_z=\pi /12\)\\ \(R_z=\sqrt {1/3}\) (ب) \((\bar {x},\bar {y},\bar {z})=(0,0,5/6)\)\\ \( I_z=\pi /14,\,R_z=\sqrt {5/14}\) 
\item [
\protect ١٤.\protect ٣٣٥)
]
 \((\bar {x},\bar {y},\bar {z})=(0,0,\tfrac {2h^2+3h}{3h+6})\)\\ \( I_z=\tfrac {\pi a^4(h^2+2h)}{4},\, R_z=\tfrac {a}{\sqrt {2}}\) 
\item [
\protect ١٤.\protect ٣٣٧)
]
 \(\tfrac {3M}{\pi R^3}\) 
\end {description}
 {\urduTechTermsfont {حصہ}} \protect ١٤.\protect ٧\hskip 1em\relax {\urduTechTermsfont {صفحہ}} \protect ١٦٦٣
\begin {description}\setlength {\parskip }{0pt} \setlength {\itemsep }{0pt plus 1pt}
\item [
\protect ١٤.\protect ٣٣٨)
]
 (ا) \(x=\tfrac {u+v}{3},\, y=\tfrac {v-2u}{3};\, \tfrac {1}{3}\)\\ (ب) تکونی خطہ جس کی سرحدیں \عددی {u=0}، \عددی {v=0} اور \عددی {u+v=3} ہیں۔ 
\item [
\protect ١٤.\protect ٣٤٠)
]
 (ا) \(x=\tfrac {1}{5}(2u-v),\,y=\tfrac {1}{10}(3v-u);\, \tfrac {1}{10}\)\\ (ب) تکونی خطہ جس کی سرحدیں \عددی {3v=u}، \عددی {v=2u} اور \عددی {3u+v=10} ہیں۔ 
\item [
\protect ١٤.\protect ٣٤٢)
]
 (ا) \(\begin {vmatrix} \cos v&-u\sin v\\ \sin v&u\cos v \end {vmatrix}=u\)\\ (ب) \(\begin {vmatrix}\sin v &u\cos v\\ \cos v&-u\sin v \end {vmatrix}=-u\) 
\item [
\protect ١٤.\protect ٣٤٦)
]
 \(\tfrac {64}{5}\) 
\item [
\protect ١٤.\protect ٣٤٨)
]
 \(\int _1^2\int _1^3(u+v)\tfrac {2u}{v}\dif u\dif v=8+\tfrac {52}{3}\ln 2\) 
\item [
\protect ١٤.\protect ٣٥٠)
]
 \(\tfrac {\pi ab(a^2+b^2)}{4}\) 
\item [
\protect ١٤.\protect ٣٥٢)
]
 \(\tfrac {1}{3}(1+\tfrac {3}{e^2})\approx 0.4687\) 
\item [
\protect ١٤.\protect ٣٥٦)
]
 \(\tfrac {4\pi abc}{3}\) 
\item [
\protect ١٤.\protect ٣٥٨)
]
 \(\int _0^3\int _0^2\int _1^2(\tfrac {v}{3}+\tfrac {vw}{3u})\dif u\dif v\dif w=2+\ln 8\) 
\end {description}
 {\urduTechTermsfont {حصہ}} \protect ١٥.\protect ١\hskip 1em\relax {\urduTechTermsfont {صفحہ}} \protect ١٦٧٥
\begin {description}\setlength {\parskip }{0pt} \setlength {\itemsep }{0pt plus 1pt}
\item [
\protect ١٥.\protect ١)
]
 شکل \حوالہ {شکل_سوال_سمتی_تکمل_مساوات_ترسیم_الف} 
\item [
\protect ١٥.\protect ٢)
]
 شکل \حوالہ {شکل_سوال_سمتی_تکمل_مساوات_ترسیم_ب} 
\item [
\protect ١٥.\protect ٣)
]
 شکل \حوالہ {شکل_سوال_سمتی_تکمل_مساوات_ترسیم_پ} 
\item [
\protect ١٥.\protect ٤)
]
 شکل \حوالہ {شکل_سوال_سمتی_تکمل_مساوات_ترسیم_ت} 
\item [
\protect ١٥.\protect ٥)
]
 شکل \حوالہ {شکل_سوال_سمتی_تکمل_مساوات_ترسیم_ٹ} 
\item [
\protect ١٥.\protect ٦)
]
 شکل \حوالہ {شکل_سوال_سمتی_تکمل_مساوات_ترسیم_ث} 
\item [
\protect ١٥.\protect ٧)
]
 شکل \حوالہ {شکل_سوال_سمتی_تکمل_مساوات_ترسیم_ج} 
\item [
\protect ١٥.\protect ٨)
]
 شکل \حوالہ {شکل_سوال_سمتی_تکمل_مساوات_ترسیم_چ} 
\item [
\protect ١٥.\protect ٩)
]
\(\sqrt {2}\)
\item [
\protect ١٥.\protect ١١)
]
\(\tfrac {13}{2}\)
\item [
\protect ١٥.\protect ١٣)
]
\(3\sqrt {14}\)
\item [
\protect ١٥.\protect ١٥)
]
\(\tfrac {1}{6}(5\sqrt {5}+9)\)
\item [
\protect ١٥.\protect ١٧)
]
\(\sqrt {3}\ln \tfrac {b}{a}\)
\item [
\protect ١٥.\protect ١٩)
]
\(\tfrac {10\sqrt {5}-2}{3}\)
\item [
\protect ١٥.\protect ٢١)
]
\(8\)
\item [
\protect ١٥.\protect ٢٣)
]
\(2\sqrt {2}-1\)
\item [
\protect ١٥.\protect ٢٥)
]
 (ا) \عددی {4\sqrt {2}-1}، (ب) \عددی {\sqrt {2}+\ln (1+\sqrt {2})} 
\item [
\protect ١٥.\protect ٢٧)
]
 \عددی {I_z=2\pi \delta a^3}، \عددی {R_z=a} 
\item [
\protect ١٥.\protect ٢٩)
]
 (ا) \عددی {I_z=2\pi \sqrt {2}\delta }، \عددی {R_z=1}؛\\ (ب) \عددی {I_z=4\pi \sqrt {2}\delta }، \عددی {R_z=1} 
\item [
\protect ١٥.\protect ٣١)
]
 \عددی {I_z=2\pi -2}، \عددی {R_z=1} 
\end {description}
 {\urduTechTermsfont {حصہ}} \protect ١٥.\protect ٢\hskip 1em\relax {\urduTechTermsfont {صفحہ}} \protect ١٦٨٩
\begin {description}\setlength {\parskip }{0pt} \setlength {\itemsep }{0pt plus 1pt}
\item [
\protect ١٥.\protect ٣٧)
]
 \(\nabla f=\tfrac {-x\ai -y\aj -z\ak }{(x^2+y^2+z^2)^{3/2}}\) 
\item [
\protect ١٥.\protect ٣٩)
]
 \(\nabla g=-\tfrac {2x}{x^2+y^2}\ai -\tfrac {2y}{x^2+y^2}\aj +e^z\ak \) 
\item [
\protect ١٥.\protect ٤١)
]
 \(\kvec {F}=\tfrac {-kx}{(x^2+y^2)^{3/2}}\ai -\tfrac {ky}{(x^2+y^2)^{3/2}}\aj , k>0\) 
\item [
\protect ١٥.\protect ٤٣)
]
 (ا) \عددی {\tfrac {9}{2}}، (ب) \عددی {\tfrac {13}{3}}، (ج) \عددی {\tfrac {9}{2}} 
\item [
\protect ١٥.\protect ٤٥)
]
 (ا) \عددی {\tfrac {1}{3}}، (ب) \عددی {-\tfrac {1}{5}}، (ج) \عددی {0} 
\item [
\protect ١٥.\protect ٤٧)
]
 (ا) \عددی {2}، (ب) \عددی {\tfrac {3}{2}}، (ج) \عددی {\tfrac {1}{2}} 
\item [
\protect ١٥.\protect ٤٩)
]
 \(\tfrac {1}{2}\) 
\item [
\protect ١٥.\protect ٥١)
]
 \(-\pi \) 
\item [
\protect ١٥.\protect ٥٣)
]
 \(\tfrac {207}{12}\) 
\item [
\protect ١٥.\protect ٥٥)
]
 \(-\tfrac {39}{2}\) 
\item [
\protect ١٥.\protect ٥٧)
]
 \(\tfrac {25}{6}\) 
\item [
\protect ١٥.\protect ٥٩)
]
 (ا) دائری بہاو اول \عددی {0}، دائری بہاو دوم \عددی {2\pi }، بہاو اول \عددی {2\pi }، بہاو دوم \عددی {0}؛ (ب) دائری بہاو اول \عددی {0}، دائری بہاو دوم \عددی {8\pi }، بہاو اول \عددی {8\pi }، بہاو دوم \عددی {0} 
\item [
\protect ١٥.\protect ٦١)
]
 دائری بہاو \عددی {0}، بہاو \عددی {a^2\pi } 
\item [
\protect ١٥.\protect ٦٣)
]
 دائری بہاو \عددی {a^2\pi }، بہاو \عددی {0} 
\item [
\protect ١٥.\protect ٦٥)
]
 (ا) \عددی {-\tfrac {\pi }{2}}، (ب) \عددی {0}، (ج) \عددی {1} 
\item [
\protect ١٥.\protect ٦٧)
]
 \begin {center} \begin {tikzpicture}[declare function={fx(\x ,\y )=-\y /sqrt((\x )^2+(\y )^2);fy(\x ,\y )=\x /sqrt((\x )^2+(\y )^2);gx(\t )=2*cos(\t );gy(\t )=2*sin(\t );}] \pgfmathsetmacro {\k }{2} \pgfmathsetmacro {\kk }{\k *cos(45)} \draw [-stealth](-2.25,0)--(2.5,0)node[right]{$x$}; \draw [-stealth](0,-2.125)--(0,2.25)node[right]{$y$}; \draw [domain=0:360] plot ({gx(\x )},{gy(\x )}); \draw [-latex](\k ,0)node[below right]{$2$}--++({fx(\k ,0)},{fy(\k ,0)}); \draw [-latex](-\k ,0)--++({fx(-\k ,0)},{fy(-\k ,0)}); \draw [-latex](0,\k )node[below left]{$2$}--++({fx(0,\k )},{fy(0,\k )}); \draw [-latex](0,-\k )--++({fx(0,-\k )},{fy(0,-\k )}); \draw [-latex](\kk ,\kk )--++({fx(\kk ,\kk )},{fy(\kk ,\kk )}); \draw [-latex](-\kk ,\kk )--++({fx(-\kk ,\kk )},{fy(-\kk ,\kk )}); \draw [-latex](\kk ,-\kk )--++({fx(\kk ,-\kk )},{fy(\kk ,-\kk )}); \draw [-latex](-\kk ,-\kk )--++({fx(-\kk ,-\kk )},{fy(-\kk ,-\kk )}); \draw [-latex](\kk ,\kk )--++({fx(\kk ,\kk )},{0}); \draw [-latex](\kk ,\kk )--++({0},{fy(\kk ,\kk )}); \draw [-latex](-\kk ,\kk )--++({fx(-\kk ,\kk )},{0}); \draw [-latex](-\kk ,\kk )--++({0},{fy(-\kk ,\kk )}); \draw [-latex](\kk ,-\kk )--++({fx(\kk ,-\kk )},{0}); \draw [-latex](\kk ,-\kk )--++({0},{fy(\kk ,-\kk )}); \draw [-latex](-\kk ,-\kk )--++({fx(-\kk ,-\kk )},{0}); \draw [-latex](-\kk ,-\kk )--++({0},{fy(-\kk ,-\kk )}); \draw (1,0.5)node[font=\small ]{$x^2+y^2=4$}; \end {tikzpicture} \end {center} 
\item [
\protect ١٥.\protect ٦٩)
]
 (ا) \عددی {\kvec {G}=-y\ai +x\aj }، \\ (ب) \عددی {\kvec {G}=\sqrt {x^2+y^2}\kvec {F}} 
\item [
\protect ١٥.\protect ٧١)
]
 \(\kvec {F}=\tfrac {-x\ai -y\aj }{\sqrt {x^2+y^2}}\) 
\item [
\protect ١٥.\protect ٧٣)
]
 \(48\) 
\item [
\protect ١٥.\protect ٧٥)
]
 \(\pi \) 
\item [
\protect ١٥.\protect ٧٧)
]
 \(0\) 
\item [
\protect ١٥.\protect ٧٩)
]
 \(\tfrac {1}{2}\) 
\end {description}
 {\urduTechTermsfont {حصہ}} \protect ١٥.\protect ٣\hskip 1em\relax {\urduTechTermsfont {صفحہ}} \protect ١٧٠٤
\begin {description}\setlength {\parskip }{0pt} \setlength {\itemsep }{0pt plus 1pt}
\item [
\protect ١٥.\protect ٨٩)
]
 بقائی 
\item [
\protect ١٥.\protect ٩١)
]
 غیر بقائی 
\item [
\protect ١٥.\protect ٩٣)
]
 غیر بقائی 
\item [
\protect ١٥.\protect ٩٥)
]
 \(f(x,y,z)=x^2+\tfrac {3y^2}{2}+2z^2+C\) 
\item [
\protect ١٥.\protect ٩٧)
]
 \(f(x,y,z)=xe^{y+2z}+C\) 
\item [
\protect ١٥.\protect ٩٩)
]
 \(f(x,y,z)=x\ln x-x+\tan (x+y)+\tfrac {1}{2}\ln (y^2+z^2)+C\) 
\item [
\protect ١٥.\protect ١٠١)
]
 \(49\) 
\item [
\protect ١٥.\protect ١٠٣)
]
 \(-16\) 
\item [
\protect ١٥.\protect ١٠٥)
]
 \(1\) 
\item [
\protect ١٥.\protect ١٠٧)
]
 \(9\ln 2\) 
\item [
\protect ١٥.\protect ١٠٩)
]
 \(0\) 
\item [
\protect ١٥.\protect ١١١)
]
 \(-3\) 
\item [
\protect ١٥.\protect ١١٥)
]
 \(\kvec {F}=\nabla (\tfrac {x^2-1}{y})\) 
\item [
\protect ١٥.\protect ١١٧)
]
 (ا) \عددی {1}، (ب) \عددی {1}، (ج) \عددی {1} 
\item [
\protect ١٥.\protect ١١٩)
]
 (ا) \عددی {2}، (ب) \عددی {2} 
\item [
\protect ١٥.\protect ١٢١)
]
 \(f(x,y,z)=\tfrac {GmM}{(x^2+y^2+z^2)^{1/2}}\) 
\item [
\protect ١٥.\protect ١٢٣)
]
 (ا) \عددی {c=b=2a}، (ب) \عددی {c=b=2} 
\item [
\protect ١٥.\protect ١٢٥)
]
 چونکہ میدان بقائی ہے لہٰذا کام راہ پر منحصر نہیں ہو گا۔ 
\end {description}
