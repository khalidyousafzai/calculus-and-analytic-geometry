\باب{سمتیات اور خلا میں تحلیلی جیومیٹری}
اس حصہ میں سمتیات اور سہ بعدی محددی نظام متعارف کئے جائیں گے۔ جیسا ایک متغیر کے تفاعل پر غور کے لئے محددی مستوی موزوں ہے، اسی طرح دو (یا دو سے زیادہ) متغیرات کے تفاعل پر غور کے لئے محددی خلاء موزوں ہے۔ ہم محددی مستوی میں ایک تیسرا محور شامل کر کے محددی خلاء پیدا کرتے ہیں۔ یہ محور \عددی{xy} مستوی سے نیچے اور اس سے اوپر فاصلہ ناپتا ہے۔

\حصہ{مستوی میں سمتیات} 
بعض چیزیں جنہیں ہم ناپتے ہیں کا تعین ان کی مقدار سے ہوتا ہے۔مثال کے طور پر کمیت، لمبائی اور وقت قلم بند کرنے کے لئے  ہم صرف ایک عدد اور موزوں اکائی لکھتے ہیں۔ اس کے برعکس قوت، ہٹاو، یا سمتی رفتار جاننے کے لئے ہمیں مزید معلوم درکار ہو گی۔ قوت کو بیان کرنے کے لئے ہمیں اس کی مقدار کے ساتھ وہ رخ بھی جاننا ہو گا جس رخ یہ عمل کرتی ہے۔ کسی جسم کا ہٹاو بیان کرنے کے لئے ہمیں اس سمت کا ذکر کرنا ہو گا جس سمت یہ جسم حرکت کرتا ہے اور ساتھ اس فاصلہ کا ذکر کرنا ہو گا جتنا یہ طے کرتا ہے۔ ایک جسم کی سمتی رفتار بیان کرنے کے لئے ہم حرکت کی سمت اور جسم کی رفتار کی بات کرتے ہیں۔

وہ مقدار جس کی جسامت اور سمت دونوں ہوں کو عموماً تیر کے نشان سے ظاہر کیا جاتا ہے جہاں مقدار کے رخ کو  تیر کا رخ  مقدار  کی جسامت کو، موزوں اکائیوں میں، تیر کی لمبائی ظاہر کرتی ہے۔

تیر کے اس نشان کو \اصطلاح{سمتیہ} کہتے ہیں۔

\ابتدا{تعریف}
ایک مستوی میں کسی مخصوص رخ خط کو \اصطلاح{سمتیہ}\فرہنگ{سمتیہ}\حاشیہب{vector}\فرہنگ{vector} کہتے ہیں۔ دو سمتیات صرف اس صورت ایک دوسرے کے برابر یا یکساں ہوں گے جب ان کی مقداریں ایک جیسی ہوں اور ان کے رخ ایک جیسے ہوں۔
\انتہا{تعریف}
%===================

یوں اگر سمتیات کو ظاہر کرنے والے تیر  آپس میں متوازی ہوں، ان کی لمبائیاں ایک جیسی ہوں اور ان کا رخ بھی ایک جیسا ہو تب یہ ایک ہی  سمتیہ کو ظاہر کرتے ہیں۔اس  کتاب میں سمتیہ کو موٹی لکھائی میں رومن حروف تہجی، مثلاً   \عددی{\kvec{v}}،  سے ظاہر کیا جائے گا\حاشیہد{قلم و کاغذ استعمال کرتے ہوئے سمتیہ کو رومن حروف تہجی پر تیر کا نشان \عددی{\vec{v}} یا نصف تیر کا نشان
$\krightharpoonup{v}$
 ڈال کر ظاہر کیا جاتا ہے۔)}۔نقطہ \عددی{A} سے نقطہ \عددی{B} تک تیر کو ہم 
$\krightharpoonup{AB}$
 لکھیں گے۔

\ابتدا{مثال}
چار تیروں کو شکل میں دکھایا گیا ہے جن کی لمبائیاں اور رخ ایک جیسی ہیں۔ یوں یہ چاروں ایک ہی سمتیہ کو ظاہر کرتے ہیں جس کو ہم درج ذیل لکھتے ہیں۔
\begin{align*}
\krightharpoonup{AB}=\krightharpoonup{CD}=\krightharpoonup{OP}=\krightharpoonup{EF}
\end{align*}
\انتہا{مثال}
%=====================

\جزوحصہء{غیر سمتیہ اور غیر سمتی مضرب}
ہم کسی سمتیہ کو مثبت حقیقی عدد سے ضرب دینے کے لئے اس کی لمبائی کو اس عدد سے ضرب دیتے ہیں۔ سمتیہ کو \عددی{2} سے ضرب دینے کے لئے ہم اس کی لمبائی دگنی کرتے ہیں۔ ایک سمتیہ کو \عددی{1.5} سے ضرب دینے کے لئے ہم اس کی لمبائی \عددی{\SI{50}{\percent}} بڑھاتے ہیں، وغیرہ، وغیرہ۔ ایک سمتیہ کو منفی  عدد سے ضرب دینے کے لئے ہم اس کا رخ الٹ کر کے اس کی لمبائی کو عدد کی مطلق قیمت سے ضرب دیتے ہیں۔

اگر \عددی{c} غیر صفر حقیقی عدد اور \عددی{\kvec{v}} ایک سمتیہ ہو تب مثبت \عددی{c} کی صورت میں \عددی{\kvec{v}} اور \عددی{c\kvec{v}} کے رخ ایک جیسے ہوں گے جبکہ منفی \عددی{c} کی صورت میں ان کے رخ ایک دوسرے کے مخالف ہوں گے۔ یہاں حقیقی اعداد تبدیلی پیمانہ کے طور پر کام کرتے ہیں اور یہ  \اصطلاح{غیر سمتی}\فرہنگ{غیر سمتی}\حاشیہب{scalar}\فرہنگ{scalar} کہلاتے ہیں جبکہ \عددی{c\kvec{v}} کے مضرب کو \عددی{\kvec{v}} کا \اصطلاح{غیر سمتی مضرب}\فرہنگ{غیر سمتی!مضرب}\حاشیہب{scalar multiple}\فرہنگ{scalar multiple} کہتے ہیں۔

صفر سے ضرب کو شامل کرنے کی خاطر ہم  اس روایت کو اپناتے ہیں جس کے  مطابق کسی بھی سمتیہ کو صفر سے ضرب دینے سے \اصطلاح{صفر سمتیہ} \عددی{\kvec{0}} حاصل ہو گا، جو ایک نقطہ پر مشتمل ہو گا جس کی لمبائی صفر ہو گی۔ دیگر سمتیہ کے برعکس صفر سمتیہ \عددی{\kvec{0}} کا کوئی رخ نہیں ہوتا ہے۔  

\جزوحصہء{جیومیٹریائی مجموعہ: قاعدہ متوازی الاضلاع}
دو غیر صفر سمتیات \عددی{\kvec{v}_1} اور \عددی{\kvec{v}_2} کا جیومیٹریائی مجموعہ لینے کی خاطر \عددی{\kvec{v}_1} کا نمائندہ، مثلاً \عددی{A} سے \عددی{B} تک، ترسیم کر کے \عددی{\kvec{v}_1} کے اختتامی نقطہ (سر) \عددی{B} پر \عددی{\kvec{v}_2}  کے نمائندہ کا ابتدائی نقطہ (دم)  رکھ کر ترسیم کریں۔ شکل میں
$\kvec{v}_2=\krightharpoonup{BC}$
ہے۔ مجموعہ \عددی{\kvec{v}_1+\kvec{v}_2} اب \عددی{\kvec{v}_1} کے دم  \عددی{A} سے \عددی{\kvec{v}_2} کے سر  \عددی{C} تک سمتیہ ہو گا۔ یوں اگر
\begin{align*}
\kvec{v}_1=\overset{\rightharpoonup}{\rule{0pt}{.9ex}\smash{AB}},\quad \kvec{v}_2=\krightharpoonup{BC}
\end{align*}
ہوں تب
\begin{align*}
\kvec{v}_1+\kvec{v}_2=\krightharpoonup{AB}+\krightharpoonup{BC}=\krightharpoonup{AC}
\end{align*}
ہو گا۔چونکہ اس عمل میں \عددی{\kvec{v}_1+\kvec{v}_2} متوازی الاضلاع کا وتر ہوتا ہے لہٰذا اس عمل کو بعض اوقات \اصطلاح{قاعدہ متوازی الاضلاع}\فرہنگ{قاعدہ!متوازی الاضلاع}\حاشیہب{parallelogram law}\فرہنگ{law!parallelogram} کہتے ہیں۔

\جزوحصہء{اجزاء}
دو سمتیات اس صورت متوازی ہوں گے جب یہ ایک دوسرے کے غیر صفر، غیر سمتی مضرب ہوں، یعنی جب ان کو ظاہر کرنے والے خطوط متوازی ہوں۔

جب بھی ایک سمتیہ \عددی{\kvec{v}} کو دو غیر متوازی سمتیات کا مجموعہ
\begin{align*}
\kvec{v}=\kvec{v}_1+\kvec{v}_2
\end{align*}
لکھنا ممکن ہو، سمتیات \عددی{\kvec{v}_1} اور \عددی{\kvec{v}_2} سمتیہ \عددی{\kvec{v}} کے اجزاء کہلائیں گے  اور ہم کہتے ہیں کہ  سمتیہ  \عددی{\kvec{v}} کو اس کے  اجزاء \عددی{\kvec{v}_1} اور \عددی{\kvec{v}_2} میں تحلیل کیا گیا ہے۔

سمتیات کے  مقبول ترین الجبرا میں ہر سمتیہ کو کارتیسی محور کے متوازی اجزاء کی صورت میں بیان کیا جاتا ہے اور یہ اجزاء از خود موزوں \اصطلاح{اساسی}\فرہنگ{اساسی}\حاشیہب{basic}\فرہنگ{basic} سمتیہ، جن کی لمبائی \عددی{1} ہوتی ہے، کے مضرب ہوتے ہیں۔ مثبت \عددی{x} محور کے رخ اساسی سمتیہ نقطہ \عددی{(0,0)} سے نقطہ \عددی{(1,0)} تک تیر سے ظاہر کیا جاتا ہے  اور اس اساسی سمتیہ کی علامت  \عددی{\ai} ہے۔ مثبت \عددی{y} محور کے رخ اساسی سمتیہ نقطہ \عددی{(0,0)} سے نقطہ \عددی{(0,1)} تک تیر سے ظاہر کیا جاتا ہے اور اس اساسی سمتیہ کی علامت  \عددی{\aj} ہے۔ اب غیر سمتی \عددی{a} کے لئے  محور \عددی{x} کے متوازی  سمتیہ \عددی{a\ai}  کی لمبائی \عددی{\abs{a}} ہو گی جبکہ اس کا رخ \عددی{a>0} کے لئے  دایاں اور  \عددی{a<0} کے لئے بایاں ہو گا۔ اس طرح غیر سمتی \عددی{b} کے لئے  محور \عددی{y} کے متوازی  سمتیہ \عددی{b\aj}  کی لمبائی \عددی{\abs{b}} ہو گی جبکہ اس کا رخ \عددی{b>0} کے لئے  اوپر اور  \عددی{b<0} کے لئے نیچے ہو گا۔ شکل میں سمتیہ 
$\kvec{v}=\krightharpoonup{AC}$
کو اجزاء \عددی{\ai} اور \عددی{\aj} میں تحلیل کیا گیا ہے:
\begin{align*}
\kvec{v}=a\ai+b\aj
\end{align*}

\ابتدا{تعریف}
اگر \عددی{\kvec{v}=a\ai+b\aj} ہو تب \عددی{\ai} اور \عددی{\aj} کے رخ، سمتیہ \عددی{\kvec{v}} کے اجزاء  سمتیات \عددی{a\ai} اور \عددی{b\aj} ہوں گے۔ اعداد \عددی{a} اور \عددی{b}، اساسی سمتیات \عددی{\ai} اور \عددی{\aj} کے رخ، سمتیہ \عددی{\kvec{v}} کے غیر سمتی اجزاء ہوں گے۔ 
\انتہا{تعریف}
%================

\ابتدا{تعریف}
سمتیات کی برابری یا یکسانیت (الجبرائی تعریف)۔
\begin{align}
a\ai+b\aj=a'\ai+b'\aj\quad \Leftrightarrow\quad a=a',\quad b=b'
\end{align}
\انتہا{تعریف}
%=======

دو سمتیات صرف اور صرف اس صورت ایک دوسرے کے برابر ہوں گے جب \عددی{\ai} اور \عددی{\aj} کے رخ، ان کے مطابقتی غیر سمتی اجزاء ایک دوسرے کے برابر ہوں۔

\جزوحصہء{الجبرائی مجموعہ}
سمتیات کے مطابقتی غیر سمتی اجزاء کا مجموعہ لے کر ان سمتیات  کا مجموعہ حاصل کیا جا سکتا ہے۔

اگر \عددی{\kvec{v}_1=a_1\ai+b_1\aj} اور \عددی{\kvec{v}_2=a_2\ai+b_2\aj} ہوں تب درج ذیل ہو گا۔ 
\begin{align*}
\kvec{v}_1+\kvec{v}_2=(a_1+a_2)\ai+(b_1+b_2)\aj
\end{align*}

\ابتدا{مثال}
\begin{align*}
(2\ai-4\aj)+(5\ai+3\aj)=(2+5)\ai+(-4+3)\aj=7\ai-\aj
\end{align*}
\انتہا{مثال}

\جزوحصہء{تفریق}
ایک سمتیہ \عددی{\kvec{v}} کا منفی سمتیہ \عددی{-\kvec{v}=(-1)\kvec{v}} ہو گا۔ اس کی لمبائی \عددی{\kvec{v}} کی لمبائی ہو گی البتہ اس کا رخ \عددی{\kvec{v}} کا مخالف ہو گا۔سمتیہ \عددی{\kvec{v}_2} کو سمتیہ \عددی{\kvec{v}_1} سے منفی کرنے کی خاطر ہم  \عددی{-\kvec{v}_2} اور \عددی{\kvec{v}_1} کا مجموعہ لیں گے۔ جیومیٹریائی طور پر ہم \عددی{\kvec{v}_1} کے سر سے \عددی{-\kvec{v}_2} کھینچ کر \عددی{\kvec{v}_1} کے دم سے \عددی{-\kvec{v}_2} کے سر تک سمتیہ ترسیم کریں گے۔ یہ عمل شکل میں دکھایا گیا ہے جہاں 
\begin{align*}
\krightharpoonup{AD}=\krightharpoonup{AB}+\krightharpoonup{BD}=\kvec{v}_1+(-\kvec{v}_2)=\kvec{v}_1-\kvec{v}_2
\end{align*}

اس کے علاوہ  \عددی{\kvec{v}_1} اور \عددی{\kvec{v}_2} کے دم مشترکہ  نقطہ پر رکھ  کر \عددی{\kvec{v}_1} اور \عددی{\kvec{v}_2} ترسیم کر کے \عددی{\kvec{v}_2} کے سر سے \عددی{\kvec{v}_1} کے سر تک سمتیہ \عددی{\kvec{v}_1-\kvec{v}_2} ہو گا۔ یہ عمل  شکل میں پیش  کیا گیا ہے جہاں درج ذیل ہے۔
 \begin{align*}
\krightharpoonup{CB}=\krightharpoonup{CA}+\krightharpoonup{AB}=-\kvec{v}_2+\kvec{v}_1=\kvec{v}_1-\kvec{v}_2
\end{align*}
مزید، \عددی{-\kvec{v}_2} کے سر سے \عددی{\kvec{v}_1} ترسیم کر کے \عددی{\kvec{v}_1-\kvec{v}_2} حاصل کیا جا سکتا ہے۔

درج ذیل قاعدہ سمتیات کی تفریق کو اجزاء کی صورت میں پیش کرتا ہے۔
\begin{align}
\kvec{v}_1-\kvec{v}_2=(a_1-a_2)\ai+(b_1-b_2)\aj
\end{align}
اس قاعدہ کے تحت دو سمتیات تفریق کرنے کی خاطر ان کے مطابقتی اجزاء تفریق کیے جائیں گے۔

\ابتدا{مثال}
\begin{align*}
(6\ai+2\aj)-(3\ai-5\aj)=(6-3)\ai+(2-(-5))\aj=3\ai+7\aj
\end{align*}
\انتہا{مثال}
%============================

ہم نقطہ \عددی{N_1(x_1,y_1)} سے نقطہ \عددی{N_2(x_2,y_2)} تک سمتیہ کے اجزاء حاصل کرنے کے لئے \عددی{\krightharpoonup{ON_1}=x_1\ai+y_1\aj} کے اجزاء کو \عددی{\krightharpoonup{ON_2}=x_2\ai+y_2\aj} کے اجزاء سے منفی کرتے ہیں۔

\عددی{N_1(x_1,y_1)} سے \عددی{N_2(x_2,y_2)} تک سمتیہ درج ذیل ہو گا۔
\begin{align}
\krightharpoonup{N_1N_2}=(x_2-x_1)\ai+(y_2-y_1)\aj
\end{align}

\ابتدا{مثال}
نقطہ \عددی{N_1(3,4)} سے نقطہ \عددی{N_2(5,1)} تک سمتیہ درج ذیل ہے۔
\begin{align*}
\krightharpoonup{N_1N_2}=(5-3)\ai+(1-4)\aj=2\ai-3\aj
\end{align*}
\انتہا{مثال}
%===============

\جزوحصہء{مقدار}
سمتیہ \عددی{\kvec{v}=a\ai+b\aj} کی \اصطلاح{لمبائی}\فرہنگ{سمتیہ!لمبائی}\حاشیہب{length}\فرہنگ{vector!length} یا \اصطلاح{مقدار}\فرہنگ{سمتیہ!مقدار}\حاشیہب{magnitude}\فرہنگ{vector!magnitude} \عددی{\abs{\kvec{v}}=\sqrt{a^2+b^2}} ہے۔  سمتیہ \عددی{\kvec{v}} اور اس کے دو سمتیہ اجزاء کے قائمہ مثلث پر مسئلہ فیثاغورث لاگو کرنے سے یہ کلیہ اخذ ہوتا ہے۔ سمتیہ کی لمبائی \عددی{\abs{\kvec{v}}} میں دو انتصابی لکیریں وہی ہیں جو مطلق قیمت کو ظاہر کرنے کے لئے استعمال کی جاتی ہیں۔ 

\begin{align}
\abs{\kvec{v}}&=\sqrt{a^2+b^2}&&\kvec{v}=a\ai+b\aj
\end{align}

\ابتدا{مثال}
آپ زمین کے ساتھ \عددی{30^{\circ}} زاویہ پر \عددی{\SI{20}{\newton}} کی قوت \عددی{\kvec{F}} سے ہاتھ ریڑھی کو دکھا لگاتے ہیں۔ قوت کا افقی جزو ریڑھی کو حرکت دیتی ہے جبکہ اس کا انتصابی جزو ریڑھی کا وزن بڑھاتا ہے۔ اس قوت کا افقی اور انتصابی جزو معلوم کریں۔

حل:\quad
ہم قوت \عددی{\kvec{F}=a\ai+b\aj} اور اس کے اجزاء کے لئے مثلث  بناتے ہیں۔ اس مثلث سے \عددی{a=10\sqrt{3}} اور \عددی{b=10} حاصل ہوتے ہیں۔ قوت کا افقی جزو \عددی{10\sqrt{3}\ai} اور انتصابی جزو \عددی{-10\aj} ہے۔یوں \عددی{\kvec{F}=10\sqrt{3}\ai-10\aj} ہو گا۔  انتصابی جزو کا رخ نیچے ہے لہٰذا یہ منفی ہے۔ 
\انتہا{مثال}
%==================

\جزوحصہء{غیر سمتی ضرب}
غیر سمتی ضرب جزو در جزو حاصل کیا جا سکتا ہے۔ اگر \عددی{c} ایک غیر سمتی اور \عددی{\kvec{v}=a\ai+b\aj} ایک سمتیہ ہو تب درج ذیل ہو گا۔
\begin{align}
c\kvec{v}=c(a\ai+b\aj)=(ca)\ai+(cb)\aj
\end{align}
سمتیہ \عددی{c\kvec{v}} کی لمبائی سمتیہ \عددی{\kvec{v}} کی لمبائی ضرب \عددی{\abs{c}} ہو گا:
\begin{align*}
\abs{c\kvec{v}}&=\abs{(ca)\ai+(cb)\aj}\\
&=\sqrt{(ca)^2+(cb)^2}\\
&=\sqrt{c^2(a^2+b^2)}\\
&=\sqrt{c^2}\sqrt{a^2+b^2}\\
&=\abs{c}\abs{\kvec{v}}
\end{align*}

یوں اگر \عددی{c} غیر سمتی ہو اور \عددی{\kvec{v}} ایک سمتیہ ہو تب \عددی{\abs{c\kvec{v}}=\abs{c}\abs{\kvec{v}}} ہو گا۔

\ابتدا{مثال}
اگر \عددی{c=-2} اور \عددی{\kvec{v}=-3\ai+4\aj} ہوں تب درج ذیل ہو گا۔
\begin{align*}
\abs{\kvec{v}}&=\abs{-3\ai+4\aj}=\sqrt{(-3)^2+(4)^2}=\sqrt{9+16}=\sqrt{25}=5\\
\abs{-2\kvec{v}}&=\abs{(-2)(-3\ai+4\aj)}=\abs{6\ai-8\aj}=\sqrt{(6)^2+(-8)^2}\\
&=\sqrt{36+64}=\sqrt{100}=10=\abs{-2}\abs{5}=\abs{c}\abs{\kvec{v}}
\end{align*}
\انتہا{مثال}
%======================

\جزوحصہء{صفر سمتیہ}
صفر سمتیہ سے مراد درج ذیل سمتیہ ہے۔
\begin{align*}
\kvec{0}=0\ai+0\aj
\end{align*}
دھیان رہے کہ صفر سمتیہ \عددی{\kvec{0}} کو ظاہر کرنے کے لئے \عددی{0} کو موٹی لکھائی میں لکھا جاتا ہے۔صفر سمتیہ وہ واحد سمتیہ ہے جس کی لمبائی صفر ہے۔ یہ حقیقت درج ذیل سے واضح ہے۔
\begin{align*}
\abs{a\ai+b\aj}=\sqrt{a^2+b^2}=0\quad \Leftrightarrow\quad a=b=0
\end{align*}

\جزوحصہء{اکائی سمتیات}
کوئی بھی سمتیہ جس کی لمبائی \عددی{1} ہو \اصطلاح{اکائی سمتیہ}\فرہنگ{سمتیہ!اکائی}\حاشیہب{unit vector}\فرہنگ{vector!unit} کہلائے گا۔ سمتیات \عددی{\ai} اور \عددی{\aj} اکائی سمتیات ہیں۔
\begin{align*}
\abs{\ai}=\abs{1\ai+0\aj}=\sqrt{1^2+0^2}=1,\quad \abs{\aj}=\abs{0\ai+1\aj}=\sqrt{0^2+1^2}=1
\end{align*}

سمتیہ \عددی{\kvec{u}} جو اکائی سمتیہ \عددی{\ai} کو  \عددی{\theta} زاویہ مثبت رخ گھما کر حاصل ہو گا،  کے سمتی اجزاء درج ذیل ہوں گے۔
\begin{align}
\kvec{u}=(\cos\theta)\ai+(\sin\theta)\aj
\end{align}
چونکہ اکائی سمتیہ کو گھمانے سے اس کی لمبائی تبدیل نہیں ہوتی لہٰذا \عددی{\kvec{u}} بھی اکائی سمتیہ ہو گا یعنی:
\begin{align*}
\abs{\kvec{u}}=\sqrt{(\cos\theta)^2+(\sin\theta)^2}=\sqrt{1^2}=1
\end{align*}
زاویہ \عددی{\theta} کو \عددی{0} تا \عددی{2\pi} کرنے سے \عددی{\kvec{u}} کا سر \عددی{N} مبدا کے گرد، گھڑی کے الٹ رخ،  دائرہ \عددی{x^2+y^2=1} پر چلتا ہے جو مستوی میں ہر ممکنہ رخ  کا اکائی سمتیہ دے گا۔

\جزوحصہء{لمبائی اور رخ}
اگر \عددی{\kvec{v}\ne \kvec{0}} ہو تب
\begin{align*}
\abs{\frac{\kvec{v}}{\abs{\kvec{v}}}}=\abs{\frac{1}{\abs{\kvec{v}}}\kvec{v}}=\frac{1}{\abs{\kvec{v}}}\abs{\kvec{v}}=1
\end{align*}
ہو گا لہٰذا \عددی{\tfrac{\kvec{v}}{\abs{\kvec{v}}}} اکائی سمتیہ ہو گا جس کا رخ \عددی{\kvec{v}} کا رخ ہو گا۔یوں ہم \عددی{\kvec{v}} کو اس کی دو اہم خواص، لمبائی اور رخ، کی صورت میں درج ذیل لکھ سکتے ہیں۔
\begin{align*}
\kvec{v}=\abs{\kvec{v}}\big(\tfrac{\kvec{v}}{\abs{\kvec{v}}}\big)
\end{align*}

یوں اگر \عددی{\kvec{u}\ne \kvec{0}} ہو تب
\begin{enumerate}[a.]
\item
\عددی{\tfrac{\kvec{v}}{\abs{\kvec{v}}}} اکائی سمتیہ ہو گا جس کا رخ \عددی{\kvec{v}} کا رخ ہو گا۔
\item
مساوات \عددی{\kvec{v}=\abs{\kvec{v}}(\tfrac{\kvec{v}}{\abs{\kvec{v}}})} سمتیہ \عددی{\kvec{v}} کو اس کی لمبائی اور رخ کی صورت میں  بیان کرتی ہے۔
\end{enumerate}

\ابتدا{مثال}
سمتیہ \عددی{\kvec{v}=3\ai-4\aj} کو اس کی لمبائی اور رخ کا حاصل ضرب لکھیں۔
 
حل:\quad
\begin{align*}
\abs{\kvec{v}}&=\sqrt{(3)^2+(-4)^2}=\sqrt{9+16}=5&&\text{\RL{\عددی{\kvec{v}} کی لمبائی}}\\
\frac{\kvec{v}}{\abs{\kvec{v}}}&=\frac{3\ai-4\aj}{5}=\frac{3}{5}\ai-\frac{4}{5}\aj&&\text{\RL{\عددی{\kvec{v}} کا رخ}}\\
\kvec{v}&=3\ai-4\aj=\underbrace{5}_{\text{لمبائی}}\big(\underbrace{\frac{3}{5}\ai-\frac{4}{5}\aj}_{\text{رخ}}\big)
\end{align*}
\انتہا{مثال}
%====================

\جزوحصہء{ڈھلوان، مماس اور عمود}
ایک سمتیہ اس صورت ایک خط کے متوازی ہو گا جب سمتیہ کو ظاہر کرنے والا قطع اور یہ خط متوازی ہوں۔ ایک غیر انتصابی سمتیہ کی ڈھلوان ان خطوط کی ڈھلوان ہو گی جو اس سمتیہ کے متوازی ہوں۔ یوں \عددی{a\ne 0} کی صورت میں سمتیہ \عددی{\kvec{v}=a\ai+b\aj} کا ڈھلوان \عددی{\tfrac{b}{a}} ہو گا۔

کسی نقطہ پر ایک منحنی کو ایک سمتیہ تب \اصطلاح{مماسی}\فرہنگ{مماس}\حاشیہب{tangent}\فرہنگ{tangent} یا \اصطلاح{عمودی}\فرہنگ{عمودی}\حاشیہب{normal}\فرہنگ{normal} ہو گا جب اس نقطہ پر منحنی کا مماس  اور یہ سمتیہ متوازی یا عمودی ہوں۔ اگلی مثال میں ایسی سمتیہ کو تلاش کرنا دکھایا گیا ہے۔

\ابتدا{مثال}
نقطہ \عددی{(1,1)} پر منحنی \عددی{y=\tfrac{x^3}{2}+\tfrac{1}{2}} کو مماسی اور عمودی اکائی سمتیات تلاش کریں۔

حل:\quad
ہم نقطہ \عددی{(1,0)} پر منحنی کے مماس کے متوازی اور عمودی اکائی سمتیات معلوم کرتے ہیں۔

اس نقطہ پر منحنی کے مماس کی ڈھلوان درج ذیل ہو گی۔
\begin{align*}
y'=\left.\frac{3x^2}{2}\right\vert_{x=1}=\frac{3}{2}
\end{align*}
ہم اتنی ڈھلوان کی اکائی سمتیہ تلاش کرتے ہیں۔ سمتیہ \عددی{\kvec{v}=2\ai+3\aj} اور اس کے ہر غیر صفر مضرب کی ڈھلوان \عددی{\tfrac{3}{2}} ہے۔ سمتیہ  \عددی{\kvec{v}} کا ایسا مضرب معلوم کرنے کے لئے جس کی لمبائی \عددی{1} ہو ہم \عددی{\kvec{v}} کو 
\begin{align*}
\abs{\kvec{v}}=\sqrt{2^2+3^2}=\sqrt{13}
\end{align*}
سے تقسیم کرتے ہیں۔ یوں درج ذیل حاصل ہو گا۔
\begin{align*}
\kvec{u}=\frac{\kvec{v}}{\abs{\kvec{v}}}=\frac{2}{\sqrt{13}}\ai+\frac{3}{\sqrt{13}}\aj
\end{align*}
سمتیہ \عددی{\kvec{u}} کی لمبائی \عددی{1} ہے اور یہ \عددی{(1,1)} پر منحنی کا مماس ہے۔ درج ذیل سمتیہ
\begin{align*}
-\kvec{u}=-\frac{2}{\sqrt{13}}\ai-\frac{3}{\sqrt{13}}\aj
\end{align*}
جو مخالف  رخ ہے بھی \عددی{(1,1)} پر منحنی کا مماس ہو گا۔ کسی اضافی شرط کے بغیر ان میں سے کسی ایک اکائی مماسی سمتیہ کو دوسری اکائی مماسی سمتیہ پر فوقیت نہیں دی جا سکتی ہے۔

نقطہ \عددی{(1,1)} پر منحنی کا عمودی سمتیہ تلاش کرنے کی خاطر ہم ایسا اکائی سمتیہ معلوم کرتے ہیں جس کی ڈھلوان \عددی{\kvec{u}} کی ڈھلوان کے بالعکس متناسب کے منفی کے برابر ہو۔ ہم \عددی{\kvec{u}} کے غیر سمتی اجزاء کے مقامات آپس میں تبدیل کر کے اور ان میں سے کسی ایک کی علامت بدل کر ایسا سمتیہ معلوم کر سکتے ہیں۔ یوں درج ذیل حاصل ہو گا۔
\begin{align*}
\kvec{n}=-\frac{3}{\sqrt{13}}\ai+\frac{2}{\sqrt{13}}\aj,\quad \text{}\quad -\kvec{n}=\frac{3}{\sqrt{13}}\ai-\frac{2}{\sqrt{13}}\aj
\end{align*}  
یہاں بھی دونوں اکائی سمتیات دیے گئے نقطہ پر منحنی کو عمودی ہیں۔ ان دو عمودی اکائی سمتیات کا رخ ایک دوسرے کے الٹ ہے لیکن دونوں \عددی{(1,1)} پر منحنی کو عمودی ہیں۔
\انتہا{مثال}
%====================

\حصہء{سوالات}
\موٹا{جیومیٹری اور حساب}\\
\ابتدا{سوال}
مستوی میں پائے جانے والے سمتیات \عددی{\kvec{A}}، \عددی{\kvec{B}} اور \عددی{\kvec{C}} کو شکل میں دکھایا گیا ہے۔ انہیں کاغذ پر اتار کر سر کے ساتھ دم جوڑ کر درج ذیل ترسیم کریں۔
\begin{multicols}{4}
\begin{enumerate}[a.]
\item
$\kvec{A}+\kvec{B}$
\item
$\kvec{A}+\kvec{B}+\kvec{C}$
\item
$\kvec{A}-2\kvec{B}$
\item
$\frac{1}{2}\kvec{A}-\kvec{C}$
\end{enumerate}
\end{multicols}
\انتہا{سوال}
%=======================
\ابتدا{سوال}
مستوی میں پائے جانے والے سمتیات \عددی{\kvec{A}}، \عددی{\kvec{B}} اور \عددی{\kvec{C}} کو شکل میں دکھایا گیا ہے۔ انہیں کاغذ پر اتار کر سر کے ساتھ دم جوڑ کر درج ذیل ترسیم کریں۔
\begin{multicols}{4}
\begin{enumerate}[a.]
\item
$\kvec{A}-\kvec{B}$
\item
$\kvec{A}+\kvec{B}+\kvec{C}$
\item
$2\kvec{A}-\frac{1}{2}\kvec{B}$
\item
$\kvec{A}-(\kvec{B}-\kvec{C})$
\end{enumerate}
\end{multicols}
\انتہا{سوال}
%=======================

سوال \حوالہ{سوال_سمتیہ_مجموعات_الف} تا سوال \حوالہ{سوال_سمتیہ_مجموعات_ب} میں \عددی{\kvec{A}=2\ai-7\aj}، \عددی{\kvec{B}=\ai+6\aj} اور \عددی{\kvec{C}=\sqrt{3}\ai-\pi\aj} لیں۔ نتائج کو \عددی{a\ai+b\aj} روپ میں لکھیں۔

\ابتدا{سوال}\شناخت{سوال_سمتیہ_مجموعات_الف}
$\kvec{A}+2\kvec{B}$
\انتہا{سوال}
%====================
\ابتدا{سوال}
$\kvec{A}+\kvec{B}-\kvec{C}$
\انتہا{سوال}
%====================
\ابتدا{سوال}
$3\kvec{A}-\frac{1}{\pi}\kvec{C}$
\انتہا{سوال}
%====================
\ابتدا{سوال}\شناخت{سوال_سمتیہ_مجموعات_ب}
$2\kvec{A}-3\kvec{B}+32\aj$
\انتہا{سوال}
%====================
\ابتدا{سوال}
مثلث \عددی{ABC}  کے اضلاع سمتیات \عددی{\kvec{u}}، \عددی{\kvec{v}} اور \عددی{\kvec{w}} دیتے ہیں۔
\begin{enumerate}[a.]
\item
\عددی{\kvec{w}} کو \عددی{\kvec{u}} اور \عددی{\kvec{v}} کی صورت میں لکھیں۔
\item
\عددی{\kvec{v}} کو \عددی{\kvec{u}} اور \عددی{\kvec{w}} کی صورت میں لکھیں۔
\end{enumerate}
\انتہا{سوال}
%===================
\ابتدا{سوال}
مثلث \عددی{ABC} کے اضلاع  سمتیات \عددی{\kvec{u}} اور \عددی{\kvec{w}} دیتے ہیں جبکہ \عددی{BC} کا وسطی نقطہ \عددی{N} ہے۔ سمتیہ \عددی{\kvec{a}} کو \عددی{\kvec{u}} اور \عددی{\kvec{w}} کی صورت میں لکھیں۔
\انتہا{سوال}
%==================

سوال \حوالہ{سوال_سمتیہ_خاکہ_الف} تا سوال \حوالہ{سوال_سمتیہ_خاکہ_ب} میں سمتیہ کو \عددی{a\ai+b\aj} روپ میں لکھیں۔ محددی سطح پر مبدا سے شروع کرتے ہوئے  انہیں ترسیم کریں۔

\ابتدا{سوال}\شناخت{سوال_سمتیہ_خاکہ_الف}
نقاط \عددی{N_1(5,7)} اور \عددی{N_2(2,9)} کے بیچ قطع \عددی{\krightharpoonup{N_1N_2}} تلاش کریں۔
\انتہا{سوال}
%======================
\ابتدا{سوال}
نقاط \عددی{N_1(1,2)} اور \عددی{N_2(-3,5)} کے بیچ قطع \عددی{\krightharpoonup{N_1N_2}}  تلاش کریں۔
\انتہا{سوال}
%======================
\ابتدا{سوال}
نقاط \عددی{A(-5,3)} اور \عددی{B(-10,8)} کے بیچ قطع \عددی{\krightharpoonup{AB}}  تلاش کریں۔
\انتہا{سوال}
%======================
\ابتدا{سوال}
نقاط \عددی{A(-7,-8)} اور \عددی{B(6,11)} کے بیچ قطع \عددی{\krightharpoonup{AB}}  تلاش کریں۔
\انتہا{سوال}
%======================
\ابتدا{سوال}
نقاط \عددی{N_1(1,3)} اور \عددی{N_2(2,-1)} کے بیچ قطع \عددی{\krightharpoonup{N_1N_2}}  تلاش کریں۔
\انتہا{سوال}
%======================
\ابتدا{سوال}
نقاط \عددی{N_3(1,3)} اور \عددی{N_4} کے بیچ قطع \عددی{\krightharpoonup{N_3N_4}}  تلاش کریں جہاں \عددی{N_1(2,-1)} اور \عددی{N_2(-4,3)} کو ملانے والے قطع کا وسطی نقطہ \عددی{N_4} ہے۔
\انتہا{سوال}
%======================
\ابتدا{سوال}
نقاط \عددی{A(1,-1)}، \عددی{B(2,0)}، \عددی{C(-1,3)} اور \عددی{D(-2,2)} دیے گئے ہیں۔ سمتیات \عددی{\krightharpoonup{CD}} اور \عددی{\krightharpoonup{AB}} کا مجموعہ تلاش کریں۔
\انتہا{سوال}
%==================
\ابتدا{سوال}\شناخت{سوال_سمتیہ_خاکہ_ب}
نقطہ \عددی{A} سے مبدا تک سمتیہ، جہاں \عددی{\krightharpoonup{AB}=4\ai-2\aj} اور \عددی{B(-2,5)} ہیں۔
\انتہا{سوال}
%=====================
\ابتدا{سوال}
سمتیہ \عددی{\krightharpoonup{AB}=3\ai-\aj} اور نقطہ \عددی{A(2,9)} دیا گیا ہے۔نقطہ \عددی{B} تلاش کریں۔
\انتہا{سوال}
%=======================
\ابتدا{سوال}
سمتیہ \عددی{\krightharpoonup{NQ}=-6\ai-4\aj} اور نقطہ \عددی{Q(3,3)} دیا گیا ہے۔نقطہ \عددی{N} تلاش کریں۔
\انتہا{سوال}
%====================

\موٹا{اکائی سمتیات}\\
سوال \حوالہ{سوال_سمتیہ_اکائی_الف} تا سوال \حوالہ{سوال_سمتیہ_اکائی_ب} میں دیے سمتیات ترسیم کریں۔ ان سمتیات کو \عددی{a\ai+b\aj} روپ میں لکھیں۔

\ابتدا{سوال}\شناخت{سوال_سمتیہ_اکائی_الف}
زاویہ \عددی{\theta=\tfrac{\pi}{6}} اور \عددی{\theta=\tfrac{2\pi}{3}} کے لئے اکائی سمتیات \عددی{\kvec{u}=(\cos\theta)\ai+(\sin\theta)\aj} ترسیم کریں۔ دائرہ \عددی{x^2+y^2=1} کی ترسیم بھی شامل کریں۔
\انتہا{سوال}
%===================
\ابتدا{سوال}
زاویہ \عددی{\theta=-\tfrac{\pi}{4}} اور \عددی{\theta=-\tfrac{3\pi}{4}} کے لئے اکائی سمتیات \عددی{\kvec{u}=(\cos\theta)\ai+(\sin\theta)\aj} ترسیم کریں۔ دائرہ \عددی{x^2+y^2=1} کی ترسیم بھی شامل کریں۔
\انتہا{سوال}
%======================
\ابتدا{سوال}
سمتیہ \عددی{\aj} کو مبدا کے گرد گھڑی کے الٹ رخ \عددی{\tfrac{3\pi}{4}} ریڈیئن  گھما کر حاصل اکائی سمتیہ ترسیم کریں۔
\انتہا{سوال}
%===================
\ابتدا{سوال}\شناخت{سوال_سمتیہ_اکائی_ب}
سمتیہ \عددی{\aj} کو مبدا کے گرد گھڑی کے رخ \عددی{\tfrac{2\pi}{3}} ریڈیئن  گھما کر حاصل اکائی سمتیہ ترسیم کریں۔
\انتہا{سوال}
%===================

سوال \حوالہ{سوال_سمتیہ_اکائی_گھوم_الف} اور سوال \حوالہ{سوال_سمتیہ_اکائی_گھوم_ب} میں اکائی سمتیہ \عددی{\kvec{u}=(\cos\theta)\ai+(\sin\theta)\aj} اسی رخ تلاش کریں۔

\ابتدا{سوال}\شناخت{سوال_سمتیہ_اکائی_گھوم_الف}
$6\ai-8\aj$
\انتہا{سوال}
%==================
\ابتدا{سوال}\شناخت{سوال_سمتیہ_اکائی_گھوم_ب}
$-\ai+3\aj$
\انتہا{سوال}
%=====================

سوال \حوالہ{سوال_سمتیہ_عمودی_مماسی_الف} تا سوال \حوالہ{سوال_سمتیہ_عمودی_مماسی_ب} میں دیے گئے نقطہ پر منحنی کے مماسی اکائی سمتیات اور عمودی اکائی سمتیات تلاش کریں۔ منحنی اور سمتیات کو ایک ساتھ ترسیم کریں۔ (سمتیات کی تعداد چار ہو گی۔)

\ابتدا{سوال}\شناخت{سوال_سمتیہ_عمودی_مماسی_الف}
$y=x^2,\quad (2,4)$
\انتہا{سوال}
%======================
\ابتدا{سوال}
$x^2+2y^2=6,\quad (2,1)$
\انتہا{سوال}
%======================
\ابتدا{سوال}
$y=\tan^{-1}x,\quad (1,\tfrac{\pi}{4})$
\انتہا{سوال}
%======================
\ابتدا{سوال}\شناخت{سوال_سمتیہ_عمودی_مماسی_ب}
$y=\sum_{n=0}^{\infty}\frac{x^n}{n!},\quad (0,1)$
\انتہا{سوال}
%======================
