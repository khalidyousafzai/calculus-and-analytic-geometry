\باب{سمتیات اور خلا میں تحلیلی جیومیٹری}
اس حصہ میں سمتیات اور سہ بعدی محددی نظام متعارف کئے جائیں گے۔ جیسا ایک متغیر کے تفاعل پر غور کے لئے محددی مستوی موزوں ہے، اسی طرح دو (یا دو سے زیادہ) متغیرات کے تفاعل پر غور کے لئے محددی خلاء موزوں ہے۔ ہم محددی مستوی میں ایک تیسرا محور شامل کر کے محددی خلاء پیدا کرتے ہیں۔ یہ محور \عددی{xy} مستوی سے نیچے اور اس سے اوپر فاصلہ ناپتا ہے۔

\حصہ{مستوی میں سمتیات} 
بعض چیزیں جنہیں ہم ناپتے ہیں کا تعین ان کی مقدار سے ہوتا ہے۔مثال کے طور پر کمیت، لمبائی اور وقت قلم بند کرنے کے لئے  ہم صرف ایک عدد اور موزوں اکائی لکھتے ہیں۔ اس کے برعکس قوت، ہٹاو، یا سمتی رفتار جاننے کے لئے ہمیں مزید معلوم درکار ہو گی۔ قوت کو بیان کرنے کے لئے ہمیں اس کی مقدار کے ساتھ وہ رخ بھی جاننا ہو گا جس رخ یہ عمل کرتی ہے۔ کسی جسم کا ہٹاو بیان کرنے کے لئے ہمیں اس سمت کا ذکر کرنا ہو گا جس سمت یہ جسم حرکت کرتا ہے اور ساتھ اس فاصلہ کا ذکر کرنا ہو گا جتنا یہ طے کرتا ہے۔ ایک جسم کی سمتی رفتار بیان کرنے کے لئے ہم حرکت کی سمت اور جسم کی رفتار کی بات کرتے ہیں۔

وہ مقدار جس کی جسامت اور سمت دونوں ہوں کو عموماً تیر کے نشان سے ظاہر کیا جاتا ہے جہاں مقدار کے رخ کو  تیر کا رخ  مقدار  کی جسامت کو، موزوں اکائیوں میں، تیر کی لمبائی ظاہر کرتی ہے۔

تیر کے اس نشان کو \اصطلاح{سمتیہ} کہتے ہیں۔

\ابتدا{تعریف}
ایک مستوی میں کسی مخصوص رخ خط کو \اصطلاح{سمتیہ}\فرہنگ{سمتیہ}\حاشیہب{vector}\فرہنگ{vector} کہتے ہیں۔ دو سمتیات صرف اس صورت ایک دوسرے کے برابر یا یکساں ہوں گے جب ان کی مقداریں ایک جیسی ہوں اور ان کے رخ ایک جیسے ہوں۔
\انتہا{تعریف}
%===================

یوں اگر سمتیات کو ظاہر کرنے والے تیر  آپس میں متوازی ہوں، ان کی لمبائیاں ایک جیسی ہوں اور ان کا رخ بھی ایک جیسا ہو تب یہ ایک ہی  سمتیہ کو ظاہر کرتے ہیں۔اس  کتاب میں سمتیہ کو موٹی لکھائی میں رومن حروف تہجی، مثلاً   \عددی{\kvec{v}}،  سے ظاہر کیا جائے گا\حاشیہد{قلم و کاغذ استعمال کرتے ہوئے سمتیہ کو رومن حروف تہجی پر تیر کا نشان \عددی{\vec{v}} یا نصف تیر کا نشان
$\overset{\rightharpoonup}{\rule{0pt}{.9ex}\smash{v}}$
 ڈال کر ظاہر کیا جاتا ہے۔)}۔نقطہ \عددی{A} سے نقطہ \عددی{B} تک تیر 
$\overset{\rightharpoonup}{\rule{0pt}{.9ex}\smash{AB}}$
 لکھا جائے گا۔

\ابتدا{مثال}
چار تیروں کو شکل میں دکھایا گیا ہے جن کی لمبائیاں اور رخ ایک جیسی ہیں۔ یوں یہ چاروں ایک ہی سمتیہ کو ظاہر کرتے ہیں جس کو ہم درج ذیل لکھتے ہیں۔
\begin{align*}
\overset{\rightharpoonup}{\rule{0pt}{.9ex}\smash{AB}}=\overset{\rightharpoonup}{\rule{0pt}{.9ex}\smash{CD}}=\overset{\rightharpoonup}{\rule{0pt}{.9ex}\smash{OP}}=\overset{\rightharpoonup}{\rule{0pt}{.9ex}\smash{EF}}
\end{align*}
\انتہا{مثال}
%=====================

\جزوحصہء{غیر سمتیہ اور غیر سمتی مضرب}
ہم کسی سمتیہ کو مثبت حقیقی عدد سے ضرب دینے کے لئے اس کی لمبائی کو اس عدد سے ضرب دیتے ہیں۔ سمتیہ کو \عددی{2} سے ضرب دینے کے لئے ہم اس کی لمبائی دگنی کرتے ہیں۔ ایک سمتیہ کو \عددی{1.5} سے ضرب دینے کے لئے ہم اس کی لمبائی \عددی{\SI{50}{\percent}} بڑھاتے ہیں، وغیرہ، وغیرہ۔ ایک سمتیہ کو منفی  عدد سے ضرب دینے کے لئے ہم اس کا رخ الٹ کر کے اس کی لمبائی کو عدد کی مطلق قیمت سے ضرب دیتے ہیں۔

اگر \عددی{c} غیر صفر حقیقی عدد اور \عددی{\kvec{v}} ایک سمتیہ ہو تب مثبت \عددی{c} کی صورت میں \عددی{\kvec{v}} اور \عددی{c\kvec{v}} کے رخ ایک جیسے ہوں گے جبکہ منفی \عددی{c} کی صورت میں ان کے رخ ایک دوسرے کے مخالف ہوں گے۔ یہاں حقیقی اعداد تبدیلی پیمانہ کے طور پر کام کرتے ہیں اور یہ  \اصطلاح{غیر سمتی}\فرہنگ{غیر سمتی}\حاشیہب{scalar}\فرہنگ{scalar} کہلاتے ہیں جبکہ \عددی{c\kvec{v}} کے مضرب کو \عددی{\kvec{v}} کا \اصطلاح{غیر سمتی مضرب}\فرہنگ{غیر سمتی!مضرب}\حاشیہب{scalar multiple}\فرہنگ{scalar multiple} کہتے ہیں۔

صفر سے ضرب کو شامل کرنے کی خاطر ہم  اس روایت کو اپناتے ہیں جس کے  مطابق کسی بھی سمتیہ کو صفر سے ضرب دینے سے \اصطلاح{صفر سمتیہ} \عددی{\kvec{0}} حاصل ہو گا، جو ایک نقطہ پر مشتمل ہو گا جس کی لمبائی صفر ہو گی۔ دیگر سمتیہ کے برعکس صفر سمتیہ \عددی{\kvec{0}} کا کوئی رخ نہیں ہوتا ہے۔  

\جزوحصہء{جیومیٹریائی مجموعہ: قاعدہ متوازی الاضلاع}
دو غیر صفر سمتیات \عددی{\kvec{v}_1} اور \عددی{\kvec{v}_2} کا جیومیٹریائی مجموعہ لینے کی خاطر \عددی{\kvec{v}_1} کا نمائندہ، مثلاً \عددی{A} سے \عددی{B} تک، ترسیم کر کے \عددی{\kvec{v}_1} کے اختتامی نقطہ \عددی{B} پر \عددی{\kvec{v}_2}  کے نمائندہ کا ابتدائی نقطہ رکھ کر ترسیم کریں۔ شکل میں
$\kvec{v}_2=\overset{\rightharpoonup}{\rule{0pt}{.9ex}\smash{BC}}$
ہے۔ مجموعہ \عددی{\kvec{v}_1+\kvec{v}_2} اب \عددی{\kvec{v}_1} کے ابتدائی نقطہ  \عددی{A} سے \عددی{\kvec{v}_2} کے اختتامی نقطہ  \عددی{C} تک کا سمتیہ ہو گا۔ یوں اگر
\begin{align*}
\kvec{v}_1=\overset{\rightharpoonup}{\rule{0pt}{.9ex}\smash{AB}},\quad \kvec{v}_2=\overset{\rightharpoonup}{\rule{0pt}{.9ex}\smash{BC}}
\end{align*}
ہوں تب
\begin{align*}
\kvec{v}_1+\kvec{v}_2=\overset{\rightharpoonup}{\rule{0pt}{.9ex}\smash{AB}}+\overset{\rightharpoonup}{\rule{0pt}{.9ex}\smash{BC}}=\overset{\rightharpoonup}{\rule{0pt}{.9ex}\smash{AC}}
\end{align*}
ہو گا۔چونکہ اس عمل میں \عددی{\kvec{v}_1+\kvec{v}_2} متوازی الاضلاع کا وتر ہوتا ہے لہٰذا اس عمل کو بعض اوقات \اصطلاح{قاعدہ متوازی الاضلاع}\فرہنگ{قاعدہ!متوازی الاضلاع}\حاشیہب{parallelogram law}\فرہنگ{law!parallelogram} کہتے ہیں۔

\جزوحصہء{اجزاء}
دو سمتیات اس صورت متوازی ہوں گے جب یہ ایک دوسرے کے غیر صفر، غیر سمتی مضرب ہوں، یعنی جب ان کو ظاہر کرنے والے خطوط متوازی ہوں۔

جب بھی ایک سمتیہ \عددی{\kvec{v}} کو دو غیر متوازی سمتیات کا مجموعہ
\begin{align*}
\kvec{v}=\kvec{v}_1+\kvec{v}_2
\end{align*}
لکھنا ممکن ہو، سمتیات \عددی{\kvec{v}_1} اور \عددی{\kvec{v}_2} سمتیہ \عددی{\kvec{v}} کے اجزاء کہلائیں گے۔ ہم یہ بھی کہیں گے کہ ہم سمتیہ  \عددی{\kvec{v}} کا اظہار اس کے اجزاء \عددی{\kvec{v}_1} اور \عددی{\kvec{v}_2} سے  کر رہے ہیں۔

سمتیات کے  مقبول ترین الجبرا میں ہر سمتیہ کو کارتیسی محور کے متوازی اجزاء کی صورت میں بیان کیا جاتا ہے اور یہ اجزاء از خود موزوں \اصطلاح{اساسی}\فرہنگ{اساسی}\حاشیہب{basic}\فرہنگ{basic} سمتیہ، جن کی لمبائی \عددی{1} ہوتی ہے، کے مضرب ہوتے ہیں۔ مثبت \عددی{x} محور کے رخ اساسی سمتیہ نقطہ \عددی{(0,0)} سے نقطہ \عددی{(1,0)} تک تیر سے ظاہر کیا جاتا ہے  اور اس اساسی سمتیہ کی علامت  \عددی{\ai} ہے۔ مثبت \عددی{y} محور کے رخ اساسی سمتیہ نقطہ \عددی{(0,0)} سے نقطہ \عددی{(0,1)} تک تیر سے ظاہر کیا جاتا ہے اور اس اساسی سمتیہ کی علامت  \عددی{\aj} ہے۔ اب غیر سمتی \عددی{a} کے لئے  محور \عددی{x} کے متوازی  سمتیہ \عددی{a\ai}  کی لمبائی \عددی{\abs{a}} ہو گی جبکہ اس کا رخ \عددی{a>0} کے لئے  دایاں اور  \عددی{a<0} کے لئے بایاں ہو گا۔ اس طرح غیر سمتی \عددی{b} کے لئے  محور \عددی{y} کے متوازی  سمتیہ \عددی{b\aj}  کی لمبائی \عددی{\abs{b}} ہو گی جبکہ اس کا رخ \عددی{b>0} کے لئے  اوپر اور  \عددی{b<0} کے لئے نیچے ہو گا۔ شکل میں سمتیہ 
$\kvec{v}=\overset{\rightharpoonup}{\rule{0pt}{.9ex}\smash{AC}}$
کا اظہار اجزاء \عددی{\ai} اور \عددی{\aj} سے  کیا گیا ہے:
\begin{align*}
\kvec{v}=a\ai+b\aj
\end{align*}

\ابتدا{تعریف}
اگر \عددی{\kvec{v}=a\ai+b\aj} ہو تب \عددی{\ai} اور \عددی{\aj} کے رخ سمتیہ \عددی{\kvec{v}} کے اجزاء  سمتیات \عددی{a\ai} اور \عددی{b\aj} ہوں گے۔ اعداد \عددی{a} اور \عددی{b} اساسی سمتیات \عددی{\ai} اور \عددی{\aj} کے رخ سمتیہ \عددی{\kvec{v}} کے غیر سمتی اجزاء ہوں گے۔ 
\انتہا{تعریف}
%================

\ابتدا{تعریف}
سمتیات کی برابری یا یکسانیت (الجبرائی تعریف)۔
\begin{align}
a\ai+b\aj=a'\ai+b'\aj\quad \Leftrightarrow\quad a=a',\quad b=b'
\end{align}
\انتہا{تعریف}
%=======

دو سمتیات صرف اور صرف اس صورت ایک دوسرے کے برابر ہوں گے جب \عددی{\ai} اور \عددی{\aj} کے رخ ان کے مطابقتی غیر سمتی اجزاء آپس میں ایک دوسرے کے برابر ہوں۔

\جزوحصہء{الجبرائی مجموعہ}
سمتیات کے مطابقتی غیر سمتی اجزاء کا مجموعہ لے کر ان سمتیات  کا مجموعہ حاصل کیا جا سکتا ہے۔

اگر \عددی{\kvec{v}_1=a_1\ai+b_1\aj} اور \عددی{\kvec{v}_2=a_2\ai+b_2\aj} ہوں تب درج ذیل ہو گا۔ 
\begin{align*}
\kvec{v}_1+\kvec{v}_2=(a_1+a_2)\ai+(b_1+b_2)\aj
\end{align*}
