
\حصہء{سوالات}
\موٹا{متغیر قوت کا کام}\\
\ابتدا{سوال}
اگر مثال \حوالہ{مثال_تکمل_استعمال_کنواں_اور_بوکا} میں بوکا کا حجم \عددی{\SI{20}{\liter}} ہو لیکن اس میں سوراخ بھی بڑا ہو تا کہ اب بھی  بوکا کو کنواں سے نکالتے ہوئے بوکا خالی ہو جاتا ہو۔ بوکا اور رسی کی کمیت کو شامل نہ کرتے ہوئے ایک بار بوکا نکالنے کے لئے درکار کام دریافت کریں۔بوکا سے پانی کے اخراج کو مستقل تصور کریں۔\\
جواب:\quad
$\SI{1960}{\joule}$
\انتہا{سوال}
%====================
\ابتدا{سوال}
فرض کریں کہ مثال \حوالہ{مثال_تکمل_استعمال_کنواں_اور_بوکا} میں بوکا کو اس رفتار سے اوپر کھینچا جاتا ہے کہ آخر میں بوکا میں \عددی{\SI{4}{\liter}} پانی ہوتا ہے۔ پانی نکالنے میں کتنا کام درکار ہو گا؟  بوکا اور رسی کی کمیت کو شامل نہ کریں اور بوکا سے پانی کے اخراج کو مستقل تصور کریں۔\\
\انتہا{سوال}
%==================
\ابتدا{سوال}
ایک کوہ پیما چٹان سے لٹکی ہوئی \عددی{\SI{50}{\meter}} رسی کو اوپر کھینچتا ہے۔ رسی کی کثافتی وزن \عددی{\SI{0.624}{\newton\per\meter}} ہے۔ کتنا کام درکار ہو گا؟
\انتہا{سوال}
%=====================
\ابتدا{سوال}
ریت کو تھیلے میں ڈال کر \عددی{\SI{6}{\meter}} بلند چھت تک برقرار رفتار سے کھینچ کر پہنچایا جاتا ہے۔ تھیلے میں سوراخ سے ریت کا اخراج ہوتا ہے جس کو مستقل تصور کیا جا سکتا ہے۔  ابتدائی طور پر تھیلا میں \عددی{\SI{50}{\kilo\gram}} ریت  ہوتی ہے جو  آخر میں آدھی رہ جاتی ہے۔ رسی اور تھیلا کی کمیت کو نظر انداز کرتے ہوئے درکار کام معلوم کریں۔
\انتہا{سوال}
%==================
\ابتدا{سوال}
آج کل بالخصوص بلند  عمارتوں میں  سیڑھیوں کے ساتھ ساتھ  \اصطلاح{رفع}\حاشیہب{lift} بھی پائے جاتے ہیں۔ رفع کو چھت پر رکھے ہوئے موٹر کی طاقت سے چلایا جاتا ہے۔  کئی لڑیوں پر مشتمل رسی کی کثافت \عددی{\SI{6}{\kilo\gram\per\meter}} ہونے کی صورت میں صرف رسی کو زمین سے  \عددی{\SI{60}{\meter}} بلند عمارت کی چھت تک اٹھانے میں موٹر کتنا کام کرے گی؟\\
جواب:\quad
$\SI{1764}{\joule}$
\انتہا{سوال}
%=====================
\ابتدا{سوال}
نقطہ \عددی{(x,0)} پر پائے جانے والے ذرہ جس کی کمیت \عددی{m} ہے پر قوت \عددی{F=\tfrac{k}{x^2}} عمل کرتی ہے جہاں \عددی{k} مستقل ہے۔ یہ ذرہ ساکن حال سے شروع ہو کر نقطہ \عددی{b} سے نقطہ \عددی{a} پہنچتا ہے جہاں \عددی{0<a<b} ہیں۔ اس ذرہ پر کتنا کام ہوا؟
\انتہا{سوال}
%==================
\ابتدا{سوال}\شناخت{سوال_تکمل_استعمال_انجن}
ایک بیلن جس کا رقبہ عمودی تراش \عددی{S} ہے میں موجود گیس پر میکانی دباو ڈالا جاتا ہے (شکل \حوالہ{شکل_سوال_تکمل_استعمال_انجن})۔ اگر گیس کا حجم \عددی{V} اور اس کا دباو \عددی{p} ہو تب دکھائیں کہ گیس کو \عددی{(p_1,V_1)} حال سے \عددی{(p_2,V_2)} حال تک پہنچانے میں درج ذیل  کام درکار ہو گا؟  
\begin{align*}
W=\int_{(p_1,V_1)}^{(p_2,V_2)}p\dif V
\end{align*}
(اشارہ: شکل \حوالہ{شکل_سوال_تکمل_استعمال_انجن} کو دیکھ کر بوکا پر قوت کو \عددی{F=pS} اور چھوٹے حجم کو \عددی{\dif V=S\dif x} لکھا جا سکتا ہے۔)
\انتہا{سوال}
%=====================
\begin{figure}
\centering
\begin{tikzpicture}
\pgfmathsetmacro{\l}{4}
\pgfmathsetmacro{\w}{1.25}
\pgfmathsetmacro{\t}{0.2}
\pgfmathsetmacro{\s}{3}
\pgfmathsetmacro{\a}{1.5}
\draw[-latex](-0.5,0)--(6,0)node[right]{$x$};
\draw[-latex](0,-1)--(0,1.5)node[above]{$y$};
\draw(\l,-\w/2)--++(-\l,0)--++(0,\w)--++(\l,0)--++(0,\t)--++(-\l-\t,0)--++(0,-\w-2*\t)--++(\l+\t,0)--++(0,\t);
\draw[draw=black,fill=lgray,opacity=0.5](\s+\a,-\t)--++(-\a,0)--++(0,-\w/2+\t)--++(-2*\t,0)--++(0,\w)--++(2*\t,0)--++(0,-1/2*\w+\t)--++(\a,0)--++(0,-2*\t);
\draw[stealth-](\l/2,\w/2+\t) to [out=45,in=180]++(0.5,0.5)node[right]{\RL{بیلن}};
\draw[stealth-](\s,\w/3) to [out=20,in=-135]++(0.5,0.75)node[right]{\RL{بوکا}};
\draw(\l/3,\w/4)node[]{گیس};
\end{tikzpicture}
\caption{گاڑی کا انجن ایک بیلن جس میں بوکا چلتا ہو پر مشتمل ہوتا ہے۔ بوکے کی حرکت سے گیس کا حجم اور دباو تبدیل ہوتے ہیں (سوال \حوالہ{سوال_تکمل_استعمال_انجن})۔}
\label{شکل_سوال_تکمل_استعمال_انجن}
\end{figure}

\ابتدا{سوال}
اگر گیس کا ابتدائی حجم \عددی{V_1=\SI{1500}{\centi\meter\cubed}}، ابتدائی دباو \عددی{\SI{103360}{\newton\per\meter\squared}} اور اختتامی حجم \عددی{\SI{200}{\centi\meter\cubed}} ہو تب سوال \حوالہ{سوال_تکمل_استعمال_انجن} کے تکمل سے کام دریافت کریں۔ یہاں آپ فرض کریں کہ گیس کا دباو ایک \اصطلاح{حرارت نا گزر عمل}\فرہنگ{حرارت نا گزر عمل}\حاشیہب{adiabatic process}\فرہنگ{adiabatic process} ہے جس میں حراری توانائی تبدیل نہیں ہوتی ہے۔ حرارت نا گزر عمل کے قانون کے تحت \عددی{pV^{1.4}=c} ہو گا جہاں \عددی{c} مستقل ہے۔ 
\انتہا{سوال}
%======================
\موٹا{اسپرنگ}\\
\ابتدا{سوال}
ایک اسپرنگ جس کی قدرتی لمبائی \عددی{\SI{2}{\meter}} ہے کی لمبائی کو \عددی{\SI{5}{\meter}} بنانے کے لئے درکار کام \عددی{\SI{1800}{\joule}} ہے۔ اس اسپرنگ کا مقیاس لچک تلاش کریں۔
\انتہا{سوال}
%==================
\ابتدا{سوال}
ایک اسپرنگ جس کی قدرتی لمبائی \عددی{\SI{30}{\centi\meter}} ہے پر \عددی{\SI{400}{\newton}} قوت لاگو کرتے ہوئے اس  کو کھینچ کر \عددی{\SI{45}{\centi\meter}} لمبائی تک پہنچایا جاتا ہے۔ (ا) مقیاس لچک تلاش کریں۔ (ب) اسپرنگ کی لمبائی کو \عددی{\SI{35}{\centi\meter}} کرنے کے لئے کتنی قوت درکار ہو گی؟ (ج) قدرتی لمبائی سے \عددی{\SI{600}{\newton}} قوت اسپرنگ کی لمبائی کو کتنا زیادہ کرتی ہے؟
\انتہا{سوال}
%====================
\ابتدا{سوال}
ایک ربڑی پٹی کی لمبائی کو \عددی{\SI{2}{\newton}} کی قوت \عددی{\SI{2}{\centi\meter}} بڑھاتی ہے۔ ربڑی پٹی پر قانون ہک کا اطلاق ہوتا ہے۔ ربڑی پٹی کی لمبائی کو \عددی{\SI{4}{\newton}} کی قوت کتنا بڑھائے گی اور یہ قوت کتنا کام کرے گی؟
\انتہا{سوال}
%=====================
\ابتدا{سوال}
اگر \عددی{\SI{90}{\newton}} کی قوت اسپرنگ کی لمبائی کو قدرتی لمبائی سے \عددی{\SI{1}{\meter}} زیادہ کرتی ہو تب اسپرنگ کی قدرتی لمبائی سے اس کی لمبائی کو \عددی{\SI{5}{\meter}} زیادہ کرنے کے لئے کتنا کام درکار ہو گا؟
\انتہا{سوال}
%===============
\ابتدا{سوال}
ریل گاڑی کے ڈبوں پر نسب اسپرنگ ان ڈبوں کو ایک دوسرے سے دور رکھتے ہیں اور ان کی ٹکراؤ کو محفوظ بناتے ہیں۔ ایسا ایک اسپرنگ جس کی قدرتی لمبائی \عددی{\SI{20}{\centi\meter}} ہے پر \عددی{\SI{100000}{\newton}} کی قوت  لاگو کرنے سے اسپرنگ کی کم سے کم لمبائی \عددی{\SI{12}{\centi\meter}} حاصل ہوتی ہے۔ (ا) اسپرنگ کا مقیاس لچک تلاش کریں۔ (ب) اسپرنگ کو پہلا \عددی{\si{\centi\meter}} دبانے کے لئے کتنا کام درکار ہو گا۔ اس کو دوسرا سنٹی میٹر دبانے کے لئے کتنا کام درکار ہو گا؟\\
جواب:\quad
(ا) \عددی{\SI{1.25e6}{\newton\per\meter}}، (ب) \عددی{\SI{62.5}{\joule}}، \عددی{\SI{187.5}{\joule}}
\انتہا{سوال}
%=================
\ابتدا{سوال}
گھریلو استعمال کے ترازو پر \عددی{\SI{74}{\kilo\gram}} کا شخص کھڑا ہونے سے ترازو \عددی{\SI{1.5}{\milli\meter}} دبتا ہے۔ فرض کریں کہ یہ ترازو قانون ہک کے تحت کام کرتا ہے۔ ایک شخص، جس کا ترازو پر کھڑا ہونے سے ترازو \عددی{\SI{3}{\milli\meter}} دبتا ہو، کا وزن کتنا ہو گا؟
\انتہا{سوال}
%====================
\موٹا{پانی کی نکاسی}\\
ثقلی اسراع کی قیمت کو عموماً \عددی{g=\SI{9.8}{\meter\per\second\squared}} لیا جاتا ہے۔ حقیقت میں سطح سمندر پر اس کی قیمت قطبین پر \عددی{\SI{9.832}{\meter\per\second\squared}} اور  عرضی خط استوا پر \عددی{\SI{9.780}{\meter\per\second\squared}} ہے۔ان دو قیمتوں میں فرق تقریباً \عددی{\SI{0.5}{\percent}} ہے۔

\ابتدا{سوال}\شناخت{سوال_تکمل_استعمال_حوض_زیر_زمین_الف}
بارانی علاقوں میں بارش کے پانی کو زیر زمین حوض میں ذخیرہ کیا جاتا ہے۔ زیر زمین حوض جس کو شکل \حوالہ{شکل_سوال_تکمل_استعمال_حوض_زیر_زمین_الف} میں دکھایا گیا ہے پانی سے بھرا ہوا ہے۔ حوض کو خالی کرتے ہوئے  پانی کو سطح زمین پر لایا جاتا ہے۔(ا) حوض کو خالی کرنے کے لئے کتنا کام کرنا ہو گا؟ (ب)  \عددی{\SI{0.25}{\kilo\watt}} کا پمپ حوض کو کتنی دیر میں خالی کرے گا؟ (ج) دکھائیں کہ ابتدائی \عددی{5} گھنٹوں میں تقریباً آدھا حوض خالی ہو جائے گا۔ (د) خط استوا پر جزو-ب کا جواب کیا ہو گا؟ قطبین پر یہ جواب کیا ہو گا؟\\
جواب:\quad
(ا) \عددی{\SI{18.375e6}{\joule}}، (ب) \عددی{20} گھنٹے اور \عددی{25} منٹ۔ (د) \عددی{20} گھنٹے اور \عددی{22.5} منٹ، \عددی{20} گھنٹے اور \عددی{29} منٹ۔
\انتہا{سوال}
%=======================
\begin{figure}
\centering
\begin{minipage}{0.45\textwidth}
\centering
\begin{tikzpicture}
\pgfmathsetmacro{\l}{2.5}
\pgfmathsetmacro{\w}{1}
\pgfmathsetmacro{\h}{2}
\pgfmathsetmacro{\ang}{50}
\draw[-latex](0,0)--++(0,-2.5)node[below]{$y$};
\draw(0,0)--++(-0.5,0)node[left]{\RL{سطح زمین}};
\draw(0,0)--++(\l,0)--++(\ang:\w)--++(-\l,0)node[pos=0.5,above]{$\SI{15}{\meter}$}--++(\ang:-\w)node[pos=0.4,left,xshift=-1mm]{$\SI{10}{\meter}$};
\draw(0,-\h)--++(\l,0)--++(\ang:\w)--++(-\l,0)--++(\ang:-\w);
\draw(0,-\h/2)node[left]{$y$}--++(\l,0)--++(\ang:\w)coordinate[](ka)--++(-\l,0)--++(\ang:-\w);
\draw(0,0)--++(0,-\h)node[left]{$\SI{5}{\meter}$};
\draw(\l,0)--++(0,-\h);
\draw(\l,0)++(\ang:\w)--++(0,-\h);
\draw(\ang:\w)--++(0,-\h);
\draw(ka)++(0.1,0)--++(0.4,0)coordinate[pos=0.5](kka);
\draw(ka)++(0.1,0.1)--++(0.4,0)coordinate[pos=0.5](kkaa);
\draw[stealth-](kka)--++(0,-0.2)--++(0.2,0)node[right]{$\Delta y$};
\draw[stealth-](kkaa)--++(0,0.2);
\end{tikzpicture}
\caption{زیر زمین حوض (سوال \حوالہ{سوال_تکمل_استعمال_حوض_زیر_زمین_الف})}
\label{شکل_سوال_تکمل_استعمال_حوض_زیر_زمین_الف}
\end{minipage}\hfill
\begin{minipage}{0.45\textwidth}
\centering
\begin{tikzpicture}
\pgfmathsetmacro{\l}{2.5}
\pgfmathsetmacro{\w}{1}
\pgfmathsetmacro{\h}{2}
\pgfmathsetmacro{\ang}{50}
\draw[-latex](0,0)--++(0,-2.5)node[below]{$y$};
\draw(1,0)--(-0.5,0)node[left]{\RL{سطح زمین}};
\draw(0,-\h/2)--++(\l,0)--++(\ang:\w)--++(-\l,0)--++(\ang:-\w);
\draw(0,-\h)--++(\l,0)node[pos=0.5,below]{$\SI{10}{\meter}$}--++(\ang:\w)node[pos=0.5,right]{$\SI{6}{\meter}$}coordinate[](ka)--++(-\l,0)--++(\ang:-\w);
\draw(0,-\h/2)node[left]{$\SI{4}{\meter}$}--++(0,-\h/2)node[left]{$\SI{7}{\meter}$};
\draw(\l,-\h/2)--++(0,-\h/2);
\draw(\l,-\h/2)++(\ang:\w)--++(0,-\h/2);
\draw(\ang:\w)++(0,-\h/2)--++(0,-\h/2);
\end{tikzpicture}
\caption{زیر زمین حوض (سوال \حوالہ{سوال_تکمل_استعمال_حوض_زیر_زمین_ب})}
\label{شکل_سوال_تکمل_استعمال_حوض_زیر_زمین_ب}
\end{minipage}
\end{figure}
\ابتدا{سوال}\شناخت{سوال_تکمل_استعمال_حوض_زیر_زمین_ب}
 زیر زمین حوض جس کو شکل \حوالہ{شکل_سوال_تکمل_استعمال_حوض_زیر_زمین_ب} میں دکھایا گیا ہے پانی سے بھرا ہوا ہے۔ حوض کا کنارہ سطح زمین سے \عددی{\SI{4}{\meter}} نیچے ہے۔ حوض کو خالی کرتے ہوئے  پانی کو سطح زمین پر لایا جاتا ہے۔(ا) حوض کو خالی کرنے کے لئے کتنا کام کرنا ہو گا؟ (ب)  \عددی{\SI{0.25}{\kilo\watt}} کا پمپ حوض کو کتنی دیر میں خالی کرے گا؟ (ج) آدھا حوض کتنی دیر میں خالی ہو گا؟ (پورا حوض خالی کرنے کے نصف دورانیہ سے کم وقت درکار ہو گا)۔ (د) خط استوا پر جزو-ب کا جواب کیا ہو گا؟ قطبین پر یہ جواب کیا ہو گا؟\\
\انتہا{سوال}
%======================
\ابتدا{سوال}
اگر حوض کے کنارے سے \عددی{\SI{4}{\meter}} بلند کی بجائے حوض کے کنارے تک پانی کو اٹھایا جائے تب مثال \حوالہ{مثال_تکمل_استعمال_نکاسی_پانی} میں کتنا کام درکار ہو گا؟\\
جواب:\quad
$\SI{38484510}{\joule}$
\انتہا{سوال}
%======================
\ابتدا{سوال}
اگر مثال \حوالہ{مثال_تکمل_استعمال_نکاسی_پانی} میں حوض آدھا بھرا ہو تب حوض کے کنارے سے \عددی{\SI{4}{\meter}} بلندی تک پانی کو پہنچانے کے لئے کتنا کام کرنا ہو گا؟\\
جواب:\quad
$\SI{19.95e6}{\joule}$
\انتہا{سوال}
%======================
\ابتدا{سوال}
ایک بیلنی حوض جس کا رداس \عددی{\SI{4}{\meter}} اور قد \عددی{\SI{10}{\meter}} ہے مٹی کے تیل سے بھرا ہوا ہے۔ مٹی کے تیل کی کثافت \عددی{\SI{0.81}{\gram\per\centi\meter}} ہے۔ تمام تیل کو حوض کے بالائی کنارے تک پمپ  کرنے کے لئے کتنا کام کرنا ہو گا؟
\انتہا{سوال}
%=====================
\ابتدا{سوال}\شناخت{سوال_بیلنی_حوض_بمع_نل}
ایک حوض جس کا قد \عددی{\SI{5}{\meter}} ہے سطح زمین پر پڑا ہوا ہے (شکل \حوالہ{شکل_سوال_بیلنی_حوض_بمع_نل})۔قدرتی پانی  سطح زمین سے \عددی{\SI{7}{\meter}} نیچے ہے۔ حوض کو اس پانی سے دو طرح بھرا جا سکتا ہے۔ (ا) پمپ کے خارجی پائپ کو حوض کے کنارے پر رکھ کر حوض کو بھرا جا سکتا ہے۔ (ب) حوض کے نچلی سر پر  موجود نل کے ذریعہ پانی کو حوض تک منتقل کیا جا سکتا ہے۔ دنوں تراکیب میں کونسا بہتر ہے؟ اپنے جواب کی وجہ پیش کریں۔     
\انتہا{سوال}
%====================
\begin{figure}
\centering
\begin{minipage}{0.45\textwidth}
\centering
\begin{tikzpicture}[font=\small]
\pgfmathsetmacro{\h}{1.5}
\pgfmathsetmacro{\r}{1.2}
\draw([shift={(180:\r cm and 1/4*\r cm)}]0,0) arc (180:360:\r cm and 1/4*\r cm);
\draw(0,\h) circle (\r cm and 1/4*\r cm);
\draw(-\r,0)--++(0,\h)  (\r,0)--++(0,\h);
\draw[fill=white](\r,0)rectangle++(0.4,0.2)node[right]{نل};
\draw[stealth-stealth](-\r-0.2,0)--++(0,\h)node[pos=0.5,left]{$\SI{5}{\meter}$};
\end{tikzpicture}
\caption{بیلنی حوض (سوال \حوالہ{سوال_بیلنی_حوض_بمع_نل})}
\label{شکل_سوال_بیلنی_حوض_بمع_نل}
\end{minipage}\hfill
\begin{minipage}{0.45\textwidth}
\centering
\begin{tikzpicture}[font=\small]
\pgfmathsetmacro{\h}{2}
\pgfmathsetmacro{\rr}{1.2}
\pgfmathsetmacro{\r}{0.9}
\pgfmathsetmacro{\k}{1.05}
\draw[-latex](\r+0.1,0)--++(1.5,0)node[right]{$x$};
\draw[-latex](0,\h)node[right]{$15$}--++(0,1)node[above]{$y$};
\draw([shift={(180:\r cm and 1/4*\r cm)}]0,0) arc (180:360:\r cm and 1/4*\r cm);
\draw(0,\h) circle (\rr cm and 1/4*\rr cm);
\draw(-\r,0)--(-\rr,\h)  (\r,0)--(\rr,\h);
\draw[gray](0,\h/2) circle (\k cm and 1/4*\k cm);
\draw[fill=white](-1/2*\rr,\h)rectangle++(0.15,0.4)node[above]{چسنا};
\draw[thick,shorten <=-0.5 cm, shorten >=-0.5 cm] (\r,0)--(\rr,\h)node[above,yshift=0.5cm,xshift=0.5cm]{$y=10x-30$};
\draw(\r,0)node[below right]{$(3,0)$} (\rr,\h)node[right]{$(4.5,15)$};
\draw[gray](0,\h/2)node[left]{$y$}--++(\k,0)coordinate[pos=0.75](ka);
\draw(ka)++(0,0.05)--++(0.5,0.25)node[right]{$x=\tfrac{y+30}{10}$};
\draw(-\k-0.2,\h/2)--++(-0.2,0)coordinate[pos=0.5](kb);
\draw(-\k-0.2,\h/2+0.1)--++(-0.2,0)coordinate[pos=0.5](kc);
\draw[stealth-](kc)--++(0,0.2);
\draw[stealth-](kb)--++(0,-0.2)--++(-0.2,0)node[left]{$\Delta y$};
\end{tikzpicture}
\caption{مخروط مقطوع ڈبیا (پیمائش سنٹی میٹروں میں ہے۔)}
\label{شکل_سوال_مخروط_ڈبیا}
\end{minipage}
\end{figure}
\ابتدا{سوال}\شناخت{سوال_مخروط_ڈبیا}
ایک مشروب جس کی کثافت \عددی{\SI{2}{\gram\per\centi\meter\cubed}} ہے سے مخروط  مقطوع ڈبیا  بھرا ہوا ہے (\حوالہ{شکل_سوال_مخروط_ڈبیا})۔اس ڈبیا کا بالائی رداس \عددی{\SI{4.5}{\centi\meter}}، زیریں رداس \عددی{\SI{3}{\centi\meter}} اور گہرائی \عددی{\SI{15}{\centi\meter}} ہے۔ مشروب کو چسنا کے ذریعہ پیا جاتا ہے جو ڈبیا کی بالائی سطح سے \عددی{\SI{2.5}{\centi\meter}} باہر نکلا ہوا ہے۔ پورا مشروب پینے کے لئے کتنا کام کرنا ہو گا۔
\انتہا{سوال}
%=======================
