\باب{تکمل کے طریقے}
ہم نے دیکھا کہ چیزوں کی ناپ اور روز مرہ زندگی کے اعمال کی نمونہ کشی تکمل کو جنم دیتے ہیں۔ ہم جانتے ہیں کہ الٹ تفرق سے تکمل کو حل کیا جا سکتا ہے۔کسی عمل کی نمونہ کشی میں زیادہ گہرائی تک جانے سے زیادہ پیچیدہ تکمل حاصل ہوتا ہے۔ ہم جاننا چاہتے ہیں کہ اس طرح کے پیچیدہ تکمل کو کس طرح سادہ صورت دی جا سکتی ہے جن کے ساتھ کام کرنا آسان ہو۔ اس باب میں ہم انجانے تکمل سے جانے پہچانے تکمل کا حصول سیکھیں گے جنہیں جدول سے دیکھا جا سکتا ہے یا جس کو کمپیوٹر سے حل کیا جا سکتا ہے۔

\حصہ{تکمل کے بنیادی کلیات}   
ہم نے حصہ \حوالہ{حصہ_تکمل_غیر_قطعی_تکمل} میں دیکھا کہ غیر قطعی تکمل کو حل کرنے کے لئے اس کے الٹ تفرق کے ساتھ مستقل جمع کرنا ہو گا۔ جدول \حوالہ{جدول_طریقے_تکمل_بنیادی_کلیات} میں ان  تکمل کی بنیادی روپ درج کی گئی ہے جنہیں اب تک ہم حل کرتے آ رہے ہیں۔ زیادہ تکملات کا جدول کتاب کی آخر میں پیش کیا گیا ہے جس پر حصہ میں غور کیا جائے گا۔

\begin{table}
\caption{تکمل کے بنیادی کلیات}
\label{جدول_طریقے_تکمل_بنیادی_کلیات}
\renewcommand{\arraystretch}{2}
\centering
\begin{tabular}{L|L}
\toprule
\text{شمار}&\text{کلیہ}\\
\midrule
1&\int \dif u=u+C\\
2&\int k\dif u=ku+C \quad \text{\RL{($k$ عدد ہے)}}\\
3&\int(\dif u+\dif v)=\int \dif u+\int \dif v\\
4&\int u^n\dif u=\frac{u^{n+1}}{n+1}+C\quad (n\ne -1)\\
5&\int\frac{\dif u}{u}=\ln \abs{u}+C\\
6&\int\sin u\dif u=-\cos u+C\\
7&\int\cos u\dif u=\sin u+C\\
8&\int \sec^2u\dif u=\tan u +C\\
9&\int \csc^2u\dif u=-\cot u+C\\
10&\int\sec u\tan u\dif u=\sec u+C\\
11&\int \csc u\cot u\dif u=-\csc u+C\\
12&\int \tan u\dif u=-\ln\abs{\cos u}+C=\ln\abs{\sec u}+C\\
13&\int \cot u\dif u=\ln \abs{\sin u}+C=-\ln\abs{\csc u}+C\\
14&\int e^u\dif u=e^u+C\\
15&\int a^u\dif u=\frac{a^u}{\ln a}+C\quad (a>0,\, a\ne 1)\\
16&\int \frac{\dif u}{\sqrt{a^2-u^2}}=\sin^{-1}(\tfrac{u}{a})+C\\
17&\int \frac{\dif u}{a^2+u^2}=\frac{1}{a}\tan^{-1}(\tfrac{u}{a})+C\\
18&\int\frac{\dif u}{u\sqrt{u^2-a^2}}=\frac{1}{a}\sec^{-1}\abs{\tfrac{u}{a}}+C\\
\bottomrule
\end{tabular}
\end{table}
\جزوحصہء{الجبرائی طریقہ}
ہمیں عموماً تکمل کو جانی پہچانی معیاری روپ میں لکھنا ہو گا۔

\ابتدا{مثال}\ترچھا{سادہ روپ حاصل کرنے کا بدل}\\
تکمل \عددی{\int\tfrac{2x-9}{\sqrt{x^2-9x+1}}\dif x} حل کریں۔

حل:
\begin{align*}
\int\frac{2x-9}{\sqrt{x^2-9x+1}}\dif x&=\int\frac{\dif u}{\sqrt{u}}&&u=x^2-9x+1\\
&=\int u^{-1/2}\dif u\\
&=\frac{u^{(-1/2)+1}}{(-1/2)+1}+C&&\text{\RL{جدول \حوالہ{جدول_طریقے_تکمل_بنیادی_کلیات} کلیہ $4$ میں $n=-1/2$}}\\
&=2u^{1/2}+C\\
&=2\sqrt{x^2-9x+1}+C
\end{align*}
\انتہا{مثال}
%================
\ابتدا{مثال}\ترچھا{تکمیل مربع}\\
تکمل \عددی{\int\tfrac{\dif x}{\sqrt{8x-x^2}}} حل کریں۔

حل:\quad
ہم مربع مکمل کرتے ہوئے زیر جذر کو لکھتے ہیں:
\begin{align*}
8x-x^2&=-(x^2-8x)=-(x^2-8x+16-16)\\
&=-(x^2-8x+16)+16=16-(x-4)^2
\end{align*}
\begin{align*}
\int\frac{\dif x}{\sqrt{8x-x^2}}&=\int\frac{\dif x}{\sqrt{16-(x-4)^2}}\\
&=\int\frac{\dif u}{\sqrt{a^2-u^2}}&&a=4,u=(x-4)\\
&=\sin^{-1}(\tfrac{u}{a})+C&&\text{\RL{جدول \حوالہ{جدول_طریقے_تکمل_بنیادی_کلیات} کلیہ $16$}}\\
&=\sin^{-1}(\tfrac{x-4}{4})+C
\end{align*}
\انتہا{مثال}
%================
\ابتدا{مثال}\ترچھا{طاقت پھیلا کر تماثل کا استعمال}\\
تکمل \عددی{\int(\sec x\tan x)^2\dif x} حل کریں۔

حل:\quad
ہم متکمل کو پھیلاتے ہیں۔
\begin{align*}
(\sec x+\tan x)^2=\sec^2x+2\sec x\tan x+\tan^2x
\end{align*}
بائیں ہاتھ پہلے دو اجزاء کا تکمل ہم جانتے ہیں البتہ \عددی{\tan^2x} کا کچھ کرنا ہو گا۔ ہم درج ذیل تماثل کے ذریعہ اس کو جانی پہچانی روپ میں تبدیل کرتے ہیں۔
\begin{align*}
\tan^2x+1=\sec^2x\quad \implies\quad \tan^2x=\sec^2x-1
\end{align*}
یوں درج ذیل ہو گا۔
\begin{align*}
\int (\sec x+\tan x)^2\dif x&=\int(\sec^2x+2\sec x\tan x+\sec^2x-1)\dif x\\
&=2\int \sec^2\dif x+2\int\sec x\tan x\dif x-\int 1\dif x\\
&=2\tan x+2\sec x-x+C
\end{align*}
\انتہا{مثال}
%=================
\ابتدا{مثال}\ترچھا{جذر سے چھٹکارا}\\
تکمل \عددی{\int_0^{\pi/4}\sqrt{1+\cos 4x}\dif x} حل کریں۔

حل:\quad
ہم تماثل
\begin{align*}
\cos^2\theta=\frac{1+\cos 2\theta}{2}\quad \implies 1+\cos 2\theta=2\cos^2\theta
\end{align*}
میں \عددی{\theta=2x} پر کر کے 
\begin{align*}
1+\cos 4x=2\cos^22x
\end{align*}
لکھتے ہیں۔یوں درج ذیل ہو گا جہاں تیسرے قدم پر وقفہ \عددی{[0,\tfrac{\pi}{4}]} پر \عددی{\cos 2x\ge 0} کی بنا \عددی{\abs{\cos 2x}=\cos 2x} ہو گا۔
\begin{align*}
\int_0^{\pi/4}\sqrt{1+\cos 4x}\dif x&=\int_0^{\pi/4}\sqrt{2}\sqrt{\cos^22x}\dif x\\
&=\sqrt{2}\int_0^{\pi/4}\abs{\cos 2x}\dif x&&\sqrt{u^2}=\abs{u}\\
&=\sqrt{2}\int_0^{\pi/4}\cos 2x\dif x\\
&=\sqrt{2}\left.\frac{\sin 2x}{2}\right\vert_0^{\pi/4}\\
&=\sqrt{2}\big[\frac{1}{2}-0\big]=\frac{\sqrt{2}}{2}
\end{align*}
\انتہا{مثال}
%================
\ابتدا{مثال}\ترچھا{غیر مناسب کسر کی مناسب کسر میں تبدیلی}\\
تکمل \عددی{\int\tfrac{3x^2-7x}{3x+2}\dif x} حل کریں۔

حل:\quad 
متکمل غیر مناسب کسر (نسب نما کی طاقت، شمار کنندہ کی طاقت سے زیادہ یا اس کے برابر ہے)  ہے۔ اس کا تکمل لینے سے پہلے ہم پہلے تقسیم کر کے حاصل تقسیم اور باقی حاصل کرتے ہیں جو مناسب کسر ہو گا:
\begin{align*}
\frac{3x^2-7x}{3x+2}=x-3+\frac{6}{3x+2}
\end{align*}
یوں درج ذیل ہو گا۔
\begin{align*}
\int\frac{3x^2-7x}{3x+2}\dif x=\int\big(x-3+\frac{6}{3x+2}\big)\dif x=\frac{x^2}{2}-3x+2\ln\abs{3x+2}+C
\end{align*}
\انتہا{مثال}
%=================
یہ ضروری نہیں ہے کہ غیر مناسب کسر کو بذریعہ تقسیم مناسب  کسر میں تبدیل کرنے سے  ہمیں ایسا تکمل حاصل ہو جسے ہم سیدھا تکمل کر سکیں۔ ایسی صورت پر حصہ میں غور کیا جائے گا۔

\ابتدا{مثال}\ترچھا{ایک کسر کی علیحدگی}\\
تکمل \عددی{\int\tfrac{3x+2}{\sqrt{1-x^2}}\dif x} حل کریں۔

حل:\quad
ہم متکمل کو دو علیحدہ کسر لکھتے ہیں۔
\begin{align*}
\int\frac{3x+2}{\sqrt{1-x^2}}\dif x=3\int\frac{x\dif x}{\sqrt{1-x^2}}+2\int\frac{\dif x}{\sqrt{1-x^2}}
\end{align*}
بائیں ہاتھ پہلے نئے تکمل میں ہم \عددی{u=1-x^2}، \عددی{\dif u=-2x\dif x} اور \عددی{x\dif x=-\frac{1}{2}\dif u} پر کرتے ہیں۔
\begin{align*}
3\int\frac{x\dif x}{\sqrt{1-x^2}}&=3\int\frac{(-1/2)\dif u}{\sqrt{u}}=-\frac{3}{2}\int u^{-1/2}\dif u\\
&=-\frac{3}{2}\cdot \frac{u^{1/2}}{1/2}+C_1=-3\sqrt{1-x^2}+C_1
\end{align*}
دوسرا نیا تکمل معیاری روپ میں ہے لہٰذا
\begin{align*}
2\int\frac{\dif x}{\sqrt{1-x^2}}=2\sin^{-1}x+C_2
\end{align*} 
ہو گا۔یوں پورا تکمل درج ذیل ہو گا جہاں \عددی{C_1+C_2=C} لکھا گیا ہے۔
\begin{align*}
\int\frac{3x+2}{\sqrt{1-x^2}}\dif x=-3\sqrt{1-x^2}+2\sin^{-1}x+C
\end{align*}

\انتہا{مثال}
%================
\ابتدا{مثال}\شناخت{مثال_طریقے_سیکنٹ_کلیہ_تکمل}\ترچھا{اکائی (\عددی{1}) کی ایک روپ سے ضرب}\\
تکمل \عددی{\int\sec x\dif x} حل کریں۔

حل:
\begin{align*}
\int\sec x\dif x&=\int (\sec x)(1)\dif x\\
&=\int \sec x \cdot \frac{\sec x+\tan x}{\sec x+\tan x}\dif x\\
&=\int \frac{\sec^2x+\sec x\tan x}{\sec x+\tan x}\dif x\\
&=\int\frac{\dif u}{u}&&u=\tan x+\sec x\\
&=\ln \abs{u}+C=\ln\abs{\sec x+\tan x}+C
\end{align*}
\انتہا{مثال}
%===================
\begin{table}
\caption{سیکنٹ اور کوسیکنٹ کے کلیات تکمل}
\label{جدول_طریقے_سیکنٹ_کوسیکنٹ_کلیات}
\centering
\renewcommand{\arraystretch}{2}
\begin{tabular}{L|L}
\toprule
\text{شمار}&\text{کلیہ}\\
\midrule
1&\int \sec u\dif u=\ln \abs{\sec u+\tan u}+C\\
2&\int\csc u\dif u=-\ln\abs{\csc u+\cot u}+C\\
\bottomrule
\end{tabular}
\end{table}

ہم مثال \حوالہ{مثال_طریقے_سیکنٹ_کلیہ_تکمل} کی ترکیب استعمال کرتے ہوئے سیکنٹ اور ٹینجنٹ کی جگہ کوسیکنٹ اور کوٹینجنٹ لیتے ہوئے   کوسیکنٹ کے تکمل کا کلیہ معلوم کر سکتے ہیں (سوال \حوالہ{سوال_طریقے_کوسیکنٹ_تکمل})۔

\موٹا{تکمل کو بنیادی کلیہ کی روپ میں لکھنے کا طریقے}
\begin{align*}
\text{مثال}&&\text{طریقہ}\\
\hline
\frac{2x-9}{\sqrt{x^2-9x+1}}\dif x&=\frac{\dif u}{u}&&\text{\RL{سادہ روپ بذریعہ بدل}}\\
\sqrt{8x-x^2}\dif x&=\sqrt{16-(x-4)^2}&&\text{\RL{تکمیل مربع}}\\
(\sec x+\tan x)^2&=\sec^2x+2\sec x\tan x+\tan^2x&&\text{\RL{تکونیاتی تماثل}}\\
&=\sec^2x+2\sec x\tan x+(\sec^2x-1)\\
&=2\sec^2x+2\sec x\tan x-1\\
\sqrt{1+\cos 4x}&=\sqrt{2\cos^22x}=\sqrt{2}\abs{\cos 2x}&&\text{\RL{جذر سے چھٹکارا}}\\
\frac{3x^2-7x}{3x+2}&=x-3+\frac{6}{3x+2}&&\text{\RL{غیر مناسب سے مناسب کسر کا حصول}}\\
\frac{3x+2}{\sqrt{1-x^2}}&=\frac{3x}{\sqrt{1-x^2}}+\frac{2}{\sqrt{1-x^2}}&&\text{\RL{کسر کی علیحدگی}}\\
\sec x&=\sec x\cdot \frac{\sec x+\tan x}{\sec x+\tan x}&&\text{\RL{اکائی ($1$) کی ایک روپ سے ضرب}}\\
&=\frac{\sec^2x+\sec x\tan x}{\sec x+\tan x}
\end{align*}


%====================================
\حصہء{سوالات}
\موٹا{بنیادی بدل}\\
سوال \حوالہ{سوال_طریقے_بدل_معیاری_روپ_حصول_الف} تا سوال \حوالہ{سوال_طریقے_بدل_معیاری_روپ_حصول_ب} میں بدل کی استعمال سے معیاری روپ حاصل کر کے تکمل حل کریں۔ 

\ابتدا{سوال}\شناخت{سوال_طریقے_بدل_معیاری_روپ_حصول_الف}
$\int\frac{16x\dif x}{\sqrt{8x^2+1}}$\\
جواب:\quad
$2\sqrt{8x^2+1}+C$
\انتہا{سوال}
%======================
\ابتدا{سوال}
$\int\frac{3\cos x\dif x}{\sqrt{1+3\sin x}}$
\انتہا{سوال}
%======================
\ابتدا{سوال}
$\int3\sqrt{\sin v}\cos v\dif v$\\
جواب:\quad
$2(\sin v)^{3/2}+C$
\انتہا{سوال}
%======================
\ابتدا{سوال}
$\int\cot^3y\csc^2y\dif y$
\انتہا{سوال}
%======================
\ابتدا{سوال}
$\int_0^1\frac{16x\dif x}{8x^2+2}$\\
جواب:\quad
$\ln 5$
\انتہا{سوال}
%======================
\ابتدا{سوال}
$\int_{\pi/4}^{\pi/3}\frac{\sec^2z}{\tan z}\dif z$
\انتہا{سوال}
%======================
\ابتدا{سوال}
$\int\frac{\dif x}{\sqrt{x}(\sqrt{x}+1)}$\\
جواب:\quad
$2\ln(\sqrt{x}+1)+C$
\انتہا{سوال}
%======================
\ابتدا{سوال}
$\int\frac{\dif x}{x-\sqrt{x}}$
\انتہا{سوال}
%======================
\ابتدا{سوال}
$\int\cot(3-7x)\dif x$\\
جواب:\quad
$-\tfrac{1}{7}\ln\abs{\sin(3-7x)}+C$
\انتہا{سوال}
%======================
\ابتدا{سوال}
$\int\csc(\pi x-1)\dif x$
\انتہا{سوال}
%======================
\ابتدا{سوال}
$\int e^{\theta}\csc(e^{\theta}+1)\dif \theta$\\
جواب:\quad
$-\ln\abs{\csc(e^{\theta}+1)+\cot(e^{\theta}+1)}+C$
\انتہا{سوال}
%======================
\ابتدا{سوال}
$\int\frac{\cot(3+\ln x)}{x}\dif x$
\انتہا{سوال}
%======================
\ابتدا{سوال}
$\int \sec\frac{t}{3}\dif t$\\
جواب:\quad
$3\ln\abs{\sec\tfrac{t}{3}+\tan\tfrac{t}{3}}+C$
\انتہا{سوال}
%======================
\ابتدا{سوال}
$\int x\sec(x^2-5)\dif x$
\انتہا{سوال}
%======================
\ابتدا{سوال}
$\int\csc(s-\pi)\dif s$\\
جواب:\quad
$-\ln\abs{\csc(s-\pi)+\cot(s-\pi)}+C$
\انتہا{سوال}
%======================
\ابتدا{سوال}
$\int \frac{1}{\theta^2}\csc\frac{1}{\theta}\dif \theta$
\انتہا{سوال}
%======================
\ابتدا{سوال}
$\int_0^{\sqrt{\ln 2}}2xe^{x^2}\dif x$\\
جواب:\quad
$1$
\انتہا{سوال}
%======================
\ابتدا{سوال}
$\int_{\pi/2}^{\pi}\sin(y)e^{\cos y}\dif y$
\انتہا{سوال}
%======================
\ابتدا{سوال}
$\int e^{\tan v}\sec^2v\dif v$\\
جواب:\quad
$e^{\tan v}+C$
\انتہا{سوال}
%======================
\ابتدا{سوال}
$\int\frac{e^{\sqrt{t}}}{\sqrt{t}}\dif t$
\انتہا{سوال}
%======================
\ابتدا{سوال}
$\int 3^{x+1}\dif x$\\
جواب:\quad
$\tfrac{e^{x+1}}{\ln 3}+C$
\انتہا{سوال}
%======================
\ابتدا{سوال}
$\int\frac{2^{\ln x}}{x}\dif x$
\انتہا{سوال}
%======================
\ابتدا{سوال}
$\int\frac{2^{\sqrt{w}}}{2\sqrt{w}}\dif w$\\
جواب:\quad
$\tfrac{2^{\sqrt{w}}}{\ln 2}+C$
\انتہا{سوال}
%======================
\ابتدا{سوال}
$\int 10^{2\theta}\dif \theta$
\انتہا{سوال}
%======================
\ابتدا{سوال}
$\int\frac{9\dif u}{1+9u^2}$\\
جواب:\quad
$3\tan^{-1}3u+C$
\انتہا{سوال}
%======================
\ابتدا{سوال}
$\int\frac{4\dif x}{1+(2x+1)^2}$
\انتہا{سوال}
%======================
\ابتدا{سوال}
$\int_0^{1/6}\frac{\dif x}{\sqrt{1-9x^2}}$\\
جواب:\quad
$\tfrac{\pi}{18}$
\انتہا{سوال}
%======================
\ابتدا{سوال}
$\int_0^1\frac{\dif t}{\sqrt{4-t^2}}$
\انتہا{سوال}
%======================
\ابتدا{سوال}
$\int\frac{2s\dif s}{\sqrt{1-s^4}}$\\
جواب:\quad
$\sin^{-1}s^2+C$
\انتہا{سوال}
%======================
\ابتدا{سوال}
$\int\frac{2\dif x}{x\sqrt{1-4\ln^2 x}}$
\انتہا{سوال}
%======================
\ابتدا{سوال}
$\int\frac{6\dif x}{x\sqrt{25x^2-1}}$\\
جواب:\quad
$6\sec^{-1}\abs{5x}+C$
\انتہا{سوال}
%======================
\ابتدا{سوال}
$\int\frac{\dif r}{r\sqrt{r^2-9}}$
\انتہا{سوال}
%======================
\ابتدا{سوال}
$\int\frac{\dif x}{e^x+e^{-x}}$\\
جواب:\quad
$\tan^{-1}e^x+C$
\انتہا{سوال}
%======================
\ابتدا{سوال}
$\int\frac{\dif y}{\sqrt{e^{2y}-1}}$
\انتہا{سوال}
%======================
\ابتدا{سوال}
$\int_{1}^{e^{\pi/3}}\frac{\dif x}{x\cos(\ln x)}$\\
جواب:\quad
$\ln(2+\sqrt{3})$
\انتہا{سوال}
%======================
\ابتدا{سوال}\شناخت{سوال_طریقے_بدل_معیاری_روپ_حصول_ب}
$\int\frac{\ln x\dif x}{x+4x\ln^2x}$
\انتہا{سوال}
%======================
\موٹا{تکمیل مربع}\\
سوال \حوالہ{سوال_طریقے_تکمیل_مربع_بدل_الف} تا سوال \حوالہ{سوال_طریقے_تکمیل_مربع_بدل_ب} میں مربع مکمل کر کے اور بدل استعمال کرتے ہوئے  معیاری روپ حاصل کر کے تکمل حل کریں۔

\ابتدا{سوال}\شناخت{سوال_طریقے_تکمیل_مربع_بدل_الف}
$\int_1^2\frac{8\dif x}{x^2-2x+2}$\\
جواب:\quad
$2\pi$
\انتہا{سوال}
%=========================
\ابتدا{سوال}
$\int_2^4\frac{2\dif x}{x^2-6x+10}$
\انتہا{سوال}
%=========================
\ابتدا{سوال}
$\int\frac{\dif t}{\sqrt{-t^2+4t-3}}$\\
جواب:\quad
$\sin^{-1}(t-2)+C$
\انتہا{سوال}
%=========================
\ابتدا{سوال}
$\int\frac{\dif\theta}{\sqrt{2\theta-\theta^2}}$
\انتہا{سوال}
%=========================
\ابتدا{سوال}
$\int\frac{\dif x}{(x+1)\sqrt{x^2+2x}}$\\
جواب:\quad
$\sec^{-1}\abs{x+1}+C\,\text{تب}\, \abs{x+1}>1\, \text{جب}$
\انتہا{سوال}
%=========================
\ابتدا{سوال}\شناخت{سوال_طریقے_تکمیل_مربع_بدل_ب}
$\int\frac{\dif x}{(x-2)\sqrt{x^2-4x+3}}$
\انتہا{سوال}
%=========================
\موٹا{تکونیاتی تماثل}\\
سوال \حوالہ{سوال_طریقے_تماثل_بدل_الف} تا سوال \حوالہ{سوال_طریقے_تماثل_بدل_ب} میں تکونیاتی تماثل اور بدل استعمال کرتے ہوئے معیاری روپ حاصل کر کے تکمل حل کریں۔

\ابتدا{سوال}\شناخت{سوال_طریقے_تماثل_بدل_الف}
$\int(\sec x+\cot x)^2\dif x$\\
جواب:\quad
$\tan x-2\ln \abs{\csc x+\cot x}-\cot x-x+C$
\انتہا{سوال}
%==========================
\ابتدا{سوال}
$\int(\csc x-\tan x)^2\dif x$
\انتہا{سوال}
%==========================
\ابتدا{سوال}
$\int \csc x\sin 3x\dif x$\\
جواب:\quad
$x+\sin 2x+C$
\انتہا{سوال}
%==========================
\ابتدا{سوال}\شناخت{سوال_طریقے_تماثل_بدل_ب}
$\int(\sin 3x\cos 2x-\cos 3x\sin 2x)\dif x$
\انتہا{سوال}
%==========================
\موٹا{غیر مناسب کسر}\\
سوال \حوالہ{سوال_طریقے_غیر_مناسب_بدل_الف} تا سوال \حوالہ{سوال_طریقے_غیر_مناسب_بدل_ب} میں غیر مناسب کسر سے مناسب کسر کے حصول اور بدل کے ذریعہ معیاری روپ حاصل کر کے تکمل حل کریں۔

\ابتدا{سوال}\شناخت{سوال_طریقے_غیر_مناسب_بدل_الف}
$\int\frac{x}{x+1}\dif x$\\
جواب:\quad
$x-\ln\abs{x+1}+C$
\انتہا{سوال}
%=======================
\ابتدا{سوال}
$\int\frac{x^2}{x^2+1}\dif x$
\انتہا{سوال}
%=======================
\ابتدا{سوال}
$\int_{\sqrt{2}}^{3}\frac{2x^3}{x^2-1}\dif x$\\
جواب:\quad
$7+\ln 8$
\انتہا{سوال}
%=======================
\ابتدا{سوال}
$\int_{-1}^{3}\frac{4x^2-7}{2x+3}\dif x$
\انتہا{سوال}
%=======================
\ابتدا{سوال}
$\int\frac{4t^3-t^2+16t}{t^2+4}\dif t$\\
جواب:\quad
$2t^2-t+2\tan^{-1}(\tfrac{t}{2})+C$
\انتہا{سوال}
%=======================
\ابتدا{سوال}\شناخت{سوال_طریقے_غیر_مناسب_بدل_ب}
$\int\frac{2\theta^3-7\theta^2+7\theta}{2\theta-5}\dif \theta$
\انتہا{سوال}
%=======================
\موٹا{کسر کی علیحدگی}\\
سوال \حوالہ{سوال_طریقے_کسر_علیحدہ_بدل_الف} تا سوال \حوالہ{سوال_طریقے_کسر_علیحدہ_بدل_ب} میں کسر علیحدہ کر کے بدل کے ذریعہ معیاری روپ حاصل کر کے تکمل حل کریں۔

\ابتدا{سوال}\شناخت{سوال_طریقے_کسر_علیحدہ_بدل_الف}
$\int\frac{1-x}{\sqrt{1-x^2}}\dif x$\\
جواب:\quad
$\sin^{-1}x+\sqrt{1-x^2}+C$
\انتہا{سوال}
%=====================
\ابتدا{سوال}
$\int\frac{x+2\sqrt{x-1}}{2x\sqrt{x-1}}\dif x$
\انتہا{سوال}
%=====================
\ابتدا{سوال}
$\int_0^{\pi/4}\frac{1+\sin x}{\cos^2x}\dif x$\\
جواب:\quad
$\sqrt{2}$
\انتہا{سوال}
%=====================
\ابتدا{سوال}\شناخت{سوال_طریقے_کسر_علیحدہ_بدل_ب}
$\int_0^{1/2}\frac{2-8x}{1+4x^2}\dif x$
\انتہا{سوال}
%=====================
\موٹا{اکائی \عددی{(1)} کی ایک روپ سے ضرب}\\
سوال \حوالہ{سوال_طریقے_اکائی_بدل_الف} تا سوال \حوالہ{سوال_طریقے_اکائی_بدل_ب} میں اکائی کی ایک روپ سے ضرب اور بدل کے ذریعہ معیاری روپ حاصل کر کے تکمل حل کریں۔

\ابتدا{سوال}\شناخت{سوال_طریقے_اکائی_بدل_الف}
$\int\frac{1}{1+\sin x}\dif x$\\
جواب:\quad
$\tan x-\sec x+C$
\انتہا{سوال}
%===================
\ابتدا{سوال}
$\int\frac{1}{1+\cos x}\dif x$
\انتہا{سوال}
%===================
\ابتدا{سوال}
$\int\frac{1}{\sec\theta+\tan\theta}\dif \theta$\\
جواب:\quad
$\ln\abs{1+\sin\theta}+C$
\انتہا{سوال}
%===================
\ابتدا{سوال}
$\int\frac{1}{\csc\theta+\cot\theta}\dif \theta$
\انتہا{سوال}
%===================
\ابتدا{سوال}
$\int\frac{1}{1-\sec x}\dif x$\\
جواب:\quad
$\cot x+x+\csc x+C$
\انتہا{سوال}
%===================
\ابتدا{سوال}\شناخت{سوال_طریقے_اکائی_بدل_ب}
$\int\frac{1}{1-\csc x}\dif x$
\انتہا{سوال}
%===================
\موٹا{جذر سے چھٹکارا}\\
سوال \حوالہ{سوال_طریقے_جذر_چھٹکارا_الف} تا سوال \حوالہ{سوال_طریقے_جذر_چھٹکارا_ب} میں جذر سے چھٹکارے کے بعد تکمل حل کریں۔

\ابتدا{سوال}\شناخت{سوال_طریقے_جذر_چھٹکارا_الف}
$\int_{0}^{2\pi}\sqrt{\frac{1-\cos x}{2}}\dif x$\\
جواب:\quad
$4$
\انتہا{سوال}
%=======================
\ابتدا{سوال}
$\int_0^{\pi}\sqrt{1-\cos 2x}\dif x$
\انتہا{سوال}
%=======================
\ابتدا{سوال}
$\int_{\pi/2}^{\pi}\sqrt{1+\cos 2t}\dif t$\\
جواب:\quad
$\sqrt{2}$
\انتہا{سوال}
%=======================
\ابتدا{سوال}
$\int_{-\pi}^0\sqrt{1+\cos t}\dif t$
\انتہا{سوال}
%=======================
\ابتدا{سوال}
$\int_{-\pi}^0\sqrt{1-\cos^2\theta}\dif \theta$\\
جواب:\quad
$2$
\انتہا{سوال}
%=======================
\ابتدا{سوال}
$\int_{\pi/2}^{\pi}\sqrt{1-\sin^2\theta}\dif\theta$
\انتہا{سوال}
%=======================
\ابتدا{سوال}
$\int_{-\pi/4}^{\pi/4}\sqrt{1+\tan^2y}\dif y$\\
جواب:\quad
$\ln\abs{\sqrt{2}+1}-\ln\abs{\sqrt{2}-1}$
\انتہا{سوال}
%=======================
\ابتدا{سوال}\شناخت{سوال_طریقے_جذر_چھٹکارا_ب}
$\int_{-\pi/4}^{0}\sqrt{\sec^2y-1}\dif y$
\انتہا{سوال}
%=======================
\موٹا{مختلف قسم کے تکمل}\\
سوال \حوالہ{سوال_طریقے_موزوں_طریقہ_الف} تا سوال \حوالہ{سوال_طریقے_موزوں_طریقہ_ب} میں کوئی بھی موزوں طریقہ استعمال کرتے ہوئے تکمل حل کریں۔

\ابتدا{سوال}\شناخت{سوال_طریقے_موزوں_طریقہ_الف}
$\int_{\pi/4}^{3\pi/4}(\csc x-\cot x)^2\dif x$\\
جواب:\quad
$4-\tfrac{\pi}{2}$
\انتہا{سوال}
%=====================
\ابتدا{سوال}
$\int_0^{\pi/4}(\sec x+4\cos x)^2\dif x$
\انتہا{سوال}
%=====================
\ابتدا{سوال}
$\int\cos\theta\csc(\sin\theta)\dif\theta$\\
جواب:\quad
$-\ln\abs{\csc(\sin\theta)+\cot(\sin\theta)}+C$
\انتہا{سوال}
%=====================
\ابتدا{سوال}
$\int(1+\tfrac{1}{x})\cot(x+\ln x)\dif x$
\انتہا{سوال}
%=====================
\ابتدا{سوال}
$\int(\csc x -\sec x)(\sin x+\cos x)\dif x$\\
جواب:\quad
$\ln\abs{\sin x}+\ln\abs{\cos x}+C$
\انتہا{سوال}
%=====================
\ابتدا{سوال}
$\int(\csc x+\sec x)(\tan x+\cot x)\dif x$
\انتہا{سوال}
%=====================
\ابتدا{سوال}
$\int\frac{6\dif y}{\sqrt{y}(1+y)}$\\
جواب:\quad
$12\tan^{-1}(\sqrt{y})+C$
\انتہا{سوال}
%=====================
\ابتدا{سوال}
$\int\frac{\dif x}{x\sqrt{4x^2-1}}$
\انتہا{سوال}
%=====================
\ابتدا{سوال}
$\int\frac{7\dif x}{(x-1)\sqrt{x^2-2x-48}}$\\
جواب:\quad
$\sec^{-1}\abs{\tfrac{x-1}{7}}+C$
\انتہا{سوال}
%=====================
\ابتدا{سوال}
$\int\frac{\dif x}{(2x+1)\sqrt{4x^2+4x}}$
\انتہا{سوال}
%=====================
\ابتدا{سوال}
$\int \sec^2t\tan(\tan t)\dif t$\\
جواب:\quad
$\ln\abs{\sec(\tan t)}+C$
\انتہا{سوال}
%=====================
\ابتدا{سوال}\شناخت{سوال_طریقے_موزوں_طریقہ_ب}
$\int\frac{\tan \theta}{2\sec\theta+1}$
\انتہا{سوال}
%=====================
\موٹا{تکونیاتی طاقت}\\
\ابتدا{سوال}
\begin{enumerate}[a.]
\item
حل کریں: \عددی{\int\cos^3\theta\dif\theta} (اشارہ: \عددی{\cos^2\theta=1-\sin^2\theta})
\item
حل کریں:  \عددی{\int\cos^5\theta\dif\theta}
\item
بغیر حل کیے بتائیں کہ آپ \عددی{\int\cos^9\theta\dif \theta} کو کس طرح حل کریں گے۔
\end{enumerate}
جواب:\quad 
(ا) \عددی{\sin\theta-\tfrac{1}{3}\sin^3\theta+C} (ب) \عددی{\sin\theta-\tfrac{2}{3}\sin^3\theta+\tfrac{1}{5}\sin^5\theta+C}\\ (ج) \عددی{\int\cos^9\theta\dif \theta=\int\cos^8\theta(\cos\theta)\dif\theta=\int(1-\sin^2\theta)^4(\cos\theta)\dif\theta}
\انتہا{سوال}
%===================
\ابتدا{سوال}
\begin{enumerate}[a.]
\item
حل کریں: \عددی{\int\sin^3\theta\dif \theta} (اشارہ: \عددی{\sin^2\theta=1-\cos^2\theta}) 
\item
حل کریں \عددی{\int\sin^5\theta\dif\theta}
\item
حل کریں: \عددی{\int\sin^7\theta\dif\theta}
\item
بغیر حل کیے بتائیں آپ \عددی{\int\sin^{13}\theta\dif\theta} کو کس طرح حل کریں گے۔
\end{enumerate}
\انتہا{سوال}
%===============
\ابتدا{سوال}
\begin{enumerate}[a.]
\item
\عددی{\int\tan^3\theta\dif\theta} کو \عددی{\int\tan\theta\dif\theta} کی صورت میں لکھ کر حل کریں۔ (اشارہ: \عددی{\tan^2\theta=\sec^2\theta-1})
\item
\عددی{\int\tan^5\theta\dif\theta} کو \عددی{\int\tan\theta^3\dif\theta} کی صورت میں لکھیں۔
\item
\عددی{\int\tan^7\theta\dif\theta} کو \عددی{\int\tan\theta^5\dif\theta} کی صورت میں لکھیں۔
\item
\عددی{\int\tan^{2k+1}\theta\dif\theta} کو \عددی{\int\tan^{2k-1}\theta\dif\theta} کی صورت میں لکھیں جہاں \عددی{k} مثبت عدد صحیح ہے
\end{enumerate}
جواب:\quad
\begin{enumerate}[a.]
\item
$\int\tan^3\theta\dif\theta=\tfrac{1}{2}\tan^2\theta-\int\tan\theta\dif\theta=\tfrac{1}{2}\tan^2\theta+\ln\abs{\cos \theta}+C$
\item
$\int\tan^5\theta\dif\theta=\tfrac{1}{4}\tan^4\theta\dif\theta-\int\tan^3\theta\dif\theta$
\item
$\int\tan^7\theta\dif\theta=\tfrac{1}{6}\tan^6\theta-\int\tan^5\theta\dif\theta$
\item
$\int\tan^{2k+1}\theta\dif\theta=\tfrac{1}{2k}\tan^{2k}\theta-\int\tan^{2k-1}\theta\dif\theta$
\end{enumerate}
\انتہا{سوال}
%===============
\ابتدا{سوال}
\begin{enumerate}[a.]
\item
\عددی{\int\cot^3\theta\dif\theta} کو \عددی{\int\cot\theta\dif\theta} کی صورت میں لکھ کر حل کریں۔ (اشارہ: \عددی{\cot^2\theta=\sec^2\theta-1})
\item
\عددی{\int\cot^5\theta\dif\theta} کو \عددی{\int\cot\theta^3\dif\theta} کی صورت میں لکھیں۔
\item
\عددی{\int\cot^7\theta\dif\theta} کو \عددی{\int\cot\theta^5\dif\theta} کی صورت میں لکھیں۔
\item
\عددی{\int\cot^{2k+1}\theta\dif\theta} کو \عددی{\int\cot^{2k-1}\theta\dif\theta} کی صورت میں لکھیں جہاں \عددی{k} مثبت عدد صحیح ہے
\end{enumerate}
\انتہا{سوال}
%===============
\موٹا{نظریہ اور استعمال}\\
\ابتدا{سوال}\شناخت{سوال_طریقے_خطہ_بیچ_دو_منحنیات_الف}
بالائی جانب \عددی{y=2\cos x} اور زیریں جانب \عددی{y=\sec x,\, -\tfrac{\pi}{4}\le x\le \tfrac{\pi}{4}} میں گھیرے ہوئے خطے کا رقبہ تلاش کریں۔\\
جواب:\quad
$2\sqrt{2}-\ln(3+2\sqrt{2})$
\انتہا{سوال}
%==============
\ابتدا{سوال}\شناخت{سوال_طریقے_خطہ_بیچ_تین_منحنیات_ب}
ایک تکونی خطہ کا بالائی سرحد \عددی{y=\csc x}، نچلا سرحد \عددی{y=\sin x,\,\tfrac{\pi}{6}\le x\le \tfrac{\pi}{2}} اور بایاں سرحد \عددی{x=\tfrac{\pi}{6}} ہیں۔ اس خطہ کا رقبہ معلوم کریں۔
\انتہا{سوال}
%======================
\ابتدا{سوال}
محور \عددی{x} کے گرد سوال \حوالہ{سوال_طریقے_خطہ_بیچ_دو_منحنیات_الف} کا خطہ گھما کر جسم طواف پیدا کیا جاتا ہے۔ اس جسم کا حجم تلاش کریں۔\\
جواب:\quad
$\pi^2$
\انتہا{سوال}
%===============
\ابتدا{سوال}
محور \عددی{x} کے گرد سوال \حوالہ{سوال_طریقے_خطہ_بیچ_تین_منحنیات_ب} کا خطہ گھما کر جسم طواف پیدا کیا جاتا ہے۔ اس جسم کا حجم تلاش کریں۔
\انتہا{سوال}
%===============
\ابتدا{سوال}
منحنی \عددی{y=\ln(\cos x),\,0\le x\le \tfrac{\pi}{3}} کی لمبائی معلوم کریں۔\\
جواب:\quad
$\ln(2+\sqrt{3})$
\انتہا{سوال}
%=======================
\ابتدا{سوال}
منحنی \عددی{y=\ln(\sec x),\,0\le x\le \tfrac{\pi}{4}} کی لمبائی معلوم کریں۔
\انتہا{سوال}
%=======================
\ابتدا{سوال}
محور \عددی{x}، قوس \عددی{y=\sec x}، لکیر \عددی{x=-\tfrac{\pi}{4}} اور \عددی{x=\tfrac{\pi}{4}} کے بیچ خطہ کا وسطانی مرکز تلاش کریں۔\\
جواب:\quad
$\bar{x}=0,\,\bar{y}=\tfrac{1}{\ln(2\sqrt{2}+3)}$
\انتہا{سوال}
%===================
\ابتدا{سوال}
محور \عددی{x}، قوس \عددی{y=\csc x}، لکیر \عددی{x=\tfrac{\pi}{6}} اور \عددی{x=\tfrac{5\pi}{6}} کے بیچ خطہ کا وسطانی مرکز تلاش کریں۔
\انتہا{سوال}
%===================
\ابتدا{سوال}\شناخت{سوال_طریقے_کوسیکنٹ_تکمل}\ترچھا{تفاعل \عددی{\csc x} کا تکمل}\\
سیکنٹ اور ٹینجنٹ کی جگہ کوسیکنٹ اور کوٹینجنٹ استعمال کرتے ہوئے مثال \حوالہ{مثال_طریقے_سیکنٹ_کلیہ_تکمل} کی طرز پر درج ذیل حاصل کریں۔
\begin{align*}
\int\csc x\dif x=-\ln\abs{\csc x+\cot x}+C
\end{align*}
\انتہا{سوال}
%====================
\ابتدا{سوال}
دکھائیں کہ تکمل
\begin{align*}
\int((x^2-1)(x+1))^{-2/3}\dif x
\end{align*}
کو درج ذیل تمام طریقوں سے حاصل کیا جا سکتا ہے۔
\begin{multicols}{3}
\begin{enumerate}[a.]
\item
$u=\tfrac{1}{x+1}$
\item
$u=\tan^{-1}x$
\item
$u=\tan^{-1}\sqrt{x}$
\item
$u=\tan^{-1}(\tfrac{x-1}{2})$
\item
$u=\cos^{-1}x$
\item
$u=\cosh^{-1}x$
\item
$u=(\tfrac{x-1}{x+1})^k$
\end{enumerate}
\end{multicols}
جہاں جزو-ز میں \عددی{k=1,\tfrac{1}{2},\tfrac{1}{3},-\tfrac{1}{3},-\tfrac{2}{3},-1} ہو سکتا ہے۔


\انتہا{سوال}
%====================
