\حصہ{تکملات بالکثرت میں بدل}
اس حصہ میں بارہا تکمل کی قیمت کا حصول بذریعہ بدل سکھایا جائے گا۔ واحد تکمل کی طرح یہاں بھی پیچیدہ تکمل کو سادہ تکمل سے بدلا  جاتا  ہے۔ بدل سے متکمل  یا تکمل کی حدوں یا ان دونوں  کی سادہ  روپ استعمال کی جاتی ہے۔


\جزوحصہء{دوہرا تکملات  میں بدل} 


ہم قطبی محدد د کی بدل کا استعمال حصہ \حوالہ{حصہ_بالکثرت_دوہرا_تکملات_قطبی_روپ} میں  دیکھ چکے ہیں جو دہرا تکملات کی بدل، جس میں متغیرات کی تبدیلی کو خطے کی تبدیلی تصور کیا جاتا ہے، کی ایک  مخصوص شکل ہے۔ 


فرض کریں مستوی \عددی{uv} کے خطہ \عددی{G} کو ایک ایک مطابقت کے ساتھ مساوات
\[x=g(u,v),\quad y=h(u,v)\]
 کے ذریعہ مستوی \عددی{xy} کے خطہ \عددی{R} میں بدلا جاتا ہے۔ ہم \عددی{R} کو اس بدل میں \عددی{G} کا \اصطلاح{عکس}\فرہنگ{عکس}\حاشیہب{image}\فرہنگ{image} اور \عددی{G} کو \عددی{R} کا  \اصطلاح{قبل عکس}\فرہنگ{عکس!قبل}\حاشیہب{preimage}\فرہنگ{image!pre} کہتے ہیں۔خطہ \عددی{R} کسی بھی تفاعل \عددی{f(x,y)} کو خطہ \عددی{G} میں معین تفاعل \عددی{f(g(u,v),h(u,v))} بھی تصور کیا جا سکتا ہے۔ خطہ \عددی{R} میں \عددی{f(x,y)} کے تکمل کا خطہ \عددی{G} میں \عددی{f(g(u,v),h(u,v))} کے تکمل کے ساتھ کیا تعلق ہو گا؟


اس کا جواب: اگر \عددی{g}، \عددی{h} اور \عددی{f} کے جزوی تفرقات استمراری ہوں اور \عددی{J(u,v)} (جس پر جلد تبصرہ کیا جائے گا) صرف  تنہا نقطوں پر صفر ہو (اگر صفر ہو بھی)  تب درج ذیل ہو گا۔ 
\begin{align}\label{مساوات_بالکثرت_یعقوبی_بدل}
\iint\limits_{R} f(x,y)\dif x\dif y=\iint\limits_G f(g(u,v),h(u,v))\abs{J(u,v)}\dif u\dif v
\end{align}
مذکورہ بالا مساوات میں \عددی{J(u,v)}، جو \ترچھا{ یعقوبی} کہلاتا ہے، کی مطلق قیمت استعمال کی گئی۔


\ابتدا{تعریف}
  \اصطلاح{یعقوبی مقطع}\فرہنگ{یہ ریاضی دان کارل   گستاف یعقوب یعقوبی کے نام سے منسوب ہے۔} یا محددی بدل \عددی{x=g(u,v)}، \عددی{y=h(u,v)} کے \اصطلاح{یعقوبی}\فرہنگ{یعقوبی}\حاشیہب{Jacobian}\فرہنگ{Jacobian} سے مراد درج ذیل ہے:
\begin{align}\label{مساوات_بالکثرت_یعقوبی_تعریف}
\renewcommand{\arraystretch}{1.5}
J(u,v)=\begin{vmatrix}
\frac{\partial x}{\partial u}&\frac{\partial x}{\partial v}\\
\frac{\partial y}{\partial u}&\frac{\partial y}{\partial v}
\end{vmatrix}=
\frac{\partial x}{\partial u}\frac{\partial y}{\partial v}-\frac{\partial y}{\partial u}\frac{\partial x}{\partial v}
\end{align}
\انتہا{تعریف}


یعقوبی کو 
\begin{align*}
J(u,v)=\frac{\partial(x,y)}{\partial(u,v)}
\end{align*}
سے بھی ظاہر کیا جاتا ہے جو ہمیں یاد دلاتا ہے کہ \عددی{x} اور \عددی{y} کی جزوی تفرقات سے یعقوبی (مساوات \حوالہ{مساوات_بالکثرت_یعقوبی_تعریف})  حاصل ہوتا ہے۔مساوات \حوالہ{مساوات_بالکثرت_یعقوبی_بدل} کی استخراج آپ کو اعلٰی احصاء کے نصاب میں ملے گی جس کو یہاں پیش نہیں کیا جائے گا۔






قطبی محدد میں میں \عددی{u} اور \عددی{v} کی جگہ \عددی{r} اور \عددی{\theta} ہوں گے لہٰذا \عددی{x=r\cos\theta} اور \عددی{y=r\sin\theta} لیتے ہوئے یعقوبی
\begin{align*}
\renewcommand{\arraystretch}{1.5}
J(r,\theta)=\begin{vmatrix}
\frac{\partial x}{\partial r}&\frac{\partial x}{\partial \theta}\\
\frac{\partial y}{\partial r}&\frac{\partial y}{\partial \theta}
\end{vmatrix}=\begin{vmatrix}
\cos\theta&-r\sin\theta\\
\sin\theta&r\cos\theta
\end{vmatrix}=
r(\cos^2\theta+\sin^2\theta)=r
\end{align*}
ہو گا اور مساوات \حوالہ{مساوات_بالکثرت_یعقوبی_بدل} درج ذیل صورت اختیار کرے گی جو حصہ  \حوالہ{حصہ_بالکثرت_دوہرا_تکملات_قطبی_روپ}کی مساوات \حوالہ{مساوات_بالکثرت_قطبی_محدد_میں_رقبہ_کی_عمومی} ہے۔
\begin{gather}
\begin{aligned}\label{مساوات_بالکثرت_قطبی_مستطیل_تکمل}
\iint\limits_R f(x,y)\dif x\dif y&=\iint\limits_G f(r\cos\theta,r\sin\theta)\abs{r}\dif r \dif \theta\\
&=\iint\limits_G f(r\cos\theta,r\sin\theta)r\dif r\dif\theta&&\text{\RL{اگر \عددی{r\ge 0} ہو}}
\end{aligned}
\end{gather}


شکل میں دکھایا گیا ہے کہ کس طرح  مستطیل \عددی{G: 0\le r\le 1,\,0\le \theta\le \pi/2} کو مساوات \عددی{x=r\cos\theta} اور \عددی{y=r\sin\theta}  ایک چوتھائی دائرہ \عددی{R}، جس کی سرحد  ربع اول میں مستوی \عددی{xy} پر \عددی{x^2+y^2=1} ہے، میں بدلتے ہیں۔

دھیان رہے کہ مساوات \حوالہ{مساوات_بالکثرت_قطبی_مستطیل_تکمل} کی دائیں ہاتھ میں قطبی محددی مستوی میں کسی خطہ پر \عددی{f(r\cos\theta,r\sin\theta)} کا تکمل نہیں  بلکہ  کارتیسی \عددی{r,\theta} مستوی کے خطہ \عددی{G} میں   \عددی{f(r\cos\theta,r\sin\theta)} اور \عددی{r} کے حاصل ضرب کا تکمل ہے۔

آئیں بدل کی دوسری مثال دیکھیں۔

 \ابتدا{مثال}
  مستوی  \عددی{uv} میں موزوں خطہ پر   بدل  
\begin{align}\label{مساوات_بالکثرت_بدل_مثال}
u=\frac{2x-y}{2},\quad v=\frac{y}{2}
\end{align}
 استعمال کرتے  ہوئے درج ذیل  تکمل کی قیمت تلاش کریں۔
\begin{align*}
\int_0^4\int_{x=y/2}^{x=y/2+1}\frac{2x-y}{2}\dif x\dif y
\end{align*}
حل:\quad
ہم مستوی  \عددی{xy} میں تکمل کے خطے کا خاکہ بنا کر اس کی سرحدوں کی نشاندہی کرتے ہیں۔

مساوات \حوالہ{مساوات_بالکثرت_یعقوبی_بدل} استعمال کرنے کی خاطر ہمیں مستوی \عددی{uv} میں مطابقتی خطہ \عددی{G} اور بدل کا یعقوبی  معلوم کرنے ہوں  گے۔ انہیں دریافت کرنے کے لئے  ہم مساوات \حوالہ{مساوات_بالکثرت_بدل_مثال} کو \عددی{x} اور \عددی{y} کے لئے \عددی{u}  اور \عددی{v} کی صورت میں حل    کرتے ہیں۔یوں درج ذیل حاصل ہو گا۔
\begin{align}\label{مساوات_بالکثرت_یو_وی_سرحد}
x=u+v,\quad y=2v
\end{align}
اس کے بعد ہم  \عددی{R} کی سرحدوں کی مساوات میں انہیں پر کر کے \عددی{G} کی سرحدیں دریافت کرتے ہیں۔
\begin{center}
\renewcommand{\arraystretch}{1.5}
\begin{tabular}{LCL}
\toprule
\begin{minipage}{2cm} \text{\RL{خطہ \عددی{R} کی سرحد}}\\  \text{\RL{کی \عددی{xy} مساواتیں}}  \end{minipage} & 
\begin{minipage}{2cm}\text{\RL{خطہ \عددی{G} کی مطابقتی}}\\ \text{\RL{سرحد کی  \عددی{uv} مساواتیں}} \end{minipage}&
\begin{minipage}{2cm} \text{\RL{\عددی{uv} مساواتوں}}\\  \text{\RL{کی سادہ صورت}} \end{minipage}\\
\midrule
x=\frac{y}{2}& u+v=\frac{2v}{2}=v& u=0\\
x=\frac{y}{2}+1& u+v=\frac{2v}{2}+1=v+1&u=1\\
y=0&2v=0&v=0\\
y=4&2v=4&v=2\\
\bottomrule
\end{tabular}
\end{center}
بدل کا یعقوبی  (مساوات \حوالہ{مساوات_بالکثرت_یو_وی_سرحد} سے) درج ذیل ہو گا۔
\begin{align*}
\renewcommand{\arraystretch}{1.5}
J(u,v)=\begin{vmatrix}
\frac{\partial x}{\partial u}&\frac{\partial x}{\partial v}\\
\frac{\partial y}{\partial u}&\frac{\partial y}{\partial v}
\end{vmatrix}=\begin{vmatrix}
\frac{\partial}{\partial u}(u+v)&\frac{\partial}{\partial v}(u+v)\\
\frac{\partial}{\partial u}(2v)&\frac{\partial}{\partial v}(2v)
\end{vmatrix}=\begin{vmatrix}
1&1\\
0&2
\end{vmatrix}=2
\end{align*}
ہم اب مساوات \حوالہ{مساوات_بالکثرت_یعقوبی_بدل} استعمال کرنے کی تمام معلومات جانتے ہیں:
\begin{align*}
\int_0^4\int_{x=y/2}^{x=y/2+1}\frac{2x-y}{2}\dif x\dif y&=\int_{v=0}^{v=2}\int_{u=0}^{u=1} u\abs{J(u,v)}\dif u\dif v\\
&=\int_0^2\int_0^1 (u)(2)\dif u\dif v=\int_0^2\big[u^2\big]_{0}^{1}\dif v=\int_0^2\dif v=2
\end{align*}
\انتہا{مثال}
%======================
\ابتدا{مثال}
درج ذیل تکمل کی قیمت تلاش کریں۔
\begin{align*}
\int_0^1\int_0^{1-x}\sqrt{x+y}(y-2x)^2\dif y\dif x
\end{align*}
حل:\quad
ہم  مستوی{xy} میں تکمل کے خطہ \عددی{R} کا خاکہ بنا  کر اس کی سرحدوں کی نشاندہی کرتے ہیں۔متکمل کو دیکھ کر ہمیں خیال آتا ہے کہ بدل \عددی{u=x+y} اور \عددی{v=y-2x}استعمال کیا جائے جنہیں \عددی{u} اور \عددی{v} کی صورت میں \عددی{x} اور \عددی{y} کے لئے حل کرتے ہوئے درج ذیل حاصل ہو گا۔
\begin{align}\label{مساوات_بالکثرت_سرحدیں_دوبارہ}
x=\frac{u}{3}-\frac{v}{3},\quad y=\frac{2u}{3}+\frac{v}{3}
\end{align}
ہم مساوات \حوالہ{مساوات_بالکثرت_سرحدیں_دوبارہ} سے مستوی \عددی{uv} میں خطہ \عددی{G} کی سرحدیں معلوم کرتے ہیں۔
\begin{center}
\renewcommand{\arraystretch}{1.5}
\begin{tabular}{LCL}
\toprule
\begin{minipage}{2cm}\text{\RL{\عددی{R} کی سرحد}}\\  \text{\RL{کی \عددی{xy} مساواتیں}}  \end{minipage}&
\begin{minipage}{2cm}\text{\RL{\عددی{G} کی مطابقتی سرحد}}\\  \text{\RL{کی \عددی{uv} مساواتیں}}  \end{minipage}&
\begin{minipage}{2cm}\text{\RL{\عددی{uv}مساواتوں}}\\  \text{\RL{کی سادہ صورت}}  \end{minipage}\\
\midrule
x+y=1&\big(\frac{u}{3}-\frac{v}{3}\big)+\big(\frac{2u}{3}+\frac{v}{3}\big)=1&u=1\\
x=0&\frac{u}{3}-\frac{v}{3}=0&v=u\\
y=0&\frac{2u}{3}+\frac{v}{3}=0&v=-2u\\
\bottomrule
\end{tabular}
\end{center}
مساوات \حوالہ{مساوات_بالکثرت_سرحدیں_دوبارہ} میں دیے بدل کا یعقوبی درج ذیل ہو گا۔
\begin{align*}
\renewcommand{\arraystretch}{1.5}
J(u,v)=\begin{vmatrix}
\frac{\partial x}{\partial u}&\frac{\partial x}{\partial v}\\
\frac{\partial y}{\partial u}&\frac{\partial y}{\partial v}
\end{vmatrix}=\begin{vmatrix}
\frac{1}{3} &-\frac{1}{3}\\
\frac{2}{3}&\frac{1}{3}
\end{vmatrix}=\frac{1}{3}
\end{align*}
 ہم مساوات \حوالہ{مساوات_بالکثرت_یعقوبی_بدل} سے تکمل کی قیمت حاصل کرتے ہیں:
\begin{align*}
\int_0^1\int_0^{1-x}&\sqrt{x+y}(y-2x)^2\dif y\dif x=\int_{u=0}^{u=1}\int_{v=-2u}^{v=u} u^{1/2}\,v^2\abs{J(u,v)}\dif v\dif u\\
&=\int_0^1\int_{-2u}^{u}u^{1/2}\,v^2\big(\frac{1}{3}\big)\dif v\dif u=\frac{1}{3}\int_0^1 u^{1/2}\big(\frac{1}{3}v^3\big)_{v=-2u}^{v=u}\dif u\\
&=\frac{1}{9}\int_0^1 u^{1/2}(u^3+8u^3)\dif u=\int_{0}^{1} u^{7/2}\dif u=\left. \frac{2}{9}u^{9/2}\right\vert_{0}^{1}=\frac{2}{9}
\end{align*}
\انتہا{مثال}
%======================

\جزوحصہء{تہرا تکملات میں بدل}
تہرا تکملات کے متغیرات کی تبدیلی کو   تین بعدی خطہ کا   بدل  تصور کرنے والے ترکیب کی خصوصی صورتیں  نلکی اور کروی محددی بدل ہیں۔ یہ ترکیب دوہرا  تکملات کی ترکیب کی طرح ہے، بس اب ہم دو کی بجائے تین بعد میں کام  کرتے ہیں۔ 

فرض کریں \عددی{uvw} فضا میں خطہ \عددی{G} کو  ایک ایک مطابقت کے  ساتھ \عددی{xyz} فضا کے خطہ \عددی{D} میں درج ذیل  روپ کی مساواتوں  سے بدلا جاتا ہے۔
\begin{align*}
x=g(u,v,w),\quad y=h(u,v,w),\quad z=k(u,v,w)
\end{align*}
تب \عددی{D} میں کسی بھی تفاعل \عددی{F(x,y,z)} کو \عددی{G} میں   معین  تفاعل
\begin{align*}
F(g(u,v,w),h(u,v,w),k(u,v,w))=H(u,v,w)
\end{align*}
تصور کیا جا سکتا ہے۔اگر \عددی{g}، \عددی{h}  اور \عددی{k} کے اول جزوی تفرقات استمراری ہوں تب \عددی{D} پر \عددی{F} کے تکمل کا \عددی{G} پر \عددی{H(u,v,w)} کے تکمل کے ساتھ تعلق درج ذیل مساوات دیگی۔
  \begin{align}\label{مساوات_بالکثرت_یعقوبی_تہرا_تکمل}
\iiint\limits_D F(x,y,z)\dif x\dif y\dif z=\iiint\limits_G H(u,v,w)\abs{J(u,v,w)}\dif u\dif v\dif w
\end{align}
اس مساوات میں \عددی{J(u,v,w)} کی مطلق قیمت استعمال کی گئی ہے جو درج ذیل \اصطلاح{یعقوبی مقطع}\فرہنگ{یعقوبی!مقطع}\حاشیہب{Jacobian determinant}\فرہنگ{Jacobian!determinant} ہے۔
\begin{align}
\renewcommand{\arraystretch}{1.5}
J(u,v,w)=\begin{vmatrix}
\frac{\partial x}{\partial u}&\frac{\partial x}{\partial v}&\frac{\partial x}{\partial w}\\
\frac{\partial y}{\partial u}&\frac{\partial y}{\partial v}&\frac{\partial y}{\partial w}\\
\frac{\partial z}{\partial u}&\frac{\partial z}{\partial v}&\frac{\partial z}{\partial w}
\end{vmatrix}=\frac{\partial(x,y,z)}{\partial(u,v,w)}
\end{align} 
متغیرات کی تبدیلی کا    کلیہ، جس کو مساوات \حوالہ{مساوات_بالکثرت_یعقوبی_تہرا_تکمل} میں پیش کیا گیا ہے، پیچیدہ ہے اور دو بعدی صورت کی طرح ، اس کی اشتقاق کو یہاں پیش نہیں کیا جائے گا۔

نلکی محدد میں \عددی{u}، \عددی{v} اور \عددی{w}  کی جگہ \عددی{\rho}، \عددی{\phi} اور \عددی{z} ہوں گے۔ کارتیسی \عددی{\rho\phi z} فضا سے کارتیسی \عددی{xyz} فضا میں بدل درج ذیل مساوات دیں گی۔
\begin{align*}
x=\rho\cos\phi,\quad y=\rho\sin\phi,\quad z=z
\end{align*}   
اس بدل کا یعقوبی
\begin{align*}
\renewcommand{\arraystretch}{1.5}
J(\rho,\phi,z)=&\begin{vmatrix}
\frac{\partial x}{\partial \rho}&\frac{\partial x}{\partial \phi}&\frac{\partial x}{\partial z}\\
\frac{\partial y}{\partial \rho}&\frac{\partial y}{\partial \phi}&\frac{\partial y}{\partial z}\\
\frac{\partial z}{\partial \rho}&\frac{\partial z}{\partial \phi}&\frac{\partial z}{\partial z}
\end{vmatrix}=\begin{vmatrix} 
\cos\phi&-\rho\sin\phi&0\\
\sin\phi&\rho\cos\phi&0\\
0&0&1
 \end{vmatrix}\\
&=\rho\cos^2\phi+\rho\sin^2\phi=\rho
\end{align*}
ہو گا۔یوں مساوات \حوالہ{مساوات_بالکثرت_یعقوبی_تہرا_تکمل} درج ذیل صورت  اختیار کریگی۔
 \begin{align}
\iiint\limits_D F(x,y,z)\dif x\dif y\dif z=\iiint\limits_G H(\rho,\phi,z)\abs{\rho}\dif \rho\dif \phi\dif z
\end{align}
    جب بھی \عددی{\rho\ge 0} ہو، ہم  مطلق کی علامت سے چھٹکارا حاصل کر سکتے ہیں۔ 


کروی  محدد میں \عددی{u}، \عددی{v} اور \عددی{w}  کی جگہ \عددی{r}، \عددی{\theta} اور \عددی{\phi} ہوں گے۔ کارتیسی \عددی{r \theta\phi} فضا سے کارتیسی \عددی{xyz} فضا میں بدل درج ذیل مساوات دیں گی۔
\begin{align*}
x=r\sin\theta\cos\phi,\quad y=r\sin\theta\sin\phi,\quad z=r\sin\theta
\end{align*}   
اس بدل کا یعقوبی
\begin{align*}
\renewcommand{\arraystretch}{1.5}
J(\rho,\phi,z)=&\begin{vmatrix}
\frac{\partial x}{\partial r}&\frac{\partial x}{\partial \theta}&\frac{\partial x}{\partial \phi}\\
\frac{\partial y}{\partial r}&\frac{\partial y}{\partial \theta}&\frac{\partial y}{\partial \phi}\\
\frac{\partial z}{\partial r}&\frac{\partial z}{\partial \theta}&\frac{\partial z}{\partial \phi}
\end{vmatrix}=r^2\sin\theta
\end{align*}
ہو گا۔یوں مساوات \حوالہ{مساوات_بالکثرت_یعقوبی_تہرا_تکمل} درج ذیل صورت  اختیار کریگی۔
 \begin{align}
\iiint\limits_D F(x,y,z)\dif x\dif y\dif z=\iiint\limits_G H(r,\theta,\phi)\abs{r^2\sin\theta}\dif r\dif \theta\dif \phi
\end{align}
 کروی محدد میں \عددی{0\le \theta\le \pi} کی بنا   \عددی{\sin\theta} کبھی بھی منفی نہیں ہو  سکتا   ہے  لہٰذا مطلق کی علامت لکھنے کی ضرورت نہیں ہے۔ 

آئیں بدل کی ایک مثال دیکھتے ہیں۔

\ابتدا{مثال}
درج ذیل بدل 
\begin{align}\label{مساوات_بالکثرت_کروی_بدل}
u=(2x-y)/2,\quad v=y/2,\quad w=z/3
\end{align}
 استعمال  کرتے ہوئے \عددی{uvw}  فضا میں موزوں خطہ پر  تکمل لے کر درج ذیل تکمل کی قیمت دریافت کریں۔ 
\begin{align*}
\int_0^3\int_0^4\int_{x=y/2}^{x=y/2+1} \big(\frac{2x-y}{2}+\frac{z}{3}\big)\dif x\dif y\dif z
\end{align*}
حل:\quad
ہم \عددی{xyz} فضا میں تکمل کے خطہ \عددی{D} کا خاکہ بنا کر اس کی سرحدوں کی نشاندہی کرتے ہیں۔ یہاں سرحدی سطحیں مستویات ہیں۔

مساوات \حوالہ{مساوات_بالکثرت_یعقوبی_تہرا_تکمل}  استعمال کرنے کے لئے ہمیں  \عددی{uvw}  فضا میں  مطابقتی خطہ \عددی{G} اور بدل کا یعقوبی جاننا ہو گا۔ ہم مساوات \حوالہ{مساوات_بالکثرت_کروی_بدل} کو \عددی{x}، \عددی{y} اور \عددی{z} کے لئے \عددی{u}، \عددی{v} اور \عددی{w} کی صورت میں حل کر کے 
\begin{align}\label{مساوات_بالکثرت_نئے_متغیرات}
x=u+v,\quad y=2v,\quad z=3w
\end{align}
حاصل کرتے ہیں۔ ہم \عددی{D} کی سرحدوں کی مساوات میں  یہ قیمتیں پر کر کے  \عددی{G} کی سرحدوں کی مساواتیں دریافت کرتے ہیں:
\begin{center}
\begin{tabular}{LCL}
\toprule
\begin{minipage}{2cm}\text{\RL{\عددی{D} کی سرحدوں}}\\ \text{\RL{کی \عددی{xyz} مساواتیں}}  \end{minipage}&
\begin{minipage}{2cm}\text{\RL{\عددی{G} کی سرحدوں کی}}\\ \text{\RL{مطابقتی \عددی{uvw} مساواتیں}}  \end{minipage}&
\begin{minipage}{2cm}\text{\RL{\عددی{uvw} مساواتوں}}\\ \text{\RL{کی سادہ صورتیں}}  \end{minipage}\\
\midrule
x=y/2&u+v=2v/2=v&u=0\\
x=y/2+1&u+v=2v/2+1=v+1&u=1\\
y=0&2v=0&v=0\\
y=4&2v=4&v=2\\
z=0&3w=0&w=0\\
z=3&3w=3&w=1
\end{tabular}
\end{center}
ہم مساوات \حوالہ{مساوات_بالکثرت_نئے_متغیرات} استعمال کرتے ہوئے یعقوبی تلاش کرتے ہیں۔
\begin{align*}
\renewcommand{\arraystretch}{1.5}
J(u,v,w)=\begin{vmatrix}
\frac{\partial x}{\partial u} &\frac{\partial x}{\partial v}&\frac{\partial x}{\partial w}\\
\frac{\partial y}{\partial u}&\frac{\partial y}{\partial v}&\frac{\partial y}{\partial w}\\
\frac{\partial z}{\partial u}&\frac{\partial z}{\partial v}&\frac{\partial z}{\partial w}
\end{vmatrix}=\begin{vmatrix}
1&1&0\\
0&2&0\\
0&0&3
\end{vmatrix}=6
\end{align*}
ہم  مساوات \حوالہ{مساوات_بالکثرت_یعقوبی_تہرا_تکمل} استعمال کرنے کے لئے درکار تمام معلوم جان چکے ہیں۔ یوں درج ذیل ہو گا۔
\begin{align*}
\int_0^3\int_0^4\int_{x=y/2}^{x=y/2+1}&\big(\frac{2x-y}{2}+\frac{z}{3}\big)\dif x\dif y\dif z\\
&=\int_0^1\int_0^2\int_0^1(u+w)\abs{J(u,v,w)}\dif u\dif v\dif w\\
&=\int_0^1\int_0^2\int_0^1(u+w)(6)\dif u\dif v\dif w=6\int_0^1\int_0^2\big[\frac{u^2}{2}+uw\big]_0^1\dif v\dif w\\
&=6\int_0^1\int_0^2\big(\frac{1}{2}+w\big)\dif v\dif w=6\int_0^1\big[\frac{v}{2}+vw\big]_0^2\dif w=6\int_0^1(1+2w)\dif w\\
&=6\big[w+w^2\big]_0^1=6(2)=12
\end{align*}
\انتہا{مثال}
%===========================

\جزوحصہء{سوالات}

