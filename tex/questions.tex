\حصہء{سوالات}
\موٹا{حد کی تلاش}\\
سوال \حوالہ{سوال_ترتیب_مرتکز_منفرج_الف} تا سوال \حوالہ{سوال_ترتیب_مرتکز_منفرج_ب} میں کون سی ترتیب \عددی{\{a_n\}}  مرتکز اور کون سی منفرج ہے؟ ہر مرتکز ترتیب کا حد تلاش کریں۔

\ابتدا{سوال}\شناخت{سوال_ترتیب_مرتکز_منفرج_الف}
$a_n=2+(0.1)^n$
\انتہا{سوال}
%=====================
\ابتدا{سوال}
$a_n=\frac{n+(-1)^n}{n}$
\انتہا{سوال}
%====================
\ابتدا{سوال}
$a_n=\frac{1-2n}{1+2n}$
\انتہا{سوال}
%====================
\ابتدا{سوال}
$a_n=\frac{2n+1}{1-3\sqrt{n}}$
\انتہا{سوال}
%====================
\ابتدا{سوال}
$a_n=\frac{1-5n^4}{n^4+8n^3}$
\انتہا{سوال}
%====================
\ابتدا{سوال}
$a_n=\frac{n+3}{n62+5n+6}$
\انتہا{سوال}
%====================
\ابتدا{سوال}
$a_n=\frac{n62-2n+1}{n-1}$
\انتہا{سوال}
%====================
\ابتدا{سوال}
$a_n=\frac{1-n^3}{70-4n^2}$
\انتہا{سوال}
%====================
\ابتدا{سوال}
$a_n=1+(-1)^n$
\انتہا{سوال}
%====================
\ابتدا{سوال}
$a_n=(-1)^n(1-\tfrac{1}{n})$
\انتہا{سوال}
%====================
\ابتدا{سوال}
$a_n=(\tfrac{n+2}{2n})(1-\tfrac{1}{n})$
\انتہا{سوال}
%====================
\ابتدا{سوال}
$a_n=(2-\tfrac{1}{2^n})(3+\tfrac{1}{2^n})$
\انتہا{سوال}
%====================
\ابتدا{سوال}
$a_n=\frac{(-1)^{n+1}}{2n-1}$
\انتہا{سوال}
%====================
\ابتدا{سوال}
$a_n=(-\tfrac{1}{2})^n$
\انتہا{سوال}
%====================
\ابتدا{سوال}
$a_n=\sqrt{\frac{2n}{n+1}}$
\انتہا{سوال}
%====================
\ابتدا{سوال}
$a_n=\frac{1}{(0.9)^n}$
\انتہا{سوال}
%====================
\ابتدا{سوال}
$a_n=\sin(\tfrac{\pi}{2}+\tfrac{1}{n})$
\انتہا{سوال}
%====================
\ابتدا{سوال}
$a_n=n\pi \cos (n\pi)$
\انتہا{سوال}
%====================
\ابتدا{سوال}
$a_n=\frac{\sin n}{n}$
\انتہا{سوال}
%====================
\ابتدا{سوال}
$a_n=\frac{\sin^2 n}{2^n}$
\انتہا{سوال}
%====================
\ابتدا{سوال}
$a_n=\frac{n}{2^n}$
\انتہا{سوال}
%====================
\ابتدا{سوال}
$a_n=\frac{3^n}{n^3}$
\انتہا{سوال}
%====================
\ابتدا{سوال}
$a_n=\frac{\ln(n+1)}{\sqrt{n}}$
\انتہا{سوال}
%====================
\ابتدا{سوال}
$a_n=\frac{\ln n}{\ln 2n}$
\انتہا{سوال}
%====================
\ابتدا{سوال}
$a_n=8^{1/n}$
\انتہا{سوال}
%====================
\ابتدا{سوال}
$a_n=(0.03)^{1/n}$
\انتہا{سوال}
%====================
\ابتدا{سوال}
$a_n=(1+\tfrac{7}{n})^n$
\انتہا{سوال}
%====================
\ابتدا{سوال}
$a_n=(1-\tfrac{1}{n})^n$
\انتہا{سوال}
%====================
\ابتدا{سوال}
$a_n=\sqrt[n]{10n}$
\انتہا{سوال}
%====================
\ابتدا{سوال}
$a_n=\sqrt[n]{n^2}$
\انتہا{سوال}
%====================
\ابتدا{سوال}
$a_n=(\tfrac{3}{n})^{1/n}$
\انتہا{سوال}
%====================
\ابتدا{سوال}
$a_n=(n+4)^{1/(n+4)}$
\انتہا{سوال}
%====================
\ابتدا{سوال}
$a_n=\frac{\ln n}{n^{1/n}}$
\انتہا{سوال}
%====================
\ابتدا{سوال}
$a_n=\ln n-\ln(n+1)$
\انتہا{سوال}
%====================
\ابتدا{سوال}
$a_n=\sqrt[n]{4^nn}$
\انتہا{سوال}
%====================
\ابتدا{سوال}
$a_n=\sqrt[n]{3^{2n+1}}$
\انتہا{سوال}
%====================
\ابتدا{سوال}
$a_n=\frac{n!}{n^n}$\quad
(اشارہ: \عددی{\tfrac{1}{n}} کے ساتھ موازنہ کریں) 
\انتہا{سوال}
%====================
\ابتدا{سوال}
$a_n=\frac{(-4)^n}{n!}$
\انتہا{سوال}
%====================
\ابتدا{سوال}
$a_n=\frac{n!}{10^{6n}}$
\انتہا{سوال}
%====================
\ابتدا{سوال}
$a_n=\frac{n!}{2^n\cdot 3^n}$
\انتہا{سوال}
%====================
\ابتدا{سوال}
$a_n=(\tfrac{1}{n})^{1/(\ln n)}$
\انتہا{سوال}
%====================
\ابتدا{سوال}
$a_n=\ln(1+\tfrac{1}{n})^n$
\انتہا{سوال}
%====================
\ابتدا{سوال}
$a_n=(\tfrac{3n+1}{3n-1})^n$
\انتہا{سوال}
%====================
\ابتدا{سوال}
$a_n=(\tfrac{n}{n+1})^n$
\انتہا{سوال}
%====================
\ابتدا{سوال}
$a_n=(\tfrac{x^n}{2n+1})^{1/n},\quad x>0$
\انتہا{سوال}
%====================
\ابتدا{سوال}
$a_n=(1-\tfrac{1}{n^2})^n$
\انتہا{سوال}
%====================
\ابتدا{سوال}
$a_n=\frac{3^n\cdot 6^n}{2^{-n}\cdot n!}$
\انتہا{سوال}
%====================
\ابتدا{سوال}
$a_n=\frac{(10/{11})^n}{(9/{10})^n+(11/{12})^n}$
\انتہا{سوال}
%====================
\ابتدا{سوال}
$a_n=\tanh n$
\انتہا{سوال}
%====================
\ابتدا{سوال}
$a_n=\sinh (\ln n)$
\انتہا{سوال}
%====================
\ابتدا{سوال}
$a_n=\frac{n^2}{2n-1}\sin\frac{1}{n}$
\انتہا{سوال}
%====================
\ابتدا{سوال}
$a_n=n(1-\cos\frac{1}{n})$
\انتہا{سوال}
%====================
\ابتدا{سوال}
$a_n=\tan^{-1}n$
\انتہا{سوال}
%====================
\ابتدا{سوال}
$a_n=\frac{1}{\sqrt{n}}\tan^{-1}n$
\انتہا{سوال}
%====================
\ابتدا{سوال}
$a_n=(\tfrac{1}{3})^n+\frac{1}{\sqrt{2^n}}$
\انتہا{سوال}
%====================
\ابتدا{سوال}
$a_n=\sqrt[n]{n^2+n}$
\انتہا{سوال}
%====================
\ابتدا{سوال}
$a_n=\frac{(\ln n)^{200}}{n}$
\انتہا{سوال}
%====================
\ابتدا{سوال}
$a_n=\frac{(\ln n)^5}{\sqrt{n}}$
\انتہا{سوال}
%====================
\ابتدا{سوال}
$a_n=n-\sqrt{n^2-n}$
\انتہا{سوال}
%====================
\ابتدا{سوال}
$a_n=\frac{1}{\sqrt{n^2-1}-\sqrt{n^2+n}}$
\انتہا{سوال}
%====================
\ابتدا{سوال}
$a_n=n-\sqrt{n^2-n}$
\انتہا{سوال}
%====================
\ابتدا{سوال}
$a_n=\frac{1}{\sqrt{n^2-1}-\sqrt{n^2+n}}$
\انتہا{سوال}
%====================
\ابتدا{سوال}
$a_n=\frac{1}{n}\int_1^n \frac{\dif x}{x}$
\انتہا{سوال}
%====================
\ابتدا{سوال}\شناخت{سوال_ترتیب_مرتکز_منفرج_ب}
$a_n=\int_1^n\frac{\dif x}{x^p},\quad p>1$
\انتہا{سوال}
%====================
\موٹا{نظریہ اور مثالیں}
