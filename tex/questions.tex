\حصہ{نلکی اور  مربع سطحیں}
واحد متغیر کے تفاعل کی احصاء میں  ہم  نے خطوط سے شروع کیا اور خطوط کے بارے میں اپنا علم  استعمال کرتے ہوئے مستوی قوسین کا مطالعہ کیا۔ہم نے مماس پر غور کیا اور دیکھا کہ  کسی بھی قابل تفرق  منحنی    کے چھوٹے حصہ کو خطی تصور کیا جا سکتا ہے۔ خاص  اہمیت کے حامل منحنیات میں مخروطی قطعات، اور دو درجی منحنیات شامل ہیں جنہیں  متغیر \عددی{x} اور \عددی{y} کے دو درجی مساوات سے ظاہر کیا جا سکتا ہے۔

ایک سے زائد متغیرات   کے تفاعل کی احصاء  کا مطالعہ کرنے کی خاطر ہم اسی طرح کی راہ پر چلتے ہیں۔ ہم  دو بعدی سطح سے شروع کر کے اس سطح کے بارے میں اپنا علم استعمال کر کر  فضا میں تین بعدی سطحوں پر غور کرتے ہیں۔ خاص اہمیت کے حامل سطحوں میں نلکیاں اور دو درجی سطحیں شامل ہیں جنہیں \عددی{x}، \عددی{y}، \عددی{z} کے دو درجی مساوات سے ظاہر کیا جا سکتا ہے۔ گزشتہ حصہ میں  دو بعدی سطحوں پر غور کیا گیا۔ اس حصہ میں ہم تین بعدی سطحوں پر غور کرتے ہیں۔

\جزوحصہء{نلکی}
\اصطلاح{نلکی}\فرہنگ{نلکی}\حاشیہب{cylinder}\فرہنگ{cylinder} سے مراد وہ سطح ہے   جو (ا) ان تمام لکیروں پر مشتمل ہو جو فضا میں کسی دی گئی  لکیر کے متوازی ہوں اور (ب)  جو دی گئی    مستوی منحنی سے گزرتے ہوں۔  اس منحنی کو نلکی کی  \اصطلاح{پیداکار منحنی}\فرہنگ{پیدا کار!منحنی}\حاشیہب{generating curve}\فرہنگ{generating!curve} کہتے ہیں۔ ٹھوس جیومیٹری میں   جہاں \ترچھا{نلکی} سے مراد \ترچھا{  دائری   نلکی} ہوتی ہے، پیدا کار منحنی ایک دائرہ ہو گی، لیکن یہاں ہم کسی بھی قسم کی پیداکار منحنی کی اجازت دیں گے۔ ہماری (درج ذیل) پہلی مثال میں  نلکی کو قطع مکافی پیدا کرتا ہے۔ 
 
نلکی یا دیگر تین بعدی سطحوں کو ترسیم کرتے ہوئے یا قلم و کاغذ سے  ان کا خاکہ بناتے ہوئے   ان سطحوں کا محددی سطحوں  کے متوازی سطحوں کے ساتھ خط تقاطع کو دیکھنا مفید ثابت ہوتا ہے۔ ان  منحنیات کو  \اصطلاح{عمودی تراش}\فرہنگ{عمودی!تراش}\حاشیہب{cross section}\فرہنگ{cross section}  کہتے ہیں۔

\ابتدا{مثال}\شناخت{مثال_سمتیہ_نلکی_قطع_مکافی}\ترچھا{قطع مکافی نلکی \عددی{y=x^2}}\\
محور \عددی{z} کے متوازی لکیروں سے حاصل اس نلکی کی مساوات تلاش کریں جو قطع مکافی \عددی{y=x^2,\,z=0} سے گزرتی ہیں۔

حل:\quad
فرض کریں مستوی \عددی{xy} میں  قطع مکافی \عددی{y=x^2} پر نقطہ \عددی{N_0(x_0,x_0^2,0)} پایا جاتا ہے۔ تب کسی بھی \عددی{z} کے لئے  چونکہ نقطہ \عددی{Q(x_0,x_0^2,z)} محور \عددی{z} کے متوازی  لکیر  \عددی{x=x_0,\,y=x_0^2}،جو نقطہ \عددی{N_0} سے گزرتی ہے،    پر پایا جائے گا  لہٰذا  \عددی{Q} اس نلکی پر پایا جائے گا۔

اس طرح  \عددی{z} کی قیمت سے قطع نظر  اس سطح پر پائے جانے والے  تمام نقاط  مساوات \عددی{y=x^2} کو مطمئن کریں گے۔ یوں \عددی{y=x^2} اس نلکی  کی مساوات ہو گی۔اس کی بنا ہم اس نلکی کو "نلکی \عددی{y=x^2}" کہتے ہیں۔
\انتہا{مثال}
%============
ہم مثال \حوالہ{مثال_سمتیہ_نلکی_قطع_مکافی} سے دیکھ سکتے ہیں کہ مستوی  \عددی{xy} میں کوئی بھی منحنی \عددی{f(x,y)=c}  محور \عددی{z} کے متوازی نلکی   دے گی  اور اس نلکی کی  مساوات  \عددی{f(x,y)=c} ہو گی۔ مساوات \عددی{x^2+y^2=1} ایک قائمہ نلکی بیان کرتی ہے جو  محور \عددی{z} کے متوازی  ان  لکیروں پر مشتمل ہے جو مستوی \عددی{xy} میں دائرہ \عددی{x^2+y^2=1} سے گزرتے ہیں۔   مساوات \عددی{x^2+4y^2=9} ایک ترخیمی  نلکی بیان کرتی ہے جو  محور \عددی{z} کے متوازی  ان  لکیروں پر مشتمل ہے جو مستوی \عددی{xy} میں ترخیم \عددی{x^2+4y^2=9} سے گزرتے ہیں۔ 

اسی طرح مستوی \عددی{xz} میں کوئی بھی منحنی \عددی{g(x,z)=c} محور \عددی{y} کے متوازی ایک نلکی دیتی ہے جس کی مساوات \عددی{g(x,z)=c} ہو گی۔ کوئی بھی مساوات  \عددی{h(y,z)=c} محور \عددی{x} کے متوازی نلکی دیتی ہے اور اس نلکی کی  مساوات بھی  \عددی{h(y,z)=c} ہو گی۔

تین   کارتیسی محوروں میں  سے کسی بھی دو  محوروں پر مبنی  مساوات ایک نلکی دیتی ہے جو تیسری کارتیسی محور کے  متوازی ہو گی۔

\جزوحصہء{مربع سطحیں}
مربع سطح سے مراد فضا میں    \عددی{x}، \عددی{y} اور \عددی{z} کی دو درجی مساوات کی  ترسیم ہے جس  کی عمومی مساوات درج ذیل ہے
\begin{align*}
Ax^2+By^2+Cz^2+Dxy+Eyz+Fxz+Gx+Hy+Jz+K=0
\end{align*}
جہاں \عددی{A}، \عددی{B}، \عددی{C}، \عددی{D}، \عددی{E}، \عددی{F}، \عددی{G}، \عددی{H}، \عددی{J}  اور \عددی{K} مستقل ہیں۔ اس مساوات کی سادہ صورت  ، حصہ \حوالہ{حصہ_مخروط_دو_درجی_مساوات_گھمانا}  میں دو بعدی صورت کی طرح، گھمانے اور منتقلی سے حاصل کی جا سکتی ہے۔ مربع سطح کی مساوات میں ایک یا ایک سے زیادہ متغیرات کا مربع پایا جاتا ہے۔  ہم صرف سادہ مساوات پر غور کریں گے۔ اگرچہ نلکی کی تعریف   یہ نہیں کہتی ہے البتہ اشکال مربع سطحوں کی بھی مثالیں   ہیں۔ ہم اب  ترخیمی سطحوں (جن میں کرہ شامل ہے)،  قطع مکافی  سطحوں،  مخروطی  سطحوں اور قطع زائد سطحوں  پر غور کرتے ہیں۔

\ابتدا{مثال}
درج ذیل\اصطلاح{ ترخیمی  سطح}  محددی محوروں کو \عددی{(\mp a,0,0)}، \عددی{(0,\mp b,0)} اور \عددی{(0,0,\mp c)} پر مس کرتا ہے۔ یہ اس مستطیل ڈبہ \عددی{\abs{x}\le a} ،
  \عددی{\abs{y}\le b}،  \عددی{\abs{z}\le c} کے اندر پایا جاتا ہے۔چونکہ اس سطح کی تعریفی مساوات میں متغیرات کا مربع پایا جاتا ہے لہٰذا   یہ سطح تینوں محددی سطحوں کے لحاظ سے تشاکلی  ہو گا۔

تینوں محددی سطحوں کا اس سطح کے ساتھ منحنی تقاطع،   ترخیمات ہوں گی۔ مثال کے طور پر محددی مستوی \عددی{z=0}  اس سطح کو درج ذیل ترخیم پر قطع کرتا ہے۔
\begin{align*}
\frac{x^2}{a^2}+\frac{y^2}{b^2}&=1&&z=0
\end{align*}
سطح \عددی{z=z_0,\,\abs{z_0}<c}  اس سطح سے درج ذیل  ترخیمی حصہ کاٹتا ہے۔
\begin{align*}
\frac{x^2}{a^2(1-z_0^2/c^2)}+\frac{y^2}{b^2(1-z_0^2/c^2)}=1
\end{align*}
اگر نصف محور \عددی{a}، \عددی{b} اور \عددی{c} میں  کوئی دو ایک دوسرے کے برابر ہوں تب یہ\اصطلاح{    ترخیمی سطح    طواف} ہو گا ۔  اگر تینوں ایک دوسرے کے برابر ہوں تب یہ  سطح  کرہ ہو گا۔
\انتہا{مثال}
%===============

\موٹا{فنیات}\quad 
\ترچھا{فضا میں ذہنی تصویر کشی}\\
فضا میں سطحوں کی تصویر کشی کمپیوٹر کی مدد سے کی جا سکتی ہے۔ یہ مختلف دو بعدی سطحوں میں  لکیریں کھینچ سکتا ہے۔ کمپیوٹر اشکال کو فضا میں گھمانے کا  نظارہ پیش کر    سکتا ہے گویا   آپ  جسم کو ہاتھ میں گھما  رہے ہوں۔کمپیوٹر اس کا خیال رکھتا ہے کہ اجسام کا سامنے  حصہ نظر آئے جب کے اس کا پچھلا حصہ آنکھوں سے اوجھل رہے۔ عمومی طور پر کمپیوٹر کو    سطحوں کی مقدار معلوم    مساوات درکار ہوں گی۔

\ابتدا{مثال}
سطح \عددی{x=0} اور سطح \عددی{y=0} کے لحاظ سے \اصطلاح{ترخیمی قطع مکافی سطح}
\begin{align}\label{مساوات_ترخیمی_قطع_مکافی_سطح}
\frac{x^2}{a^2}+\frac{y^2}{b^2}=\frac{z}{c}
\end{align}
تشاکلی ہو گا۔ صرف مبدا پر محور ی  تقاطع پایا جاتا ہے۔ مستقل \عددی{c}  کی علامت  تعین کرتی ہے کہ یہ  مکمل  سطح \عددی{xy} سے نیچے یا اس سے اوپر پایا جائے گا۔ محددی سطح اس سے درج ذیل حصے کاٹتے ہیں۔
\begin{gather}
\begin{aligned}
x&=0:&& z=\frac{c}{b^2}y^2\,\,\text{قطع مکافی}\\
y&=0:&&z=\frac{c}{a^2}x^2\,\,\text{قطع مکافی}\\
z&=0:&&(0,0,0)\,\,\text{نقطہ}
\end{aligned}
\end{gather}
مستوی \عددی{xy} سے اوپر ہر سطح \عددی{z=z_0}  اسے  درج ذیل ترخیم میں کاٹتا ہے۔
\begin{align*}
\frac{x^2}{a^2}+\frac{y^2}{b^2}=\frac{z_0}{c}
\end{align*}
 \انتہا{مثال}
%=======================
\ابتدا{مثال}
\اصطلاح{دائری قطع مکافی سطح } یا \اصطلاح{قطع مکافی سطح طواف}
\begin{align}
\frac{x^2}{a^2}+\frac{y^2}{a^2}=\frac{z}{c}
\end{align}
کو مساوات \حوالہ{مساوات_ترخیمی_قطع_مکافی_سطح} میں \عددی{b=a} پر کر کے حاصل کیا جاتا ہے۔ محور \عددی{z}  کے عمودی سطحوں  کی عمودی تراش  سے دائرے حاصل ہوں گے جن کا  مرکز محور \عددی{z} پر  ہو گا۔ان سطحوں کی عمودی تراش جن میں   محور \عددی{z} پایا جاتا ہو،  مماثل قطع مکافی ہوں گی  جن کا مشترک  ماسکہ \عددی{(0,0,\tfrac{a^2}{4c})} ہو گا۔

دائری قطع مکافی سطحوں سے حصے تراش  کر بطور  ریڈیو دور بین،  مصنوعی سیارے  کے  تعاقب کار،  اور خورد امواج ریڈیو   کے اینٹینا  استعمال کئے جاتے ہیں۔
\انتہا{مثال}
%=================
\ابتدا{مثال}
\اصطلاح{ترخیمی  مخروط}
\begin{align}\label{مساوات_سمتیہ_ترخیمی_مخروط}
\frac{x^2}{a^2}+\frac{y^2}{b^2}=\frac{z^2}{c^2}
\end{align}
تینوں محددی سطحوں کے لحاظ سے تشاکلی ہے۔ محددی سطحیں اس سے درج ذیل حصے کاٹتے ہیں۔
\begin{align}
x&=0:&&z=\pm\frac{c}{b}y\,\,\text{خط}\label{مساوات_سمتیہ_ترخیمی_مخروط_الف}\\
y&=0:&&z=\pm \frac{c}{a}x\,\,\text{خط}\label{مساوات_سمتیہ_ترخیمی_مخروط_ب}\\
z&=0:&&(0,0,0)\,\,\text{نقطہ}\nonumber
\end{align}
مستوی \عددی{xy} سے اوپر اور اس سے نیچے سطحیں   \عددی{z=z_0}،   اس سے ترخیمات  کاٹتے ہیں جن کے مراکز محور \عددی{z} پر اور راس  مساوات \حوالہ{مساوات_سمتیہ_ترخیمی_مخروط_الف} اور مساوات \حوالہ{مساوات_سمتیہ_ترخیمی_مخروط_ب} میں دی گئی خطوط  پر پائے جاتے ہیں۔

اگر \عددی{a=b} ہو تب یہ   مخروط ایک قائمہ دائری مخروط ہو گا۔
\انتہا{مثال}
%===================
\ابتدا{مثال}
\اصطلاح{یک چادری قطع زائد سطح}
 \begin{align}\label{مساوات_سمتیہ_یک_چادری_قطع_زائد}
\frac{x^2}{a^2}+\frac{y^2}{b^2}-\frac{z^2}{c^2}=1
\end{align}
تینوں محددی سطحوں کے لحاظ سے تشاکلی ہو گا۔ محددی سطحیں اس سے درج ذیل حصے کاٹتے  ہیں۔
\begin{gather}
\begin{aligned}\label{مساوات_سمتیہ_قطع_مکافی_سطح}
x&=0:&&\frac{y^2}{b^2}-\frac{z^2}{c^2}=1\,\,\text{\RL{قطع زائد}}\\
y&=0:&&\frac{x^2}{a^2}-\frac{z^2}{c^2}=1\,\,\text{\RL{قطع زائد}}\\
z&=0:&&\frac{x^2}{a^2}+\frac{y^2}{b^2}=1\,\,\text{\RL{قطع زائد}}
\end{aligned}
\end{gather}
سطح \عددی{z=z_0} اس   کو ترخیم میں کاٹتا ہے جس کا مرکز محور \عددی{z} پر اور راسیں    مساوات \حوالہ{مساوات_سمتیہ_قطع_مکافی_سطح} میں دی گئی  قطع مکافی میں سے ایک پر پائی جاتی ہیں۔

یہ پوری سطح آپس میں جڑی ہوئی ہے یعنی اس سطح پر چل کر  کسی ایک نقطہ سے دوسرے نقطہ تک پہنچا جا سکتا ہے۔ اسی  لئے اس کو یک چادری قطع مکافی سطح کہتے ہیں۔ اگلی مثال  میں دو چادر  کی سطح پائی جاتی ہے۔

اگر \عددی{a=b} ہو تب  یہ قطع زائد سطح  ایک سطح طواف ہو گا۔
\انتہا{مثال}
%=====================
\ابتدا{مثال}
\اصطلاح{دو چادری قطع مکافی سطح }
\begin{align}\label{مساوات_سمتیہ_دو_چادری_قطع_زائد}
\frac{z^2}{c^2}-\frac{x^2}{a^2}-\frac{y^2}{b^2}=1
\end{align}
تینوں محددی سطحوں کے لحاظ سے تشاکلی ہے۔ سطح  \عددی{z=0} اس کو قطع  نہیں کرتا ہے۔ درحقیقت ایک افقی سطح اس صورت اس کو قطع کرتا ہے جب \عددی{\abs{z}\ge c} ہو۔ قطع زائد حصوں
 \begin{align*}
x&=0:\quad\frac{z^2}{c^2}-\frac{y^2}{b^2}=1,&& y=0:\quad \frac{z^2}{c^2}-\frac{x^2}{a^2}=1
\end{align*}
کے راس اور ماسکے محور \عددی{z} پر پائے  جاتے ہیں۔ یہ سطح دو حصوں میں تقسیم ہے۔ پہلا حصہ سطح \عددی{z=c} سے اوپر اور دوسرا حصہ سطح \عددی{z=-c} سے نیچے پایا جاتا ہے۔ اسی لئے اس کو دو چادری سطح کہتے ہیں۔

مساوات \حوالہ{مساوات_سمتیہ_یک_چادری_قطع_زائد} اور مساوات \حوالہ{مساوات_سمتیہ_دو_چادری_قطع_زائد} میں منفی اجزاء کی تعداد ایک جیسی نہیں ہے۔ دونوں  صورتوں میں منفی اجزاء کی تعداد اور چادروں کی تعداد ایک جیسی ہے۔   مساوات  \حوالہ{مساوات_سمتیہ_یک_چادری_قطع_زائد}یا  مساوات \حوالہ{مساوات_سمتیہ_دو_چادری_قطع_زائد} میں دائیں ہاتھ \عددی{1} کی جگہ  \عددی{0} پر کرنے سے ترخیمی مخروط کی مساوات
\begin{align*}
\frac{x^2}{a^2}+\frac{y^2}{b^2}=\frac{z^2}{c^2}
\end{align*}
حاصل ہوتی ہے (مساوات \حوالہ{مساوات_سمتیہ_ترخیمی_مخروط})۔ قطع زائد سطحیں اس مخروط کے متقارب ہیں۔ یہ بالکل ایسا ہی ہے جیسے قطع زائد
\begin{align*}
\frac{x^2}{a^2}-\frac{y^2}{b^2}=\mp 1
\end{align*}
مستوی \عددی{xy} میں  خط
\begin{align*}
\frac{x^2}{a^2}-\frac{y^2}{b62}=0
\end{align*}
کے متقارب ہیں۔
\انتہا{مثال}
%======================
