\حصہء{سوالات}
\موٹا{مقدار معلوم مساوات سے کارتیسی مساوات کا حصول}\\
سوال \حوالہ{سوال_مخروط_مقدار_معلوم_سے_کارتیسی_الف} تا سوال \حوالہ{سوال_مخروط_مقدار_معلوم_سے_کارتیسی_ب} میں \عددی{xy} مستوی میں ایک ذرہ کی حرکت کی مقدار معلوم مساوات دی گئی ہیں۔ اس ذرے کی راہ کی کارتیسی مساوات حاصل کرتے ہوئے راہ کو پہچانئے۔ کارتیسی مساوات کو ترسیم کرتے ہوئے اس پر ذرے کی راہ اور رخ دکھائیں۔ 

\ابتدا{سوال}\شناخت{سوال_مخروط_مقدار_معلوم_سے_کارتیسی_الف}
$x=\cos t,\quad y=\sin t,\quad 0\le t\le \pi$
\انتہا{سوال}
%========================
\ابتدا{سوال}
$x=\cos 2t,\quad y=\sin 2t,\quad 0\le t\le \pi$
\انتہا{سوال}
%========================
\ابتدا{سوال}
$x=\sin(2\pi(1-t)),\quad y=\cos(2\pi(1-t)),\quad 0\le t\le 1$
\انتہا{سوال}
%========================
\ابتدا{سوال}
$x=\cos(\pi-t),\quad y=\sin(\pi-t),\quad 0\le t\le \pi$
\انتہا{سوال}
%========================
\ابتدا{سوال}
$x=4\cos t,\quad y=2\sin t,\quad 0\le t\le 2\pi$
\انتہا{سوال}
%========================
\ابتدا{سوال}
$x=4\sin t,\quad y=2\cos t,\quad 0\le t\le \pi$
\انتہا{سوال}
%========================
\ابتدا{سوال}
$x=4\cos t,\quad y=5\sin t,\quad 0\le t\le \pi$
\انتہا{سوال}
%========================
\ابتدا{سوال}
$x=4\sin t,\quad y=5\cos t,\quad 0\le t\le 2\pi$
\انتہا{سوال}
%========================
\ابتدا{سوال}
$x=3t,\quad y=9t^2,\quad -\infty<t<\infty$
\انتہا{سوال}
%========================
\ابتدا{سوال}
$x=-\sqrt{t},\quad y=t,\quad t\ge 0$
\انتہا{سوال}
%========================
\ابتدا{سوال}
$x=t,\quad y=\sqrt{t},\quad t\ge 0$
\انتہا{سوال}
%========================
\ابتدا{سوال}
$x=\sec^2t-1,\quad y=\tan t,\quad -\tfrac{\pi}{2}<t<\tfrac{\pi}{2}$
\انتہا{سوال}
%========================
\ابتدا{سوال}
$x=-\sec t,\quad y=\tan t,\quad -\tfrac{\pi}{2}<t<\tfrac{\pi}{2}$
\انتہا{سوال}
%========================
\ابتدا{سوال}
$x=\csc t,\quad y=\cot t,\quad 0<t<\pi$
\انتہا{سوال}
%========================
\ابتدا{سوال}
$x=2t-5,\quad y=4t-7,\quad -\infty<t<\infty$
\انتہا{سوال}
%========================
\ابتدا{سوال}
$x=1-t,\quad y=1+t,\quad -\infty<t<\infty$
\انتہا{سوال}
%========================
\ابتدا{سوال}
$x=t,\quad y=1-t,\quad 0\le t\le 1$
\انتہا{سوال}
%========================
\ابتدا{سوال}
$x=3-3t,\quad y=2t,\quad 0\le t\le 1$
\انتہا{سوال}
%========================
\ابتدا{سوال}
$x=t,\quad y=\sqrt{1-t^2},\quad -t\le t\le 0$
\انتہا{سوال}
%========================
\ابتدا{سوال}
$x=t,\quad y=\sqrt{4-t^2},\quad 0\le t\le 2$
\انتہا{سوال}
%========================
\ابتدا{سوال}
$x=t^2,\quad y=\sqrt{t^4+1},\quad t\ge 0$
\انتہا{سوال}
%========================
\ابتدا{سوال}
$x=\sqrt{t+1},\quad y=\sqrt{t},\quad t\ge 0$
\انتہا{سوال}
%========================
\ابتدا{سوال}
$x=-\cosh t,\quad y=\sinh t,\quad -\infty<t<\infty$
\انتہا{سوال}
%========================
\ابتدا{سوال}\شناخت{سوال_مخروط_مقدار_معلوم_سے_کارتیسی_ب}
$x=2\sinh t,\quad y=2\cosh t,\quad -\infty<t<\infty$
\انتہا{سوال}
%========================

\موٹا{مقدار معلوم مساوات کا حصول}

\ابتدا{سوال}
ایک ذرہ \عددی{(a,0)} سے ابتدا کرتے ہوئے دائرہ \عددی{x^2+y^2=a^2}  پر (ا) ایک بار گھڑی کے رخ، (ب) ایک بار گھڑی کے الٹ رخ، (ج) دو بار گھڑی کے رخ یا (د) دو بار گھڑی کے الٹ رخ صفر کرتا ہے۔ ہر ایک صورت میں اس ذرے کی راہ کی مقدار معلوم مساوات اور حرکت کا وقفہ تلاش کریں۔ (اس کو حل کرنے کے کئی طریقے ہیں لہٰذا آپ کا  جواب دیے گئے جواب سے مختلف ہو سکتا ہے۔)
\انتہا{سوال}
%=======================
\ابتدا{سوال}
ایک ذرہ \عددی{(a,0)} سے ابتدا کرتے ہوئے ترخیم \عددی{\tfrac{x^2}{a^2}+\tfrac{y^2}{b^2}=1}  پر  (ا) ایک بار گھڑی کے رخ، (ب) ایک بار گھڑی کے الٹ رخ، (ج) دو بار گھڑی کے رخ یا (د) دو بار گھڑی کے الٹ رخ صفر کرتا ہے۔ ہر ایک صورت میں اس ذرے کی راہ کی مقدار معلوم مساوات اور حرکت کا وقفہ تلاش کریں۔ (اس کو حل کرنے کے کئی طریقے ہیں لہٰذا آپ کا  جواب دیے گئے جواب سے مختلف ہو سکتا ہے۔)
\انتہا{سوال}
%======================
\ابتدا{سوال}
درج ذیل نصف دائرے کی مقدار معلوم مساوات تلاش کریں۔ ایسا کرتے  ہوئے \عددی{(x,y)} پر منحنی کے مماس کی ڈھلوان \عددی{t=\tfrac{\dif y}{\dif x}} کو مقدار معلوم لیں۔
\begin{align*}
x^2+y^2=a^2,\quad y>0
\end{align*}
\انتہا{سوال}
%======================
\ابتدا{سوال}
دائرہ \عددی{x^2+y^2=a^2} پر نقطہ \عددی{(a,0)} سے نقطہ \عددی{(x,y)} تک گھڑی کے الٹ رخ  فاصلہ \عددی{s} کو مقدار معلوم لیتے ہوئے  اس دائرے کی مقدار معلوم مساوات حاصل کریں۔
\انتہا{سوال}
%====================
\ابتدا{سوال}\شناخت{سوال_مخروط_چڑیل}
ایک دائرہ جس کا رداس \عددی{1} اور مرکز \عددی{(0,1)} ہو کو شکل \حوالہ{شکل_سوال_مخروط_چڑیل} میں دکھایا گیا ہے۔ ساتھ ہی لکیر \عددی{y=2} بھی دکھائی گئی ہے۔ اس لکیر پر کوئی نقطہ \عددی{A} لیں اور اس کو مبدا \عددی{O} کے ساتھ سیدھی لکیر سے ملائیں۔ خط \عددی{OA} اکائی دائرہ کو نقطہ \عددی{N} پر قطع کرتا ہے۔ نقطہ \عددی{A} کو لکیر \عددی{y=2} پر چلانے سے نقطہ \عددی{N} جس راہ پر چلتا ہے اس کو مریا اگنیسی کی چڑیل کہتے ہیں۔ مریا اگنیسی کی چڑیل کی مقدار معلوم مساوات اور اس کا مقدار معلوم وقفہ تلاش کریں۔ قطع \عددی{OA} اور مثبت \عددی{x} محور کے بیچ زاویہ \عددی{t} کو مقدار معلوم لیں جہاں \عددی{t} کو ریڈیئن میں ناپا جاتا ہے۔درج ذیل مساوات فرض کرنے سے آپ کو مدد مل سکتی ہے۔
\begin{align*}
x=AQ,\quad y=2-AB\sin t,\quad AB\cdot OA=(AQ)^2
\end{align*}
\انتہا{سوال}
%=====================
\begin{figure}
\centering
\begin{minipage}{0.45\textwidth}
\centering
\begin{tikzpicture}
\draw[name path=c](0,1)node[circ]{}node[left]{$(0,1)$} circle (1);
\draw[-latex](-1.5,0)--(1.5,0)node[right]{$x$};
\draw[-latex](0,-0.2)node[left]{$O$}--(0,2.5)node[above]{$y$};
\draw(0,2)node[above left]{$Q$};
\draw(-2,2)node[above]{$y=2$}--(2,2)coordinate[pos=0.85](ka)node[pos=0.85,above]{$A$}coordinate[pos=0.85](kA);
\draw[thick](0,2) to [out=0,in=140] (2,0.77);
\draw(0,2) to [out=180,in=40] (-2,0.77);
\draw[name path=st](ka)--(0,0);
\path[name intersections={of={st and c}}] (intersection-1)node[left]{$B$}coordinate(kB)--++(1,0)coordinate(kBB);
\path(kA)--($(kB)!(kA)!(kBB)$)coordinate(kR);
\draw[dashed](kB)--(kR)node[circ]{}node[right]{$N(x,y)$}--(kA);
\draw[-stealth]([shift={(0:0.5)}]0,0) arc (0:55:0.5)node[right,xshift=0.5ex]{$t$};
\end{tikzpicture}
\caption{ترسیم برائے سوال \حوالہ{سوال_مخروط_چڑیل}}
\label{شکل_سوال_مخروط_چڑیل}
\end{minipage}\hfill
\begin{minipage}{0.45\textwidth}
\centering
\begin{tikzpicture}[declare function={f(\x)=cos(deg(\x))+\x*sin(deg(\x));g(\x)=sin(deg(\x))-\x*cos(deg(\x));}]
\draw[-latex](-1.25,0)--(2.5,0)node[right]{$x$};
\draw[-latex](0,-1.2)--(0,1.5)node[above]{$y$};
\draw(0,0) circle (1);
\draw[name path=c,->-=0.75,thick,domain=0:2.5] plot ({f(\x)},{g(\x)});
\path[name path=t](70:1)--++(-20:2);
\draw[name intersections={of=c and t}](intersection-1)node[circ]{}node[right]{$N(x,y)$}--(70:1)node[above]{$Q$}node[pos=0.5,above,sloped]{دھاگہ};
\draw(0,0)node[below left]{$O$}--(70:1);
\draw[-stealth]([shift={(0:0.3)}]0,0) arc (0:70:0.3)node[right,xshift=0.5ex]{$t$};
\draw(1,0)node[below right]{$(1,0)$};
\end{tikzpicture}
\caption{اکائی دائرے کا در پیچیدہ (سوال \حوالہ{سوال_دائرے_کا_در_پیچیدہ})}
\label{شکل_سوال_دائرے_کا_در_پیچیدہ}
\end{minipage}
\end{figure}
\ابتدا{سوال}\شناخت{سوال_دائرے_کا_در_پیچیدہ}\ترچھا{دائرے کا در پیچیدہ}\\
ایک غیر تغیر پذیر دائرہ کے گرد لپٹے گئے دھاگے کو تان کر، دائرے کی مستوی میں رہتے ہوئے،  کھولنے سے دھاگے کا سر \عددی{N} جس راہ پر چلتا ہے، اس کو دائرے کا \اصطلاح{در پیچیدہ}\فرہنگ{در پیچیدہ}\حاشیہب{involute}\فرہنگ{involute} کہتے ہیں۔  شکل \حوالہ{سوال_دائرے_کا_در_پیچیدہ} میں دائرہ \عددی{x^2+y^2=1} اور ابتدائی نقطہ \عددی{(1,0)} ہے۔ کھولا گیا دھاگہ \عددی{Q} پر دائرے کا مماس ہے۔ قطع \عددی{OQ} اور مثبت \عددی{x} محور کے بیچ زاویہ \عددی{t} ہے۔ نقطہ \عددی{N(x,y)} کے محدد \عددی{x} اور \عددی{y} کو \عددی{t} کی روپ میں لکھ کر در پیچیدہ کی مقدار معلوم مساوات تلاش کریں۔ در پیچیدہ کی مقدار معلوم کا  وقفہ \عددی{t\ge 0} لیں۔ 
\انتہا{سوال}
%==================
\ابتدا{سوال}\شناخت{سوال_مخروط_مقدار_معلوم_سیدھی_لکیر} \ترچھا{مستوی میں لکیروں کی مقدار معلوم روپ}\\
(الف) دکھائیں کہ درج ذیل ایک ایسی لکیر کو ظاہر کرتی ہیں جو نقطہ \عددی{(x_0,y_0)} اور \عددی{(x_1,y_1)} سے گزرتی ہے (شکل \حوالہ{شکل_سوال_مخروط_مقدار_معلوم_سیدھی_لکیر})۔
\begin{align*}
x=x_0+(x_1-x_0)t,\quad y=y_0+(y_1-y_0)t,\quad -\infty<t<\infty
\end{align*}
 (ب)اسی مقدار معلوم وقفہ کو استعمال کرتے ہوئے نقطہ \عددی{(x_1,y_1)} اور مبدا سے گزرتی لکیر کی مقدار معلوم مساوات لکھیں۔ (ج) اسی مقدار معلوم وقفہ کے لئے \عددی{(-1,0)} اور \عددی{(0,1)} سے گزرتی لکیر کی مقدار معلوم مساوات معلوم کریں۔ شکل \حوالہ{شکل_سوال_مخروط_مقدار_معلوم_سیدھی_لکیر} میں تیر کا نشان بڑھتے \عددی{t} کا رخ ظاہر کرتا ہے۔
\انتہا{سوال}
%====================
\begin{figure}
\centering
\begin{minipage}{0.45\textwidth}
\centering
\begin{tikzpicture}[font=\scriptsize]
\draw[->-=0.85](0,0)--(3,2)node[pos=0.1,circ]{}node[pos=0.1,above left]{$t=0$}node[pos=0.1,below right]{$(x_0,y_0)$}node[pos=0.3,circ]{}node[pos=0.3,above left]{$t=1$}node[pos=0.3,below right]{$(x_1,y_1)$}node[pos=0.7,circ]{}node[pos=0.7,above left,font=\scriptsize]{$N(x_0+(x_1-x_0)t,y_0+(y_1-y_0)t)$};
\end{tikzpicture}
\caption{مستوی میں سیدھی لکیر (سوال \حوالہ{سوال_مخروط_مقدار_معلوم_سیدھی_لکیر})۔}
\label{شکل_سوال_مخروط_مقدار_معلوم_سیدھی_لکیر}
\end{minipage}\hfill
\begin{minipage}{0.45\textwidth}
\centering
\begin{tikzpicture}
\draw[-latex,name path=kx](-0.25,0)--(4,0)node[right]{$x$};
\draw[-latex](0,-1.25)--(0,1.5)node[above]{$y$};
\draw[thick,name path=kr](0,-0.75)--(25:3.75)node[right]{$N$};
\draw[-stealth,name intersections={of={kx and kr}}] ([shift={(0:0.5)}]intersection-1) arc (0:35:0.5)node[right,shift={(0.5ex,-0.25ex)}]{$\theta$};
\draw($(0,-0.75)!0.5!(intersection-1)$) node[below right]{$R$};
\draw(intersection-1)node[circ]{};
\draw[fill=gray](0,-0.75) circle (0.1);
\draw(0,-0.75) circle (0.15);
\draw[fill=gray](intersection-1) circle (0.1);
\draw(intersection-1) circle (0.15);
\end{tikzpicture}
\caption{آرشمیدسی روک برائے سوال \حوالہ{سوال_مخروط_پہیا_سلاخ}}
\label{شکل_سوال_مخروط_پہیا_سلاخ}
\end{minipage}
\end{figure}
\ابتدا{سوال}\شناخت{سوال_مخروط_پہیا_سلاخ}\ترچھا{آرشمیدسی روک}\\
آرشمیدسی روک کو شکل \حوالہ{شکل_سوال_مخروط_پہیا_سلاخ} میں دکھایا گیا ہے جو ایک مضبوط سلاخ جس کی لمبائی \عددی{L} ہو پر مشتمل ہے۔محور \عددی{x} اور \عددی{y}  کے ساتھ اس کو پہیوں کے ساتھ منسلک کیا گیا ہے۔ اس سلاخ کا آزاد سر \عددی{N} ہے۔ سلاخ اور مثبت \عددی{x} محور کے بیچ زاویہ \عددی{\theta} ہے۔
\begin{enumerate}[a.]
\item
مقدار معلوم \عددی{\theta} کی صورت میں \عددی{N} کی راہ کی مساوات تلاش کریں۔
\item
\عددی{N} کی راہ کی کارتیسی مساوات تلاش کر کے اس کی ترسیم کو پہچانیں۔
\end{enumerate}
\انتہا{سوال}
%====================
\ابتدا{سوال}\شناخت{سوال_مخروط_ستارہ_نما}\ترچھا{فلک تدویر}\\
ایک غیر تغیر پذیر دائرے کے محیط کی اندرون پر چلتے ہوئے دائرہ کی محیط پر کسی بھی نقطہ \عددی{N} کی راہ \اصطلاح{فلک تدویر}\فرہنگ{تدویر!فلک}\فرہنگ{فلک تدویر}\حاشیہب{hypocycloid}\فرہنگ{hypocycloid} کہلاتی ہے۔ غیر تغیر پذیر دائرہ \عددی{x^2+y^2=a^2} لیں جبکہ دوسرے دائرے کا رداس \عددی{b} ہے۔ \عددی{N} کا ابتدائی مقام نقطہ \عددی{A(a,0)} لیں۔ دونوں دائروں کے مراکز کو ملانے والے خط اور مثبت \عددی{x} محور کے بیچ زاویہ \عددی{\theta} ہے۔  فلک تدویر کی مقدار معلوم مساوات تلاش کریں جہاں مقدار معلوم \عددی{\theta} ہے۔ بالخصوص \عددی{b=\tfrac{a}{4}} کی صورت میں فلک تدویر درج ذیل \اصطلاح{ستارہ نم}\فرہنگ{ستارہ نما}\حاشیہب{stroid}\فرہنگ{astroid} ہو گا (شکل \حوالہ{شکل_سوال_مخروط_ستارہ_نما})۔
\begin{align*}
x=a\cos^3\theta,\quad y=a\sin^3\theta
\end{align*}
\انتہا{سوال}
%====================
\begin{figure}
\centering
\begin{minipage}{0.30\textwidth}
\centering
\begin{tikzpicture}[font=\scriptsize,scale=1.5,declare function={f(\x)=(cos(deg(\x)))^3;g(\x)=(sin(deg(\x)))^3;}]
\pgfmathsetmacro{\ang}{25}
\draw[-latex](-1.25,0)--(1.5,0)node[right]{$x$};
\draw[-latex](0,0)--(0,1.25)node[above]{$y$};
\draw(0,0) circle (1);
\draw(0,0)node[below]{$O$}--(\ang:1);
\draw[name path=kc](\ang:0.75)node[shift={(\ang+90:0.2)}]{$C$} circle (0.25);
\draw[name path=kf,thick,domain=0:pi]plot ({f(\x)},{g(\x)});
\draw[name intersections={of={kc and kf}}] (\ang:0.75)--(intersection-2)node[circ]{}node[below,yshift=-0.5ex]{$N$};
\draw(1,0)node[below right,xshift=-0.5ex]{$A(a,0)$};
\draw[-stealth]([shift={(0:0.3)}]0,0) arc (0:\ang:0.3)node[shift={(1ex,-0.5ex)}]{$\theta$};
\end{tikzpicture}
\caption{ستارہ نما۔ سوال \حوالہ{سوال_مخروط_ستارہ_نما}}
\label{شکل_سوال_مخروط_ستارہ_نما}
\end{minipage}\hfill
\begin{minipage}{0.30\textwidth}
\centering
\begin{tikzpicture}[scale=0.5]
\draw(0,0) circle (2);
\draw(1,0) circle (1);
\draw(0,0)--(1,0)node[circ]{}node[pos=0.5,above]{$a$}--(2,0)node[right]{$N$}node[pos=0.5,above]{$a$};
\end{tikzpicture}
\caption{تدویر برائے سوال \حوالہ{سوال_مخروط_مزید_تدویر}}
\label{شکل_سوال_مخروط_مزید_تدویر}
\end{minipage}\hfill
\begin{minipage}{0.30\textwidth}
\centering
\begin{tikzpicture}[font=\scriptsize,declare function={f(\x)=(sin(deg(\x)))^2*tan(deg(\x));g(\x)=(sin(deg(\x)))^2;}]
\pgfmathsetmacro{\r}{0.75}
\draw[-latex](-1.5,0)--(1.5,0)node[right]{$x$};
\draw[-latex](0,0)node[below]{$O$}--(0,2.15)node[above]{$y$};
\draw[name path=k,->-=0.95,-<-=0.7](-1.5,2*\r)--(1.5,2*\r);
\draw[name path=A](0,\r)circle (\r);
\draw[name path=B,thick,domain=0:1]plot ({2*\r*f(\x)},{2*\r*g(\x)});
\draw[thick,domain=0:1]plot ({-2*\r*f(\x)},{2*\r*g(\x)});
\draw(0,2*\r)node[above left]{$A(0,a)$};
\draw[name path=C](0,0)--(1,2*\r)node[circ]{}node[above]{$N$}node[pos=0.3,circ]{}node[pos=0.3,xshift=1ex,yshift=-0.5ex]{$P$};
\draw[name intersections={of={A and C}}](intersection-1)node[xshift=1ex]{$M$};
\draw[-stealth]([shift={(90:0.75)}]0,0) arc (90:55:0.75);
\draw(75:0.9)node[]{$t$};
\end{tikzpicture}
\caption{ترسیمات برائے سوال \حوالہ{سوال_مخروط_عجیب_صورت}}
\label{شکل_سوال_مخروط_عجیب_صورت}
\end{minipage}
\end{figure}
\ابتدا{سوال}\شناخت{سوال_مخروط_مزید_تدویر}\ترچھا{تدویر پر مزید معلومات}\\
ایک دائرہ جس کا رداس \عددی{2a} ہے کے اندر دوسرا دائرہ جس کا رداس \عددی{a} ہے شکل \حوالہ{شکل_سوال_مخروط_مزید_تدویر} میں دکھایا گیا ہے۔ نقطہ \عددی{N} پر یہ دائرے آپس میں ملتے ہیں۔ اندرونی دائرہ بیرونی دائرے کے اندر محیط پر چلتا ہے۔ نقطہ \عددی{N} کے مقام کی مساوات تلاش کریں۔
\انتہا{سوال}
%=======================
\ابتدا{سوال}\شناخت{سوال_مخروط_عجیب_صورت}
نقطہ \عددی{N} لکیر \عددی{y=a} پر چلتا ہے (شکل \حوالہ{شکل_سوال_مخروط_عجیب_صورت})۔ نقطہ \عددی{P} یوں حرکت کرتا ہے کہ \عددی{OP=MN} ہو۔زاویہ \عددی{t} کو مقدار معلوم لیتے ہوئے نقطہ \عددی{P} کی مقدار معلوم مساوات معلوم کریں۔لکیر \عددی{ON} اور \عددی{y} محور کے بیچ زاویہ \عددی{t} ہے۔
\انتہا{سوال}
%===================
\ابتدا{سوال}\شناخت{تدویری راہ}
ایک پہیا جس کا رداس \عددی{a} ہے ایک سیدھی لکیر پر بغیر پھسلے چل رہا ہے۔پہیے کے مرکز سے \عددی{b} اکائی دور نقطہ \عددی{N} کی راہ کی مقدار معلوم مساوات تلاش کریں۔ پہیا جتنا زاویہ (\عددی{\theta}) گھومتا ہے، اس کو مقدار معلوم لیں۔   
\انتہا{سوال}
%=====================

\موٹا{مقدار معلوم مساوات سے فاصلے کا حصول}

\ابتدا{سوال}
قطع مکافی \عددی{x=t,\, y=t^2,\, -\infty<t<\infty} پر \عددی{(2,\tfrac{1}{2})} کا قریبی نقطہ تلاش کریں۔ (اشارہ: فاصلہ کے مربع کا \عددی{t} کے لحاظ سے تفرق لیں۔)
\انتہا{سوال}
%==============
\ابتدا{سوال}
ترخیم \عددی{x=2\cos t,\, y=\sin t,\, 0\le t\le 2\pi} پر \عددی{(\tfrac{3}{4},0)} کا قریبی نقطہ تلاش کریں۔  (اشارہ: فاصلہ کے مربع کا \عددی{t} کے لحاظ سے تفرق لیں۔)
\انتہا{سوال}
%=============

\موٹا{کمپیوٹر کا استعمال}

درج ذیل مقدار معلوم مساوات کو کمپیوٹر پر ترسیم کریں۔

\ابتدا{سوال}
ترخیم \عددی{x=4\cos t,\, y=2\sin t} وقفہ (ا) \عددی{0\le t\le 2\pi}، (ب) \عددی{0\le t\le \pi}، (ج) \عددی{-\tfrac{\pi}{2}\le t\le \tfrac{\pi}{2}}
\انتہا{سوال}
%=================
\ابتدا{سوال}
قطع زائد \عددی{x=\sec t,\, y=\tan t} کا ایک بازو وقفہ (ا) \عددی{-1.5\le t\le 1.5}، (ب) \عددی{-0.5\le t\le 0.5}، (ج) \عددی{-0.1\le t\le 0.1}
\انتہا{سوال}
%===============
\ابتدا{سوال}
قطع مکافی \عددی{x=2t+3,\, y=t^2-1} وقفہ \عددی{-2\le t\le 2}
\انتہا{سوال}
%====================
\ابتدا{سوال}
تدویر \عددی{x=t-\sin t,\, y=1-\cos t} وقفہ (ا) \عددی{0\le t\le 2\pi}، (ب) \عددی{0\le t\le 4\pi}، (ج) \عددی{\pi\le t\le 3\pi}
\انتہا{سوال}
%==================
\ابتدا{سوال}
ستارہ نما \عددی{x=\cos^3t,\, y=\sin^3t} وقفہ (ا) \عددی{0\le t\le 2\pi}، (ب) \عددی{-\tfrac{\pi}{2}\le t \le \tfrac{\pi}{2}}، (ج) \عددی{}
\انتہا{سوال}
%==============
\ابتدا{سوال}\ترچھا{ایک خوبصورت منحنی یا \اصطلاح{مثلثی}\فرہنگ{مثلثی}\حاشیہب{deltoid}\فرہنگ{deltoid}}\\
اگر درج ذیل مساوات کے \عددی{x} اور \عددی{y} میں \عددی{2} کی جگہ \عددی{-2} ہو تب کیا ہو گا؟
\begin{align*}
x=2\cos t+\cos 2t,\quad y=2\sin t-\sin 2t,\quad 0\le t\le 2\pi
\end{align*}
اس نئی مساوات کو  ترسیم کر کے دریافت کریں۔
\انتہا{سوال}
%================
\ابتدا{سوال}\ترچھا{مزید خوبصورت منحنی}\\
اگر درج ذیل مساوات کے \عددی{x} اور \عددی{y} میں \عددی{3} کی جگہ \عددی{-3} ہو تب کیا ہو گا؟
\begin{align*}
x=3\cos t+\cos 3t,\quad y=3\sin t-\sin 3t,\quad 0\le t\le 2\pi
\end{align*}
اس نئی مساوات کو  ترسیم کر کے دریافت کریں۔

\انتہا{سوال}
%========================
\ابتدا{سوال}\ترچھا{گولا}\\
توپ کے  گولے کا مدار درج ذیل ہے۔
\begin{align*}
x=(64\cos \alpha)t,\quad y=-4.9t^2+(64\sin\alpha)t,\quad 0\le t\le 4\sin \alpha
\end{align*}
توپ کے مدار کو زاویہ (ا) \عددی{\alpha=\tfrac{\pi}{4}}، (ب) \عددی{\alpha=\tfrac{\pi}{6}}، (ج) \عددی{\alpha=\tfrac{\pi}{3}} اور (د) \عددی{\alpha=\tfrac{\pi}{2}}  کے لئے ترسیم کریں۔
\انتہا{سوال}
%==================
\ابتدا{سوال}\ترچھا{تین خوبصورت منحنیات}\\
\begin{enumerate}[a.]
\item
بر تدویر:
\begin{align*}
x=9\cos t-\cos 9t,\quad y=9\sin t-\sin 9t,\quad 0\le t\le 2\pi
\end{align*}
\item
فلک تدویر:
\begin{align*}
x=8\cos t+2\cos 4t,\quad y=8\sin t-2\sin 4t,\quad 0\le t\le 2\pi
\end{align*}
\item
زیر تدویر:
\begin{align*}
x=\cos t+5\cos 3t,\quad y=6\cos t-5\sin 3t,\quad 0\le t\le 2\pi
\end{align*}
\end{enumerate}
\انتہا{سوال}
%=================
\ابتدا{سوال}\ترچھا{خوبصورت ترین منحنیات}\\
\begin{enumerate}[a.]
\item
$x=6\cos t+5\cos 3t,\quad y=6\sin t-5\sin 3t,\quad 0\le t\le 2\pi$
\item
$x=6\cos 2t+5\cos 6t,\quad y=6\sin 2t-5\sin 6t,\quad 0\le t\le \pi$
\item
$x=6\cos t+5\cos 3t,\quad y=6\sin 2t-5\sin 3t,\quad 0\le t\le 2\pi$
\item
$x=6\cos 2t+5\cos 6t,\quad y=6\sin 4t-5\sin 6t,\quad 0\le t\le \pi$
\end{enumerate}
\انتہا{سوال}
%======================
