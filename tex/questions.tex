\جزوحصہء{سوالات}
\موٹا{نقطہ پر ڈھلوان کا حصول}\\
سوال \حوالہ{سوال_کثیرالمتغیر_ڈھلوان_ہم_قد_الف} تا سوال \حوالہ{سوال_کثیرالمتغیر_ڈھلوان_ہم_قد_ب} میں دیے نقطہ پر تفاعل کی ڈھلوان  تلاش کریں۔اس  نقطہ پر ڈھلوان اور اس نقطہ سے گزرتی   ہم قد منحنی  ترسیم کریں۔

\ابتدا{سوال}\شناخت{سوال_کثیرالمتغیر_ڈھلوان_ہم_قد_الف}
$f(x,y)=y-x,\quad (2,1)$
\انتہا{سوال}
%====================
\ابتدا{سوال}
$f(x,y)=\ln(x^2+y^2),\quad (1,1)$
\انتہا{سوال}
%===================
\ابتدا{سوال}
$g(x,y)=y-x^2,\quad (-1,0)$
\انتہا{سوال}
%===================
\ابتدا{سوال}\شناخت{سوال_کثیرالمتغیر_ڈھلوان_ہم_قد_ب}
$g(x,y)=\frac{x^2}{2}-\frac{y^2}{2},\quad (\sqrt{2},1)$
\انتہا{سوال}
%===================
سوال \حوالہ{سوال_کثیرالمتغیر_ڈھلوان_الف} تا سوال \حوالہ{سوال_کثیرالمتغیر_ڈھلوان_ب} میں دیے نقطہ پر \عددی{\nabla f} تلاش کریں۔

\ابتدا{سوال}\شناخت{سوال_کثیرالمتغیر_ڈھلوان_الف}
$f(x,y,z)=x^2+y^2-2z^2+z\ln x,\quad (1,1,1)$
\انتہا{سوال}
%================
\ابتدا{سوال}
$f(x,y,z)=2z^3-3(x^2+y^2)z+\tan^{-1}xz,\quad (1,1,1)$
\انتہا{سوال}
%===================
\ابتدا{سوال}
$f(x,y,z)=(x^2+y^2+z^2)^{-1/2}+\ln(xyz),\quad (-1,2,-2)$
\انتہا{سوال}
%===================
\ابتدا{سوال}\شناخت{سوال_کثیرالمتغیر_ڈھلوان_ب}
$f(x,y,z)=e^{x+y}\cos z+(y+1)\sin^{-1}x,\quad (0,0,\pi/6)$
\انتہا{سوال}
%===================

\موٹا{مستوی \عددی{xy} میں رخی تفرق کی تلاش}\\
سوال \حوالہ{سوال_کثیرالمتغیر_رخی_تفرق_الف} تا سوال \حوالہ{سوال_کثیرالمتغیر_رخی_تفرق_ب} میں \عددی{N_0} پر \عددی{\kvec{A}} کے رخ تفاعل کا رخی تفرق دریافت کریں۔

\ابتدا{سوال}\شناخت{سوال_کثیرالمتغیر_رخی_تفرق_الف}
$f(x,y)=2xy-3y^2,\quad N_0(5,5),\quad \kvec{A}=4\ai+3\aj$
\انتہا{سوال}
%======================
\ابتدا{سوال}
$f(x,y)=2x^2+y^2,\quad N_0(-1,1),\quad \kvec{A}=3\ai-4\aj$
\انتہا{سوال}
%==========================
\ابتدا{سوال}
$g(x,y)=x-\frac{y^2}{x}+\sqrt{3}\sec^{-1}(2xy),\quad N_0(1,1),\quad \kvec{A}=12\ai+5\aj$
\انتہا{سوال}
%==========================
\ابتدا{سوال}
$h(x,y)=\tan^{-1}\frac{y}{x}+\sqrt{3}\sin^{-1}\frac{xy}{2},\quad N_0(1,1),\quad \kvec{A}=3\ai-2\aj$
\انتہا{سوال}
%==========================
\ابتدا{سوال}
$f(x,y,z)=xy+yz+zx,\quad N_0(1,-1,2),\quad \kvec{A}=3\ai+6\aj-2\ak$
\انتہا{سوال}
%==========================
\ابتدا{سوال}
$f(x,y,z)=x^2+2y^2-3z^2,\quad N_0(1,1,1),\quad \kvec{A}=\ai+\aj+\ak$
\انتہا{سوال}
%==========================
\ابتدا{سوال}
$g(x,y,z)=3e^x\cos yz,\quad N_0(0,0,0),\quad \kvec{A}=2\ai+\aj-2\ak$
\انتہا{سوال}
%==========================
\ابتدا{سوال}\شناخت{سوال_کثیرالمتغیر_رخی_تفرق_ب}
$h(x,y,z)=\cos xy+e^{yz}+\ln zx,\quad N_0(1,0,1/2),\quad \kvec{A}=\ai+2\aj+2\ak$
\انتہا{سوال}
%==========================

\موٹا{تیز بڑھنے اور گھٹنے کے رخ}\\
سوال \حوالہ{سوال_کثیرالمتغیر_تیز_ترین_بڑھنا_الف} تا سوال \حوالہ{سوال_کثیرالمتغیر_تیز_ترین_بڑھنا_ب} میں نقطہ \عددی{N_0}  پر  وہ رخ تلاش کریں جس رخ تفاعل کے بڑھنے اور گھٹنے کی تبدیلی تیز ترین ہو۔ ان رخ تفاعل کے تفرق دریافت کریں۔

\ابتدا{سوال}\شناخت{سوال_کثیرالمتغیر_تیز_ترین_بڑھنا_الف}
$f(x,y)=x^2+xy+y^2,\quad N_0(-1,1)$
\انتہا{سوال}
%==================
\ابتدا{سوال}
$f(x,y)=x^2y+e^{xy}\sin y,\quad N_0(1,0)$
\انتہا{سوال}
%==================
\ابتدا{سوال}
$f(x,y,z)=\frac{x}{y}-yz,\quad N_0(4,1,1)$
\انتہا{سوال}
%======================
\ابتدا{سوال}
$g(x,y,z)=xe^y+z^2,\quad N_0(1,\ln 2,1/2)$
\انتہا{سوال}
%======================
\ابتدا{سوال}
$f(x,y,z)=\ln xy+\ln yz+\ln xz,\quad N_0(1,1,1)$
\انتہا{سوال}
%======================
\ابتدا{سوال}\شناخت{سوال_کثیرالمتغیر_تیز_ترین_بڑھنا_ب}
$h(x,y,z)=\ln(x^2+y^2-1)+y+6z,\quad N_0(1,1,0)$
\انتہا{سوال}
%======================

\موٹا{تبدیلی کا اندازہ}\\
\ابتدا{سوال}
نقطہ \عددی{N(x,y,z)}  کو \عددی{N_0(3,4,12)} سے\عددی{3\ai+6\aj-2\ak} رخ   \عددی{\dif s=0.1} اکائیاں دور منتقل کرنے سے  \عددی{f(x,y,z)=\ln\sqrt{x^2+y^2+z^2}} کی قیمت میں کتنی تبدیلی رونما ہو گی؟
\انتہا{سوال}
%=============
\ابتدا{سوال}
نقطہ \عددی{N(x,y,z)}  کو مبدا  سے\عددی{2\ai+2\aj-2\ak} رخ   \عددی{\dif s=0.1} اکائیاں دور منتقل کرنے سے تفاعل   \عددی{f(x,y,z)=e^x\cos yz} کی قیمت میں کتنی تبدیلی رونما ہو گی؟
\انتہا{سوال}
%=============
\ابتدا{سوال}
نقطہ \عددی{N(x,y,z)}  کو  \عددی{N_0(2,-1,0)}سے  \عددی{N_1(0,1,2)} جانب      \عددی{\dif s=0.2} اکائیاں دور منتقل کرنے سے تفاعل   \عددی{g(x,y,z)=x+x\cos z-y\sin z+y} کی قیمت میں کتنی تبدیلی رونما ہو گی؟
\انتہا{سوال}
%=============
\ابتدا{سوال}
نقطہ \عددی{N(x,y,z)}  کو\عددی{N_0(-1,-1,-1)} سے مبدا  کے   رخ   \عددی{\dif s=0.1} اکائیاں دور منتقل کرنے سے تفاعل   \عددی{h(x,y,z)=\cos(\pi xy)+xz^2} کی قیمت میں کتنی تبدیلی رونما ہو گی؟
\انتہا{سوال}
%=============

\موٹا{سطح کا مماسی مستوی اور  عمودی خط}\\
سوال \حوالہ{سوال_کثیرالمتغیر_مماسی_مستوی_عمودی_خط_الف} تا سوال \حوالہ{سوال_کثیرالمتغیر_مماسی_مستوی_عمودی_خط_ب} میں  نقطہ \عددی{N_0} پر دیے گئے   سطح کا  (ا) مماسی مستوی اور (ب) عمودی خط تلاش کریں۔

\ابتدا{سوال}\شناخت{سوال_کثیرالمتغیر_مماسی_مستوی_عمودی_خط_الف}
$x^2+y^2+z^2=3,\quad N_0(1,1,1)$
\انتہا{سوال}
%=================
\ابتدا{سوال}
$x^2+y^2-z^2=18,\quad N_0(3,5,-4)$
\انتہا{سوال}
%===================
\ابتدا{سوال}
$2z-x^2=0,\quad N_0(2,0,2)$
\انتہا{سوال}
%===================
\ابتدا{سوال}
$x^2+2xy-y^2+z^2=7,\quad N_0(1,-1,3)$
\انتہا{سوال}
%===================
\ابتدا{سوال}
$\cos \pi x-x^2y+e^{xz}+yz=4,\quad N_0(0,1,2)$
\انتہا{سوال}
%===================
\ابتدا{سوال}
$x^2-xy-y^2-z=0,\quad N_0(1,1,-1)$
\انتہا{سوال}
%===================
\ابتدا{سوال}
$x+y+z=1,\quad N_0(0,1,0)$
\انتہا{سوال}
%===================
\ابتدا{سوال}\شناخت{سوال_کثیرالمتغیر_مماسی_مستوی_عمودی_خط_ب}
$x^2+y^2-2xy-x+3y-z=-4,\quad N_0(2,-3,18)$
\انتہا{سوال}
%===================

سوال \حوالہ{سوال_کثیرالمتغیر_مماسی_مستوی_کی_مساوات_الف} تا سوال \حوالہ{سوال_کثیرالمتغیر_مماسی_مستوی_کی_مساوات_ب} میں دیے گئے نقطہ پر سطح کے مماسی مستوی کی مساوات تلاش کریں۔

\ابتدا{سوال}\شناخت{سوال_کثیرالمتغیر_مماسی_مستوی_کی_مساوات_الف}
$z=\ln(x^2+y^2),\quad (1,0,0)$
\انتہا{سوال}
%=====================
\ابتدا{سوال}
$z=e^{-(x^2+y^2)},\quad (0,0,1)$
\انتہا{سوال}
%======================
\ابتدا{سوال}
$z=\sqrt{y-x},\quad (1,2,1)$
\انتہا{سوال}
%======================
\ابتدا{سوال}\شناخت{سوال_کثیرالمتغیر_مماسی_مستوی_کی_مساوات_ب}
$z=4x^2+y^2,\quad (1,1,5)$
\انتہا{سوال}
%======================

\موٹا{منحنیات کے مماسی خط}\\
سوال \حوالہ{سوال_کثیرالمتغیر_مماسی_خط_نیبلا_الف} تا سوال \حوالہ{سوال_کثیرالمتغیر_مماسی_خط_نیبلا_ب} میں منحنی  \عددی{f(c,y)=c}  ترسیم کریں۔ ساتھ ہی دیے گئے نقطہ پر  \عددی{\nabla f} اور مماسی خط ترسیم کریں۔

\ابتدا{سوال}\شناخت{سوال_کثیرالمتغیر_مماسی_خط_نیبلا_الف}
$x^2+y^2=4,\quad (\sqrt{2},\sqrt{2})$
\انتہا{سوال}
%===================
\ابتدا{سوال}
$x^2-y=1,\quad (\sqrt{2},1)$
\انتہا{سوال}
%=================
\ابتدا{سوال}
$xy=-4,\quad (2,-2)$
\انتہا{سوال}
%=================
\ابتدا{سوال}\شناخت{سوال_کثیرالمتغیر_مماسی_خط_نیبلا_ب}
$x^2-xy+y^2=7,\quad (-1,2)$\quad
(یہ مثال \حوالہ{مثال_تفرق_خفی_تفاعل_د} کی منحنی ہے۔)
\انتہا{سوال}
%=================

سوال \حوالہ{سوال_کثیرالمتغیر_متقاطع_منحنی_مماسی_خط_الف} تا سوال \حوالہ{سوال_کثیرالمتغیر_متقاطع_منحنی_مماسی_خط_ب} میں دیے نقطہ پر  سطحوں کی متقاطع منحنی  کے مماسی خط کی مقدار معلوم مساوات

\ابتدا{سوال}\شناخت{سوال_کثیرالمتغیر_متقاطع_منحنی_مماسی_خط_الف}
$x+y^2+2z=4,\quad x=1;\quad (1,1,1)$
\انتہا{سوال}
%==================
\ابتدا{سوال}
$xyz=1,\quad x^2+2y^2+3z^2=6;\quad (1,1,1)$
\انتہا{سوال}
%======================
\ابتدا{سوال}
$x^2+2y+2z=4,\quad y=1;\quad (1,1,1/2)$
\انتہا{سوال}
%======================
\ابتدا{سوال}
$x+y^2+z=2,\quad y=1;\quad (1/2,1,1/2)$
\انتہا{سوال}
%======================
\ابتدا{سوال}
$x^3+3x^2y^2+y^3+4xy-z^2=0,\quad x^2+y^2+z^2=11;\quad (1,1,3)$
\انتہا{سوال}
%======================
\ابتدا{سوال}\شناخت{سوال_کثیرالمتغیر_متقاطع_منحنی_مماسی_خط_ب}
$x^2+y^2=4,\quad x^2+y^2-z=0;\quad (\sqrt{2},\sqrt{2},4)$
\انتہا{سوال}
%======================

\موٹا{نظریہ اور مثالیں}\\
\ابتدا{سوال}
نقطہ \عددی{N(3,2)} پر کس رخ \عددی{f(x,y)=xy+y^2} کا تفرق صفر ہو گا؟
\انتہا{سوال}
%===============
\ابتدا{سوال}
نقطہ \عددی{N(1,1)} پر کن دو رخ تفاعل \عددی{f(x,y)=\tfrac{x^2-y^2}{x^2+y^2}} کے تفرقات  صفر ہوں گے؟
\انتہا{سوال}
%=============
\ابتدا{سوال}
کیا \عددی{N(1,2)} پر ایسا کوئی  رخ \عددی{\kvec{A}} ہے جس رخ \عددی{f(x,y)=x^2-3xy+4y^2}  کا تفرق \عددی{14} کے برابر ہو؟ اپنے جواب کی وجہ پیش کریں۔
\انتہا{سوال}
%=============
\ابتدا{سوال}
کیا کسی رخ \عددی{\kvec{A}}،  نقطہ \عددی{N(1,-1,1)} پر  درجہ حرارت  \عددی{T(x,y,z)=2xy-yz}   کی شرح تبدیلی \عددی{\SI{-3}{\celsius\per\meter}} ہو گی؟ اپنے جواب کی وجہ پیش کریں۔
\انتہا{سوال}
%==========
\ابتدا{سوال}
نقطہ \عددی{N_0(1,2)} پر \عددی{\ai+\aj} رخ  \عددی{f(x,y)} کا تفرق \عددی{2\sqrt{2}}  اور \عددی{-2\aj} رخ \عددی{-3} ہے۔   سمتیہ \عددی{-\ai-2\aj}  رخ \عددی{f} کا تفرق کیا ہو گا؟ اپنے جواب کی وجہ پیش کریں۔
\انتہا{سوال}
%==========
\ابتدا{سوال}
کسی نقطہ پر \عددی{f(x,y,z)} کے تفرق کی قیمت \عددی{\kvec{A}=\ai+\aj-\ak} رخ  زیادہ سے زیادہ ہے۔ اس رخ تفرق کی قیمت \عددی{2\sqrt{3}} ہے۔ (ا) اس نقطہ پر \عددی{\nabla f} کتنا ہو گا؟ اپنے جواب کی وجہ پیش کریں۔ (ب) سمتیہ \عددی{\ai+\aj}   کے رخ  اس نقطہ پر \عددی{f}  کا تفرق کیا ہو گا؟
\انتہا{سوال}
%=========
\ابتدا{سوال}\ترچھا{دائرہ پر درجہ حرارت کی تبدیلی}\\
فرض کریں مستوی \عددی{xy} میں  نقطہ \عددی{(x,y)}  پر درجہ حرارت   \عددی{T(x,y)=x\sin 2y} ہے۔  ایک ذرہ  ایک میٹر رداس کے دائرہ پر  گھڑی کے رخ  \عددی{\SI{2}{\meter\per\second}} کی  رفتار سے حرکت کرتا ہے۔ اس دائرے کا مرکز مبدا پر ہے۔ (ا)  نقطہ \عددی{N(1/2,\sqrt{3}/2)} پر یہ ذرہ  کس شرح حرارت   \عددی{\si{\celsius\per\meter}} سے گزرتا ہے؟ (ب)     نقطہ \عددی{N(1/2,\sqrt{3}/2)} پر یہ ذرہ  کس شرح حرارت   \عددی{\si{\celsius\per\second}} سے گزرتا ہے؟ 
\انتہا{سوال}
%==================
\ابتدا{سوال}\ترچھا{دائرہ کے درپیچیدہ  پر تبدیلی}\\
درج ذیل منحنی  کے   اکائی مماسی سمتیہ  کے رخ تفاعل \عددی{f(x,y)=x^2+y^2} کا تفرق   تلاش کریں۔
\begin{align*}
\kvec{r}(t)=(\cos t+t\sin t)\ai+(\sin t-t\cos t)\aj,\,t>0
\end{align*}
\انتہا{سوال}
%=========
\ابتدا{سوال}\ترچھا{پیچدار منحنی کے ساتھ ساتھ تبدیلی}\\
نقاط  \عددی{t=-\pi/4,0,\pi/4} پر پیچ دار  منحنی \عددی{\kvec{r}(t)=(\cos t)\ai+(\sin t)\aj+t\ak}  کے اکائی مماسی سمتیات کے رخ     تفاعل \عددی{f(x,y,z)=x^2+y^2+z^2} کے تفرقات   تلاش کریں۔ یہ تفاعل مبدا سے پیچدار منحنی کے نقطہ \عددی{N(x,y,z)} کے  فاصلہ کا مربع  دیتا ہے۔اس منحنی پر چلتے ہوئے   تفرق   \عددی{t} کے لحاظ سے فاصلے کے مربع کی شرح تبدیلی دے گا۔
\انتہا{سوال}
%============
\ابتدا{سوال}
فضا میں درجہ حرارت \عددی{T(x,y,z)=2x^2-xyz} ہے۔ ایک  متحرک ذرے کا مقام لمحہ \عددی{t} پر  \عددی{x=2t^2,\,y=3t,\,z=-t^2} ہے، جہاں وقت کی اکائی سیکنڈ اور فاصلہ کی اکائی میٹر ہے۔ (ا)  نقطہ \عددی{N(8,6,-4)} پر یہ ذرہ کس شرح تبدیلی \عددی{\si{\celsius\per\meter}} سے گزرتا ہے؟  (ب)  نقطہ \عددی{N(8,6,-4)} پر یہ ذرہ کس شرح تبدیلی \عددی{\si{\celsius\per\second}} سے گزرتا ہے؟ 
\انتہا{سوال}
%=============
\ابتدا{سوال}\شناخت{سوال_کثیرالمتغیر_خط_کی_مساوات}
دکھائیں کہ مستوی \عددی{xy} میں  نقطہ \عددی{(x_0,y_0)}   پر سمتیہ \عددی{\kvec{N}=A\ai+B\aj}  کے عمودی خط کی مساوات \عددی{A(x-x_0)+B(y-y_0)=0} ہو گی۔
\انتہا{سوال}
%=================
\ابتدا{سوال}\ترچھا{عمودی منحنیات  اور مماسی منحنیات}\\
ایک منحنی اس صورت ایک سطح \عددی{f(x,y,z)=c} کو نقطہ تقاطع پر عمودی ہو گی جب منحنی کا سمتیہ رفتار اس نقطہ پر \عددی{\nabla f} کا مستقل مضرب ہو۔ نقطہ تقاطع پر ایک منحنی اس صورت  سطح \عددی{f} کی مماسی منحنی ہو گی جب  منحنی کا سمتیہ رفتار  اس نقطہ پر \عددی{\nabla f} کو عمودی ہو۔ (ا) دکھائیں کہ منحنی \عددی{\kvec{r}(t)=\sqrt{t}\ai+\sqrt{t}\aj-\tfrac{1}{4}(t+3)\ak}  نقطہ \عددی{t=1} پر سطح \عددی{x^2+y^2-z=3}   کو عمودی ہے۔ (ب) دکھائیں کہ منحنی \عددی{\kvec{r}(t)=\sqrt{t}\ai+\sqrt{t}\aj+(2t-1)\ak} نقطہ \عددی{t=1} پر سطح \عددی{x^2+y^2-z=1}  کو مماسی ہے۔
\انتہا{سوال}
%=============
\ابتدا{سوال}\ترچھا{ہم قد منحنیات اور ڈھلوان ایک دوسرے کے عمودی کیوں ہوتے ہیں۔ دوسرا نقطہ نظر۔}\\
فرض کریں \عددی{t} کی تمام قیمتوں کے لئے قابل تفرق منحنی \عددی{x=g(t),\, y=h(t)} پر  قابل تفرق تفاعل \عددی{f(x,y)} کی قیمت  مستقل \عددی{c} ہو۔ مساوات \عددی{f(g(t),h(t))=c} کے دونوں اطراف کا \عددی{t} کے لحاظ سے تفرق لے کر دکھائیں کہ ہر نقطہ پر منحنی کا مماس اور  \عددی{\nabla f} آپس میں عمودی ہیں۔ 
\انتہا{سوال}
%==================
\ابتدا{سوال}\ترچھا{تفاعل \عددی{f(x,y)} کی خط بندی، مماسی مستوی تخمین ہو گی۔}\\
دکھائیں کہ نقطہ \عددی{N_0(x_0,y_0,f(x_0,y_0))} پر سطح \عددی{z=f(x,y)}  کا مماسی مستوی  درج ذیل ہو گا جہاں \عددی{f} قابل تفرق ہے۔
\begin{align*}
&f_x(x_0,y_0)(x-x_0)+f_y(x_0,y_0)(y-y_0)-(z-f(x_0,y_0))=0\\
z&=f(x_0,y_0)+f_x(x_0,y_0)(x-x_0)+f_y(x_0,y_0)(y-y_0)&&\text{یعنی}
\end{align*} 
یوں \عددی{N_0} پر مماسی مستوی  نقطہ \عددی{N_0} پر \عددی{f} کی خط بندی کی ترسیم ہو گی۔
\انتہا{سوال}
%============
\ابتدا{سوال}\ترچھا{رخی تفرقات اور غیر سمتی اجزاء}\\
نقطہ \عددی{N_0} پر اکائی سمتیہ \عددی{\uu}  کے رخ قابل تفرق تفاعل \عددی{f(x,y,z)}  کے تفرقات  کا  \عددی{\uu} کے رخ \عددی{(\nabla f)_{N_0}}  کے غیر سمتی اجزاء کے ساتھ کیا  تعلق ہے؟ اپنے جواب کی وجہ پیش کریں۔
\انتہا{سوال}
%==============
\ابتدا{سوال}\ترچھا{رخی تفرقات اور جزوی تفرقات}\\
فرض کریں کہ \عددی{f(x,y,z)} کے مطلوبہ تفرقات موجود ہیں۔ تفرقات \عددی{D_{\ai}f}، \عددی{D_{\aj}f} اور \عددی{D_{\ak}f} کا تفرقات \عددی{f_x}، \عددی{f_y} اور \عددی{f_z} کے ساتھ کیا تعلق ہے؟ اپنے جواب کی وجہ پیش کریں۔
\انتہا{سوال}
%=======================
\ابتدا{سوال}\ترچھا{الجبرائی قواعد برائے ڈھلوان}\\
مستقل \عددی{k} اور ڈھلوان \عددی{\nabla f=\tfrac{\partial f}{\partial x}\ai+\tfrac{\partial f}{\partial y}\aj+\tfrac{\partial f}{\partial z}\ak} اور  \عددی{\nabla g=\tfrac{\partial g}{\partial x}\ai+\tfrac{\partial g}{\partial y}\aj+\tfrac{\partial g}{\partial z}\ak}  دیے گئے ہیں۔  غیر سمتی مساوات
\begin{align*}
\frac{\partial}{\partial x}(kf)=k\frac{\partial f}{\partial x},\quad \frac{\partial}{\partial x}(f\mp g)=\frac{\partial f}{\partial x}\mp \frac{\partial g}{\partial x},\\
\frac{\partial}{\partial x}(fg)=f\frac{\partial g}{\partial x}+g\frac{\partial f}{\partial x},\quad \frac{\partial}{\partial x}\big(\frac{f}{g}\big)=\frac{g\frac{\partial f}{\partial x}-f\frac{\partial g}{\partial x}}{g^2},
\end{align*}
وغیرہ استعمال کرتے ہوئے درج ذیل قواعد کی تصدیق کریں۔
\begin{enumerate}[a.]
\item
$\nabla(kf)=k\nabla f$\quad
جہاں \عددی{k} مستقل ہے
\item
$\nabla (f+g)=\nabla f+\nabla g$
\item
$\nabla (f-g)=\nabla f-\nabla g$
\item
$\nabla (fg)=f\nabla g+g\nabla f$
\item
$\nabla\big(\frac{f}{g}\big)=\frac{g\nabla f-f\nabla g}{g^2}$
\end{enumerate}
\انتہا{سوال}
%=============
