\حصہ{غیر منفی اجزاء کے تسلسل کے پرکھ موازنہ}
گزشتہ حصہ کے ضمنی نتیجہ \حوالہ{ضمنی_تسلسل_نتیجہ_الف} کی استعمال میں اصل سوال یہ  معلوم کرنا ہے کہ \عددی{s_n} اوپر سے محدود ہے۔ بعض اوقات ہم دکھا پاتے ہیں کہ چونکہ دیے گئے تسلسل کا ہر جزوی مجموعہ \عددی{s_n} کسی مرتکز تسلسل کے  مطابقتی جزوی مجموعہ سے کم ہے لہٰذا دیا گیا تسلسل مرتکز ہے۔

\ابتدا{مثال}\شناخت{مثال_تسلسل_پرکھ_موازنہ_استعمال}
درج ذیل تسلسل
\begin{align*}
\sum\limits_{n=0}^{\infty}\frac{1}{n!}=1+\frac{1}{1!}+\frac{1}{2!}+\frac{1}{3!}+\cdots
\end{align*}
 اس لئے مرتکز ہے کہ اس کے تمام اجزاء مثبت اور  درج ذیل تسلسل کے مطابقتی اجزاء سے کم ہیں۔
\begin{align*}
1+\sum_{n=0}^{\infty}\frac{1}{2^n}=1+1+\frac{1}{2}+\frac{1}{2^2}+\cdots
\end{align*}
آئیں دیکھتے ہیں کہ یہ تعلق \عددی{\sum_{n=0}^{\infty}\tfrac{1}{n!}}  کے جزوی مجموعات کو کیسے اوپر سے محدود بناتا ہے۔ درج ذیل فرض کر کے
\begin{align*}
s_n=1+\frac{1}{1!}+\frac{1}{2!}+\cdots+\frac{1}{n!}
\end{align*}
ہم دیکھتے ہیں کہ ہر \عددی{n} کے لئے
\begin{align*}
s_n\le 1+1+\frac{1}{2}+\frac{1}{2^2}+\cdots+\frac{1}{2^{n-1}}<1+\sum_{n=0}^{\infty}\frac{1}{2^n}=1+\frac{1}{1-(1/2)}=3
\end{align*}
ہو گا۔ یوں \عددی{\sum_{n=0}^{\infty}\tfrac{1}{n!}} کے تمام جزوی مجموعات \عددی{3} سے کم ہیں لہٰذا \عددی{\sum_{n=0}^{\infty}\tfrac{1}{n!}} مرتکز ہو گا۔

\عددی{\sum_{n=0}^{\infty}\tfrac{1}{n!}} کے جزوی مجموعات کی بالائی حد بندی \عددی{3} ہونے کا یہ مطلب نہیں کہ یہ تسلسل \عددی{3} پر مرتکز ہو گا۔ جیسا ہم آگے اس باب میں دیکھیں گے یہ تسلسل \عددی{e} پر مرتکز ہے۔ 
\انتہا{مثال}
%======================

\جزوحصہء{بلا واسطہ پرکھ موازنہ}
ہم نے مثال \حوالہ{مثال_تسلسل_پرکھ_موازنہ_استعمال} میں ارتکاز کو پرکھ موازنہ سے ثابت کیا۔ہم نے دیے گئے تسلسل کے اجزاء کا ایک مرتکز تسلسل کے مطابقتی اجزاء کے ساتھ موازنہ کرتے ہوئے ایسا کیا۔ اس طریقہ کار سے کئی تراکیب حاصل کئے جا سکتے ہیں جنہیں \اصطلاح{پرکھ موازنہ}\فرہنگ{پرکھ!موازنہ}\حاشیہب{comparison tests}\فرہنگ{test!comparison} کہتے ہیں۔

\ابتدا{پرکھ}\موٹا{غیر منفی اجزاء کے تسلسل کا بلا واسطہ پرکھ موازنہ}\\
فرض کریں \عددی{\sum a_n} ایک ایسا تسلسل ہے جس میں کوئی منفی جزو نہیں پایا جاتا ہے۔
\begin{enumerate}[a.]
\item
اگر ایسا مرتکز تسلسل \عددی{\sum c_n} پایا جاتا ہو کہ  تمام \عددی{n>N}، جہاں \عددی{N} کوئی عدد صحیح ہے،  کے لئے \عددی{a_n\le c_n} ہو تب تسلسل \عددی{\sum a_n} مرتکز ہو گا۔
\item
اگر غیر منفی اجزاء کا ایسا منفرج تسلسل \عددی{\sum d_n} پایا جاتا ہو کہ  تمام \عددی{n>N}، جہاں \عددی{N} کوئی عدد صحیح ہے،  کے لئے \عددی{a_n\ge d_n} ہو تب  تسلسل \عددی{\sum a_n} منفرج ہو گا۔
\end{enumerate}
\انتہا{پرکھ}
%===========================
\ابتدا{ثبوت پرکھ}
جزو-الف میں جزوی مجموعات \عددی{\sum a_n} کو درج ذیل اوپر سے محدود کرتا ہے
\begin{align*}
M=a_1+a_2+\cdots+a_n+\sum_{n=N+1}^{\infty}c_n
\end{align*}
لہٰذا یہ غیر گھٹتا ترتیب دیتے ہیں جس کا حد \عددی{L\le M} ہے۔

جزو-ب میں جزوی مجموعات \عددی{\sum a_n} اوپر سے محدود نہیں ہیں۔ اگر یہ اوپر سے محدود ہوتے تب جزوی مجموعہ \عددی{\sum d_n}  کو درج ذیل اوپر سے محدود کرتا
\begin{align*}
M'=d_1+d_2+\cdots+d_N+\sum_{n=N+1}^{\infty}a_n
\end{align*} 
اور \عددی{\sum d_n} کو انفراج کی بجائے مرتکز ہونا ہوتا۔
\انتہا{ثبوت پرکھ}
%========================

بلا واسطہ پرکھ موازنہ کو تسلسل پر لاگو کرنے کے لئے ہمیں تسلسل کے ابتدائی اجزاء شامل کرنے ہوں گے۔ ہم کسی بھی اشاریہ \عددی{N} سے پرکھ شروع کر سکتے ہیں جب تک ہم اس کے بعد کے تمام اجزاء شامل کریں۔

\ابتدا{مثال}
کیا درج ذیل تسلسل مرتکز ہے؟
\begin{align*}
5+\frac{2}{3}+1+\frac{1}{7}+\frac{1}{2}+\frac{1}{3!}+\frac{1}{4!}+\cdots+\frac{1}{k!}+\cdots
\end{align*}
حل:\quad
ہم ابتدائی چار اجزاء نظر انداز کر کے باقی اجزاء کا مرتکز ہندسی تسلسل \عددی{\sum_{n=1}^{\infty}\tfrac{1}{2^n}} کے اجزاء کے ساتھ موازنہ کرتے ہیں۔  ہم درج ذیل دیکھتے ہیں۔
\begin{align*}
\frac{1}{2}+\frac{1}{3!}+\frac{1}{4!}+\cdots\le \frac{1}{2}+\frac{1}{4}+\frac{1}{8}+\cdots
\end{align*}
یوں دیا گیا تسلسل بلا واسطہ پرکھ موازنہ کے تحت مرتکز ہو گا۔
\انتہا{مثال}
%===================

بلا واسطہ پرکھ موازنہ استعمال کرنے کی خاطر ہمارے پاس  مرتکز اور منفرج تسلسل کی فہرست ہونی چاہیے۔  اب تک ہم درج ذیل جانتے ہیں:
\begin{center}
\renewcommand{\arraystretch}{2.5}
\begin{tabular}{r|r}
\toprule
مرتکز تسلسل& منفرج تسلسل\\
\midrule
ہندسی تسلسل جس میں \عددی{\abs{r}<1} ہو& ہندسی تسلسل جس میں \عددی{\abs{r}\ge 1} ہو\\
دوربینی تسلسل مثلاً \عددی{\sum_{n=1}^{\infty}\tfrac{1}{n(n+1)}} & ہارمونی تسلسل \عددی{\sum_{n=1}^{\infty}\tfrac{1}{n}} \\
تسلسل \عددی{\sum_{n=0}^{\infty}\tfrac{1}{n!}} & 
\begin{minipage}{0.45\textwidth}
کوئی بھی تسلسل \عددی{\sum a_n} جس کے لئے \عددی{\lim_{n\to\infty}a_n} غیر موجود ہو یا \عددی{\lim_{n\to\infty}a_n\ne 0} ہو
\end{minipage}\\
\عددی{p} تسلسل \عددی{\sum_{n=1}^{\infty}\tfrac{1}{n^p}} جہاں \عددی{p>1} ہو & \عددی{p} تسلسل
 \عددی{\sum_{n=1}^{\infty}\tfrac{1}{n^p}} جہاں \عددی{p\le1} ہو\\
\bottomrule
 \end{tabular}
\end{center}

\جزوحصہء{پرکھ موازنہ حد}
ہم اب ایسے پرکھ موازنہ پر غور کرتے ہیں  جس کا استعمال ان تسلسل میں بالخصوص آسان ثابت ہوتا ہے جن میں \عددی{a_n} اشاریہ \عددی{n} کا ناطق تفاعل ہو۔

فرض کریں ہم درج ذیل تسلسل کے ارتکاز پر غور کرنا چاہتے ہیں۔
\begin{align*}
\sum_{n=2}^{\infty}\frac{8n^3+100n^2+1000}{2n^6-n+5} \quad \text{\RL{(ب)}}\quad \quad \sum_{n=2}^{\infty}\frac{2n}{n^2-n+1}\quad \text{\RL{(الف)}}
\end{align*}
ارتکاز یا انفراج جاننے میں صرف دم کارآمد ہوتی ہے۔ جب \عددی{n} بہت بڑا ہو تب نسب نما اور شمار کنندہ میں \عددی{n} کی بلند ترین طاقت سب سے زیادہ اہم ہوں گے۔ یوں (الف) میں بڑے \عددی{n}  کے لئے
\begin{align*}
a_n=\frac{2n}{n^2-n+1}
\end{align*}
کا رویہ \عددی{\tfrac{2n}{n^2}=\tfrac{2}{n}} کی طرح کا ہو گا۔ چونکہ \عددی{\sum\tfrac{1}{n}} منفرج ہے لہٰذا ہم توقع کرتے ہیں کہ \عددی{\sum a_n} بھی منفرج ہو گا۔

اسی طرح (ب) میں بڑے \عددی{n} کے لئے
\begin{align*}
\frac{8n^3+100n^2+1000}{2n^6-n+5} 
\end{align*}
 کا رویہ \عددی{\tfrac{8n^3}{2n^6}=\tfrac{4}{n^3}} کی طرح کا ہو گا۔ چونکہ \عددی{\sum\tfrac{4}{n^3}} مرتکز  ہے(یہ مرتکز \عددی{p} تسلسل کا چار گننا ہے)  لہٰذا ہم توقع کرتے ہیں کہ تسلسل  \عددی{\sum a_n} بھی مرتکز ہو گا۔

درج ذیل پرکھ کے تحت ہماری توقعات دونوں صورتوں میں درست ہیں۔

