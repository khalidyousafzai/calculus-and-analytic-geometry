\حصہ{یولر کی اعدادی ترکیب؛ میدان ڈھلوان}
بعض اوقات ہم ابتدائی قیمت مسئلہ \عددی{y=f(x,y),\,y(x_0)=y_0} کا بالکل درست حل معلوم نہیں کر سکتے ہیں یا نہیں کرنا چاہتے ہیں۔ ایسی صورت میں ہم عموماً کمپیوٹر استعمال کرتے ہوئے موزوں وقفہ پر ہر \عددی{x} کے لئے \عددی{y} کی تخمینی قیمت   تلاش کر سکتے ہیں۔ ایسے حل کو ہم \اصطلاح{اعدادی حل}\فرہنگ{اعدادی!حل}\حاشیہب{numerical solution}\فرہنگ{numerical!solution} کہتے ہیں اور اس حل کو حاصل کرنے کے طریقہ کو \اصطلاح{اعدادی ترکیب}\فرہنگ{اعدادی!ترکیب}\حاشیہب{numerical method}\فرہنگ{numerical!method} کہتے ہیں۔ اعدادی تراکیب عموماً بہت کم وقت میں  درست نتائج دیتے ہیں اور جہاں بھی تحلیلی حل نا ممکن، غیر ضروری یا پیچیدہ ہو، وہاں اعدادی تراکیب کو ترجیح دی جاتی ہے۔ اس حصہ میں ہم ایسے ایک ترکیب پر غور کرتے ہیں جس کو \اصطلاح{ترکیب یولر}\فرہنگ{ترکیب یولر}\حاشیہب{Euler' method}\فرہنگ{Euler's method} کہتے ہیں۔ 

\جزوحصہء{میدان ڈھلوان}
ابتدائی معلومات \عددی{y(x_0)=y_0} تفرقی مساوات \عددی{y'=f(x,y)} پر یہ شرط مسلط کرتی ہے کہ تفرقی مساوات کا حل نقطہ  \عددی{(x_0,y_0)} سے گزرے گا اور اس نقطہ ہر حل کا ڈھلوان \عددی{f(x_0,y_0)} ہو گا۔ ہم \عددی{f} کے دائرہ کار میں منتخب نقطوں \عددی{(x,y)} پر \عددی{f(x,y)} ڈھلوان والی قلیل سیدھے خطوط بنا کر اس ڈھلوان کو تصویری جامہ پہنا سکتے ہیں۔ نقطہ \عددی{(x,y)} سے گزرتے ہوئے حل کی ڈھلوان اس نقطہ پر بنائے گئے قلیل خط کی ڈھلوان کے برابر ہو گا لہٰذا یہ خط اس نقطہ پر حل کا مماس ہو گا۔ہم ان مماس پر نظر دوڑا کر حل کے رویہ کو جان سکتے ہیں۔

قلم و کاغذ کے ساتھ میدان ڈھلوان بنانا تھکا دینے والا کام ہے۔ اس کتاب میں تمام مثالوں میں میدان ڈھلوان کمپیوٹر کی مدد سے بنائے گئے۔ آئیں  کمپیوٹر کی مدد سے منحنی حل  کے حصول پر غور کریں۔

\جزوحصہء{خط بندی کا استعمال}
دیے گئے تفرقی مساوات \عددی{\tfrac{\dif y}{\dif x}=f(x,y)} اور ابتدائی معلومات \عددی{y(x_0)=y_0} سے ہم منحنی حل \عددی{y=y(x)} کی تخمین درج ذیل خط بندی
\begin{align*}
L(x)=y(x_0)+\left.\frac{\dif y}{\dif x}\right\vert_{x=x_0} (x-x_0)
\end{align*}
یا
\begin{align*}
L(x)=y_0+f(x_0,y_0)(x-x_0)
\end{align*}
سے  کر سکتے ہیں۔ \عددی{x_0} کی بالکل پڑوس میں تفاعل \عددی{L(x)} اصل حل \عددی{y(x)} کا اچھا تخمین ہو گا۔ ترکیب یولر میں اس طرح کے خط بندیوں کو آپس میں جوڑ کر زیادہ لمبے فاصلہ کے لئے حل تلاش کیا جاتا ہے۔ اب اس ترکیب پر غور کرتے ہیں۔

ہم جانتے ہیں کہ نقطہ \عددی{(x_0,y_0)} منحنی حل پر پایا جاتا ہے۔ فرض کریں ہم غیر تابع متغیر کی ایک نئی قیمت \عددی{x_0=x_0+\dif x} منتخب کرتے ہیں۔  اگر بڑھوتری \عددی{\dif x} بہت کم ہو تب
\begin{align*}
y_1=L(x_1)=y_0+f(x_0,y_0)\dif x
\end{align*}
اصل حل \عددی{y=y(x_1)} کا اچھا تخمین ہو گا۔ یوں نقطہ \عددی{(x_0,y_0)} سے، جو ٹھیک منحنی حل پر پایا جاتا ہے، ہم نقطہ \عددی{(x_1,y_1)} حاصل کر پائیں ہیں جو منحنی حل پر نقطہ \عددی{(x_1,y(x_1))} کے بہت قریب ہو گا۔

نقطہ \عددی{(x_1,y_1)} اور ڈھلوان \عددی{f(x_1,y_1)} لیتے ہوئے ہم دوسرا قدم لیتے ہیں۔ ہم \عددی{x_2=x_1+\dif x} لے کر 
\begin{align*}
y_2=y_1+f(x_1,y_1)\dif x
\end{align*}
سے، منحنی حل \عددی{y=y(x)} کے ساتھ، دوسرا تخمینی نقطہ \عددی{(x_2,y_2)} حاصل کرتے ہیں۔ اسی طرح چلتے ہوئے تیسرے قدم پر  ہم نقطہ \عددی{(x_2,y_2)} اور ڈھلوان \عددی{f(x_2,y_2)} سے
\begin{align*}
y_3=y_2+f(x_2,y_2)\dif x
\end{align*}
حاصل کرتے ہیں، وغیرہ وغیرہ۔

\ابتدا{مثال}\شناخت{مثال_ماورائی_ترکیب_یولر_الف}
درج ذیل ابتدائی قیمت مسئلے کے لئے ترکیب یولر کی مدد سے  ابتدائی تین تخمین \عددی{y_1}، \عددی{y_2} اور \عددی{y_3}  حاصل کریں۔
\begin{align*}
y'=1+y,\quad y(0)=1
\end{align*}
ابتدا \عددی{x_0=0} سے کریں اور \عددی{\dif x=0.1} لیں۔

حل:\quad
\موٹا{قدم اول:}
\begin{align*}
y_1&=y_0+f(x_0,y_0)\dif x\\
&=y_0+(1+y_0)\dif x\\
&=1+(1+1)(0.1)=1.2
\end{align*}
\موٹا{قدم دوم:}
\begin{align*}
y_2&=y_1+f(x_1,y_1)\dif x\\
&=y_1+(1+y_1)\dif x\\
&=1.2+(1+1.2)(0.1)=1.42
\end{align*}
\موٹا{قدم سوم:}
\begin{align*}
y_3&=y_2+f(x_2,y_2)\dif x\\
&=y_2+(1+y_2)\dif x\\
&=1.42+(1+1.42)(0.1)=1.662
\end{align*}
\انتہا{مثال}
%==================

\جزوحصہء{ترکیب یولر}
ترکیب یولر سے ابتدائی قیمت مسئلہ
\begin{align*}
y'=f(x,y),\quad y(x_0)=y_0
\end{align*}
کے حل کی تخمینی قیمتیں حاصل ہوتی ہیں۔ اگر ہم غیر تابع متغیر کے منتخب کردہ قیمتوں کے بیچ یکساں فاصلہ رکھیں اور \عددی{n} عدد ایسے نقطے منتخب کریں تب درج ذیل ہوں گے۔
\begin{gather}
\begin{aligned}\label{مساوات_ماورائی_ترکیب_یولر_ایکس}
x_1&=x_0+\dif x\\
x_2&=x_1+\dif x\\
&\vdots\\
x_n&=x_{n-1}+\dif x
\end{aligned}
\end{gather}
اس کے بعد ہم متواتر درج ذیل حاصل کرتے ہیں۔
\begin{gather}
\begin{aligned}\label{مساوات_ماورائی_ترکیب_یولر_وائے}
y_1&=y_0+f(x_0,y_0)\dif x\\
y_2&=y_1+f(x_1,y_1)\dif x\\
&\vdots\\
y_n&=y_{n-1}+f(x_{n-1},y_{n-1})\dif x
\end{aligned}
\end{gather}
ہم قدموں کی تعداد \عددی{n} جتنی چاہیں رکھ سکتے ہیں، البتہ، \عددی{n} کو بہت بڑا رکھنے سے نتائج میں خلل جمع ہو گا۔

\ابتدا{مثال}\شناخت{مثال_ماورائی_یولر_موازنہ}
درج ذیل ابتدائی قیمت مسئلہ کے لئے وقفہ \عددی{0\le x\le 1} پر  ترکیب یولر کی درستگی پر غور کریں۔
\begin{align*}
y'=1+y,\quad y(0)=1
\end{align*}
\عددی{x_0=0} سے شروع کریں اور \عددی{\dif x=0.1} لیں۔

حل:\quad
اس مسئلے کا بالکل درست تحلیلی حل \عددی{y=2e^x-1} ہے۔ جدول \حوالہ{جدول_ماورائی_یولر_موازنہ} میں، مساوات \حوالہ{مساوات_ماورائی_ترکیب_یولر_ایکس} اور مساوات \حوالہ{مساوات_ماورائی_ترکیب_یولر_وائے} استعمال کرتے ہوئے، ترکیب یولر سے حاصل \عددی{4} اعشاریہ درست تخمینی نتائج کا تحلیل حل کے ساتھ موازنہ کیا گیا ہے۔ دس قدم بعد \عددی{x=1} تک پہنچ کر ترکیب یولر سے حاصل تخمینی حل میں \عددی{\SI{5.6}{\percent}} خلل پایا جاتا ہے۔
\begin{table}
\caption{تحلیل حل اور ترکیب یولر سے حاصل تخمینی حل کا موازنہ (مثال \حوالہ{مثال_ماورائی_یولر_موازنہ})}
\label{جدول_ماورائی_یولر_موازنہ}
\centering
\begin{tabular}{LLLL}
\toprule
x&y_{\text{تخمینی}}&y_{\text{تحلیلی}}&y_{\text{تحلیلی}}-y_{\text{تخمینی}}=\text{\RL{خلل}}\\
\midrule
0&1&1&0\\
0.1&1.2&1.2103&0.0103\\
0.2&1.42&1.4428&0.0228\\
0.3&1.662&1.6997&0.0377\\
0.4&1.9282&1.9836&0.0554\\
0.5&2.2210&2.2974&0.0764\\
0.6&2.5431&2.6442&0.1011\\
0.7&2.8974&3.0275&0.1301\\
0.8&3.2872&3.4511&0.1639\\
0.9&3.7159&3.9192&0.2033\\
1.0&4.1875&4.4366&0.2491\\
\bottomrule
\end{tabular}
\end{table}
\انتہا{مثال}
%==================

\ابتدا{مثال}

\انتہا{مثال}
%=====================
