\حصہ{الٹ تکونیاتی تفاعل کے تفرق؛ تکمل}
الٹ تکونیاتی تفاعل مختلف اقسام کے تفاعل، جو انجینئری، طبیعیات اور ریاضیات میں رونما ہوتے ہیں،  کے الٹ تفرق مہیا کرتے ہیں۔ اس حصہ میں ہم الٹ تکونیاتی تفاعل کے  تفرق حاصل کرتے ہیں اور متعلقہ تکملات پر غور کرتے ہیں۔

\ابتدا{مثال}
\begin{align*}
\frac{\dif}{\dif x}\sin^{-1}(x^2)&=\frac{1}{\sqrt{1-(x^2)^2}}\cdot \frac{\dif}{\dif x}(x^2)=\frac{2x}{\sqrt{1-x^2}}&&\text{\RL{(ا)}}\\
\frac{\dif}{\dif x}\tan^{-1}\sqrt{x+1}&=\frac{1}{1+(\sqrt{x+1})^2}\cdot \frac{\dif}{\dif x}(\sqrt{x+1})&&\text{\RL{(ب)}}\\
&=\frac{1}{x+2}\cdot\frac{1}{2\sqrt{x+1}}=\frac{1}{2\sqrt{x+1}(x+2)}\\
\frac{\dif}{\dif x}\sec^{-1}(-3x)&=\frac{1}{\abs{-3x}\sqrt{(-3x)^2-1}}\cdot\frac{\dif}{\dif x}(-3x)&&\text{\RL{\text{\RL{(ج)}}}}\\
&=\frac{-3}{\abs{3x}\sqrt{9x^2-1}}=\frac{-1}{\abs{x}\sqrt{9x^2-1}}
\end{align*}
\انتہا{مثال}
%========================
\ابتدا{مثال}
\begin{align*}
\int_0^1\frac{e^{\tan x}}{1+x^2}\dif x=\int_0^{\pi/4}e^u\dif u&&u=\tan^{-1}u\\
&=\left.e^u\right]_0^{\pi/4}=e^{\pi/4}-1
\end{align*}
\انتہا{مثال}
%======================

الٹ تکونیاتی تفاعل کے تفرق درج ذیل ہیں۔
\begin{align}
\frac{\dif\, (\sin^{-1}u)}{\dif x}&=\frac{\frac{\dif u}{\dif x}}{\sqrt{1-u^2}},\quad \abs{u}<1\label{مساوات_ماورائی_الٹ_تکونیاتی_تفرق_الف}\\
\frac{\dif\, (\cos^{-1}u)}{\dif x}&=-\frac{\frac{\dif u}{\dif x}}{\sqrt{1-u^2}},\quad \abs{u}<1\label{مساوات_ماورائی_الٹ_تکونیاتی_تفرق_ب}\\
\frac{\dif\,(\tan^{-1}u)}{\dif x}&=\frac{\frac{\dif u}{\dif x}}{1+u^2}\label{مساوات_ماورائی_الٹ_تکونیاتی_تفرق_پ}\\
\frac{\dif\, (\cot^{-1}u)}{\dif x}&=-\frac{\frac{\dif u}{\dif x}}{1+u^2}\label{مساوات_ماورائی_الٹ_تکونیاتی_تفرق_ت}\\
\frac{\dif\, (\sec^{-1}u)}{\dif x}&=\frac{\frac{\dif u}{\dif x}}{\abs{u}\sqrt{u^2-1}},\quad \abs{u}>1\label{مساوات_ماورائی_الٹ_تکونیاتی_تفرق_ٹ}\\
\frac{\dif\, (\csc^{-1}u)}{\dif x}&=\frac{-\frac{\dif u}{\dif x}}{\abs{u}\sqrt{u^2-1}},\quad \abs{u}>1\label{مساوات_ماورائی_الٹ_تکونیاتی_تفرق_ث}
\end{align}

آئیں مساوات \حوالہ{مساوات_ماورائی_الٹ_تکونیاتی_تفرق_الف} اور مساوات \حوالہ{مساوات_ماورائی_الٹ_تکونیاتی_تفرق_ٹ} کو حاصل کرتے ہیں۔ مساوات \حوالہ{مساوات_ماورائی_الٹ_تکونیاتی_تفرق_پ} کو بھی اسی طرح حاصل کیا جا سکتا ہے۔ مساوات \حوالہ{مساوات_ماورائی_الٹ_تکونیاتی_تفرق_ب}، مساوات \حوالہ{مساوات_ماورائی_الٹ_تکونیاتی_تفرق_ت} اور مساوات \حوالہ{مساوات_ماورائی_الٹ_تکونیاتی_تفرق_ث} کو موزوں تماثل تفرق کر کے حاصل کیا جا سکتا ہے۔

\جزوحصہء{تفاعل \عددی{y=\sin^{-1}u} کا تفرق}
ہم جانتے ہیں کہ وقفہ \عددی{-\tfrac{y}{2}<y<\tfrac{y}{2}} میں تفاعل \عددی{x=\sin y} قابل تفرق ہے اور اس کا تفرق، یعنی کوسائن، اس وقفہ پر مثبت ہے۔ یوں مسئلہ \حوالہ{مسئلہ_ماورائی_الٹ_تفرق_قاعدہ} ہمیں یقین دہانی کراتا ہے  کہ پورے وقفہ \عددی{-1<x<1} پر الٹ تفاعل \عددی{y=\sin^{-1}x} قابل تفرق ہو گا۔ چونکہ نقطہ \عددی{x=1} اور \عددی{x=-1} پر اس کے ترسیم کے مماس انتصابی ہیں (شکل \حوالہ{شکل_ماورائی_انتصابی_مماس})لہٰذا ان نقطوں  پر ہم  الٹ تفاعل \عددی{y=\sin^{-1}x} کو قابل تفرق تصور نہیں کر سکتے ہیں۔ 
\begin{figure}
\centering
\begin{minipage}{0.45\textwidth}
\centering
\begin{tikzpicture}[font=\small,declare function={f(\x)=sin(deg(\x));}]
\pgfmathsetmacro{\k}{pi/2}
\begin{axis}[clip=false,small,axis lines=middle,xlabel={$x$},ylabel={$y$},xtick={-1,1}, ytick={-\k,\k}, yticklabels={$-\tfrac{\pi}{2}$,$\tfrac{\pi}{2}$}, xlabel style={at={(current axis.right of origin)},anchor=west}, ylabel style={at={(current axis.above origin)},anchor=south},enlargelimits=true]
\addplot[domain=-pi/2:pi/2]({f(x)},x)node[pos=0,circ]{}node[pos=1,circ]{};
\addplot[dashed,domain=-pi/2:-pi]({f(x)},x);
\addplot[dashed,domain=pi/2:pi]({f(x)},x)node[pos=0.5,above]{$x=\sin y$};
\draw(1,\k-1)--(1,\k+1);
\draw(-1,-\k-0.75)--(-1,-\k+0.75);
\draw(-0.3,\k)node[above left]{$\begin{aligned} y=\sin^{-1}x\\  -1\le x\le 1\quad \text{\RL{دائرہ کار}}\\ -\tfrac{\pi}{2}\le y\le \tfrac{\pi}{2}\quad \text{\RL{سعت}} \end{aligned}$};
\end{axis}
\end{tikzpicture}
\caption{تفاعل \عددی{y=\sin^{-1}x} کی ترسیم کے مماس نقطہ \عددی{x=-1} اور \عددی{x=1} پر انتصابی ہیں۔}
\label{شکل_ماورائی_انتصابی_مماس}
\end{minipage}\hfill
\begin{minipage}{0.45\textwidth}
\centering
\begin{tikzpicture}[font=\small]
\pgfmathsetmacro{\len}{4}
\pgfmathsetmacro{\ang}{30}
\draw(0,0)--++(\ang:\len)coordinate[](kT)node[pos=0.5,above left]{$1$}--($(0,0)!(kT)!(2,0)$)node[pos=0.5,right]{$x$}--(0,0)node[pos=0.5,below]{$\sqrt{1-x^2}$};
\draw[-stealth]([shift={(0:0.8)}]0,0) arc (0:\ang:0.8);
\draw(\ang/2:1.1)node[]{$y$};
\end{tikzpicture}
\caption{
اس حوالہ مثلث میں \عددی{\sin y=\tfrac{x}{1}=x} اور \عددی{\cos y=\tfrac{\sqrt{1-x^2}}{1}=\sqrt{1-x^2}} ہو گا۔
}
\label{شکل_ماورائی_حوالہ_مثلث_الٹ_تکونیاتی}
\end{minipage}
\end{figure}

ہم \عددی{y=\sin^{-1}x} کا تفرق درج ذیل طریقہ سے حاصل کرتے ہی:
\begin{align*}
\sin y&=x&&y=\sin^{-1}x\Leftrightarrow\sin y=x\\
\frac{\dif}{\dif x}(\sin y)&=1&&\text{\RL{دونوں اطراف کا $x$ کے لحاظ سے تفرق}}\\
\cos y\frac{\dif y}{\dif x}&=1&&\text{\RL{زنجیری قاعدہ}}\\
\frac{\dif y}{\dif x}&=\frac{1}{\cos y}&&\text{\RL{وقفہ پر $\cos y>0$ ہے لہٰذا تقسیم کیا جا سکتا ہے}}\\
&=\frac{1}{\sqrt{1-x^2}}&&\text{\RL{شکل \حوالہ{شکل_ماورائی_حوالہ_مثلث_الٹ_تکونیاتی}}}
\end{align*}
یوں \عددی{x} کے لحاظ سے \عددی{y=\sin^{-1}x} کو تفرق درج ذیل ہو گا۔
\begin{align*}
\frac{\dif}{\dif x}(\sin^{-1}x)=\frac{1}{\sqrt{1-x^2}}
\end{align*}
اگر \عددی{x} کے لحاظ سے \عددی{u} قابل تفرق تفاعل ہو تب \عددی{y=\sin^{-1}u} کو زنجیری قاعدہ 
\begin{align*}
\frac{\dif y}{\dif x}=\frac{\dif y}{\dif u}\frac{\dif u}{\dif x}
\end{align*}
کی اطلاق سے درج ذیل حاصل ہو گا۔
\begin{align*}
\frac{\dif}{\dif x}(\sin^{-1}u)&=\frac{1}{\sqrt{1-u^2}}\frac{\dif u}{\dif x}&&\abs{u}<1
\end{align*}

\جزوحصہء{تفاعل \عددی{y=\sec^{-1}u} کا تفرق}
ہم \عددی{y=\sec^{-1}x,\, \abs{x}>1} کا تفرق بھی اسی طرح حاصل کرتے ہیں۔
\begin{align*}
\sec y&=x&&y=\sec^{-1}x\Leftrightarrow \sec y=x\\
\frac{\dif}{\dif x}(\sec y)&=1&&\text{\RL{$x$ کے لحاظ سے دونوں اطراف کا تفرق}}\\
\sec y\tan y\frac{\dif y}{\dif x}&=1&&\text{\RL{زنجیری قاعدہ}}\\
\frac{\dif y}{\dif x}&=\frac{1}{\sec y\tan y}\\
&=\mp\frac{1}{x\sqrt{x^2-1}}&&\text{\RL{شکل \حوالہ{شکل_ماورائی_تکونیاتی_تفرق_سیکنٹ}}}
\end{align*}
درج بالا میں تیسرے قدم پر چونکہ \عددی{\abs{x}>1} ہے لہٰذا \عددی{y} وقفہ \عددی{(0,\pi/2)\cup(\pi/2,\pi)} میں پایا جائے گا جس کی بنا \عددی{\sec y\tan y\ne 0} ہو گا لہٰذا دونوں اطراف کو غیر صفر \عددی{\sec y\tan y} سے تقسیم کیا جا سکتا ہے۔

\begin{figure}
\centering
\begin{subfigure}{0.45\textwidth}
\centering
\begin{tikzpicture}[font=\small]
\pgfmathsetmacro{\len}{3}
\pgfmathsetmacro{\ang}{30}
\draw[-latex](-0.25,0)--(4,0)node[right]{$x$};
\draw[-latex](0,-0.25)--(0,2)node[above]{$y$};
\draw[thick](0,0)--++(\ang:\len)coordinate[](kT)node[pos=0.5,above left]{$x$}--($(0,0)!(kT)!(4,0)$)node[pos=0.5,right]{$\sqrt{x^2-1}$}--(0,0)node[pos=0.5,below]{$1$};
\draw[-stealth] ([shift={(0:0.8)}]0,0) arc (0:\ang:0.8);
\draw(\ang/2:1)node[]{$y$};
\end{tikzpicture}
\end{subfigure}\hfill
\begin{subfigure}{0.45\textwidth}
\centering
\begin{tikzpicture}[font=\small]
\pgfmathsetmacro{\len}{3}
\pgfmathsetmacro{\ang}{150}
\draw[-latex](-4,0)--(0.75,0)node[right]{$x$};
\draw[-latex](0,-0.25)--(0,2)node[above]{$y$};
\draw[thick](0,0)--++(\ang:\len)coordinate[](kT)node[pos=0.75,above right,yshift=-1ex]{$-x(x<0)$}--($(0,0)!(kT)!(4,0)$)node[pos=0.5,left]{$\sqrt{x^2-1}$}--(0,0)node[pos=0.5,below]{$-1$};
\draw[-stealth] ([shift={(0:0.5)}]0,0) arc (0:\ang:0.5);
\draw(\ang/3:0.8)node[]{$y$};
\end{tikzpicture}
\end{subfigure}
\caption{
دونوں ربع میں \عددی{\sin y=x} ہے۔ ربع اول میں \عددی{\tan y=\tfrac{\sqrt{x^2-1}}{1}=\sqrt{x^2-1}} ہے جبکہ ربع دوم میں \عددی{\tan y=\tfrac{\sqrt{x^2-1}}{(-1)}=-\sqrt{x^2-1}} ہے۔ 
}
\label{شکل_ماورائی_تکونیاتی_تفرق_سیکنٹ}
\end{figure}

علامت کے بارے میں ہم کیا کر سکتے ہیں؟ ہم دیکھتے ہیں (شکل \حوالہ{شکل_ماورائی_ڈھلوان_مثبت_سیکنٹ}) کہ \عددی{\abs{x}>1} کے لئے \عددی{y=\sec^{-1}x} کی ترسیم کی ڈھلوان مثبت رہتی ہے لہٰذا درج ذیل لکھا جا سکتا ہے۔
\begin{align}\label{مساوات_ماورائی_سیکنٹ_کلیہ_الف}
\frac{\dif}{\dif x}(\sec^{-1}x)&=
\begin{cases}
\phantom{-}\frac{1}{x\sqrt{x^2-1}}&x>1\\
-\frac{1}{x\sqrt{x^2-1}}&x<-1
\end{cases}
\end{align}
مطلق قیمت استعمال کرتے ہوئے  ہم مساوات \حوالہ{مساوات_ماورائی_سیکنٹ_کلیہ_الف} کو درج ذیل لکھ سکتے ہیں۔
\begin{align*}
\frac{\dif}{\dif x}(\sec^{-1}x)&=\frac{1}{\abs{x}\sqrt{x^2-1}}&&\abs{x}>1
\end{align*}
اگر \عددی{\abs{u}>1} ہو اور \عددی{x} کا \عددی{u} قابل تفرق تفاعل ہو تب زنجیری قاعدہ کے استعمال سے درج ذیل حاصل ہو گا۔
\begin{align*}
\frac{\dif}{\dif x}(\sec^{-1}u)&=\frac{1}{\abs{u}\sqrt{u^2-1}}\frac{\dif u}{\dif x}&&\abs{u}>1
\end{align*}

\begin{figure}
\centering
\begin{tikzpicture}[declare function={f(\x)=sec(deg(\x));}]
\begin{axis}[clip=false,small,axis lines=middle,xlabel={$x$},ylabel={$y$},xtick={-1,1}, ytick={3.142}, yticklabels={$\pi$}, xlabel style={at={(current axis.right of origin)},anchor=west}, ylabel style={at={(current axis.above origin)},anchor=south},enlargelimits=true]
\addplot[domain=0:pi/2-0.2,smooth]({f(x)},x);
\addplot[domain=pi:pi/2+0.2,smooth]({f(x)},x);
\draw({f(pi/2+0.2)},pi/2)--({f(pi/2-0.2)},pi/2);
\draw(0,3/4*pi)node[right]{$y=\sec^{-1}x$};
\end{axis}
\end{tikzpicture}
\caption{
قوس \عددی{y=\sec^{-1}x} کی ڈھلوان \عددی{x<-1} اور \عددی{x>1} دونوں کے لئے مثبت ہے۔
}
\label{شکل_ماورائی_ڈھلوان_مثبت_سیکنٹ}
\end{figure}

\جزوحصہء{کلیات تکمل}
ہم مساوات \حوالہ{مساوات_ماورائی_الٹ_تکونیاتی_تفرق_الف}، مساوات \حوالہ{مساوات_ماورائی_الٹ_تکونیاتی_تفرق_پ} اور مساوات \حوالہ{مساوات_ماورائی_الٹ_تکونیاتی_تفرق_ٹ} جہاں \عددی{a=1} ہے،  سے تکمل کے درج ذیل تین اہم کلیات حاصل ہوتے ہیں جہاں  \عددی{a\ne 0}  مستقل ہے۔
\begin{align}
\int\frac{\dif u}{\sqrt{a^2-u^2}}&=\sin^{-1}(\tfrac{u}{a})+C&&\text{\RL{$u^2<a^2$ کے لئے درست ہے}}\label{مساوات_ماورائی_کلیات_تکمل_الف}\\
\int\frac{\dif u}{a^2+u^2}&=\frac{1}{a}\tan^{-1}(\tfrac{u}{a})+C&&\text{\RL{تمام $u$ کے لئے درست ہے}}\label{مساوات_ماورائی_کلیات_تکمل_ب}\\
\int\frac{\dif u}{u\sqrt{u^2-a^2}}&=\frac{1}{a}\sec^{-1}\abs{\tfrac{u}{a}}+C&&\text{\RL{$u^2>a^2$ کے لئے درست ہے}}\label{مساوات_ماورائی_کلیات_تکمل_پ}
\end{align}
تکمل کے درج بالا کلیات  کے دائیں ہاتھ کا تفرق لے کر ان کی تصدیق کی جا سکتی ہے۔

\ابتدا{مثال}
\begin{align*}
\int_{\sqrt{2}/2}^{\sqrt{3}/2}\frac{\dif x}{\sqrt{1-x^2}}&=\left.\sin^{-1}(x)\right]_{\sqrt{2}/2}^{\sqrt{3}/2}\\
&=\sin^{-1}(\tfrac{\sqrt{3}}{2})-\sin^{-1}(\tfrac{\sqrt{2}}{2})=\frac{\pi}{3}-\frac{\pi}{4}=\frac{\pi}{12}
&&\text{\RL{(ا)}}\\
\int_0^1\frac{\dif x}{1+x^2}&=\left.\tan^{-1}(x)\right]_0^1=\tan^{-1}(1)-\tan^{-1}(0)=\frac{\pi}{4}-0=\frac{\pi}{4}&&\text{\RL{(ب)}}\\
\int_{\tfrac{2}{\sqrt{3}}}^{\sqrt{2}}\frac{\dif x}{x\sqrt{x^2-1}}&=\left.\sec^{-1}(x)\right]_{2/\sqrt{3}}^{\sqrt{2}}=\frac{\pi}{4}-\frac{\pi}{6}=\frac{\pi}{12}&&\text{\RL{(ج)}}
\end{align*}
\انتہا{مثال}
%======================== 
\ابتدا{مثال}
\begin{align*}
\text{\RL{(ا)}}\quad\int\frac{\dif x}{\sqrt{9-x^2}}&=\int\frac{\dif x}{\sqrt{(3)^2-x^2}}=\sin^{-1}(\tfrac{x}{3})+C&&\text{\RL{مساوات \حوالہ{مساوات_ماورائی_کلیات_تکمل_الف} میں $u=x$ اور $a=3$}}\\
\text{\RL{(ب)}}\quad\int\frac{\dif x}{\sqrt{3-4x^2}}&=\frac{1}{2}\int\frac{\dif u}{\sqrt{a^2-u^2}}&&a=\sqrt{3},\, u=2x\\
&=\frac{1}{2}\sin^{-1}(\tfrac{u}{a})+C&&\text{\RL{مساوات \حوالہ{مساوات_ماورائی_کلیات_تکمل_الف}}}\\
&=\frac{1}{2}\sin^{-1}(\tfrac{2x}{\sqrt{3}})+C
\end{align*}
\انتہا{مثال}
%==================
\ابتدا{مثال}
تکمل \عددی{\int\frac{\dif x}{\sqrt{4x-x^2}}} کو حل کریں۔

حل:\quad
یہاں ریاضی فقرہ \عددی{\sqrt{4x-x^2}} مساوات \حوالہ{مساوات_ماورائی_کلیات_تکمل_الف} تا مساوات \حوالہ{مساوات_ماورائی_کلیات_تکمل_پ} میں سے کسی بھی تکمل میں پائے جانے والے ریاضی فقرے کی طرح نہیں ہے لہٰذا ہم اس کو مربع کے روپ میں لکھتے ہیں:
\begin{align*}
4x-x^2=-(x^2-4x)=-(x^2-4x+4)+4=3-(x-2)^2
\end{align*}
اس کے بعد \عددی{a=2}،  \عددی{u=x-2} اور \عددی{\dif u=\dif x}  لیتے ہوئے درج ذیل حاصل ہوتا ہے۔
\begin{align*}
\int\frac{\dif x}{\sqrt{4x-x^2}}&=\int\frac{\dif x}{\sqrt{4-(x-2)^2}}\\
&=\int\frac{\dif u}{\sqrt{a^-u^2}}&&a=2,\, u=x-2\\
&=\sin^{-1}(\tfrac{u}{a})+C&&\text{\RL{مساوات \حوالہ{مساوات_ماورائی_کلیات_تکمل_الف}}}\\
&=\sin^{-1}(\tfrac{x-2}{2})+C
\end{align*} 
\انتہا{مثال}
%=================
\ابتدا{مثال}
\begin{align*}
\int\frac{\dif x}{10+x^2}&=\frac{1}{\sqrt{10}}\tan^{-1}(\tfrac{x}{\sqrt{10}})+C&& a=\sqrt{10},\,u=x,\,\text{\RL{مساوات \حوالہ{مساوات_ماورائی_کلیات_تکمل_ب}}}\\
\int\frac{\dif x}{7+3x^2}&=\frac{1}{\sqrt{3}}\int\frac{\dif u}{a^2+u^2}&&a=\sqrt{7},\,u=\sqrt{3}x\\
&=\frac{1}{\sqrt{3}}\cdot\frac{1}{a}\tan^{-1}(\tfrac{u}{a})+C&&\text{\RL{مساوات \حوالہ{مساوات_ماورائی_کلیات_تکمل_ب}}}\\
&=\frac{1}{\sqrt{3}}\cdot\frac{1}{\sqrt{7}}\tan^{-1}(\tfrac{\sqrt{3}x}{\sqrt{7}})+C\\
&=\frac{1}{\sqrt{21}}\tan^{-1}(\tfrac{\sqrt{3}x}{\sqrt{7}})+C
\end{align*}
\انتہا{مثال}
%=====================
