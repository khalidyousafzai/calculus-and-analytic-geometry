\حصہ{نلکی اور دو درجی سطحیں}
واحد متغیر کے تفاعل کی احصاء میں  ہم  نے خطوط سے شروع کیا اور خطوط کے بارے میں اپنا علم  استعمال کرتے ہوئے مستوی قوسین کا مطالعہ کیا۔ہم نے مماس پر غور کیا اور دیکھا کہ  کسی بھی قابل تفرق  منحنی    کے چھوٹے حصہ کو خطی تصور کیا جا سکتا ہے۔ خاص  اہمیت کے حامل منحنیات میں مخروطی قطعات، اور دو درجی منحنیات شامل ہیں جنہیں  متغیر \عددی{x} اور \عددی{y} کے دو درجی مساوات سے ظاہر کیا جا سکتا ہے۔

ایک سے زائد متغیرات   کے تفاعل کی احصاء  کا مطالعہ کرنے کی خاطر ہم اسی طرح کی راہ پر چلتے ہیں۔ ہم  دو بعدی سطح سے شروع کر کے اس سطح کے بارے میں اپنا علم استعمال کر کر  فضا میں تین بعدی سطحوں پر غور کرتے ہیں۔ خاص اہمیت کے حامل سطحوں میں نلکیاں اور دو درجی سطحیں شامل ہیں جنہیں \عددی{x}، \عددی{y}، \عددی{z} کے دو درجی مساوات سے ظاہر کیا جا سکتا ہے۔


