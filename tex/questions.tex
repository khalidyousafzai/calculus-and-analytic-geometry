\جزوحصہء{سوالات}
\موٹا{ارتکاز کے وقفے}\\
سوال \حوالہ{سوال_تسلسل_رداس_وقفہ_ارتکاز_الف} تا سوال \حوالہ{سوال_تسلسل_رداس_وقفہ_ارتکاز_ب} میں (الف) تسلسل کا رداس اور وقفہ ارتکاز تلاش کریں۔  \عددی{x} کی کن قیمتوں کے لئے تسلسل (ب) مطلق مرتکز(ج) مشروط مرتکز ہے؟

\ابتدا{سوال}\شناخت{سوال_تسلسل_رداس_وقفہ_ارتکاز_الف}
  $\sum\limits_{n=0}^{\infty}x^n$
\انتہا{سوال}
%====================== 
\ابتدا{سوال}
  $\sum\limits_{n=0}^{\infty}(x+5)^n$
\انتہا{سوال}
%====================== 
\ابتدا{سوال}
  $\sum\limits_{n=0}^{\infty}(-1)^n(4x+1)^n$
\انتہا{سوال}
%====================== 
\ابتدا{سوال}
  $\sum\limits_{n=1}^{\infty}\frac{(3x-2)^n}{n}$
\انتہا{سوال}
%====================== 
\ابتدا{سوال}
  $\sum\limits_{n=0}^{\infty}\frac{(x-2)^n}{10^n}$
\انتہا{سوال}
%====================== 
\ابتدا{سوال}
  $\sum\limits_{n=0}^{\infty}(2x)^n$
\انتہا{سوال}
%====================== 
\ابتدا{سوال}
  $\sum\limits_{n=0}^{\infty}\frac{nx^n}{n+2}$
\انتہا{سوال}
%====================== 
\ابتدا{سوال}
  $\sum\limits_{n=0}^{\infty}\frac{(-1)^n(x+2)^n}{n}$
\انتہا{سوال}
%====================== 
\ابتدا{سوال}
  $\sum\limits_{n=1}^{\infty}\frac{x^n}{n\sqrt{n}3^n}$
\انتہا{سوال}
%====================== 
\ابتدا{سوال}
  $\sum\limits_{n=1}^{\infty}\frac{(x-1)^n}{\sqrt{n}}$
\انتہا{سوال}
%====================== 
\ابتدا{سوال}
  $\sum\limits_{n=0}^{\infty}\frac{(-1)^nx^n}{n!}$
\انتہا{سوال}
%====================== 
\ابتدا{سوال}
  $\sum\limits_{n=0}^{\infty}\frac{3^nx^n}{n!}$
\انتہا{سوال}
%====================== 
\ابتدا{سوال}
  $\sum\limits_{n=0}^{\infty}\frac{x^{2n+1}}{n!}$
\انتہا{سوال}
%====================== 
\ابتدا{سوال}
  $\sum\limits_{n=0}^{\infty}\frac{(2x+3)^{2n+1}}{n!}$
\انتہا{سوال}
%====================== 
\ابتدا{سوال}
  $\sum\limits_{n=0}^{\infty}\frac{x^n}{\sqrt{n^2+3}}$
\انتہا{سوال}
%====================== 
\ابتدا{سوال}
  $\sum\limits_{n=0}^{\infty}\frac{(-1)^nx^n}{\sqrt{n^2+3}}$
\انتہا{سوال}
%====================== 
\ابتدا{سوال}
  $\sum\limits_{n=0}^{\infty}\frac{n(x+3)^n}{5^n}$
\انتہا{سوال}
%====================== 
\ابتدا{سوال}
  $\sum\limits_{n=0}^{\infty}\frac{nx^n}{4^n(n^2+1)}$
\انتہا{سوال}
%====================== 
\ابتدا{سوال}
  $\sum\limits_{n=0}^{\infty}\frac{\sqrt{n}x^n}{3^n}$
\انتہا{سوال}
%====================== 
\ابتدا{سوال}
  $\sum\limits_{n=1}^{\infty}\sqrt[n]{n}(2x+5)^n$
\انتہا{سوال}
%====================== 
\ابتدا{سوال}
  $\sum\limits_{n=1}^{\infty}\big(1+\frac{1}{n}\big)^nx^n$
\انتہا{سوال}
%====================== 
\ابتدا{سوال}
  $\sum\limits_{n=1}^{\infty}(\ln x)x^n$
\انتہا{سوال}
%====================== 
\ابتدا{سوال}
  $\sum\limits_{n=1}^{\infty}n^nx^n$
\انتہا{سوال}
%====================== 
\ابتدا{سوال}
  $\sum\limits_{n=0}^{\infty}n!(x-4)^n$
\انتہا{سوال}
%====================== 
\ابتدا{سوال}
  $\sum\limits_{n=1}^{\infty}\frac{(-1)^{n+1}(x+2)^n}{n2^n}$
\انتہا{سوال}
%====================== 
\ابتدا{سوال}
  $\sum\limits_{n=0}^{\infty}(-2)^n(n+1)(x-1)^n$
\انتہا{سوال}
%====================== 
\ابتدا{سوال}
  $\sum\limits_{n=2}^{\infty}\frac{x^n}{n(\ln n)^2}$\quad
آپ سوال \حوالہ{سوال_تسلسل_لوگارتھمی_پی_تسلسل} کی مدد لے سکتے ہیں۔
\انتہا{سوال}
%====================== 
\ابتدا{سوال}
  $\sum\limits_{n=2}^{\infty}\frac{x^n}{n\ln n}$\quad
آپ سوال \حوالہ{سوال_کوشی_پرکھ_جمود_کی_مدد} کی مدد لے سکتے ہیں۔
\انتہا{سوال}
%====================== 
\ابتدا{سوال}
  $\sum\limits_{n=1}^{\infty}\frac{(4x-5)^{2n+1}}{n^{3/2}}$
\انتہا{سوال}
%====================== 
\ابتدا{سوال}
  $\sum\limits_{n=1}^{\infty}\frac{(3x+1)^{n+1}}{2n+2}$
\انتہا{سوال}
%====================== 
\ابتدا{سوال}
  $\sum\limits_{n=1}^{\infty}\frac{(x+\pi)^n}{\sqrt{n}}$
\انتہا{سوال}
%====================== 
\ابتدا{سوال}\شناخت{سوال_تسلسل_رداس_وقفہ_ارتکاز_ب}
  $\sum\limits_{n=0}^{\infty}\frac{(x-\sqrt{2})^{2n+1}}{2^n}$
\انتہا{سوال}
%====================== 
سوال \حوالہ{سوال_تسلسل_مجموعہ_بطور_تفاعل_الف} تا سوال \حوالہ{سوال_تسلسل_مجموعہ_بطور_تفاعل_ب} میں تسلسل کی ارتکاز کا وقفہ تلاش کریں اور اس وقفہ میں تسلسل کے مجموعہ کو \عددی{x} کا تفاعل لکھیں۔

\ابتدا{سوال}\شناخت{سوال_تسلسل_مجموعہ_بطور_تفاعل_الف}
$\sum\limits_{n=0}^{\infty}\frac{(x-1)^{2n}}{4^n}$
\انتہا{سوال}
%========================
\ابتدا{سوال}
$\sum\limits_{n=0}^{\infty}\frac{(x+1)^{2n}}{9^n}$
\انتہا{سوال}
%========================
\ابتدا{سوال}
$\sum\limits_{n=0}^{\infty}\big(\frac{\sqrt{x}}{2}-1\big)^n$
\انتہا{سوال}
%========================
\ابتدا{سوال}
$\sum\limits_{n=0}^{\infty}(\ln x)^n$
\انتہا{سوال}
%========================
\ابتدا{سوال}
$\sum\limits_{n=0}^{\infty}\big(\frac{x^2+1}{3}\big)^n$
\انتہا{سوال}
%========================
\ابتدا{سوال}\شناخت{سوال_تسلسل_مجموعہ_بطور_تفاعل_ب}
$\sum\limits_{n=0}^{\infty}\big(\frac{x^2-1}{2}\big)^n$
\انتہا{سوال}
%========================
\موٹا{نظریہ اور مثالیں}

\ابتدا{سوال}\شناخت{سوال_تسلسل_نظریہ_اور_مثالیں_تفرق_نیا_تسلسل_الف}
درج ذیل تسلسل \عددی{x} کی کن قیمتوں کے لئے مرتکز ہے؟
\begin{align*}
1-\frac{1}{2}(x-3)+\frac{1}{4}(x-3)^2+\cdots+(-\frac{1}{2})^n(x-3)^n+\cdots
\end{align*}
اس کا مجموعہ کتنا ہے؟ اس تسلسل کا جزو در جزو تفرق لینے سے کونسا تسلسل حاصل ہوتا ہے؟ یہ نیا تسلسل \عددی{x} کی کن قیمتوں کے لئے مرتکز ہو گا؟ اس کا مجموعہ کیا ہے؟
\انتہا{سوال}
%=====================
\ابتدا{سوال}
اگر آپ سوال \حوالہ{سوال_تسلسل_نظریہ_اور_مثالیں_تفرق_نیا_تسلسل_الف} کا تسلسل جزو در جزو تکمل کریں تب کونسا تسلسل حاصل ہو گا؟ \عددی{x} کی کن قیمتوں کے لئے یہ نیا تسلسل مرتکز ہو گا؟ اس مجموعے کا دوسرا نام کیا ہے؟
\انتہا{سوال}
%==========================
\ابتدا{سوال}
درج ذیل تسلسل تمام \عددی{x} کے لئے \عددی{\sin x} پر مرکوز ہے۔
\begin{align*}
\sin x=x-\frac{x^3}{3!}+\frac{x^5}{5!}-\frac{x^7}{7!}+\frac{x^9}{9!}-\frac{x^{11}}{11!}+\cdots
\end{align*}
\begin{enumerate}[a.]
\item
 \عددی{\cos x} کے تسلسل کے ابتدائی چھ اجزاء دریافت کریں۔ \عددی{x} کی کن قیمتوں کے لئے حاصل تسلسل مرتکز ہو گا۔  
\item
\عددی{\sin x} کے تسلسل میں \عددی{x} کی جگہ \عددی{2x} پر کرنے ایسا تسلسل حاصل کریں جو تمام \عددی{x} کے لئے \عددی{\sin 2x} پر مرتکز ہو۔
\item
ضرب تسلسل اور جزو-الف کا نتیجہ استعمال کرتے ہوئے \عددی{2\sin x\cos x} کے تسلسل کے ابتدائی چھ اجزاء حاصل کریں۔ جزو-ب کے نتیجہ کے ساتھ موازنہ کریں۔
\end{enumerate}
\انتہا{سوال}
%========================
\ابتدا{سوال}
درج ذیل تسلسل تمام \عددی{x} کے لئے \عددی{e^x} پر مرکوز ہے۔
\begin{align*}
e^x=1+x+\frac{x^2}{2!}+\frac{x^3}{3!}+\frac{x^4}{4!}+\frac{x^5}{5!}+\cdots
\end{align*}
\begin{enumerate}[a.]
\item
\عددی{\tfrac{\dif}{\dif x}e^x} کا تسلسل دریافت کریں۔ کیا آپ کو \عددی{e^x} کا تسلسل دوبارہ حاصل ہوتا ہے؟ وجہ پیش کریں۔
\item
\عددی{\int e^x\dif x} کا تسلسل دریافت کریں۔ کیا آپ کو \عددی{e^x} کا تسلسل دوبارہ حاصل ہوتا ہے؟ وجہ پیش کریں۔
\item
\عددی{e^x} کے تسلسل میں \عددی{x} کی جگہ \عددی{-x} پر کر کے \عددی{e^{-x}} کا تسلسل حاصل کریں۔ اب \عددی{e^x} کے تسلسل کو  \عددی{e^{-x}} کے تسلسل کے ساتھ ضرب کر کے \عددی{e^x\cdots e^{-x}} کے تسلسل کے ابتدائی چھ اجزاء تلاش کریں۔
\end{enumerate}
\انتہا{سوال}
%=========================
\ابتدا{سوال}
درج ذیل تسلسل \عددی{-\tfrac{\pi}{2}<x<\tfrac{\pi}{2}} کے لئے \عددی{\tan x} پر مرتکز ہے۔
\begin{align*}
\tan x=x+\frac{x^3}{3}+\frac{2x^5}{15}+\frac{17x^7}{315}+\frac{62x^9}{2835}+\cdots
\end{align*}
\begin{enumerate}[a.]
\item
\عددی{\ln\abs{\sec x}} کے تسلسل کے ابتدائی پانچ اجزاء تلاش کریں۔ \عددی{x} کی کن  قیمتوں کے لئے یہ تسلسل مرتکز ہو گا؟
\item
\عددی{\sec^2x} کے تسلسل کے ابتدائی پانچ اجزاء تلاش کریں۔ \عددی{x} کی کن  قیمتوں کے لئے یہ تسلسل مرتکز ہو گا؟
\item
اگلے سوال میں \عددی{\sec x} کے تسلسل کا مربع تلاش کرتے ہوئے جزو-ب کے نتیجہ کی تصدیق کریں۔
\end{enumerate}
\انتہا{سوال}
%======================
\ابتدا{سوال}
درج ذیل تسلسل \عددی{-\tfrac{\pi}{2}<x<\tfrac{\pi}{2}} کے لئے \عددی{\sec x} پر مرتکز ہے۔
\begin{align*}
\sec x=1+\frac{x^2}{2}+\frac{5}{24}x^4+\frac{61}{720}x^6+\frac{277}{8064}x^8+\cdots
\end{align*}
\begin{enumerate}[a.]
\item
\عددی{\ln\abs{\sec x+\tan x}} کے تسلسل کے ابتدائی پانچ اجزاء تلاش کریں۔ \عددی{x} کی کن  قیمتوں کے لئے یہ تسلسل مرتکز ہو گا؟
\item
\عددی{\sec x\tan x} کے تسلسل کے ابتدائی چار اجزاء تلاش کریں۔ \عددی{x} کی کن  قیمتوں کے لئے یہ تسلسل مرتکز ہو گا؟
\item
گزشتہ سوال میں \عددی{\tan x} کے تسلسل کو \عددی{\sec x} کے تسلسل کے ساتھ ضرب کرتے ہوئے جزو-ب کے نتیجہ کی تصدیق کریں۔
\end{enumerate}
\انتہا{سوال}
%============
\ابتدا{سوال}\ترچھا{مرتکز طاقتی تسلسل کی یکتائی}
\begin{enumerate}[a.]
\item
دکھائیں کہ کھلے وقفہ \عددی{(-c,c)} میں تمام \عددی{x} کے لئے مرتکز اور ایک دوسرے کے برابر تسلسل \عددی{\sum_{n=0}^{\infty}a_nx^n} اور \عددی{\sum_{n=0}^{\infty}b_nx^n} کی صورت میں تمام \عددی{n} کے لئے \عددی{a_n=b_n} ہو گا۔ (اشارہ: فرض کریں \عددی{\sum_{n=0}^{\infty}a_nx^n=\sum_{n=0}^{\infty}b_nx^n} ہے۔ جزو در جزو تفرق لے کر ثابت کریں کہ \عددی{a_n} اور \عددی{b_n} دونوں \عددی{\tfrac{f^{(n)}(0)}{n!}} کے برابر ہیں۔)
\item
دکھائیں کہ کھلے وقفہ \عددی{(-c,c)} میں تمام \عددی{x} کے لئے \عددی{\sum_{n=0}^{\infty}a_nx^n=0} کی صورت میں تمام \عددی{n} کے لئے \عددی{a_n=0} ہو گا۔
\end{enumerate}
\انتہا{سوال}
%======================
\ابتدا{سوال}
تسلسل \عددی{\sum_{n=0}^{\infty}\tfrac{n^2}{2^n}} کا مجموعہ تلاش کرنے کی خاطر \عددی{\tfrac{1}{1-x}} کو ہندسی تسلسل کی صورت میں لکھ کر دونوں اطراف کا \عددی{x} کے ساتھ تفرق لیں، دونوں اطراف کو \عددی{x} سے ضرب دے کر دونوں اطراف کا تفرق لیں اور آخر کار دونوں اطراف کو \عددی{x} سے ضرب کریں۔ اب \عددی{x=\tfrac{1}{2}} پر کریں۔ کیا حاصل ہوتا ہے؟
\انتہا{سوال}
%=================
\ابتدا{سوال}\ترچھا{آخری نقطوں پر ارتکاز}\\
ایک مثال سے دکھائیں کہ ایک طاقتی تسلسل کے وقفہ ارتکاز کے آخری سروں پر اس تسلسل کا ارتکاز مشروط یا مطلق ہو سکتا ہے۔
\انتہا{سوال}
%=====================
\ابتدا{سوال}
ایسے طاقتی تسلسل بنائیں جن کے وقفہ ارتکاز درج ذیل ہوں۔
\begin{multicols}{3}
\begin{enumerate}[a.]
\item
$(-3,3)$
\item
$(-2,0)$
\item
$(1,5)$
\end{enumerate}
\end{multicols}
\انتہا{سوال}
%======================

\حصہ{ٹیلر اور مکلارن تسلسل}
اس حصہ میں دکھایا جائے گا کہ وہ تفاعل جو لامتناہی گنّا قابل تفرق ہوں طاقتی تسلسل پیدا کرتے ہیں جنہیں ٹیلر تسلسل کہتے ہیں۔  عموماً ایسے تسلسل، پیداکار تفاعل کے کارآمد تخمینی کثیر رکنیاں پیش کرتے ہیں۔

\جزوحصہء{تسلسلی اظہار}
ہم جانتے ہیں کہ اپنے وقفہ ارتکاز کے اندر طاقتی تسلسل کا مجموعہ استمراری تفاعل ہوتا ہے جس کے تفرقات  ہر درجے کے پائے جاتے ہیں۔ لیکن کیا ہم یہی کچھ دوسری رخ بھی کہہ سکتے ہیں؟ یعنی کیا ایسا تفاعل \عددی{f(x)} جس کے وقفہ \عددی{I} ہر درجہ کے تفرقات پائے جاتے ہوں کو \عددی{ِI} میں طاقتی تسلسل سے ظاہر کرنا ممکن ہو گا؟ اگر ایسا ممکن ہو، تب اس تسلسل کے عددی سر کیا ہوں گے؟

ہم \عددی{f(x)}  کو مثبت رداس ارتکاز کے طاقت تسلسل کا مجموعہ 
\begin{align*}
f(x)&=\sum_{n=0}^{\infty}a_n(x-a)^n\\
&=a_0+a_1(x-a)+a_2(x-a)^2+\cdots+a_n(x-a)^n+\cdots
\end{align*}
تصور کر کے اس آخری سوال کا جواب با آسانی دے سکتے ہیں۔وقفہ \عددی{I} میں  بار بار جزو در جزو تفرق لینے سے 
\begin{align*}
f'(x)&=a_1+2a_2(x-a)+3a_3(x-a)^2+\cdots+na_n(x-a)^{n-1}+\cdots\\
f''(x)&=1\cdot 2a_2+2\cdot 3a_3(x-a)+3\cdot 4a_4(x-a)^2+\cdots\\
f'''(x)&=1\cdot2\cdot3a_3+2\cdot3\cdot4a_4(x-a)+3\cdot4\cdot5a_5(x-a)^2+\cdots
\end{align*}
حاصل ہو گا۔یوں تمام \عددی{n} کے لئے \عددی{n} گنّا تفرق درج ذیل ہو گا۔
\begin{align*}
f^{(n)}(x)=n!a_n+\text{\RL{اجزاء کا مجموعہ جن میں جزو ضربی $(x-a)$ پایا جاتا ہے}}
\end{align*}
چونکہ یہ تمام مساوات \عددی{x=a} پر کارآمد ہیں لہٰذا 
\begin{align*}
f'(a)&=a_1\\
f''(a)&=1\cdot 2 a_2\\
f'''(a)&=1\cdot 2\cdot 3a_3
\end{align*}
یا عمومی طور پر
\begin{align*}
f^{(n)}(a)=n!a_n
\end{align*}
حاصل ہوتا ہے۔ یہ کلیہ وقفہ \عددی{I} پر \عددی{f} کو مرتکز کسی بھی طاقتی تسلسل \عددی{\sum_{n=0}^{\infty}a_n(x-a)^n} کے عددی سروں کا ایک حیرت کن نقش پیش کرتا ہے۔ اگر ایسا تسلسل پایا جاتا ہو (جو ہم اب تک نہیں جانتے کہ پایا جاتا ہے) تب ایسا تسلسل صرف ایک ہو سکتا ہے جس کے \عددی{n} عددی سر درج ذیل ہوں گے۔
\begin{align*}
a_n=\frac{f^{(n)}(a)}{n!}
\end{align*}
اگر \عددی{f} کا تسلسلی روپ پایا جاتا ہو تب یہ تسلسل لازماً درج ذیل ہو گا۔
\begin{align}\label{مساوات_تسلسل_عمومی_تسلسل}
f(x)&=f(a)+f'(a)(x-a)+\frac{f''(a)}{2!}(x-a)^2+\cdots+\frac{f^{(n)}(a)}{n!}(x-a)^n+\cdots
\end{align}
لیکن کیا وقفہ \عددی{I}، جس کا مرکز \عددی{x=a} ہو، پر  لامتناہی گنّا قابل تفرق اختیاری تفاعل \عددی{f} سے شروع کر کے مساوات \حوالہ{مساوات_تسلسل_عمومی_تسلسل} کا تسلسل پیدا کر کے \عددی{I} کی اندرون میں ہر \عددی{x} پر \عددی{f(x)} کو مرکوز تسلسل حاصل ہو گا؟ جیسا ہم دیکھیں گے  بعض تفاعل کے لئے ایسا ہو گا اور بعض کے لئے ایسا نہیں ہو گا۔

\جزوحصہء{ٹیلر اور مکلارن تسلسل}

