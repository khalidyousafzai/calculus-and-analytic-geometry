\حصہ{راہ سے آزادی، تفاعل خفی توانائی، اور بقائی میدان}
ثقلی اور برقی میدان میں کمیت یا بار کو ایک نقطہ سے دوسرے نقطہ منتقل کرنے کے لئے درکار کام صرف ابتدائی اور اختتامی نقطوں پر منحصر ہوتا ہے نا کہ منتقلی کی راہ پر۔اس حصہ میں تکمل کام کی راہ سے آزادی کے تصور پر غور کیا جائے گا اور ایسے میدانوں کی خواص پر غور کیا جائے گا جن میں تکمل کام کی قیمت راہ کے تابع نہیں ہوتا۔

\جزوحصہء{راہ سے آزادی}
فضا میں کھلا خطہ \عددی{D} پر معین میدان \عددی{\kvec{F}} ایک ذرہ کو \عددی{D} میں  نقطہ \عددی{A} سے نقطہ \عددی{B} منتقل کرتا ہے۔ عموماً   تکمل کام \عددی{\int\kvec{F}\cdot\dif\kvec{r}} کی قیمت منتقلی کی راہ پر منحصر ہو گی، البتہ ایسے میدان پائے جاتے ہیں جن میں تکمل کام کی قیمت صرف ابتدائی اور اختتامی نقطوں پر منحصر ہو گی نا کہ منتقلی کی راہ پر۔ اگر \عددی{D} میں تمام \عددی{A} اور \عددی{B} کے لئے ایسا ہو تب یہ میدان بقائی میدان کہلائے گا اور ہم کہیں گے کہ \عددی{D} میں \عددی{\int\kvec{F}\cdot\dif\kvec{r}} راہ سے آزاد ہے اور \عددی{D} پر \عددی{\kvec{F}} بقائی ہے۔

\ابتدا{تعریف}
فضا میں کھلا خطہ \عددی{D} پر \عددی{\kvec{F}} کو ایک  معین میدان لیتے ہوئے تصور کریں کہ \عددی{D} میں ہر دو نقطوں \عددی{A} اور \عددی{B} کے بیچ ہر ممکنہ راہ پر تکمل کام \عددی{\int_A^B\kvec{F}\cdot\dif\kvec{r}} کی قیمت ایک جیسی ہے۔ تب تکمل \عددی{\int\kvec{F}\cdot\dif\kvec{r}} خطہ \عددی{D} میں \اصطلاح{راہ سے آزاد}\فرہنگ{راہ سے آزاد}\حاشیہب{path independent}\فرہنگ{path independent} ہو گا اور میدان \عددی{\kvec{F}} خطہ \عددی{D} پر \اصطلاح{بقائی}\فرہنگ{بقائی}\حاشیہب{conservative}\فرہنگ{conservative} ہو گا۔
\انتہا{تعریف}
%======================

عملی زندگی میں عموماً میدان \عددی{\kvec{F}} صرف اور صرف اس صورت بقائی ہو گا جب  \عددی{D} پر \عددی{\kvec{F}=\nabla f} ہو جہاں \عددی{f} ایک غیر سمتی تفاعل ہے۔ ایسی صورت میں تفاعل \عددی{f} کو \عددی{\kvec{F}} کا مخفی قوہ تفاعل کہتے ہیں۔   

\ابتدا{تعریف}
اگر \عددی{D}پر میدان \عددی{\kvec{F}} معین ہو اور \عددی{\kvec{F}=\nabla f} ہو جہاں \عددی{f} خطہ \عددی{D} پر ایک غیر سمتی تفاعل ہو تب \عددی{f} کو \عددی{\kvec{F}} کا \اصطلاح{مخفی قوہ تفاعل}\فرہنگ{مخفی قوہ تفاعل}\حاشیہب{potential function}\فرہنگ{potential function} کہتے ہیں۔
\انتہا{تعریف}
%======================

برقی مخفی قوہ ایک غیر سمتی تفاعل ہے جس کا میدان ڈھلوان  ایک برقی میدان ہوتا ہے۔  ثقلی مخفی قوہ ایک غیر سمتی تفاعل ہے جس کا میدان ڈھلوان ایک  ثقلی میدان ہوتا ہے، وغیرہ وغیرہ۔ جیسا ہم اب دیکھیں گے،  میدان \عددی{\kvec{F}} کا مخفی قوہ تفاعل \عددی{f} جاننے کے بعد \عددی{\kvec{F}} کی دائرہ کار میں تمام تکملات کام  کی قیمتیں درج ذیل سے حاصل کی جا سکتی ہیں۔
\begin{align}\label{مساوات_سمتی_تکمل_مخفی_قوہ_استعمال}
\int_A^B\kvec{F}\cdot\dif\kvec{r}=\int_A^B\nabla f\cdot\dif \kvec{r}=f(B)-f(A)
\end{align}
اگر آپ واحد متغیر کے تفرق \عددی{f'} کی طرح  \عددی{\nabla f} کو متعدد متغیرات کے تفاعل کے لئے فرض کریں تب مساوات \حوالہ{مساوات_سمتی_تکمل_مخفی_قوہ_استعمال} کو احصاء کے بنیادی کلیہ
\begin{align*}
\int_a^bf'(x)\dif x=f(b)-f(a)
\end{align*}  
کا مطابقتی سمتی احصاء کا کلیہ تصور کیا جا سکتا ہے۔

بقائی میدان کی دیگر قابل ذکر خواص پر، آگے چلتے ہوئے ساتھ ساتھ، غور کیا جائے گا۔ مثلاً، \عددی{D} پر بقائی \عددی{\kvec{F}} کی صورت میں \عددی{D} میں ہر بند راہ پر تکمل کام صفر ہو گا۔مساوات \حوالہ{مساوات_سمتی_تکمل_مخفی_قوہ_استعمال} اور اس کی مضمرات کی درستگی برقرار رکھنے  کی خاطر ہمیں اس مساوات میں مستعمل منحنی، میدان، اور دائرہ کار پر شرائط مسلط کرنی ہوں گے۔

ہم فرض کرتے ہیں کہ تمام منحنیات \اصطلاح{ٹکڑوں میں ہموار}\حاشیہب{piecewise smooth} ہیں،یعنی، انہیں متناہی تعداد کی ہموار  منحنیات کو ایک دوسرے کے ساتھ جوڑ کر، حصہ \حوالہ{حصہ_سمتی_تفاعل_سمتی_قیمت_تفاعل_اور_فضائی_منحنیات} کی طرح ،حاصل کیا گیا ہے۔مزید ہم فرض کرتے ہیں کہ \عددی{\kvec{F}} کے اجزاء کے یک رتبی استمراری تفرقات  پائے جاتے ہیں۔استمرار کی اس شرح  کے بعد \عددی{\kvec{F}=\nabla f} کی صورت میں مخفی قوہ تفاعل \عددی{f} کے مدغم تفرقات ایک دوسرے کے برابر ہوں گے، جو بقائی میدان \عددی{\kvec{F}} کے خواص پر غور کے دوران آفشاں انگیز ثابت ہو گا۔

ہم فضا میں \عددی{D} کو ایک \ترچھا{کھلا} خطہ فرض کرتے ہیں۔ یوں \عددی{D} میں ہر نقطہ ایک ایسے گیند کے مرکز پر پایا جائے گا جو مکمل طور پر \عددی{D} میں پایا جاتا ہو۔ مزید ہم فرض کرتے ہیں کہ \عددی{D} \اصطلاح{تعلق (دار)}\فرہنگ{تعلق!دار}\حاشیہب{connected}\فرہنگ{connected} خطہ ہے۔  کھلا خطہ میں تعلق دار سے مراد  ایسا خطہ ہے، جس میں ہر دو نقطوں کو ایک ایسی مسلسل راہ سے جوڑا جا سکتا ہے جو مکمل طور پر اس خطہ میں پائی جاتی ہو۔ 

\جزوحصہء{بقائی میدان میں لکیری تکملات}
بقائی میدان میں لکیری تکملات کی قیمتیں درج ذیل نتیجہ کی مدد سے باآسانی حاصل کی جا سکتی ہیں۔اس نتیجہ کے تحت تکمل کی قیمت صرف ابتدائی اور اختتام نقطوں پر منحصر ہو گی نا کہ منتقلی کی راہ پر۔

\ابتدا{مسئلہ}\شناخت{مسئلہ_سمتی_تکمل_بنیادی_مسئلہ}\موٹا{لکیری تکملات کا بنیادی مسئلہ}\\
\begin{enumerate}[1.]
\item
فرض کریں فضا میں کھلے  تعلق دار خطہ \عددی{D} میں سمتی میدان  \عددی{\kvec{F}=M\ai+N\aj+P\ak} کے اجزاء  استمراری ہیں۔تب صرف اور صرف اس صورت جب  \عددی{D} میں تمام نقاط \عددی{A} اور \عددی{B} کے لئے تکمل \عددی{\int_A^B\kvec{F}\cdot\dif\kvec{r}} کی قیمت، \عددی{D} کے اندر رہتے ہوئے \عددی{A} اور \عددی{B} کے بیچ تمام راہ سے آزاد ہو،  ایسا قابل تفرق تفاعل \عددی{f} موجود ہو گا جو درج ذیل پر پورا اترتا ہو۔
\begin{align*}
\kvec{F}=\nabla f=\frac{\partial f}{\partial x}\ai+\frac{\partial f}{\partial y}\aj+\frac{\partial f}{\partial z}\ak
\end{align*}
\item
اگر تکمل کی قیمت \عددی{A} اور \عددی{B} کے بیچ راہ سے آزاد ہو تب تکمل کی قیمت درج ذیل ہو گی۔
\begin{align*}
\int_A^B\kvec{F}\cdot\dif\kvec{r}=f(B)-f(A)
\end{align*}
\end{enumerate}
\انتہا{مسئلہ}
%========= 
\ابتدا{ثبوت}\موٹا{کہ \عددی{\kvec{F}=\nabla f} سے مراد تکمل کی قیمت کا راہ سے آزاد ہونا ہے}\\
فرض کریں \عددی{D} میں \عددی{A} اور \عددی{B} دو نقطے ہیں اور \عددی{D} میں \عددی{A} سے \عددی{B} تک
 ہموار راہ \عددی{C:\, \kvec{r}(t)=g(t)\ai+h(t)\aj+k(t)\ak,\, a\le t\le b} ہے۔ منحنی کی ہمراہ \عددی{t} کے لحاظ سے  \عددی{C} قابل تفرق ہے اور درج ذیل ہو گا۔
\begin{align}
\frac{\dif f}{\dif t}&=\frac{\partial f}{\partial x}\frac{\dif x}{\dif t}+\frac{\partial f}{\partial y}\frac{\dif y}{\dif t}+\frac{\partial f}{\partial z}\frac{\dif z}{\dif t}&&\text{\RL{زنجیری قاعدہ}}\nonumber\\
&=\nabla f\cdot \big(\frac{\dif x}{\dif t}\ai+\frac{\dif y}{\dif t}\aj+\frac{\dif z}{\dif t}\ak\big)=\nabla f\cdot \frac{\dif\kvec{r}}{\dif t}=\kvec{F}\cdot\frac{\dif\kvec{r}}{\dif t}\label{مساوات_سمتی_تکمل_ثبوت_بقائی}
\end{align}
یوں درج ذیل ہو گا۔
\begin{align*}
\int_C\kvec{F}\cdot\dif\kvec{r}&=\int_{t=a}^{t=b}\kvec{F}\cdot\frac{\dif \kvec{r}}{\dif t}\dif t=\int_a^b\frac{\dif f}{\dif t}\dif t&&\text{\RL{مساوات \حوالہ{مساوات_سمتی_تکمل_ثبوت_بقائی}}}\\
&=f(g(t),h(t),k(t))\big]_a^b=f(B)-f(A)
\end{align*}
اس طرح تکمل کام کی قیمت \عددی{A} اور \عددی{B} پر \عددی{f} کی قیمتوں پر منحصر ہو گی نا کہ ان کے بیچ راہ پر۔ یوں مسئلہ کے دوسرا  جزو  کے ساتھ ساتھ پہلا مضمر جزو بھی ثابت ہوتا ہے۔ ہم الٹ مضمر  کا زیادہ پیچیدہ ثبوت پیش نہیں کرتے ہیں۔
\انتہا{ثبوت}
%==================

\ابتدا{مثال}
نقاط \عددی{(-1,3,9)} اور \عددی{(1,6,-4)} کے بیچ ہموار منحنی \عددی{C} پر چلتے  ہوئے درج ذیل بقائی میدان کا کم تلاش کریں۔
\begin{align*}
\kvec{F}=yz\ai+xz\aj+xy\ak=\nabla(xyz)
\end{align*}

حل:\quad
\عددی{f(x,y,z)=xyz} لیتے ہوئے درج ذیل ہو گا۔
\begin{align*}
\int_A^B\kvec{F}\cdot\dif\kvec{r}&=\int_A^B\nabla f\cdot \dif \kvec{r}&&\kvec{F}=\nabla f\\
&=f(B)-f(A)&&\text{\RL{مسئلہ \حوالہ{مسئلہ_سمتی_تکمل_بنیادی_مسئلہ} کا جزو دوم}}\\
&=\left. xyz\right\vert_{(1,6,-4)}-\left. xyz\right\vert_{(-1,3,9)}\\
&=(1)(6)(-4)-(-1)(3)(9)\\
&=-24+27=3
\end{align*}
\انتہا{مثال}
%=====================
