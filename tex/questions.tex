\جزوحصہء{سوالات}
\ابتدا{سوالات}
\موٹا{مقامی انتہا کی تلاش}\\
سوال \حوالہ{سوال_کثیرالمتغیر_تمام_زیادہ_کم_زین_الف} تا سوال  \حوالہ{سوال_کثیرالمتغیر_تمام_زیادہ_کم_زین_ب} میں تفاعل کے  تمام مقامی زیادہ سے زیادہ قیمت کے نقاط، مقامی کم سے کم قیمت کے نقاط  اور نقاط زین تلاش کریں۔

\ابتدا{سوال}\شناخت{سوال_کثیرالمتغیر_تمام_زیادہ_کم_زین_الف}
$f(x,y)=x^2+xy+y^2+3x-3y+4$
\انتہا{سوال}
%==================
\ابتدا{جواب}
\wf{\unexpanded{
$f(-3,-3)=-5$
مقامی کم سے کم قیمت نقطہ
}}
\انتہا{جواب}
%======================
\ابتدا{سوال}
$f(x,y)=x^2+3xy+3y^2-6x+3y-6$
\انتہا{سوال}
%====================
\ابتدا{سوال}
$f(x,y)=2xy-5x^2-2y^2+4x+4y-4$
\انتہا{سوال}
%====================
\ابتدا{جواب}
\wf{\unexpanded{
$f(\tfrac{2}{3},\tfrac{2}{3})=0$
مقامی زیادہ سے زیادہ قیمت نقطہ
}}
\انتہا{جواب}
%======================
\ابتدا{سوال}
$f(x,y)=2xy-5x^2-2y^2+4x-4$
\انتہا{سوال}
%====================
\ابتدا{سوال}
$f(x,y)=x^2+xy+3x+2y+5$
\انتہا{سوال}
%====================
\ابتدا{جواب}
\wf{\unexpanded{
$f(-2,1)$
نقطہ زین
}}
\انتہا{جواب}
%======================
\ابتدا{سوال}
$f(x,y)=y^2+xy-2x-2y+2$
\انتہا{سوال}
%====================
\ابتدا{سوال}
$f(x,y)=5x-7x^2+3x-6y+2$
\انتہا{سوال}
%====================
\ابتدا{جواب}
\wf{\unexpanded{
$f(\tfrac{6}{5},\tfrac{69}{25})$
نقطہ زین
}}
\انتہا{جواب}
%======================
\ابتدا{سوال}
$f(x,y)=2xy-x^2-2y^2+3x+4$
\انتہا{سوال}
%====================
\ابتدا{سوال}
$f(x,y)=x^2-4xy+y^2+6y+2$
\انتہا{سوال}
%====================
\ابتدا{جواب}
\wf{\unexpanded{
$f(2,1)$
نقطہ زین
}}
\انتہا{جواب}
%======================
\ابتدا{سوال}
$f(x,y)=3x^2+6xy+7y^2-2x+4y$
\انتہا{سوال}
%====================
\ابتدا{سوال}
$f(x,y)=2x^2+3xy+4y^2-5x+2y$
\انتہا{سوال}
%====================
\ابتدا{جواب}
\wf{\unexpanded{
$f(2,-1)=-6$
مقامی کم سے کم قیمت نقطہ
}}
\انتہا{جواب}
%======================
\ابتدا{سوال}
$f(x,y)=4x^2-6xy+5y^2-20x+26y$
\انتہا{سوال}
%====================
\ابتدا{سوال}
$f(x,y)=x^2-y^2-2x+4y+6$
\انتہا{سوال}
%====================
\ابتدا{جواب}
\wf{\unexpanded{
$f(1,2)$
نقطہ زین
}}
\انتہا{جواب}
%======================
\ابتدا{سوال}
$f(x,y)=x^2-2xy+2y^2-2x+2y+1$
\انتہا{سوال}
%====================
\ابتدا{سوال}
$f(x,y)=x^2+2xy$
\انتہا{سوال}
%====================
\ابتدا{جواب}
\wf{\unexpanded{
\عددی{f(0,0)} نقطہ زین؛ 
}}
\انتہا{جواب}
%======================
\ابتدا{سوال}
$f(x,y)=3+2x+2y-2x^2-2xy-y^2$
\انتہا{سوال}
%====================
\ابتدا{سوال}
$f(x,y)=x^3-y^3-2xy+6$
\انتہا{سوال}
%====================
\ابتدا{جواب}
\wf{\unexpanded{
\عددی{f(0,0)}نقطہ زین؛ \عددی{f(-\tfrac{2}{3},\tfrac{2}{3})=\tfrac{170}{27}} مقامی زیادہ سے زیادہ قیمت نقطہ
}}
\انتہا{جواب}
%======================
\ابتدا{سوال}
$f(x,y)=x^3+3xy+y^3$
\انتہا{سوال}
%====================
\ابتدا{سوال}
$f(x,y)=6x^2-2x^3+3y^2+6xy$
\انتہا{سوال}
%====================
\ابتدا{جواب}
\wf{\unexpanded{
\عددی{f(0,0)=0} مقامی کم سے کم قیمت نقطہ؛ \عددی{f(1,-1)}  نقطہ زین
}}
\انتہا{جواب}
%======================
\ابتدا{سوال}
$f(x,y)=3y^2-2y^3-3x^2+6xy$
\انتہا{سوال}
%====================
\ابتدا{سوال}
$f(x,y)=9x^3+\tfrac{y^3}{3}-4xy$
\انتہا{سوال}
%====================
\ابتدا{جواب}
\wf{\unexpanded{
\عددی{f(0,0)}نقطہ زین؛ \عددی{f(\tfrac{4}{9},\tfrac{4}{3})=-\tfrac{64}{81}} مقامی کم سے کم  قیمت نقطہ
}}
\انتہا{جواب}
%======================
\ابتدا{سوال}
$f(x,y)=8x^3+y^3+6xy$
\انتہا{سوال}
%====================
\ابتدا{سوال}
$f(x,y)=x^3+y^3+3x^2-3y^2-8$
\انتہا{سوال}
%====================
\ابتدا{جواب}
\wf{\unexpanded{
\عددی{f(0,0)}نقطہ زین؛ \عددی{f(0,2)=-12} مقامی  کم سے کم  قیمت نقطہ؛  \عددی{f(-2,0)=-4} مقامی زیادہ سے زیادہ قیمت نقطہ؛ \عددی{f(-2,-2)} نقطہ زین
}}
\انتہا{جواب}
%======================
\ابتدا{سوال}
$f(x,y)=2x^3+2y^3-9x^2+3y^2-12y$
\انتہا{سوال}
%====================
\ابتدا{سوال}
$f(x,y)=4xy-x^4-y^4$
\انتہا{سوال}
%====================
\ابتدا{جواب}
\wf{\unexpanded{
\عددی{f(0,0)}نقطہ زین؛ \عددی{f(1,1)=2}، \عددی{f(-1,-1)=2}   مقامی زیادہ سے زیادہ قیمت نقطہ
}}
\انتہا{جواب}
%======================
\ابتدا{سوال}
$f(x,y)=x^4+y^4+4xy$
\انتہا{سوال}
%====================
\ابتدا{سوال}
$f(x,y)=\frac{1}{x^2+y^2-1}$
\انتہا{سوال}
%====================
\ابتدا{جواب}
\wf{\unexpanded{
\عددی{f(0,0)=-1}  مقامی زیادہ سے زیادہ قیمت نقطہ
}}
\انتہا{جواب}
%======================
\ابتدا{سوال}
$f(x,y)=\frac{1}{x}+xy+\frac{1}{y}$
\انتہا{سوال}
%====================
\ابتدا{سوال}
$f(x,y)=y\sin x$
\انتہا{سوال}
%====================
\ابتدا{جواب}
\wf{\unexpanded{
\عددی{f(n\pi,0)}نقطہ زین؛ ہر \عددی{n} پر  \عددی{f(n\pi,0)=0}
}}
\انتہا{جواب}
%======================
\ابتدا{سوال}\شناخت{سوال_کثیرالمتغیر_تمام_زیادہ_کم_زین_ب}
$f(x,y)=e^{2x}\cos y$
\انتہا{سوال}
%====================

\موٹا{مطلق انتہا کی تلاش}\\
سوال \حوالہ{سوال_کثیرالمتغیر_مطلق_انتہا_الف} تا سوال \حوالہ{سوال_کثیرالمتغیر_مطلق_انتہا_ب} میں تفاعل کی مطلق انتہا تلاش کریں۔

\ابتدا{سوال}\شناخت{سوال_کثیرالمتغیر_مطلق_انتہا_الف}
ربع اول میں بند تکون، جس کے  سرحد   \عددی{x=0}، \عددی{y=2} اور \عددی{y=2x} ہیں، میں تفاعل  \عددی{f(x,y)=2x^2-4x+y^2-4y+1} ہے۔
\انتہا{سوال}
%=================
\ابتدا{جواب}
\wf{\unexpanded{
\عددی{(0,0)} پر مطلق زیادہ سے زیادہ قیمت \عددی{1} جبکہ  \عددی{(1,2)} پر مطلق کم سے کم قیمت \عددی{-5}
}}
\انتہا{جواب}
%======================
\ابتدا{سوال}
ربع اول میں بند تکون، جس کے اطراف \عددی{x=0}، \عددی{y=4} اور \عددی{y=x} ہیں،  میں تفاعل \عددی{f(x,y)=x^2-xy+y^2+1} ہے۔
\انتہا{سوال}
%===================
\ابتدا{سوال}
ربع اول میں بند تکون، جس کے اطراف \عددی{x=0}، \عددی{y=0} اور \عددی{y+2x=2} ہیں، میں تفاعل \عددی{f(x,y)=x^2+y^2} ہے۔ 
\انتہا{سوال}
%===========
\ابتدا{جواب}
\wf{\unexpanded{
\عددی{(0,2)} پر مطلق زیادہ سے زیادہ قیمت \عددی{4} جبکہ  \عددی{(0,0)} پر مطلق کم سے کم قیمت \عددی{0}
}}
\انتہا{جواب}
%======================
\ابتدا{سوال}
مستطیل پٹی \عددی{0\le x\le 5,\, -3\le y\le 3} پر تفاعل \عددی{T(x,y)=x^2+xy+y^2-6x} ہے۔
\انتہا{سوال}
%==================
\ابتدا{سوال}
مستطیل \عددی{0]le x\le 5,\, -3\le y\le 0} پر تفاعل \عددی{T(x,y)=x^2+xy+y^2-6x+2} ہے۔
\انتہا{سوال}
%=================
\ابتدا{جواب}
\wf{\unexpanded{
\عددی{(0,-3)} پر مطلق زیادہ سے زیادہ قیمت \عددی{11} جبکہ  \عددی{(4,-2)} پر مطلق کم سے کم قیمت \عددی{-10}
}}
\انتہا{جواب}
%======================
\ابتدا{سوال}
مستطیل \عددی{0\le x\le 1,\, 0\le y\le 1} پر تفاعل \عددی{f(x,y)=48xy-32x^3-24y^2} ہے۔
\انتہا{سوال}
%=================
\ابتدا{سوال}
مستطیل \عددی{1\le x\le 3,\, -\tfrac{\pi}{4}\le \tfrac{\pi}{4}} پر تفاعل \عددی{f(x,y)=(4x-x^2)\cos y} ہے۔
\انتہا{سوال}
%===============
\ابتدا{جواب}
\wf{\unexpanded{
\عددی{(2,0)} پر مطلق زیادہ سے زیادہ قیمت \عددی{4}  جبکہ  \عددی{(3,-\tfrac{\pi}{4})}، \عددی{(3,\tfrac{\pi}{4})}، \عددی{(1,-\tfrac{\pi}{4})} اور \عددی{(1,\tfrac{\pi}{4})}  پر مطلق کم سے کم قیمت \عددی{(\tfrac{3\sqrt{2}}{2})}
}}
\انتہا{جواب}
%======================
\ابتدا{سوال}\شناخت{سوال_کثیرالمتغیر_مطلق_انتہا_ب}
 تفاعل \عددی{f(x,y)=4x-8xy+2y+1}،   اضلاع \عددی{x=0}، \عددی{y=0} اور \عددی{x+y=1} میں بند خطہ میں ہے۔
\انتہا{سوال}
%===================
\ابتدا{سوال}
دو ایسے  اعداد \عددی{a} اور \عددی{b}، جہاں \عددی{a\le b} ہے،  تلاش کریں تا کہ درج ذیل کی قیمت زیادہ سے زیادہ ہو۔
\begin{align*}
\int_a^b(6-x-x^2)\dif x
\end{align*}
\انتہا{سوال}
%====================
\ابتدا{جواب}
\wf{\unexpanded{
\عددی{a=-3,\, b=2}
}}
\انتہا{جواب}
%======================
\ابتدا{سوال}
دو ایسے  اعداد \عددی{a} اور \عددی{b}، جہاں \عددی{a\le b} ہے،  تلاش کریں تا کہ درج ذیل کی قیمت زیادہ سے زیادہ ہو۔
\begin{align*}
\int_a^b(24-2x-x^2)^{1/3}\dif x
\end{align*}
\انتہا{سوال}
%====================
\ابتدا{سوال}\ترچھا{درجہ حرارت}\\
ایک دائری پٹی \عددی{x^2+y^2\le 1}  اور اس کی سرحد \عددی{x^2+y^2=1} کو یوں گرم کیا جاتا ہے کہ نقطہ \عددی{(x,y)} پر درجہ حرارت \عددی{T(x,y)x^2+2y^2-x} ہو۔اس پٹی پر زیادہ سے زیادہ اور کم سے کم درجہ حرارت  تلاش کریں۔
\انتہا{سوال}
%=============
\ابتدا{جواب}
\wf{\unexpanded{
\عددی{(-\tfrac{1}{2},\tfrac{\sqrt{3}}{2})} اور \عددی{(-\tfrac{1}{2},-\tfrac{\sqrt{3}}{2})} پر گرم ترین \عددی{ \SI{2.25}{\celsius}} جبکہ \عددی{(\tfrac{1}{2},0)} پر سرد  ترین \عددی{\SI{-0.25}{\celsius}}
}}
\انتہا{جواب}
%======================
\ابتدا{سوال}
کھلا  ربع اول \عددی{x>0,\, y>0} میں  \عددی{f(x,y)=xy+2x-\ln x^2y} کا نقطہ  فاصل تلاش کریں اور دکھائیں کہ اس نقطہ پر تفاعل کی قیمت کم سے کم ہو گی۔
\انتہا{سوال}
%==========

\موٹا{نظریہ اور مثالیں}\\
\ابتدا{سوال}
درج ذیل معلومات استعمال کرتے ہوئے    زیادہ سے زیادہ قیمت کے نقاط، کم سے کم قیمت کے نقاط اور نقاط زین، اگر موجود ہوں، تلاش کریں۔
\begin{enumerate}[a.]
\item
$f_x=2x-4y,\, f_y=2y-4x$
\item
$f_x=2x-2,\, f_y=2y-4$
\item
$f_x=9x^2-9,\, f_y=2y+4$
\end{enumerate}
ہر جواب کی وجہ بیان کریں۔
\انتہا{سوال}
%===============
\ابتدا{جواب}
\wf{\unexpanded{
(ا) \عددی{f(0,0)} نقطہ زین، (ب)  \عددی{f(1,2)} مقامی کم سے کم  قیمت نقطہ، (ج) \عددی{f(1,-2)} مقامی کم سے کم قیمت نقطہ، (د)  \عددی{f(-1,-2)} نقطہ زین
}}
\انتہا{جواب}
%======================
\ابتدا{سوال}
درج ذیل تفاعل کے لئے مبدا پر ممیز \عددی{f_{xx}f_{yy}-f_{xy}^2}   صفر ہے لہٰذا دو رتبی تفرقی پرکھ غیر  فیصلہ کن ہو گا۔ مبدا پر سطح \عددی{z=f(x,y)} کی ذہنی  تصویر کشی کرتے ہوئے  دریافت کریں کہ مبدا پر زیادہ سے زیادہ قیمت کا نقطہ، کم سے کم قیمت کا نقطہ یا نقطہ زین پایا جاتا ہے۔ ہر جواب کی وجہ پیش کریں۔
\begin{multicols}{2}
\begin{enumerate}[a.]
\item
$f(x,y)=x^2y^2$
\item
$f(x,y)=1-x^2y^2$
\item
$f(x,y)=xy^2$
\item
$f(x,y)=x^3y^2$
\item
$f(x,y)=x^3y^3$
\item
$f(x,y)=x^4y^4$
\end{enumerate}
\end{multicols}
\انتہا{سوال}
%============
\ابتدا{سوال}
دکھائیں کہ \عددی{k}  کی ہر قیمت کے لئے  \عددی{(0,0)} تفاعل \عددی{f(x,y)=x^2+kxy+y^2} کا نقطہ فاصل ہو گا۔ (اشارہ: دو صورتوں پر غور کریں: \عددی{k=0} اور \عددی{k\ne 0})
\انتہا{سوال}
%==================
\ابتدا{سوال}
مستقل \عددی{k} کی کن قیمتوں کے لئے دو رتبی تفرقی پرکھ ضمانت دیتا ہے کہ  \عددی{(0,0)} پر \عددی{f(x,y)=x^2+kxy+y^2} کا (ا)  نقطہ زین (ب) مقامی کم سے کم قیمت  کا نقطہ   پایا جائے گا؟  مستقل \عددی{k} کی کن قیمتوں کے لئے دو رتبی تفرقی پرکھ غیر  فیصلہ کن ہو گا؟ اپنے جوابات کی وجہ پیش کریں۔
\انتہا{سوال}
%==============
\ابتدا{سوال}
(ا) کیا \عددی{f_x(a,b)=f_y(a,b)=0} ہوتے ہوئے ہر   صورت \عددی{(a,b)} پر \عددی{f} کا مقامی زیادہ سے ی زیادہ قیمت کا نقطہ یا کم سے کم قیمت کا نقطہ پایا جائے گا؟ اپنے جواب کی وجہ پیش کریں۔   (ب)  اگر ایک قرص  میں ، جس کا مرکز  \عددی{(a,b)} ہو،  ہر نقطہ پر \عددی{f}  اور اس کے یک رتبی اور دو رتبی جزوی تفرقات استمراری ہوں، اور \عددی{f_{xx}(a,b)} اور \عددی{f_{yy}(a,b)} کی علامتیں ایک دوسرے سے مختلف ہوں  تب کیا \عددی{f} کے بارے میں کچھ کہنا ممکن ہو گا؟ اپنے جواب کی وجہ پیش کریں۔
\انتہا{سوال}
%=================
\ابتدا{سوال}
نقطہ \عددی{(a,b)} پر  \عددی{f} کا   مقامی  زیادہ سے زیادہ قیمت کا نقطہ ہونے کی صورت میں  مسئلہ \حوالہ{مسئلہ_کثیرالمتغیر_مقامی_انتہائی_قیمت_یک_رتبی_تفرقی_پرکھ} کا دیا گیا  ثبوت استعمال کرتے ہوئے  اس مسئلہ کو \عددی{(a,b)} پر مقامی کم سے کم قیمت کا نقطہ ہونے کی صورت کے لئے ثابت کریں۔
\انتہا{سوال}
%================
\ابتدا{سوال}
مستوی \عددی{x+2y+3z=0}  سے   زیادہ بلندی  پر  \عددی{z=10-x^2-y^2} کی ترسیم کے تمام نقاط میں وہ نقطہ تلاش کریں جو مستوی سے دور ترین ہو۔
\انتہا{سوال}
%=================
\ابتدا{جواب}
\wf{\unexpanded{
$(\tfrac{1}{6},\tfrac{1}{3},\tfrac{355}{36})$
}}
\انتہا{جواب}
%======================
\ابتدا{سوال}
مستوی \عددی{x+2y-z=0} سے \عددی{z=x^2+y^2+10} کی ترسیم کا قریب ترین نقطہ تلاش کریں۔
\انتہا{سوال}
%===============
\ابتدا{سوال}
بند ربع اول \عددی{x\ge 0,\, y\ge 0} میں تفاعل \عددی{f(x,y)=x+y} کی کوئی مطلق زیادہ سے زیادہ قیمت نہیں پائی جاتی ہے۔ کیا  اس حقیقت میں اور    کتاب میں  مطلق انتہا کی تلاش پر کی گئی گفتگو  میں تضاد پایا جاتا ہے؟ اپنے جواب کی وجہ پیش کریں۔
\انتہا{سوال}
%============
\ابتدا{سوال}
مربع \عددی{0\le x\le 1,\, 0\le y\le 1} میں تفاعل \عددی{f(x,y)=x^2+y^2+2xy-x-y+1} پر غور کریں۔
\begin{enumerate}[a.]
\item
دکھائیں کہ  اس مربع میں خطی قطع \عددی{2x+2y=1} پر \عددی{f} کی مطلق کم سے کم  قیمت پائی جاتی ہے۔ اس کم سے کم قیمت کو تلاش کریں۔
\item
مربع پر \عددی{f} کی مطلق زیادہ سے زیادہ قیمت تلاش کریں۔
\end{enumerate}
\انتہا{سوال}
%==============

\موٹا{مقدار معلوم منحنیات پر انتہائی قیمتیں}\\
منحنی \عددی{x=x(t),\,y=y(t)} پر تفاعل \عددی{f(x,y)}  کی انتہائی قیمتیں تلاش کرنے کی خاطر  ہم \عددی{f} کو واحد متغیر \عددی{t} کا تفاعل تصور کر کے  زنجیری قاعدہ سے \عددی{\tfrac{\dif f}{\dif t}}  معلوم کر کے صفر کے برابر  رکھتے  ہیں۔کسی بھی واحد متغیر تفاعل کی طرح، انتہائی قیمتوں کو درج ذیل نقطوں پر تلاش کیا جاتا ہے۔
\begin{enumerate}[a.]
\item
نقطہ فاصل پر  (جہاں \عددی{\tfrac{\dif f}{\dif t}} صفر ہو یا غیر موجود ہو)،   اور
\item
مقدار معلوم دائرہ کار کے آخری سروں پر۔
\end{enumerate}
سوال \حوالہ{سوال_کثیرالمتغیر_تفاعل_منحنی_انتہائی_الف} تا سوال \حوالہ{سوال_کثیرالمتغیر_تفاعل_منحنی_انتہائی_ب} میں تفاعل کی مطلق زیادہ سے زیادہ قیمت اور مطلق کم سے کم قیمتیں دی گئی منحنیات پر  دریافت کریں۔

\ابتدا{سوال}\شناخت{سوال_کثیرالمتغیر_تفاعل_منحنی_انتہائی_الف}
تفاعل:
\begin{multicols}{3}
\begin{enumerate}[a.]
\item
$f(x,y)=x+y$
\item
$g(x,y)=xy$
\item
$h(x,y)=2x^2+y^2$
\end{enumerate}
\end{multicols}
منحنیات:
\begin{multicols}{2}
\begin{enumerate}[a.]
\item
$x^2+y^2=4,\,y\ge 0$
\item
$x^2+y^2=4,\,x\ge 0,\,y\ge 0$
\end{enumerate}
\end{multicols}
مقدار معلوم مساوات \عددی{x=2\cos t,\, y=2\sin t} استعمال کریں۔
\انتہا{سوال}
%====================
\ابتدا{جواب}
\wf{\unexpanded{
(ا) نصف دائرہ : \عددی{t=\tfrac{\pi}{4}} پر مقامی زیادہ سے زیادہ قیمت \عددی{f=2\sqrt{2}} جبکہ \عددی{t=\pi} پر مقامی کم سے کم قیمت \عددی{f=-2}؛ چوتھائی  دائرہ: \عددی{t=\tfrac{\pi}{4}} پر مقامی زیادہ سے زیادہ قیمت \عددی{f=2\sqrt{2}} جبکہ \عددی{t=0} اور \عددی{t=\tfrac{\pi}{2}} پر مقامی کم سے کم قیمت \عددی{f=2} (ب)  نصف دائرہ: \عددی{t=\tfrac{\pi}{4}} پر مقامی زیادہ سے زیادہ قیمت \عددی{g=2} جبکہ \عددی{t=\tfrac{3\pi}{4}} پر مقامی کم سے کم قیمت \عددی{g=-2}؛  چوتھائی دائرہ: \عددی{t=\tfrac{\pi}{4}} پر مقامی زیادہ سے زیادہ قیمت \عددی{g=2} جبکہ \عددی{t=0} اور \عددی{t=\tfrac{\pi}{2}}   پر مقامی کم سے کم قیمت \عددی{g=0}؛ (ج) نصف دائرہ: \عددی{t=0} اور \عددی{t=\pi} پر مقامی زیادہ سے زیادہ قیمت  \عددی{h=8} جبکہ \عددی{t=\tfrac{\pi}{2}} پر کم سے کم قیمت \عددی{h=4}؛ چوتھائی دائرہ: \عددی{t=0} پر مقامی زیادہ سے زیادہ قیمت \عددی{h=8}  جبکہ \عددی{t=\tfrac{\pi}{2}} پر مقامی کم سے کم قیمت \عددی{h=4}
}}
\انتہا{جواب}
%======================
\ابتدا{سوال}
تفاعل:
\begin{multicols}{3}
\begin{enumerate}[a.]
\item
$f(x,y)=2x+3y$
\item
$g(x,y)=xy$
\item
$h(x,y)=x^2+3y^2$
\end{enumerate}
\end{multicols}
منحنیات:
\begin{multicols}{2}
\begin{enumerate}[a.]
\item
$\frac{x^2}{9}+\frac{y^2}{4}=1,\,y\ge 0$
\item
$\frac{x^2}{9}+\frac{y^2}{4}=1,\,x\ge 0,\,y\ge 0$
\end{enumerate}
\end{multicols}
مقدار معلوم مساوات \عددی{x=3\cos t,\, y=2\sin t} استعمال کریں۔
\انتہا{سوال}
%====================
\ابتدا{سوال}
تفاعل:
$f(x,y)=xy$
منحنیات:
\begin{multicols}{2}
\begin{enumerate}[a.]
\item
$x=2t,\,y=t+1$
\item
$x=2t,\,y=t+1,\,-1\le t\le 0$
\item
$x=2t,\,y=t+1,\,0\le t\le 1$
\end{enumerate}
\end{multicols}
\انتہا{سوال}
%====================
\ابتدا{جواب}
\wf{\unexpanded{
(ا) \عددی{t=-\tfrac{1}{2}} پر کم سے کم قیمت \عددی{f=-\tfrac{1}{2}} جبکہ کوئی  زیادہ سے زیادہ قیمت نہیں پائی جاتی ہے۔ (ب)  \عددی{t=-1} اور \عددی{t=0} پر زیادہ سے زیادہ  \عددی{f=0} جبکہ \عددی{t=-\tfrac{1}{2}} پر کم سے کم \عددی{f=-\tfrac{1}{2}}  پائی جاتی ہے۔ (ج) \عددی{t=1} پر زیادہ سے زیادہ \عددی{f=4} جبکہ \عددی{t=0} پر کم سے کم  قیمت \عددی{f=0} پائی جاتی ہے۔
}}
\انتہا{جواب}
%======================
\ابتدا{سوال}\شناخت{سوال_کثیرالمتغیر_تفاعل_منحنی_انتہائی_ب}
تفاعل:
\begin{multicols}{2}
\begin{enumerate}[a.]
\item
$f(x,y)=x^2+y^2$
\item
$g(x,y)=\frac{1}{x^2+y^2}$
\end{enumerate}
\end{multicols}
منحنیات:
\begin{multicols}{2}
\begin{enumerate}[a.]
\item
$x=t,\,y=2-2t$
\item
$x=t,\,y=2-2t,\,0\le t\le 0$
\end{enumerate}
\end{multicols}
\انتہا{سوال}
%====================
\موٹا{کمتر مربعے  اور خطوط  رجعت }\\
 اعدادی  نقاط  مواد \عددی{(x_1,y_1),(x_2,y_2),\cdots,(x_n,y_n)}  پر  سیدھا خط \عددی{y=mx+b} بٹھاتے ہوئے  ہم  نقطوں سے خط تک افقی فاصلوں  کے مربع کے مجموعہ کو کم سے کم رکھتے ہیں۔ ایسا کرنے کی خاطر ہمیں \عددی{m} اور \عددی{b} کی وہ قیمتیں تلاش کرنی ہوں گی جو درج ذیل کی قیمت کم سے کم کرتے ہوں۔
\begin{align}\label{مساوات-کثیرالمتغیر_خط_رجعت_الف}
w=(mx_1+b-y_1)^2+\cdots+(mx_n+b-y_n)^2
\end{align}
یک رتبی اور دو رتبی تفرقی پرکھ سے\عددی{m} اور \عددی{b} کی   مطلوبہ    قیمتیں
\begin{align}
m&=\frac{(\sum x_k)(\sum y_k)-n\sum x_ky_k}{(\sum x_k)^2-n\sum x_k^2}\label{مساوات-کثیرالمتغیر_خط_رجعت_ب}\\
b&=\frac{1}{n}(\sum y_k-m\sum x_k)\label{مساوات-کثیرالمتغیر_خط_رجعت_پ}
\end{align} 
حاصل ہوتی ہیں جہاں تمام مجموعے \عددی{k=1} تا \عددی{k=n} لئے گئے ہیں۔ عموماً کیلکولیٹر میں یہ کلیات دیے گئے ہوں گے۔

وہ خط  \عددی{y=mx+b} جس میں \عددی{m} اور \عددی{b}  کی مذکورہ بالا قیمتیں استعمال کی گئی ہوں  زیر غور مواد کا  \اصطلاح{خط رجعت}\فرہنگ{رجعت!خط}\فرہنگ{خط رجعت}\حاشیہب{regression line}\فرہنگ{line!regression}  کہلاتا ہے۔  کمتر مربعی خط کی مدد سے (ا) آپ مواد کو ایک سادہ مساوات سے ظاہر کر  پاتے  ہیں، (ب) متغیر \عددی{x} کی دیگر قیمتوں کے لئے \عددی{y} کی قیمت کی پیش گوئی کر پاتے ہیں، (ج) اور مواد پر  تحلیلی غور کر سکتے ہیں۔  مثال کے طور پر نقاط \عددی{(0,1)}، \عددی{(1,3)}، \عددی{(2,2)}، \عددی{(3,4)}، \عددی{(4,5)}  کو    جدول
\begin{center}
\begin{tabular}{RRRRR}
k&x_k&y_k&x_k^2&x_ky_k\\
\midrule
1&0&1&0&0\\
2&1&3&1&3\\
3&2&2&4&4\\
4&3&4&9&12\\
5&4&5&16&20\\
\bottomrule
\sum&10&15&30&39
\end{tabular}
\end{center}
کی صورت میں لکھ کر
\begin{align*}
m&=\frac{(10)(15)-5(39)}{(10)^2-5(30)}=0.9&&\text{\RL{مساوات \حوالہ{مساوات-کثیرالمتغیر_خط_رجعت_ب}}}\\
b&=\frac{1}{5}(15-(0.9)(10))=1.2&&\text{\RL{مساوات \حوالہ{مساوات-کثیرالمتغیر_خط_رجعت_پ}}}
\end{align*}
حاصل ہوں گے۔ یوں ان  نقاط مواد  کا  خط رجعت \عددی{y=0.9x+1.2} ہو گا۔

سوال \حوالہ{سوال_کثیرالمتغیر_خط_رجعت_الف} تا سوال \حوالہ{سوال_کثیرالمتغیر_خط_رجعت_ب} میں مساوات \حوالہ{مساوات-کثیرالمتغیر_خط_رجعت_ب} اور مساوات \حوالہ{مساوات-کثیرالمتغیر_خط_رجعت_پ}  استعمال کرتے ہوئے  ہر نقاط مواد کے لئے خط رجعت تلاش کریں۔یوں حاصل خطی مساوات استعمال کرتے ہوئے \عددی{x=4} کے لئے \عددی{y} کی قیمت کی پیش گوئی کریں۔

\ابتدا{سوال}\شناخت{سوال_کثیرالمتغیر_خط_رجعت_الف}
$(-1,2),(0,1),(3,-4)$
\انتہا{سوال}
%====================
\ابتدا{جواب}
\wf{\unexpanded{
$y=-\frac{20}{13}x+\frac{9}{13},\quad \left.y\right\vert_{x=4}=-\frac{71}{13}$
}}
\انتہا{جواب}
%======================
\ابتدا{سوال}
$(-2,0),(0,2),(2,3)$
\انتہا{سوال}
%====================
\ابتدا{سوال}
$(0,0),(1,2),(2,3)$
\انتہا{سوال}
%====================
\ابتدا{جواب}
\wf{\unexpanded{
$y=\frac{3}{2}x+\frac{1}{6},\quad \left.y\right\vert_{x=4}=\frac{37}{6}$
}}
\انتہا{جواب}
%======================
\ابتدا{سوال}\شناخت{سوال_کثیرالمتغیر_خط_رجعت_ب}
$(0,1),(2,2),(3,2)$
\انتہا{سوال}
%====================
\ابتدا{سوال}\شناخت{سوال_کثیرالمتغیر_پیداوار_پانی}
جدول \حوالہ{جدول_سوال_کثیرالمتغیر_پیداوار_پانی}   میں  پانی کی مقدار (گہرائی)  بالمقابل لوسن  کی اوسط   پیداوار (کلوگرام فی ایکڑ) دی گئی ہے۔ اس کی خطی مساوات تلاش کریں۔ اس مواد کو اور  خطی مساوات کو  ترسیم کریں۔
\انتہا{سوال}
%=========
\ابتدا{جواب}
\wf{\unexpanded{
$y=51.3x+3467$
}}
\انتہا{جواب}
%======================
\begin{table}
\centering
\begin{subtable}{0.45\textwidth}
\caption{لوسن کی پیداوار}
\label{جدول_سوال_کثیرالمتغیر_پیداوار_پانی}
\centering
\begin{tabular}{CC}
\text{\RL{پانی کی گہرائی}}&\text{\RL{فی ایکڑ پیداوار}}\\
x,\,[\si{\centi\meter}]&y,\,[\si{\kilo\gram}]\\
\midrule
30&5270\\
45&5680\\
60&6250\\
75&7210\\
90&8200\\
100&8710\\
\bottomrule
\end{tabular}
\end{subtable}\hfill
\begin{subtable}{0.45\textwidth}
\caption{مریخ پر گڑھے}
\label{جدول_سوال_کثیرالمتغیر_مریخ_گڑھے}
\centering
\begin{tabular}{CCC}
\toprule
&\text{\RL{جماعتی وقفہ کی }}&\\
\text{قطر}&\text{\RL{بایاں قیمت کے لئے}}&\text{تعدد}\\
D\,[\si{\kilo\meter}]&\frac{1}{D^2}&F\\
\midrule
32-45&0.001&51\\
45-64&0.0005&22\\
64-90&0.00024&14\\
90-128&0.000123&4\\
\bottomrule
\end{tabular}
\end{subtable}
\end{table}
\ابتدا{سوال}\شناخت{سوال_کثیرالمتغیر_مریخ_گڑھے}\ترچھا{مریخ  پر گڑھے}\\
ایک نظریہ کہتا ہے کہ  گڑھوں کی تعدد قطر کے مربع کی بالعکس متناسب ہو گی۔ مریخ کی سطح کی تصویر سے جدول \حوالہ{جدول_سوال_کثیرالمتغیر_مریخ_گڑھے} میں دی گئی معلومات حاصل کی جاتی ہے۔ اس مواد پر \عددی{F=m(1/D^2)+b}  طرز کی لکیر بٹھائیں۔ مواد اور اس لکیر کو ترسیم کریں۔
\انتہا{سوال}
%=======================

\موٹا{کمپیوٹر کا استعمال}\\
سوال \حوالہ{سوال_کثیرالمتغیر_کمپیوٹر_نقاط_زین_فاصل_ہم_قد_الف} تا سوال \حوالہ{سوال_کثیرالمتغیر_کمپیوٹر_نقاط_زین_فاصل_ہم_قد_ب} میں تفاعل پر غور کرتے ہوئے  ان کے مقامی انتہائی نقاط تلاش کرنا مقصود ہے۔ کمپیوٹر استعمال کرتے ہوئے درج ذیل اقدام کریں۔
\begin{enumerate}[a.]
\item
دیے گئے  مستطیل پر تفاعل ترسیم کریں۔
\item
مستطیل میں چند ہم قد منحنیات ترسیم کریں۔
\item
تفاعل کا یک رتبی تفرق لے کر  کمپیوٹر کی مدد سے اس مساوات کو حل کر کے  نقاط فاصل تلاش کریں۔ نقاط فاصل اور ہم قد منحنیات کے بیچ  کیا تعلق نظر آتا ہے؟ کون سے نقاط فاصل پر نقاط زین پائے جاتے ہیں؟ اپنے جوابات کی وجہ پیش کریں۔ 
\item
تفاعل کے دو رتبی تفرقات تلاش کر کے ممیز \عددی{f_{xx}f_{yy}-f_{xy}^2} معلوم کریں۔
\item
زیادہ سے زیادہ اور کم سے کم پرکھ استعمال کرتے ہوئے جزو-ج کے نقاط فاصل کی جماعت بندی کریں۔ کیا  یہ   معلومات آپ کی جزو-ج میں دی گئی وجوہات سے مطابقت رکھتی ہے۔
\end{enumerate}

\ابتدا{سوال}\شناخت{سوال_کثیرالمتغیر_کمپیوٹر_نقاط_زین_فاصل_ہم_قد_الف}
$f(x,y)=x^2+y^3-3xy,\quad -5\le x\le 5,\, -5\le y\le 5$
\انتہا{سوال}
%===============
\ابتدا{سوال}
$f(x,y)=x^3-3xy^2+y^2,\quad -2\le x\le 2,\quad -2\le y\le 2$
\انتہا{سوال}
%===============
\ابتدا{سوال}
$f(x,y)=x^4+y^2-8x^2-6y+16,\quad -3\le x\le 3,\quad -6\le y\le 6$
\انتہا{سوال}
%===============
\ابتدا{سوال}
$f(x,y)=2x^4+y^4-2x^2-2y^2+3,\quad -\frac{3}{2}\le x\le \frac{3}{2},\quad -\frac{3}{2}\le y\le \frac{3}{2}$
\انتہا{سوال}
%===============
\ابتدا{سوال}
$f(x,y)=5x^6+18x^5-30x^4+30xy^2-120x^3,$\\
$ -4\le x\le 3,\quad -2\le y\le 2$
\انتہا{سوال}
%===============
\ابتدا{سوال}\شناخت{سوال_کثیرالمتغیر_کمپیوٹر_نقاط_زین_فاصل_ہم_قد_ب}
\begin{align*}
&f(x,y)=\begin{cases}x^5\ln(x^2+y^2) &(x,y)\ne (0,0)\\ 0&(x,y)=(0,0)\end{cases}\\
&-2\le x\le 2,\quad -2\le y\le y
\end{align*}
\انتہا{سوال}
%===============
\انتہا{سوالات}
%%%%%%%%%%%%%%%%%%%
