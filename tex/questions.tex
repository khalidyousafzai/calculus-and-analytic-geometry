\حصہ{گولا کے حرکت کی نمونہ کشی}
ایک گولا چلانے سے پہلے  ہم جاننا چاہیں گے کہ آیا وہ حدف کو مار سکے گا (کیا حدف تک پہنچے گا)؟   یہ گولا کس  بلندی  تک  پہنچے گا (کیا یہ  پہاڑی کو پار کر پائے گا)؟ اور  یہ حدف پر کتنی دیر میں پہنچے گا (نتائج   کب حاصل  ہوں گے)؟ یہ تمام معلومات  گولے کی ابتدائی سمتیہ رفتار سے نیوٹن کے دوسرے قانون کی مدد سے حاصل کی جا سکتی ہیں۔

\جزوحصہء{گولا کی حرکت کی مقدار معلوم مثالی  مساوات}
حرکت  گولا کی مثالی مساوات حاصل کرتے ہوئے ہم  فرض کرتے ہیں کہ یہ ایک ذرہ کی مانند مستوی میں حرکت کرتا ہے اور اس پر صرف  مستقل   قوت کشش   سیدھا نیچے رخ عمل   کرتی ہے۔ حقیقت میں یہ مفروضے  درست نہیں  ہیں۔ زمین گھومنے کی بنا گولے کے نیچے    زمین  حرکت میں ہوتی  ہے، ہوائی  رگڑ  جو  گولے کی رفتار اور بلندی پر منحصر ہے گولا    پر عمل کرتی ہے، اور قوت کشش ایک مستقل نہیں ہے بلکہ اس کی قیمت  گولا کی بلندی پر منحصر ہے۔اگرچہ ان تمام کے اثرات کو بھی دیکھنا ہو گا، ہم یہاں انہیں نظر انداز کرتے ہیں۔

ہم فرض کرتے ہیں کہ لمحہ \عددی{t=0} پر ابتدائی سمتیہ رفتار \عددی{\kvec{v}_0} کے ساتھ  مبدا سے گولا      ربع اول میں مارا   جاتا ہے۔ اگر  افقی زمین کے ساتھ \عددی{\kvec{v}_0} کا زاویہ \عددی{\alpha} ہو تب
\begin{align}
\kvec{v}_0=(\abs{\kvec{v}_0}\cos\alpha)\ai+(\abs{\kvec{v}_0}\sin\alpha)\aj
\end{align}
ہو گا۔اس میں \عددی{\abs{\kvec{v}_0}} کو سادہ علامت   \عددی{v_0} سے ظاہر کرتے ہوئے درج ذیل لکھا جا سکتا ہے۔
 \begin{align}
\kvec{v}_0=(v_0\cos\alpha)\ai+(v_0\sin\alpha)\aj
\end{align}
گولا کا ابتدائی مقام  درج ذیل ہو گا۔
\begin{align*}
\kvec{r}_0=0\ai+0\aj=\kvec{0}
\end{align*}
