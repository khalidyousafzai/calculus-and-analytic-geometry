\جزوحصہء{سوالات}
\ابتدا{سوال}
سکائے  لیب  4 کا نصف محور اکبر \عددی{a=\SI{6808}{\kilo\meter}} ہے۔کپلر کے تیسرے قانون  میں زمین کی کمیت کو \عددی{M} لیتے ہوئے دوری عرصہ  معلوم کیا جا سکتا ہے۔اس کا حساب لگائیں۔ جدول \حوالہ{جدول_سمتی_تفاعل_مصنوعی_سیارے_مواد} میں دی گئی  قیمت کے ساتھ موازنہ کریں۔
\انتہا{سوال}
%===============
\ابتدا{سوال}
حضیض شمسی پر سورج سے زمین کا فاصلہ تقریباً \عددی{\SI{149577000}{\kilo\meter}} ہو تا ہے اور  سورج کے گرد زمین کے مدار کی سنک \عددی{0.0167} ہے۔ حضیض شمس پر زمین کی رفتار \عددی{v_0} تلاش کریں (مساوات \حوالہ{مساوات_سمتی_تفاعل_سنک_رفتار_رداس} استعمال کریں)۔
\انتہا{سوال}
%==============
\ابتدا{سوال}
روس نے جولائی \سن{1965} میں پروٹان  1، مصنوعی سیارہ مدار میں چھوڑا جس کی کمیت  (چھوڑتے وقت ) \عددی{\SI{12200}{\kilo\gram}}،  بلندی حضیض \عددی{\SI{183}{\kilo\meter}}، بلندی اوج \عددی{\SI{589}{\kilo\meter}} اور دوری عرصہ \عددی{92.25} منٹ  تھا۔  زمین کی کمیت اور تجاذبی مستقل  کی قیمتیں استعمال کر کے مساوات \حوالہ{مساوات_سمتی_تفاعل_کپلر_تیسرا_الف} سے  نصف محور اکبر \عددی{a}  تلاش کریں۔اس کا موازنہ اس عدد سے کریں جو حضیض اور اوج کے مجموعہ کے ساتھ  زمین کا قطر جمع  کرنے سے حاصل ہو گا۔
\انتہا{سوال}
%==============
\ابتدا{سوال}
(ا)  وائکنگ 1 مصنوعی سیارہ ، جس کا دوری عرصہ \عددی{1639} منٹ تھا ، نے   اگست 1975 تا جون 1976 مریخ کا جائزہ  کیا۔  مریخ کی کمیت \عددی{\SI{6.418e23}{\kilo\gram}} لیتے ہوئے  وائکنگ 1 کا نصف محور اکبر تلاش کریں۔ (ب) مریخ کی سطح  سے وائکنگ 1 کا کم سے کم فاصلہ \عددی{\SI{1499}{\kilo\meter}} اور زیادہ سے زیادہ فاصلہ \عددی{\SI{35800}{\kilo\meter}}  تھا۔ ان حقائق اور جزو-ا میں حاصل نتائج کو  استعمال کرتے ہوئے مریخ کے اوسط  قطر کی   اندازاً  قیمت معلوم کریں۔
\انتہا{سوال}
%============
\ابتدا{سوال}
وائکنگ 2  مصنوعی سیارہ نے ستمبر 1975 تا اگست 1976  مریخ کا جائزہ  کیا۔ اس کے نصف محور اکبر \عددی{\SI{22030}{\kilo\meter}} تھا۔ اس کا دوری عرصہ دریافت کریں۔
\انتہا{سوال}
%==============
\ابتدا{سوال}\ترچھا{ہم عصر  مدار}\\
زمین کی استوائی مستوی میں کئی مصنوعی سیاروں  کے مدار تقریباً دائری ہے اور ان کا دوری عرصہ عین  ایک دن کے برابر ہے۔  یوں یہ بلندی پر رہتے ہوئے   سطح زمین کے اوپر ساکن نظر آتے ہیں۔ ایسے مدار کو \اصطلاح{ہم عصر مدار}\فرہنگ{مدار!ہم عصر}\حاشیہب{geosynchronous orbit, geostationary orbit}\فرہنگ{orbit!geosynchronous}\فرہنگ{orbit!geostationary} کہتے ہیں۔
\begin{enumerate}[a.]
\item
ہم عصر مصنوعی سیارے کا نصف محور اکبر تقریباً کتنا ہو گا؟ اپنے جواب کی وجہ پیش کریں۔
\item
زمین کی سطح سے ہم عصر مدار کتنی  بلندی پر ہو گا؟
\item
جدول  \حوالہ{جدول_سمتی_تفاعل_مصنوعی_سیارے_مواد} میں دیے گئے مصنوعی سیاروں میں کس کا مدار تقریباً ہم عصر ہے؟
\end{enumerate}
\انتہا{سوال}
%===================
\ابتدا{سوال}
مریخ کی کمیت \عددی{\SI{6.418e23}{\kilo\gram}} ہے  جبکہ مریخ کا ایک  دن \عددی{1477.4} منٹ  ہے۔مریخ کے گرد مدار میں ایک مصنوعی سیارہ جس کا  دوری عرصہ مریخی دن کے برابر ہو، سطح مریخ سے کتنی بلندی پر ہو گا؟ اپنے جواب کی وجہ پیش کریں۔
\انتہا{سوال}
%===========
\ابتدا{سوال}
زمین کے گرد چاند  کا دوری عرصہ \عددی{\num{2.36055e6}} سیکنڈ  ہے۔چاند کتنا دور ہے؟
\انتہا{سوال}
%==============
\ابتدا{سوال}
زمین کے گرد ایک مصنوعی سیارہ دائری مدار میں حرکت کرتا  ہے۔ مصنوعی سیارے کی رفتار کو مدار کے رداس کا تفاعل لکھیں۔
\انتہا{سوال}
%==============
\ابتدا{سوال}
نظام شمسی میں سیاروں کا \عددی{\tfrac{T^2}{a^3}} کتنا ہو گا؟ زمین کے گرد مصنوعی سیاروں کے لئے کتنا ہو گا؟ چاند کے گرد مصنوعی سیاروں کے لئے یہ کتنا ہو گا؟ (چاند کی کمیت \عددی{\SI{7.354e22}{\kilo\gram}} ہے۔)
\انتہا{سوال}
%==============
\موٹا{بغیر کیلکولیٹر استعمال کئے قلم و کاغذ سے حل کریں}\\
\ابتدا{سوال}
مساوات \حوالہ{مساوات_سمتی_تفاعل_سنک_رفتار_رداس} میں  \عددی{v_0} کی کس قیمت کے لئے  مساوات \حوالہ{مساوات_سمتی_تفاعل_سنک_رداس} کا مدار دائری ہو گا؟ ترخیمی ہو گا؟ قطع مکافی ہو گا؟ قطع زائد ہو گا؟
\انتہا{سوال}
%==========
\ابتدا{سوال}
دکھائیں کہ دائری مدار میں سیارہ یکساں رفتار سے حرکت کرتا ہے۔ (اشارہ: یہ قوانین کپلر کی بدولت ہو گا۔)
\انتہا{سوال}
%================
\ابتدا{سوال}
فرض کریں  ایک مستوی میں متحرک ذرے کا تعین گر سمتیہ \عددی{\kvec{r}} ہے اور  یہ  سمتیہ \عددی{\tfrac{\dif S}{\dif t}} کی شرح سے رقبہ واضح کرتا ہے۔محدد متعارف کئے بغیر اور مطلوبہ تفرقات  کی موجودگی تصور کرتے ہوئے، بڑھوتری اور حد  پر مبنی  درج ذیل مساوات کی  جیومیٹریائی جواز پیش کریں۔ 
\begin{align*}
\frac{\dif S}{\dif t}=\frac{1}{2}\abs{\kvec{r}\times \dot{\kvec{r}}}
\end{align*}
\انتہا{سوال}
%============
\ابتدا{سوال}\شناخت{سوال_سمتی_تفاعل_اشتقاق_کپلر_تیسرا}
کپلر کے تیسرے قانون  کا اشتقاق  پورا کریں (مساوات \حوالہ{مساوات_سمتی_تفاعل_کپلر_تیسرا_ب} کے بعد  حصہ۔)
\انتہا{سوال}
%============
\ابتدا{سوال}
کسی ستارہ کے گرد دو سیارے دائری مدار میں  طواف  کرتے ہیں۔سیارہ \عددی{A} ستارے کے  قریب ہے جبکہ سیارہ \عددی{B}ستارہ سے  زیادہ فاصلہ پر ہے۔فرض کریں لمحہ \عددی{t} پر ان کے مقام  بالترتیب 
\begin{align*}
\kvec{r}_A(t)&=2\cos(2\pi t)\ai+2\sin(2\pi t)\aj\\
\kvec{r}_B(t)&=3\cos(\pi t)\ai+3\sin(\pi t)\aj
\end{align*}
ہیں جہاں ستارہ کا مقام مبدا ہے اور فاصلوں کو فلکیاتی اکائیوں میں ناپا گیا ہے۔ (دھیان رہے کہ سیارہ \عددی{A} کی رفتار سیارہ \عددی{B} سے زیادہ ہے۔)

سیارہ \عددی{A} پر رہائش پذیر  لوگوں کا خیال ہے کہ ان کا سیارہ، ان کے شمسی نظام کا مرکز ہے۔
\begin{enumerate}[a.]
\item
سیارہ \عددی{A} کو نئی محددی نظام کا  مبدا  تصور کرتے ہوئے سیارہ \عددی{B} کے مقام کی مقدار معلوم مساوات تلاش کریں۔ اپنا جواب \عددی{\cos(\pi t)} اور \عددی{\sin(\pi t)}کی صورت میں لکھیں۔
\item
سیارہ \عددی{A} کو مبدا  تصور کرتے ہوئے  سیارہ \عددی{B} کی راہ ترسیم کریں۔

آپ دیکھ سکتے ہیں کہ ان لوگوں کو سیاروں کی حرکت سمجھنے میں کتنی دشواری ہو گی۔ کپلر سے پہلے یہی حال ہمارا تھا۔ 
\end{enumerate}
\انتہا{سوال}
%================
\ابتدا{سوال}
کپلر نے دریافت کیا  کہ سورج کے گرد زمین ترخیمی راہ  پر  طواف کرتی ہے اور سورج اس کے ایک ماسکہ پر پایا جاتا ہے۔ سورج کے مرکز سے زمین کے مرکز تک لمحہ \عددی{t} پر تعین گر  سمتیہ \عددی{\kvec{r}(t)} لیں۔ زمین کے جنوبی قطب  سے شمالی قطب تک سمتیہ \عددی{\kvec{w}} لیں۔ہم جانتے ہیں کہ \عددی{\kvec{w}} مستقل ہے اور ترخیم  کے مستوی  کو عمودی نہیں ہے (زمین کا محور  جھکا  ہے)۔ سمتیات \عددی{\kvec{w}} اور \عددی{\kvec{r}(t)} کے روپ میں (ا)  حضیض شمسی، (ب) اوج شمسی، (ج)  اعتدالین  (جب دن اور رات ایک دوسرے کے برابر ہوں)، (د) لمبا ترین دن (گرم ترین دن)، (ہ) چھوٹا ترین دن ( سرد ترین دن) کے  ریاضی  معنی  پیش کریں۔
\انتہا{سوال}
%========
