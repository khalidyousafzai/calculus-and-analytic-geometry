\حصہ{سمتی میدان، کام، دورانیت، اور بہاو}
ان طبیعی  مظہر  کے مطالعہ کے دوران، جنہیں سمتیات سے ظاہر کیا جاتا ہے، بند راہ پر تکملات کی بجائے سمتی میدان میں راہ پر تکملات استعمال کیے جاتے ہیں۔متغیر قوت کے خلاف ایک مقام سے دوسری مقام کسی جسم کو منتقل کرنے (جیسا قوت ثقل کے خلاف خلاء میں سواری بھیجنے)   یا سمتی میدان میں ایک جسم کو کسی راہ پر حرکت دینے (جیسا مسرع کسی ذرے کی توانائی بڑھاتا ہو)  کے لئے درکار کام   اس طرح کے تکملات سے حاصل کیے  جاتے ہیں۔  منحنیات کے پار سیال کے بہاو کی شرح بھی لکیری تکملات سے حاصل کی جاتی ہے۔

\جزوحصہء{سمتی میدان}
مستوی یا فضا میں دائرہ کار پر  \اصطلاح{سمتی میدان}\فرہنگ{میدان!سمتی}\حاشیہب{vector field}\فرہنگ{field!vector} سے مراد ایسا تفاعل ہے جو دائرہ کار کے ہر نقطہ کو ایک سمتیہ مختص کرتا ہو۔سہ ابعادی سمتیات کے میدان کا ایک کلیہ درج ذیل ہو سکتا ہے۔
\begin{align*}
\kvec{F}(x,y,z)=M(x,y,z)\ai+N(x,y,z)\aj+P(x,y,z)\ak
\end{align*}
استمراری جزوی تفاعل \عددی{M}، \عددی{N}، \عددی{P} کی صورت میں یہ میدان استمراری ہو گا، قابل تفرق \عددی{M}، \عددی{N}، \عددی{P} کی صورت میں یہ میدان قابل تفرق ہو گا، وغیرہ وغیرہ۔ دو ابعادی سمتیات کے میدان کا ایک کلیہ درج ذیل ہو سکتا ہے۔
\begin{align*}
\kvec{F}(x,y)=M(x,y)\ai+N(x,y)\aj
\end{align*}
گول انداز کی گزرگاہ کے مستوی میں  گزرگاہ کے ہر نقطہ کے ساتھ گول انداز کی سمتی رفتار کا سمتیہ منسلک کرنے سے گزرگاہ  کے ساتھ ساتھ دو ابعادی میدان حاصل ہو گا۔  غیر سمتی تفاعل کے ہم قد سطح کے ہر نقطہ کے ساتھ  تفاعل کا سمتیہ ڈھلوان منسلک کرنے سے سطح پر سہ ابعادی میدان حاصل ہو گا۔   متحرک سیال کے ہر نقطہ کے ساتھ سمتی رفتار کا سمتیہ منسلک کرنے سے فضا میں اس خطہ پر سہ ابعادی  میدان حاصل ہو گا۔ بشمول ان  کے چند  میدان  شکل میں دکھائے گئے ہیں جہاں کچھ میدانوں کے کلیات بھی دیے گئے ہیں۔
 

وہ میدان ترسیم کرنے کے لئے جن کے کلیات معلوم ہوں، ہم دائرہ کار میں چند نقطے منتخب کر کے ان نقطوں پر نقطوں کے ساتھ منسلک سمتیات  کا خاکہ بناتے ہیں۔ دھیان رہے کہ روایتی طور پر اس نقطہ پر، جہاں سمتی تفاعل کی قیمت حاصل کی گئی ہو،  سمتیہ ظاہر کرنے والی تیر دار لکیر  کی دم  رکھی جاتی ہے نا کہ سر۔ تعین گر سمتیات (باب \حوالہ{سمتی_قیمت_تفاعل_اور_فضا_میں_حرکت}) کے لئے ایسا نہیں کیا جاتا ہے بلکہ تعین گر سمتیات کو ظاہر کرنے والی تیر دار لکیر کی دم کو مبدا پر رکھا جاتا ہے جبکہ اس کا سر  سیارہ یا گول انداز کے مقام پر رکھا جاتا ہے۔

\جزوحصہء{میدان ڈھلوان}
\ابتدا{تعریف}
قابل تفرق تفاعل \عددی{f(x,y,z)} کے \اصطلاح{میدان ڈھلوان}\فرہنگ{میدان!ڈھلوان}\حاشیہب{gradient field}\فرہنگ{gradient!field} سے مراد سمتیات ڈھلوان
\begin{align*}
\nabla f=\frac{\partial f}{\partial x}\ai+\frac{\partial f}{\partial y}\aj+\frac{\partial f}{\partial z}\ak
\end{align*}
 کا میدان ہے۔
\انتہا{تعریف}
%=======================
\ابتدا{مثال}
تفاعل \عددی{f(x,y,z)=xyz} کا میدان ڈھلوان تلاش کریں۔

حل:\quad
تفاعل \عددی{f} کے میدان ڈھلوان سے مراد میدان \عددی{\kvec{F}=\nabla f=yz\ai+xz\aj+xy\ak} ہے۔
\انتہا{مثال}
%=====================

 ہم  دیکھیں گے کہ انجینئری، ریاضیات، طبیعیات، وغیرہ میں میدان ڈھلوان  خصوصی اہمیت رکھتے ہیں۔

\جزوحصہء{فضا میں منحنی پر چلتے  ہوئے قوت کا کام}
فرض کریں  فضا کے ایک خطہ  میں سمتی میدان \عددی{\kvec{F}=M(x,y,z)\ai+N(x,y,z)\aj+P(x,y,z)\ak}  ایک قوت کو ظاہر کرتا ہے (یہ قوت ثقل  یا کسی قسم کی برقناطیسی قوت ہو سکتی ہے) جبکہ اس خطہ میں درج ذیل ایک ہموار منحنی ہے۔
\begin{align*}
\kvec{r}(t)=g(t)\ai+h(t)\aj+k(t)\ak,\quad a\le t\le b
\end{align*}
ایسی صورت میں منحنی پر \عددی{\kvec{F}\cdot\kvec{T}}،  اکائی مماسی سمتیہ کے رخ \عددی{\kvec{F}} کا غیر سمتی جزو، کے تکمل کو \عددی{a} تا \عددی{b} \عددی{\kvec{F}} کا کام کہتے ہیں۔

\ابتدا{تعریف}
ہموار منحنی \عددی{\kvec{r}(t)=g(t)\ai+h(t)\aj+k(t)\ak} پر \عددی{a} تا \عددی{b} قوت
 \عددی{\kvec{F}=M(x,y,z)\ai+N(x,y,z)\aj+P(x,y,z)\ak} کا کام \عددی{W} درج ذیل ہو گا۔
\begin{align}\label{مساوات_سمتی_میدان_کام_کی_تعریف}
W=\int_{t=1}^{t=b}\kvec{F}\cdot \kvec{T}\dif s
\end{align}
\انتہا{تعریف}
%=======================

 مثبت محور \عددی{x} رخ  مقدار \عددی{F(x)} کی استمراری قوت کے کام کا کلیہ  \عددی{W=\int_a^bF(x)\dif x} حصہ \حوالہ{حصہ_تکمل_استعمال_کام} میں اخذ کیا گیا۔مساوات \حوالہ{مساوات_سمتی_میدان_کام_کی_تعریف} کے حصول کے دلائل وہی ہیں۔ ہم منحنی کو چھوٹے قطعات میں تقسیم کر کے، ہر قطع پر کام کو تخمیناً مستقل قوت ضرب فاصلہ لے کر،  نتائج کے مجموعہ کو منحنی پر کام کی تخمینی قیمت حاصل کرتے ہیں، اور قطعات کی تعداد زیادہ سے زیادہ کر کے ہر قطع کی لمبائی کم سے کم کرتے ہوئے، تخمینی مجموعات کی تحدیدی قیمت کو کام کی تعریف لیتے ہیں۔ یہ جاننے کے لئے کہ تحدیدی تکمل کی قیمت کیا ہو گی، ہم قطع \عددی{ِ=[a,b]} کی خانہ بندی عمومی طرح  کر کے ہر ذیلی قطع \عددی{[t_k,t_{k+1}]} پر نقطہ \عددی{c_k} منتخب کرتے ہیں۔ قطع \عددی{I} کی خانہ بندی  منحنی کی خانہ بندی تعین (پیدا) کرتی ہے، جہاں تعین گر سمتیہ \عددی{\kvec{r}} کا سر نقطہ \عددی{P_k} پر ہو گا اور ذیلی قطع \عددی{P_kP_{k+1}} کی لمبائی \عددی{\Delta s_k} ہو گی۔ اگر منحنی پر \عددی{t=c_k} کے مطابقتی نقطہ  پر \عددی{\kvec{F}} کی قیمت \عددی{\kvec{F}_k} اور منحنی کا مماسی سمتیہ \عددی{\kvec{T}_k} ہو تب \عددی{t=c_k}  پر \عددی{\kvec{T}} کے رخ \عددی{\kvec{F}} کا غیر سمتی جزو \عددی{\kvec{F}_k\cdot\kvec{T}_k} ہو گا۔ منحنی کے قطع \عددی{P_kP_{k+1}} پر چلتے ہوئے \عددی{\kvec{F}} کا کام تخمیناً درج ذیل ہو گا۔
\begin{align*}
(\text{\RL{حرکت کے رخ قوت کا جزو}})\times (\text{\RL{طے فاصلہ}})=\kvec{F}_k\cdot\kvec{T}_k\Delta s_k
\end{align*}
منحنی پر چلتے ہوئے \عددی{t=a} تا \عددی{t=b} کام تخمیناً درج ذیل ہو گا۔
\begin{align*}
\sum_{k=1}^{n}\kvec{F}_k\cdot\kvec{T}_k\Delta s_k
\end{align*}
جیسا جیسا \عددی{[a,b]} کے خانہ بندی کا معیار صفر کے قریب سے قریب ہوتا  ہے، ویسے ویسے منحنی کی پیدا کردہ خانہ بندی کا معیار بھی صفر کے قریب سے قریب ہوتا ہے اور مجموعہ درج ذیل لکیری تکمل کو پہنچتا ہے۔
\begin{align*}
\int_{t=a}^{t=b}\kvec{F}\cdot\kvec{T}\dif s
\end{align*} 
اس تکمل سے حاصل عدد کی علامت، \عددی{t} بڑھانے سے حاصل، منحنی پر چلنے کے رخ پر منحصر ہو گی۔ منحنی پر چلنے کا رخ الٹ کرنے سے \عددی{\kvec{T}} کا رخ الٹ ہو گا جس کی بنا \عددی{\kvec{F}\cdot\kvec{T}} اور تکمل کی علامت الٹ ہو گی۔ 

\جزوحصہء{علامتیت اور قیمت کا حصول}
تکمل کام (مساوات \حوالہ{مساوات_سمتی_میدان_کام_کی_تعریف}) کو لکھنے کے چھ طریقے درج ذیل ہیں۔
\begin{gather}
\begin{aligned}\label{مساوات_سمتی_تکمل_متعدد_روپ_کام}
W&=\int_{t=a}^{t=b}\kvec{F}\cdot\kvec{T}\dif s&&\text{\RL{تعریف}}\\
&=\int_{t=a}^{t=b}\kvec{F}\cdot\dif \kvec{r}&&\text{\RL{مختصر تفریقی روپ}}\\
&=\int_{t=a}^{t=b}\kvec{F}\cdot\frac{\dif \kvec{r}}{\dif t}\dif t&&
\text{\RL{پھیلا کر \عددی{\dif t}  شامل کیا گیا ہے}}\\
&=\int_{t=a}^{t=b}\big(M\frac{\dif g}{\dif t}+N\frac{\dif h}{\dif t}+P\frac{\dif k}{\dif t}\big)\dif t&&\text{\RL{مقدار معلوم \عددی{t} اور سمتی رفتار \عددی{\frac{\dif\kvec{r}}{\dif t}} اجاگر کیے گئے}}\\
&=\int_{t=a}^{t=b}\big(M\frac{\dif x}{\dif t}+N\frac{\dif y}{\dif t}+P\frac{\dif z}{\dif t}\big)\dif t&&\text{\RL{جزوی تفاعل اجاگر کیے گئے}}\\
&=\int_{t=a}^{t=b} M\dif x+N\dif y+P\dif z&&\text{\RL{\عددی{\dif t} منسوخ کر کے عمومی روپ حاصل کی گئی}}
\end{aligned}
\end{gather}

مساوات \حوالہ{مساوات_سمتی_تکمل_متعدد_روپ_کام} کے کلیات کی  قیمتوں کا حصول،  بظاہر مختلف روپ  کے باوجود،  ایک ہی طرح کیا جاتا ہے۔ 
