\حصہ{لمبائی قوس اور اکائی مماسی سمتیہ \عددی{\kvec{T}}}
قابل تفرق منحنیات جن کا   پہلا اور دوسرا استمراری تفرق پایا جاتا ہو خلاء  میں حرکت کو ظاہر کرنے  کے لئے  اہم  ہیں۔ ان پر تفصیلاً غور کیا گیا ہے۔اس حصہ میں اور اگلے حصہ میں ہم ان کے چند  ایسے   خدوخال پر غور کریں گے جن کہ بنا  ایسے   منحنیات کی اہم ہیں۔  

\جزوحصہء{منحنی پر لمبائی قوس}
ہموار فضائی منحنیات کی ایک خاص خاصیت یہ ہے کہ ان کی لمبائی  قابل ناپ ہوتی ہے۔ یوں ہم  منحنی پر کسی نقطہ کو  ابتدائی نقطہ   تصور کرتے ہوئے،  ابتدائی نقطہ  سے کسی بھی نقطہ \عددی{N}  تک منحنی پر فاصلہ \عددی{s}،  سے نقطہ     \عددی{ N} کی نشاندہی کر سکتے ہیں۔یہ محددی مستوی پر  مبدا سے نقطہ کا فاصلہ دینے کے مترادف ہے۔متحرک جسم کی سمتی رفتار اور اسراع  پر غور کے لئے وقت ایک فطری متغیر ہے جبکہ \عددی{s} منحنی کی صورت  پر غور کرنے کے لئے ایک  فطری  متغیر ہے۔ فضا میں حرکت پر غور کے دوران ان دونوں متغیرات کی ضرورت پیش آتی ہے۔

فضا میں ہموار منحنی پر فاصلہ ناپنے کی خاطر  ہم مستوی میں منحنی کے کلیہ میں جزو \عددی{z} شامل کرتے ہیں۔

\ابتدا{تعریف}
ہموار منحنی \عددی{\kvec{r}(t)=f(t)\ai+g(t)\aj+h(t)\ak,\,\,a\le t\le b} جس پر  \عددی{t=a}  تا \عددی{t=b}صرف ایک بار چلا جاتا ہو،   کی لمبائی  درج ذیل ہو گی۔
\begin{gather}
\begin{aligned}\label{مساوات_سمتی_تفاعل_لمبائی_قوس_الف}
L&=\int_a^b\sqrt{\big(\frac{\dif f}{\dif t}\big)^2+\big(\frac{\dif g}{\dif t}\big)^2+\big(\frac{\dif h}{\dif t}\big)^2}\dif t\\
&=\int_a^b\sqrt{\big(\frac{\dif x}{\dif t}\big)^2+\big(\frac{\dif y}{\dif t}\big)^2+\big(\frac{\dif z}{\dif t}\big)^2}\dif t
\end{aligned}
\end{gather} 
\انتہا{تعریف}
%===============

مستوی منحنیات کی طرح، ہم فضا میں منحنی کی لمبائی معلوم کرتے ہوئے منحنی کی کوئی بھی مقدار معلوم مساوات، جو دیے گئے شرائط کو پورا کرتے ہوں، استعمال کر سکتے ہیں۔ اس کا ثبوت پیش نہیں کیا جائے گا۔

مساوات \حوالہ{مساوات_سمتی_تفاعل_لمبائی_قوس_الف} میں جذز، سمتی رفتار سمتیہ \عددی{\tfrac{\dif \kvec{r}}{\dif t}} کی لمبائی  \عددی{\abs{\kvec{v}}} ہے۔ یوں لمبائی قوس کا کلیہ مختصراً
\begin{align}\label{مساوات_سمتی_تفاعل_لمبائی_قوس_ب}
L=\int_a^b\abs{\kvec{v}}\dif t
\end{align}
لکھا جا سکتا ہے۔

\ابتدا{مثال}\شناخت{مثال_سمتی_تفاعل_مقدار_معلوم_لمبائی_قوس_الف}
درج ذیل پیچ دار منحنی کے ایک چکر کی لمبائی تلاش کریں۔
\begin{align*}
\kvec{r}(t)=(\cos t)\ai+(\sin t)\aj+t\ak
\end{align*}
حل:\quad
پیچ دار منحنی \عددی{t=0} سے \عددی{t=2\pi} تک ایک چکر مکمل کرتی ہے۔اس حصہ کی لمبائی
\begin{align*}
L&=\int_a^b\abs{\kvec{v}}\dif t=\int_0^{2\pi}\sqrt{(-\sin t)^2+(\cos t)^2+(1)^2}\dif t\\
&=\int_0^{2\pi}\sqrt{2}\dif t=2\pi\sqrt{2}
\end{align*}
ہو گی جو مستوی \عددی{xy} میں اس دائرہ کے لمبائی کا \عددی{\sqrt{2}} گنّا ہے جس پر پیچ دار منحنی کھڑی ہے۔
\انتہا{مثال}
%===============

اگر ہم ہموار منحنی \عددی{C}، جس کی مقدار معلوم مساوات کا متغیر \عددی{t} ہو،     پر نقطہ \عددی{N_0} کو  ابتدائی نقطہ تصور کریں  تب \عددی{t} کی ہر قیمت  \عددی{C} پر ایک نقطہ \عددی{N(t)=(x(t),y(t),z(t))}  اور  \اصطلاح{سمت بند فاصلہ}
\begin{align}\label{مساوات_سمتی_تفاعل_لمبائی_قوس_پ}
s(t)=\int_{t_0}^t\abs{\kvec{v}(\tau)}\dif \tau
\end{align}
دے گی  جو \عددی{N_0} سے \عددی{C} پر چلتے ہوئے  ناپا جائے گا۔ اگر \عددی{t>t_0} ہو تب \عددی{N(t_0)} سے \عددی{N(t)} تک فاصلہ \عددی{s(t)} ہو گا۔ اگر \عددی{t<t_0} ہو تب \عددی{s(t)}،  فاصلہ کے نفی کا برابر ہو گا۔ \عددی{} کی ہر ایک قیمت  \عددی{} پر ایک نقطہ تعین کرتی ہے لہٰذا یوں \عددی{s} کے لحاض سے \عددی{C} کی مقدار معلوم روپ حاصل ہوتی ہے۔ ہم \عددی{} کو منحنی کا \اصطلاح{مقدار معلوم  لمبائی قوس} کہتے ہیں جس کی قیمت  بڑھتے \عددی{t} کے رخ بڑھتی ہے۔

\موٹا{ابتدائی نقطہ \عددی{N(t_0)} لیتے ہوئے مقدار معلوم لمبائی قوس}
\begin{align}\label{مساوات_سمتی_تفاعل_لمبائی_قوس_ت}
s(t)=\int_{t_0}^t \sqrt{[x'(\tau)]^2+[y'(\tau)]^2+[z'(\tau)]^2}\dif \tau=\int_{t_0}^t \abs{\kvec{v}(\tau)}\dif \tau
\end{align}

\ابتدا{مثال}
اگر \عددی{t_0=0} ہو تب پیچ دار منحنی
\begin{align*}
\kvec{r}(t)=(\cos t)\ai+(\sin t)\aj+t\ak
\end{align*}
  پر  \عددی{t_0} سے \عددی{t} تک چلتے ہوئے  مقدار معلوم لمبائی قوس درج ذیل ہو گی۔
\begin{align*}
s(t)&=\int_{t_0}^t\abs{\kvec{v}(\tau)}\dif \tau&&\text{\RL{مساوات \حوالہ{مساوات_سمتی_تفاعل_لمبائی_قوس_ت}}}\\
&=\int_0^{t}\sqrt{2}\dif \tau&&\text{\RL{مساوات \حوالہ{مثال_سمتی_تفاعل_مقدار_معلوم_لمبائی_قوس_الف} کی قیمتیں}}\\
&=\sqrt{2}t
\end{align*}
یوں \عددی{s(2\pi)=2\pi\sqrt{2}}، \عددی{s(-2\pi)=-2\pi\sqrt{2}}، وغیرہ،  ہوں گے۔
\انتہا{مثال}
%============
\ابتدا{مثال}\ترچھا{ایک لکیر پر لمبائی}\\
دکھائیں   اگر \عددی{\kvec{u}=u_1\ai+u_2\aj+u_3\ak}  اکائی سمتیہ ہو،  تب لکیر
\begin{align*}
\kvec{r}(t)=(x_0+tu_1)\ai+(y_0+tu_2)\aj+(z_0+tu_3)\ak
\end{align*} 
 پر ،  نقطہ \عددی{N_0(x_0,y_0,z_0)}   جہاں  \عددی{t=0} ہو گا، سے   سمت بند لمبائی  از خود \عددی{t} کے برابر  ہو گی۔

حل:\quad
\begin{align*}
\kvec{v}&=\frac{\dif}{\dif t}(x_0+tu_1)\ai+\frac{\dif}{\dif t}(y_0+tu_2)\aj+\frac{\dif}{\dif t}(z_0+tu_3)\ak=u_1\ai+u_2\aj+u_3\ak
\end{align*}
سے درج ذیل حاصل ہو گا۔
\begin{align*}
s(t)=\int_0^t \abs{\kvec{v}}\dif \tau=\int_0^t\abs{\kvec{u}}\dif \tau=\int_0^t 1\dif \tau=t
\end{align*}
\انتہا{مثال}
%================

\جزوحصہء{ہموار منحنی پر رفتار}
چونکہ مساوات \حوالہ{مساوات_سمتی_تفاعل_لمبائی_قوس_ت} میں جذر کے اندر تفرقات  استمراری   (ہموار منحنی)  ہیں احصاء کے بنیادی مسئلہ کے تحت \عددی{s} متغیر \عددی{t} کا قابل تفرق تفاعل ہو گا     اور یہ تفرق درج ذیل ہو گا۔
\begin{align}\label{مساوات_سمتی_تفاعل_لمبائی_قوس_ٹ}
\frac{\dif s}{\dif t}=\abs{\kvec{v}(t)}
\end{align}
جیسا ہم توقع  کریں گے،کسی بھی راہ پر  ایک ذرے کی رفتار \عددی{\kvec{v}} کی مقدار ہوتی ہے۔ 

 دھیان رہے کہ اگرچہ \عددی{s} تعین کرنے میں ابتدائی نقطہ  \عددی{N(t_0)} کا کردار پایا جاتا ہے،  \عددی{N(t_0)}  کا مساوات \حوالہ{مساوات_سمتی_تفاعل_لمبائی_قوس_ٹ} میں کوئی کردار نہیں پایا جاتا ہے۔ ایک راہ پر چلتے ہوئے جس رفتار سے ایک  ذرہ فاصلہ طے کرتا ہے، اس کا ابتدائی نقطہ کے ساتھ کوئی تعلق نہیں پایا جاتا ہے۔

ساتھ ہی اس بات کو ذہن نشین کریں کہ چونکہ تعریف کی رو سے ہموار منحنی کے لئے  \عددی{\abs{\kvec{v}}} غیر صفر   ہے لہٰذا \عددی{\tfrac{\dif s}{\dif t}>0} ہو گا۔ہم ایک بار دوبارہ دیکھتے ہیں کہ \عددی{s} متغیر \عددی{t} کا بڑھتا تفاعل ہے۔

\جزوحصہء{اکائی مماسی سمتیہ \عددی{\kvec{T}}}
چونکہ زیر بحث منحنیات کے لئے \عددی{\tfrac{\dif s}{\dif t}>0} ہے  لہٰذا \عددی{s} ایک ایک مطابقت رکھتا ہے اور اس کا الٹ پایا جائے گا جو \عددی{t}  کو بطور \عددی{s} کا قابل تفرق تفاعل دے گا (حصہ \حوالہ{حصہ_استعمال_تکمل_الٹ_تفاعل_اور_ان_کے_تفرق})۔ اس الٹ کا تفرق درج ذیل ہو گا۔
\begin{align}\label{مساوات_سمتی_تفاعل_لمبائی_قوس_ث}
\frac{\dif t}{\dif s}=\frac{1}{\dif s/\dif t}=\frac{1}{\abs{\kvec{v}}}
\end{align}
یوں \عددی{\kvec{r}} متغیر \عددی{s} کا قابل تفرق تفاعل ہو گا جس کے تفرق کو زنجیری قاعدہ سے حاصل کیا جا سکتا ہے:
\begin{align}\label{مساوات_سمتی_تفاعل_لمبائی_قوس_ج}
\frac{\dif \kvec{r}}{\dif s}=\frac{\dif \kvec{r}}{\dif t}\frac{\dif t}{\dif s}=\kvec{v}\frac{1}{\abs{\kvec{v}}}=\frac{\kvec{v}}{\abs{\kvec{v}}}
\end{align}

مساوات \حوالہ{مساوات_سمتی_تفاعل_لمبائی_قوس_ج} کہتی ہے کہ \عددی{\tfrac{\dif \kvec{r}}{\dif t}}ایک  اکائی سمتیہ ہے جو   \عددی{\kvec{v}} کے رخ ہے۔ہم \عددی{\tfrac{\dif \kvec{r}}{\dif s}} کو \عددی{\kvec{r}} کی منحنی راہ کا اکائی مماسی سمتیہ کہتے ہیں اور اس کو \عددی{\kvec{T}} سے ظاہر کرتے ہیں۔

\ابتدا{تعریف}
قابل تفرق تفاعل \عددی{\kvec{r}(t)} کا اکائی مماسی سمتیہ درج ذیل ہو گا۔
\begin{align}
\kvec{T}=\frac{\dif \kvec{r}}{\dif s}=\frac{\dif \kvec{r}/\dif t}{\dif s/\dif t}=\frac{\kvec{v}}{\abs{\kvec{v}}}
\end{align}
\انتہا{تعریف}
%============

جہاں بھی \عددی{\kvec{v}} متغیر \عددی{t} کا قابل تفرق  تفاعل ہو وہاں اکائی مماسی سمتیہ  \عددی{\kvec{T}} بھی \عددی{t} کا قابل تفرق تفاعل ہو گا۔جیسا ہم اگلے حصہ میں دیکھیں گے، فضا میں اجسام کی حرکت پر غور  میں   مستعمل،   متحرک \اصطلاح{حوالہ  چھوکٹ}\فرہنگ{حوالہ چھوکٹ}\حاشیہب{reference frame}\فرہنگ{reference frame}، کے تین اکائی سمتیات میں سے ایک اکائی سمتیہ \عددی{\kvec{T}} ہے۔

\ابتدا{مثال}
درج ذیل پیچ دار منحنی کا اکائی مماسی سمتیہ تلاش کریں۔
\begin{align*}
\kvec{r}(t)&=(\cos t+t\sin t)\ai+(\sin t-t\cos t)\aj&&t>0
\end{align*}
حل:\quad
\begin{align*}
\kvec{v}&=\frac{\dif \kvec{r}}{\dif t}=(-\sin t+\sin t+t\cos t)\ai+(\cos t-\cos t+t\sin t)\aj\\
&=(t\cos t)\ai+(t\sin t)\aj\\
\abs{\kvec{v}}&=\sqrt{t^2\cos^2t+t^2\sin^2t}=\sqrt{t^2}=t\quad \quad \text{\small\RL{\عددی{t>0} کی بنا \عددی{\abs{t}=t} ہے}}\\
\kvec{T}&=\frac{\kvec{v}}{\abs{\kvec{v}}}=\frac{\kvec{v}}{t}=(\cos t)\ai+(\sin t)\aj
\end{align*}
\انتہا{مثال}
%=================
\ابتدا{مثال}
اکائی دائرہ 
\begin{align*}
\kvec{v}=(-\sin t)\ai+(\cos t)\aj
\end{align*}
کے گرد گھڑی  کے مخالف رخ حرکت
\begin{align*}
\kvec{r}(t)=(\cos t)\ai+(\sin t)\aj
\end{align*}
کا اکائی مماسی سمتیہ \عددی{\kvec{T}=\kvec{v}} ہے۔
\انتہا{مثال}
%============

\حصہء{سوالات}

