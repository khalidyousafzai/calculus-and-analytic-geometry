\ابتدا{مثال}
ایک  نشانہ  باز \عددی{\SI{2}{\meter}}  بلندی سے  \عددی{\SI{30}{\meter}} دور  درخت   پر     \عددی{\SI{20}{\meter}} بلندی   پر لگائی گئی نشانی  کو   تیر کا نشانہ بناتا ہے۔ تیر نشانہ پر عین اس لمحہ پہنچتا ہے جب اس کی بلندی زیادہ سے زیادہ ہو۔  (ا) ابتدائی رفتار \عددی{v_0} اور زاویہ  \عددی{\alpha} کی صورت میں زیادہ سے زیادہ بلندی \عددی{y_{\text{بلندتر}}} لکھیں۔ (ب) اگر \عددی{y_{\text{بلندتر}}=\SI{20}{\meter}} ہو  تب جزو-ا کے نتیجہ سے \عددی{v_0\sin\alpha} معلوم کریں۔ (ج) تیر \عددی{\SI{30}{\meter}} افقی فاصلہ طے کر کے  درخت تک پہنچتا ہے۔ اس حقیقت کو استعمال کرتے ہوئے \عددی{v_0\cos\alpha} کی قیمت معلوم کریں۔ (د)  تیر کا ابتدائی زاویہ تلاش کریں۔

حل:\quad
(ا) ہم  نشانہ باز کو مبدا پر تصور کرتے ہیں۔ یوں\عددی{t=0} پر  \عددی{x_0=0} اور \عددی{y_0=2} ہو گا لہٰذا درج ذیل لکھا جا سکتا ہے۔
\begin{align*}
y&=y_0+(v_0\sin \alpha)t-\frac{1}{2}gt^2&&\text{\RL{مساوات \حوالہ{مساوات_سمتی_تفاعل_گولا_خ}}}\\
&=2+(v_0\sin \alpha)t-\frac{1}{2}gt^2&&y_0=2
\end{align*}
ہم  \عددی{\tfrac{\dif y}{\dif t}=0} سے وہ لمحہ \عددی{t} تلاش کرتے ہیں جب تیر زیادہ سے زیادہ بلندی  پر ہو گا:
\begin{align*}
t=\frac{v_0\sin\alpha}{g}
\end{align*} 
اس لمحہ پر \عددی{y}    کی قیمت درج ذیل ہو گی۔
\begin{align*}
y_{\text{بلندتر}}=2+(v_0\sin\alpha)\big(\frac{v_0\sin\alpha}{g}\big)-\frac{1}{2}g\big(\frac{v_0\sin\alpha}{g}\big)^2=2+\frac{(v_0\sin\alpha)^2}{2g}
\end{align*}
(ب)  ہم مساوات  \حوالہ{مساوات_سمتی_تفاعل_گولا_خ}  میں \عددی{g=9.8} اور \عددی{y_{\text{بلندتر}}=20} پر  کر کے جزو-ا سے
\begin{align*}
20=2+\frac{(v_0\sin\alpha)^2}{2(9.8)}
\end{align*}
یعنی
\begin{align*}
v_0\sin\alpha=\sqrt{(18)(19.6)}
\end{align*}
(ج) ہم جزو-ا میں حاصل زیادہ سے زیادہ بلندی تک پہنچنے کے لئے درکار وقت \عددی{t}  اور  افقی فاصلہ \عددی{x=\SI{30}{\meter}}  مساوات \حوالہ{مساوات_سمتی_تفاعل_گولا_خ} میں پر کرتے ہیں۔
\begin{align*}
x&=x_0(\cos \alpha)t&&\text{\RL{مساوات \حوالہ{مساوات_سمتی_تفاعل_گولا_خ}}}\\
30&=0+(v_0\cos\alpha)t&&x=30,\, x_0=0\\
&=(v_0\cos\alpha)\big(\frac{v_0\sin\alpha}{g}\big)&&t=(v_0\sin\alpha)/g
\end{align*}
اس مساوات کو \عددی{v_0\cos\alpha} کے حل کر کے اس میں جزو-ب کا نتیجہ پر کر کے درج ذیل حاصل کرتے ہیں۔
\begin{align*}
v_0\cos\alpha=\frac{30g}{v_0\sin\alpha}
\end{align*}
(د) ہم جزو-ب اور جزو-ج سے
\begin{align*}
\tan\alpha&=\frac{(18)(19.6)}{(30)(9.8)}\approx 0.876
\end{align*}
یعنی
\begin{align*}
\alpha\approx \tan^{-1}(0.876)=50.2^{\circ}
\end{align*}
حاصل کرتے ہیں۔
\انتہا{مثال}
%================


\حصہء{سوالات}
درج ذیل سوالات میں گولا کی حرکت کو مثالی تصور کیا جائے۔ تمام زاویات افقی میدان سے ناپے جائیں گے۔ جہاں  اس کے برعکس ذکر نہ کیا گیا ہو، گولا کو مبدا سے افقی  میدان میں چلایا جاتا ہے۔

\ابتدا{سوال}
ایک گولا \عددی{60^{\circ}} زاویہ پر \عددی{\SI{840}{\meter\per\second}} رفتار سے داغا  جاتا ہے۔یہ حدف کے رخ کتنی دیر میں \عددی{\SI{21}{\kilo\meter}} فاصلہ طے کرے گا؟
\انتہا{سوال}
%==============
\ابتدا{سوال}
ایک توپ  کی زیادہ سے زیادہ فاصلہ مار \عددی{\SI{24.5}{\kilo\meter}} ہے۔ اس کے نالی میں  گولے کی   رفتار معلوم کریں۔
\انتہا{سوال}
%=============
\ابتدا{سوال}
ایک گولا \عددی{45^{\circ}} زاویہ پر \عددی{\SI{500}{\meter\per\second}} رفتار سے داغا  جاتا ہے۔ (ا)  اس کا فاصلہ مار کتنا ہو گا؟ (ب) افقی رخ \عددی{\SI{5}{\kilo\meter}} فاصلہ پر گولا کتنا بلند ہو گا؟ (ج)  یہ گولا کتنی بلندی تک پہنچے گا؟
\انتہا{سوال}
%===================
\ابتدا{سوال}
ایک گیند \عددی{\SI{10}{\meter}} کی بلندی سے \عددی{30^{\circ}} زاویہ اور \عددی{\SI{10}{\meter\per\second}}کی  رفتار سے پھینکی جاتی ہے۔ یہ گیند کب اور کتنے  فاصلہ پر زمین کو مس کرے گی؟ 
\انتہا{سوال}
%==================
\ابتدا{سوال}\شناخت{سوال_سمتی_تفاعل_کھلاڑی_الف}
ایک کھلاڑی  \عددی{\SI{7}{\kilo\gram}} کا گولا \عددی{\SI{2}{\meter}} بلندی سے \عددی{45^{\circ}} زاویہ پر \عددی{\SI{13.4}{\meter\per\second}} رفتار سے پھینکتا ہے۔ یہ گیند کتنی دیر بعد اور کتنے فاصلہ پر زمین پر گرے گی؟

جواب:\quad
$t\approx \SI{2.1257}{\second},\quad x\approx \SI{20.14}{\meter}$
\انتہا{سوال}
%===============
\ابتدا{سوال}
اگر  سوال \حوالہ{سوال_سمتی_تفاعل_کھلاڑی_الف} میں گیند \عددی{40^{\circ}} پر پھینکی جاتی تب یہ نسبتاً زیادہ دور گرتی۔ فاصلہ میں اضافہ کتنا ہو گا؟
\انتہا{سوال}
%===========
\ابتدا{سوال}
ایک گیند کو  \عددی{45^{\circ}} زاویہ پر پھینکا جاتا ہے۔یہ \عددی{\SI{10}{\meter}} فاصلہ پر گرتا ہے۔ اس کی ابتدائی رفتار تلاش کریں۔ کن دو  زاویات  پر پھینکنے سے اس گیند کا فاصلہ مار \عددی{\SI{6}{\meter}} ہو گا؟
\انتہا{سوال}
%================
\ابتدا{سوال}
ٹیلی ویژن  کے ٹیوب میں  ایک الیکٹران \عددی{\SI{5e6}{\meter\per\second}} رفتار سے \عددی{\SI{40}{\centi\meter}} دور    ٹیلی ویژن کے شیشہ  کے رخ   افقی سمت  خارج ہوتا ہے۔ یہ الیکٹران شیشہ پر لگنے سے پہلے کتنا نیچے گرتا ہے؟
\انتہا{سوال}
%==================
\ابتدا{سوال}
تجربہ گاہ  میں گالف  گیند کو پرکھنے کے دوران   \اصطلاح{\عددی{100} داب}\حاشیہب{100 compression} کی گیند کو \عددی{\SI{160}{\kilo\meter\per\hour}} رفتار پر چلتے ہوئے لاٹھ سے مار کر  \عددی{9^{\circ}} زاویہ پر پھینکا جاتا ہے۔ یہ گیند    \عددی{\SI{227}{\meter}}  دور گرتا ہے۔ گیند کی ابتدائی رفتار کتنی تھی؟  

جواب:
$\SI{174}{\kilo\meter\per\hour}$
\انتہا{سوال}
%===========
\ابتدا{سوال}
ایک  تماش گاہ   میں  ایک مسخرہ  کو توپ سے \عددی{\SI{25}{\meter\per\second}} رفتار سے داغا جاتا  ہے۔ توقع کی جاتی ہے کہ یہ \عددی{\SI{60}{\meter}} دور نرم  گدی  پر جا گرے گا۔ یہ تماش گاہ  ایک بڑے کمرہ میں منعقد ہوتا ہے جس کی چھت \عددی{\SI{23}{\meter}} بلند ہے۔  کیا مسخرہ کو یوں داغا جا سکتا ہے کہ یہ چھت کو نہ لگے؟ اگر ایسا ممکن ہو تب توپ کا زاویہ کتنا ہونا چاہیے؟
\انتہا{سوال}
%====================
