
\begin{figure}
\centering
\begin{subfigure}{0.45\textwidth}
\centering
\begin{tikzpicture}
\draw[-stealth](-0.25,0)--++(4.5,0)node[right]{$u$};
\draw[-stealth](1,-1.25)--++(0,3.5)node[left]{$v$};
\draw[fill=lgray,smooth cycle,opacity=0.5]plot coordinates {(0,0)(1,-1)(2,-1)(4,1)(3,2)};
\draw(2,1)node[circ]{}node[below]{$(u,v)$};
\draw(3,1.5)node[]{$G$};
\end{tikzpicture}
\caption{کارتیسی \(uv\) مستوی}
\end{subfigure}\hfill
\begin{subfigure}{0.45\textwidth}
\centering
\begin{tikzpicture}
\draw[-stealth](-0.25,0)--++(4,0)node[right]{$x$};
\draw[-stealth](1,-1.5)--++(0,3)node[left]{$y$};
\draw[fill=lgray,smooth cycle,opacity=0.5]plot coordinates {(0,0)(0.5,-1)(2,-1)(3,0)(2,1)(1,1)};
\draw(1.5,0.7)node[circ]{}node[below]{$(x,y)$};
\draw(2,-0.5)node[]{$R$};
\draw(2,1.5)node[right]{$\begin{aligned} x&=g(u,v)\\ y&=h(u,v) \end{aligned}$};
\end{tikzpicture}
\caption{کارتیسی \(xy\) مستوی}
\end{subfigure}
\caption{
مستوی \عددی{xy} میں خطہ \عددی{R} پر تکمل کا تبادلہ مساوات \عددی{x=g(u,v)}، \عددی{y=h(u,v)}  مستوی \عددی{uv} میں خطہ \عددی{G} پر کرتی ہیں۔ 
}
\label{شکل_کثیر_ایک_خطہ_سے_دوسرے_خطہ_تکمل_تبادلہ}
\end{figure}


\begin{figure}
\centering
\begin{subfigure}{0.45\textwidth}
\centering
\begin{tikzpicture}
\pgfmathsetmacro{\kpi}{3.142/2}
\begin{axis}[small,axis lines=middle, xlabel={$\rho$},ylabel={$\phi$},enlargelimits=true,xlabel style={at={(current axis.right of origin)},anchor=west},ylabel style={at={(current axis.above origin)},anchor=south},xtick={1},ytick={\kpi},yticklabels={$\frac{\pi}{2}$}]
\addplot[fill=lgray,opacity=0.5]plot coordinates {(0,0)(1,0)(1,pi/2)(0,pi/2)(0,0)};
\addplot[]plot coordinates {(0.4,0)(0.4,pi/2)};
\addplot[]plot coordinates {(0.7,0)(0.7,pi/2)};
\addplot[]plot coordinates {(0,pi/4)(1,pi/4)};
\addplot[]plot coordinates {(0,pi/5)(1,pi/5)};
\addplot[fill=gray]plot coordinates {(0.4,pi/5)(0.7,pi/5)(0.7,pi/4)(0.4,pi/4)(0.4,pi/5)};
\addplot[]plot coordinates{(0.8,pi/3)}node[]{$G$};
\end{axis}
\end{tikzpicture}
\caption{کارتیسی \(\rho\phi\) مستوی}
\end{subfigure}\hfill
\begin{subfigure}{0.45\textwidth}
\centering
\begin{tikzpicture}[declare function={fx(\r,\p)=\r*cos(\p);fy(\r,\p)=\r*sin(\p);}]
\begin{axis}[clip=false,small,axis lines=middle, xlabel={$x$},ylabel={$y$},enlargelimits=true,xlabel style={at={(current axis.right of origin)},anchor=west},ylabel style={at={(current axis.above origin)},anchor=south},xtick={1},ytick={1},axis on top,xmax=1.25,ymax=1.25]
\addplot[name path=fun,domain=0:90] ({fx(1,x)},{fy(1,x)});
\addplot[domain=0:90] ({fx(0.4,x)},{fy(0.4,x)});
\addplot[domain=0:90] ({fx(0.7,x)},{fy(0.7,x)});
\addplot[domain=0:1] ({fx(x,36)},{fy(x,36)});
\addplot[domain=0:1] ({fx(x,45)},{fy(x,45)});
\path[name path=xaxis] (axis cs:0,0)--(axis cs:1,0);
\addplot[fill=llgray] fill between[of=fun and xaxis];
%
\addplot[draw=none,name path=ka,domain=36:45] ({fx(0.4,x)},{fy(0.4,x)});
\addplot[draw=none,name path=kb,domain=36:45] ({fx(0.7,x)},{fy(0.7,x)});
\addplot[fill=gray] fill between[of=ka and kb];
\addplot[data cs=polar]plot coordinates {(60,0.8)}node[]{$R$};
\addplot[data cs=polar]plot coordinates {(0,1)}node[above right]{$\phi=0$};
\addplot[data cs=polar]plot coordinates {(90,1)}node[above right]{$\phi=\frac{\pi}{2}$};
\end{axis}
\end{tikzpicture}
\caption{کارتیسی \(xy\) مستوی}
\end{subfigure}
\caption{
خطہ \عددی{G} کا تبادلہ مساوات \عددی{x=\rho\cos\phi}، \عددی{y=\rho\sin\phi} خطہ \عددی{R} میں کرتی ہیں۔
}
\label{شکل_کثیر_چکور_سے_دائری_خطہ}
\end{figure}



\begin{figure}
\centering
\begin{subfigure}{0.45\textwidth}
\centering
\begin{tikzpicture}
\pgfmathsetmacro{\kpi}{3.142/2}
\begin{axis}[clip=false,small,axis lines=middle, xlabel={$u$},ylabel={$v$},enlargelimits=true,xlabel style={at={(current axis.right of origin)},anchor=west},ylabel style={at={(current axis.above origin)},anchor=south},xtick={1},ytick={2},axis on top]
\addplot[fill=lgray,opacity=0.5]plot coordinates {(0,0)(1,0)(1,2)(0,2)(0,0)};
\addplot[]plot coordinates {(0.5,1)}node[]{$G$};
\addplot[]plot coordinates {(0,1)}node[left]{$u=0$};
\addplot[]plot coordinates {(1,1)}node[right]{$u=1$};
\addplot[]plot coordinates {(0.5,0)}node[below]{$v=0$};
\addplot[]plot coordinates {(0.5,2)}node[above]{$v=2$};
\end{axis}
\end{tikzpicture}
\end{subfigure}\hfill
\begin{subfigure}{0.45\textwidth}
\centering
\begin{tikzpicture}[declare function={fx(\r,\p)=\r*cos(\p);fy(\r,\p)=\r*sin(\p);}]
\begin{axis}[clip=false,small,axis lines=middle, xlabel={$x$},ylabel={$y$},enlargelimits=true,xlabel style={at={(current axis.right of origin)},anchor=west},ylabel style={at={(current axis.above origin)},anchor=south},xtick={1},ytick={4},axis on top]
\addplot[fill=lgray,opacity=0.5]plot coordinates {(0,0)(1,0)(3,4)(2,4)(0,0)};
\addplot[]plot coordinates {(1.5,2)}node[]{$R$};
\addplot[]plot coordinates {(1.75,1.5)}node[right]{$y=2x-2$};
\addplot[]plot coordinates {(1.5,3)}node[left]{$y=2x$};
\end{axis}
\end{tikzpicture}
\end{subfigure}
\caption{
خطہ \عددی{G} کا تبادلہ مساوات \عددی{x=u+v}، \عددی{y=2v} خطہ \عددی{R} میں کرتی  ہیں۔ ان مساوات کو \عددی{u=(2x-y)/2}، \عددی{v=y/2} لکھ کر \عددی{R} کا تبادلہ \عددی{G} میں  کیا جا سکتا ہے۔
}
\label{شکل_کثیر_مستطیل_سے_چار_ضلعی}
\end{figure}


\begin{figure}
\centering
\begin{subfigure}{0.45\textwidth}
\centering
\begin{tikzpicture}
\pgfmathsetmacro{\kpi}{3.142/2}
\begin{axis}[clip=false,small,axis lines=middle, xlabel={$u$},ylabel={$v$},enlargelimits=true,xlabel style={at={(current axis.right of origin)},anchor=west},ylabel style={at={(current axis.above origin)},anchor=south},xtick={1},xticklabels ={$\rlap{1}$},ytick={-2,1},axis on top]
\addplot[fill=lgray,opacity=0.5]plot coordinates {(0,0)(1,-2)(1,1)(0,0)};
\addplot[]plot coordinates {(0.5,0)}node[below]{$G$};
\addplot[]plot coordinates {(1,-1)}node[right]{$u=1$};
\addplot[]plot coordinates {(0.5,0.5)}node[above left]{$v=u$};
\addplot[]plot coordinates {(0.5,-1)}node[below left]{$v=-2u$};
\end{axis}
\end{tikzpicture}
\end{subfigure}\hfill
\begin{subfigure}{0.45\textwidth}
\centering
\begin{tikzpicture}[]
\begin{axis}[clip=false,small,axis lines=middle, xlabel={$x$},ylabel={$y$},enlargelimits=true,xlabel style={at={(current axis.right of origin)},anchor=west},ylabel style={at={(current axis.above origin)},anchor=south},xtick={1},ytick={1},axis on top]
\addplot[draw=black,fill=lgray,opacity=0.5]plot coordinates {(0,0)(1,0)(0,1)(0,0)};
\addplot[]plot coordinates {(0.5,0.25)}node[]{$R$};
\addplot[]plot coordinates {(0.5,0)}node[below]{$y=0$};
\addplot[]plot coordinates {(0.5,0.5)}node[above right]{$x+y=1$};
\addplot[]plot coordinates {(0,0.5)}node[left]{$x=0$};
\end{axis}
\end{tikzpicture}
\end{subfigure}
\caption{
خطہ \عددی{G} کا تبادلہ مساوات \عددی{x=\tfrac{u}{3}-\tfrac{v}{3}}، \عددی{y=\tfrac{2u}{3}+\tfrac{v}{3}} خطہ \عددی{R} میں کرتی  ہیں۔ ان مساوات  کو \عددی{u=x+y}، \عددی{v=y-2x} لکھ کر \عددی{R} کا تبادلہ \عددی{G} میں  کیا جا سکتا ہے۔
}
\label{شکل_کثیر_مثلث_سے_مثلث}
\end{figure}


\begin{figure}
\centering
\begin{subfigure}{0.45\textwidth}
\centering
\begin{tikzpicture}
\pgfmathsetmacro{\kpi}{3.142/2}
\begin{axis}[view/h=110,clip=false,small,axis lines=middle, xlabel={$u$},ylabel={$v$},zlabel={$w$},enlargelimits=true,xtick={\empty},ytick={\empty},ztick={\empty}]
\addplot3[fill=gray,opacity=0.5]plot coordinates {(2,1,1)(2,2,1)(2,2,2)(2,1,2)(2,1,1)};
\addplot3[fill=lgray,opacity=0.5]plot coordinates {(2,2,1)(1,2,1)(1,2,2)(2,2,2)(2,2,1)};
\addplot3[fill=llgray,opacity=0.5]plot coordinates {(2,2,2)(1,2,2)(1,1,2)(2,1,2)(2,2,2)};
\addplot3[]plot coordinates {(1,1.5,2)}node[above]{$G$};
\end{axis}
\end{tikzpicture}
\caption{کارتیسی \عددی{uvw} فضا۔}
\end{subfigure}\hfill
\begin{subfigure}{0.45\textwidth}
\centering
\begin{tikzpicture}[]
\begin{axis}[view/h=110,clip=false,small,axis lines=middle, xlabel={$x$},ylabel={$y$},zlabel={$z$},enlargelimits=true,xtick={\empty},ytick={\empty},ztick={\empty}]
\addplot3[fill=gray,opacity=0.5]plot coordinates {(2,1,1)(2,2,1)(2,1,2)(2,1,1)};
\addplot3[fill=lgray,opacity=0.5]plot coordinates {(2,2,1)(1,2,1)(1,1,2)(2,1,2)(2,2,1)};
\addplot3[]plot coordinates {(1,1.5,1.6)}node[above]{$D$};
\end{axis}
\end{tikzpicture}
\caption{کارتیسی \عددی{xyz} فضا۔}
\end{subfigure}
\caption{
کارتیسی \عددی{xyz} فضا میں خطہ \عددی{D} پر تکمل کا تبادلہ مساوات \عددی{x=g(u,v,w)}، \عددی{y=h(u,v,w)}، \عددی{z=k(u,v,w)}  کارتیسی \عددی{uvw} فضا میں خطہ \عددی{G} پر تکمل میں کرتی ہیں۔
}
\label{شکل_کثیر_مکعب_سے_منشور}
\end{figure}


\begin{figure}
\centering
\begin{subfigure}{0.45\textwidth}
\centering
\begin{tikzpicture}
\pgfmathsetmacro{\kpi}{3.142/2}
\begin{axis}[view/h=110,clip=false,small,axis lines=middle, xlabel={$\rho$},ylabel={$\phi$},zlabel={$z$},enlargelimits=true,xtick={\empty},ytick={\empty},ztick={\empty},zlabel style={anchor=east}]
\addplot3[fill=gray,opacity=0.5]plot coordinates {(2,1,1)(2,2,1)(2,2,2)(2,1,2)(2,1,1)};
\addplot3[fill=lgray,opacity=0.5]plot coordinates {(2,2,1)(1,2,1)(1,2,2)(2,2,2)(2,2,1)};
\addplot3[fill=llgray,opacity=0.5]plot coordinates {(2,2,2)(1,2,2)(1,1,2)(2,1,2)(2,2,2)};
\addplot3[]plot coordinates {(1,2,2)}node[right]{$G$};
\addplot3[]plot coordinates {(1,1.3,2)}node[above]{\begin{minipage}{0.25\textwidth} \text{\RL{ایک مکعب جس کے اطراف محددی}}\\
 \text{\RL{محوروں کے متوازی ہیں۔}}  \end{minipage}};
\end{axis}
\end{tikzpicture}
\caption{کارتیسی \عددی{\rho\phi z} فضا۔}
\end{subfigure}\hfill
\begin{subfigure}{0.45\textwidth}
\centering
\begin{tikzpicture}[declare function={fx(\r,\p)=\r*cos(\p);fy(\r,\p)=\r*sin(\p);}]
\pgfmathsetmacro{\angS}{45}
\pgfmathsetmacro{\angE}{60}
\pgfmathsetmacro{\angC}{50}
\begin{axis}[view/h=110,clip=false,small,axis lines=middle, xlabel={$x$},ylabel={$y$},zlabel={$z$},enlargelimits=true,xtick={\empty},ytick={\empty},ztick={\empty},zlabel style={anchor=east}]
\addplot3[name path=kin,domain=\angS:\angE,variable =\p,samples y=0]({fx(1,p)},{fy(1,p)},{2});
\addplot3[name path=kout,domain=\angS:\angE,variable =\p,samples y=0]({fx(2,p)},{fy(2,p)},{2});
\addplot3[dashed]plot coordinates {(0,0,2)({fx(1,\angS)},{fy(1,\angS)},{2})};
\addplot3[dashed]plot coordinates {(0,0,2)({fx(1,\angE)},{fy(1,\angE)},{2})};
\addplot3[dashed]plot coordinates {(0,0,1)({fx(1,\angS)},{fy(1,\angS)},{1})};
\addplot3[dashed]plot coordinates {(0,0,1)({fx(1,\angE)},{fy(1,\angE)},{1})};
\addplot3[]plot coordinates {({fx(1,\angS)},{fy(1,\angS)},{2})({fx(2,\angS)},{fy(2,\angS)},{2})};
\addplot3[]plot coordinates {({fx(1,\angE)},{fy(1,\angE)},{2})({fx(2,\angE)},{fy(2,\angE)},{2})};
\addplot[llgray]fill between[of=kout and kin];
\addplot3[fill=lgray]plot coordinates {({fx(1,\angS)},{fy(1,\angS)},{2})({fx(1,\angS)},{fy(1,\angS)},{1})({fx(2,\angS)},{fy(2,\angS)},{1})({fx(2,\angS)},{fy(2,\angS)},{2})({fx(1,\angS)},{fy(1,\angS)},{2})};
\addplot3[name path=klow,domain=\angS:\angE,variable=\p]({fx(2,p)},{fy(2,p)},{1});
\addplot3[fill=lgray]fill between[of=kout and klow];
\addplot3[]plot coordinates {({fx(2,\angE)},{fy(2,\angE)},{1})({fx(2,\angE)},{fy(2,\angE)},{2})};
\addplot3[]plot coordinates {({fx(2,\angE)},{fy(2,\angE)},{2})}node[right]{$D$};
\addplot3[]plot coordinates {({fx(1.5,\angC)},{fy(1.5,\angC)},{2})}node[pin=45:{\text{\RL{مستقل $z$}}}]{};
\addplot3[]plot coordinates {({fx(2,\angC)},{fy(2,\angC)},{1.5})}node[pin={[pin distance=1cm]-45:{\text{\RL{مستقل $\rho$}}}}]{};
\addplot3[]plot coordinates {({fx(1.75,\angS)},{fy(1.75,\angS)},{1.25})}node[pin={-135:{\text{\RL{مستقل $\phi$}}}}]{};
\end{axis}
\end{tikzpicture}
\caption{کارتیسی \عددی{xyz} فضا۔}
\end{subfigure}
\caption{
خطہ \عددی{G} کا تبادلہ مساوات \عددی{x=\rho\cos\phi}، \عددی{y=\rho\sin\phi}، \عددی{z=z} خطہ \عددی{D} میں کرتی ہیں۔
}
\label{شکل_کثیر_مکعب_نلکی_محدد}
\end{figure}


\begin{figure}
\centering
\begin{subfigure}{0.45\textwidth}
\centering
\begin{tikzpicture}
\pgfmathsetmacro{\kpi}{3.142/2}
\begin{axis}[view/h=110,clip=false,small,axis lines=middle, xlabel={$r$},ylabel={$\theta$},zlabel={$\phi$},enlargelimits=true,xtick={\empty},ytick={\empty},ztick={\empty},zlabel style={anchor=east}]
\addplot3[fill=gray,opacity=0.5]plot coordinates {(2,1,1)(2,2,1)(2,2,2)(2,1,2)(2,1,1)};
\addplot3[fill=lgray,opacity=0.5]plot coordinates {(2,2,1)(1,2,1)(1,2,2)(2,2,2)(2,2,1)};
\addplot3[fill=llgray,opacity=0.5]plot coordinates {(2,2,2)(1,2,2)(1,1,2)(2,1,2)(2,2,2)};
\addplot3[]plot coordinates {(1,2,2)}node[right]{$G$};
\addplot3[]plot coordinates {(1,1.3,2)}node[above]{\begin{minipage}{0.25\textwidth} \text{\RL{ایک مکعب جس کے اطراف محددی}}\\
 \text{\RL{محوروں کے متوازی ہیں۔}}  \end{minipage}};
\end{axis}
\end{tikzpicture}
\caption{کارتیسی \عددی{r\theta\phi} فضا۔}
\end{subfigure}\hfill
\begin{subfigure}{0.45\textwidth}
\centering
\begin{tikzpicture}[declare function={fx(\r,\t,\p)=\r*sin(\t)*cos(\p);fy(\r,\t,\p)=\r*sin(\t)*sin(\p);fz(\r,\t,\p)=\r*cos(\t);}]
\pgfmathsetmacro{\rS}{2}
\pgfmathsetmacro{\rE}{3}
\pgfmathsetmacro{\tS}{30}
\pgfmathsetmacro{\tE}{45}
\pgfmathsetmacro{\pS}{45}
\pgfmathsetmacro{\pE}{60}
\pgfmathsetmacro{\rC}{1/2*(\rS+\rE)}
\pgfmathsetmacro{\tC}{1/2*(\tS+\tE)}
\pgfmathsetmacro{\pC}{1/2*(\pS+\pE)}
\pgfmathsetmacro{\pCC}{\pS+1/3*(\pE-\pS)}
\begin{axis}[view/h=110,clip=false,small,axis lines=middle, xlabel={$x$},ylabel={$y$},zlabel={$z$},enlargelimits=true,xtick={\empty},ytick={\empty},ztick={\empty},zlabel style={anchor=east},colormap={}{gray(0cm)=(0.6);gray(1cm)=(0.9);}]
\addplot3[name path=fin,domain=\tS:\tE,variable=\t,samples y=0]({fx(\rS,t,\pS)},{fy(\rS,t,\pS)},{fz(\rS,t,\pS)});
\addplot3[name path=fout,domain=\tS:\tE,variable=\t,samples y=0]({fx(\rE,t,\pS)},{fy(\rE,t,\pS)},{fz(\rE,t,\pS)});\addplot3[domain=\rS:\rE,variable=\r,samples y=0]({fx(r,\tS,\pS)},{fy(r,\tS,\pS)},{fz(r,\tS,\pS)});
\addplot3[domain=\rS:\rE,variable=\r,samples y=0]({fx(r,\tE,\pS)},{fy(r,\tE,\pS)},{fz(r,\tE,\pS)});
\addplot[fill=lgray] fill between [of=fin and fout];
\addplot3[domain=\pS:\pE,variable=\p,samples y=0]({fx(\rE,\tS,p)},{fy(\rE,\tS,p)},{fz(\rE,\tS,p)});
\addplot3[name path=outHigh,domain=\pS:\pE,variable=\p,samples y=0]({fx(\rE,\tE,p)},{fy(\rE,\tE,p)},{fz(\rE,\tE,p)});
\addplot3[domain=\tS:\tE,variable=\t,samples y=0]({fx(\rE,t,\pE)},{fy(\rE,t,\pE)},{fz(\rE,t,\pE)});
\addplot3[domain=\rS:\rE,variable=\r,samples y=0]({fx(r,\tE,\pE)},{fy(r,\tE,\pE)},{fz(r,\tE,\pE)});
\addplot3[name path=outLow,domain=\pS:\pE,variable=\p,samples y=0]({fx(\rS,\tE,p)},{fy(\rS,\tE,p)},{fz(\rS,\tE,p)});
\addplot[fill=llgray]fill between [of=outLow and outHigh];
\addplot3[]plot coordinates {({fx(\rE,\tC,\pC)},{fy(\rE,\tC,\pC)},{fz(\rE,\tC,\pC)})}node[pin=45:{\text{\RL{مستقل $r$}}}]{};
\addplot3[]plot coordinates {({fx(\rC,\tC,\pS)},{fy(\rC,\tC,\pS)},{fz(\rC,\tC,\pS)})}node[pin=-110:{\text{\RL{مستقل $\phi$}}}]{};
\addplot3[]plot coordinates {({fx(\rC,\tE,\pCC)},{fy(\rC,\tE,\pCC)},{fz(\rC,\tE,\pCC)})}node[pin={-45:{\text{\RL{مستقل $\theta$}}}}]{};
\end{axis}
\end{tikzpicture}
\caption{کارتیسی \عددی{xyz} فضا۔}
\end{subfigure}
\caption{
خطہ \عددی{G} کا تبادلہ مساوات \عددی{x=r\sin\theta\cos\phi}، \عددی{y=r\sin\theta\sin\phi}، \عددی{z=r\cos\theta} خطہ \عددی{D} میں کرتی ہیں۔
}
\label{شکل_کثیر_مکعب_کروی_محدد}
\end{figure}




\begin{figure}
\centering
\begin{subfigure}{0.45\textwidth}
\centering
\begin{tikzpicture}
\begin{axis}[view/h=110,clip=false,small,axis lines=middle, xlabel={$u$},ylabel={$v$},zlabel={$w$},enlargelimits=true,xtick={1},ytick={2},ztick={1},xlabel style={anchor=north east},ylabel style={anchor=west},zlabel style={anchor=south}]
\addplot3[fill=gray,opacity=0.5]plot coordinates {(1,0,0)(1,2,0)(1,2,1)(1,0,1)(1,0,0)};
\addplot3[fill=lgray,opacity=0.5]plot coordinates {(1,2,0)(0,2,0)(0,2,1)(1,2,1)(1,2,0)};
\addplot3[fill=llgray,opacity=0.5]plot coordinates {(1,0,1)(1,2,1)(0,2,1)(0,0,1)(1,0,1)};
\addplot3[]plot coordinates {(0,2,1)}node[right]{$G$};
\end{axis}
\end{tikzpicture}
\end{subfigure}\hfill
\begin{subfigure}{0.45\textwidth}
\centering
\begin{tikzpicture}[]
\begin{axis}[view/h=110,clip=false,small,axis lines=middle, xlabel={$x$},ylabel={$y$},zlabel={$z$},enlargelimits=true,xtick={1},ytick={4},ztick={3},ylabel style={anchor=west},zlabel style={anchor=south}]
\addplot3[fill=gray,opacity=0.5]plot coordinates {(1,0,0)(3,4,0)(3,4,3)(1,0,3)(1,0,0)};
\addplot3[fill=lgray,opacity=0.5]plot coordinates {(3,4,0)(2,4,0)(2,4,3)(3,4,3)(3,4,0)};
\addplot3[fill=llgray,opacity=0.5]plot coordinates {(1,0,3)(3,4,3)(2,4,3)(0,0,3)(1,0,3)};
\addplot3[dashed]plot coordinates {(0,0,0)(2,4,0)};
\addplot3[dashed]plot coordinates {(2,4,0)(0,4,0)};
\addplot3[]plot coordinates {(2,4,3)}node[right]{$D$};
\end{axis}
\end{tikzpicture}
\end{subfigure}
\caption{
خطہ \عددی{G}  کا تبادلہ مساوات \عددی{x=u+v}، \عددی{y=2v}، \عددی{z=3w} خطہ \عددی{D} میں  کرتی ہیں جبکہ ان کی مخالف مساوات \عددی{u=\tfrac{2x-y}{2}}، \عددی{v=\tfrac{y}{2}}، \عددی{w=\tfrac{z}{3}} خطہ \عددی{D} کا تبادلہ \عددی{G} میں  کرتی ہیں۔
}
\label{شکل_کثیر_مکعب_سے_مستطیل}
\end{figure}

