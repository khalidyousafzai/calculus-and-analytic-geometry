\حصہ{راہ سے آزادی، تفاعل خفی توانائی، اور بقائی میدان}
ثقلی اور برقی میدان میں کمیت یا بار کو ایک نقطہ سے دوسرے نقطہ منتقل کرنے کے لئے درکار کام صرف ابتدائی اور اختتامی نقطوں پر منحصر ہوتا ہے نا کہ منتقلی کی راہ پر۔اس حصہ میں تکمل کام کی راہ سے آزادی کے تصور پر غور کیا جائے گا اور ایسے میدانوں کی خواص پر غور کیا جائے گا جن میں تکمل کام کی قیمت راہ کے تابع نہیں ہوتا۔

\جزوحصہء{راہ سے آزادی}
فضا میں کھلا خطہ \عددی{D} پر معین میدان \عددی{\kvec{F}} ایک ذرہ کو \عددی{D} میں  نقطہ \عددی{A} سے نقطہ \عددی{B} منتقل کرتا ہے۔ عموماً   تکمل کام \عددی{\int\kvec{F}\cdot\dif\kvec{r}} کی قیمت منتقلی کی راہ پر منحصر ہو گی، البتہ ایسے میدان پائے جاتے ہیں جن میں تکمل کام کی قیمت صرف ابتدائی اور اختتامی نقطوں پر منحصر ہو گی نا کہ منتقلی کی راہ پر۔ اگر \عددی{D} میں تمام \عددی{A} اور \عددی{B} کے لئے ایسا ہو تب یہ میدان بقائی میدان کہلائے گا اور ہم کہیں گے کہ \عددی{D} میں \عددی{\int\kvec{F}\cdot\dif\kvec{r}} راہ سے آزاد ہے اور \عددی{D} پر \عددی{\kvec{F}} بقائی ہے۔

\ابتدا{تعریف}
فضا میں کھلا خطہ \عددی{D} پر \عددی{\kvec{F}} کو ایک  معین میدان لیتے ہوئے تصور کریں کہ \عددی{D} میں ہر دو نقطوں \عددی{A} اور \عددی{B} کے بیچ ہر ممکنہ راہ پر تکمل کام \عددی{\int_A^B\kvec{F}\cdot\dif\kvec{r}} کی قیمت ایک جیسی ہے۔ تب تکمل \عددی{\int\kvec{F}\cdot\dif\kvec{r}} خطہ \عددی{D} میں \اصطلاح{راہ سے آزاد}\فرہنگ{راہ سے آزاد}\حاشیہب{path independent}\فرہنگ{path independent} ہو گا اور میدان \عددی{\kvec{F}} خطہ \عددی{D} پر \اصطلاح{بقائی}\فرہنگ{بقائی}\حاشیہب{conservative}\فرہنگ{conservative} ہو گا۔
\انتہا{تعریف}
%======================

عملی زندگی میں عموماً میدان \عددی{\kvec{F}} صرف اور صرف اس صورت بقائی ہو گا جب  \عددی{D} پر \عددی{\kvec{F}=\nabla f} ہو جہاں \عددی{f} ایک غیر سمتی تفاعل ہے۔ ایسی صورت میں تفاعل \عددی{f} کو \عددی{\kvec{F}} کا مخفی قوہ تفاعل کہتے ہیں۔   

\ابتدا{تعریف}
اگر \عددی{D}پر میدان \عددی{\kvec{F}} معین ہو اور \عددی{\kvec{F}=\nabla f} ہو جہاں \عددی{f} خطہ \عددی{D} پر ایک غیر سمتی تفاعل ہو تب \عددی{f} کو \عددی{\kvec{F}} کا \اصطلاح{مخفی قوہ تفاعل}\فرہنگ{مخفی قوہ تفاعل}\حاشیہب{potential function}\فرہنگ{potential function} کہتے ہیں۔
\انتہا{تعریف}
%======================

برقی مخفی قوہ ایک غیر سمتی تفاعل ہے جس کا میدان ڈھلوان  ایک برقی میدان ہوتا ہے۔  ثقلی مخفی قوہ ایک غیر سمتی تفاعل ہے جس کا میدان ڈھلوان ایک  ثقلی میدان ہوتا ہے، وغیرہ وغیرہ۔ جیسا ہم اب دیکھیں گے،  میدان \عددی{\kvec{F}} کا مخفی قوہ تفاعل \عددی{f} جاننے کے بعد \عددی{\kvec{F}} کی دائرہ کار میں تمام تکملات کام  کی قیمتیں درج ذیل سے حاصل کی جا سکتی ہیں۔
\begin{align}\label{مساوات_سمتی_تکمل_مخفی_قوہ_استعمال}
\int_A^B\kvec{F}\cdot\dif\kvec{r}=\int_A^B\nabla f\cdot\dif \kvec{r}=f(B)-f(A)
\end{align}
اگر آپ واحد متغیر کے تفرق \عددی{f'} کی طرح  \عددی{\nabla f} کو متعدد متغیرات کے تفاعل کے لئے فرض کریں تب مساوات \حوالہ{مساوات_سمتی_تکمل_مخفی_قوہ_استعمال} کو احصاء کے بنیادی کلیہ
\begin{align*}
\int_a^bf'(x)\dif x=f(b)-f(a)
\end{align*}  
کا مطابقتی سمتی احصاء کا کلیہ تصور کیا جا سکتا ہے۔

بقائی میدان کی دیگر قابل ذکر خواص پر، آگے چلتے ہوئے ساتھ ساتھ، غور کیا جائے گا۔ مثلاً، \عددی{D} پر بقائی \عددی{\kvec{F}} کی صورت میں \عددی{D} میں ہر بند راہ پر تکمل کام صفر ہو گا۔مساوات \حوالہ{مساوات_سمتی_تکمل_مخفی_قوہ_استعمال} اور اس کی مضمرات کی درستگی برقرار رکھنے  کی خاطر ہمیں اس مساوات میں مستعمل منحنی، میدان، اور دائرہ کار پر شرائط مسلط کرنی ہوں گے۔

ہم فرض کرتے ہیں کہ تمام منحنیات \اصطلاح{ٹکڑوں میں ہموار}\حاشیہب{piecewise smooth} ہیں،یعنی، انہیں متناہی تعداد کی ہموار  منحنیات کو ایک دوسرے کے ساتھ جوڑ کر، حصہ \حوالہ{حصہ_سمتی_تفاعل_سمتی_قیمت_تفاعل_اور_فضائی_منحنیات} کی طرح ،حاصل کیا گیا ہے۔مزید ہم فرض کرتے ہیں کہ \عددی{\kvec{F}} کے اجزاء کے یک رتبی استمراری تفرقات  پائے جاتے ہیں۔استمرار کی اس شرح  کے بعد \عددی{\kvec{F}=\nabla f} کی صورت میں مخفی قوہ تفاعل \عددی{f} کے مدغم تفرقات ایک دوسرے کے برابر ہوں گے، جو بقائی میدان \عددی{\kvec{F}} کے خواص پر غور کے دوران آفشاں انگیز ثابت ہو گا۔

ہم فضا میں \عددی{D} کو ایک \ترچھا{کھلا} خطہ فرض کرتے ہیں۔ یوں \عددی{D} میں ہر نقطہ ایک ایسے گیند کے مرکز پر پایا جائے گا جو مکمل طور پر \عددی{D} میں پایا جاتا ہو۔ مزید ہم فرض کرتے ہیں کہ \عددی{D} \اصطلاح{تعلق (دار)}\فرہنگ{تعلق!دار}\حاشیہب{connected}\فرہنگ{connected} خطہ ہے۔  کھلا خطہ میں تعلق دار سے مراد  ایسا خطہ ہے، جس میں ہر دو نقطوں کو ایک ایسی مسلسل راہ سے جوڑا جا سکتا ہے جو مکمل طور پر اس خطہ میں پائی جاتی ہو۔ 

\جزوحصہء{بقائی میدان میں لکیری تکملات}
بقائی میدان میں لکیری تکملات کی قیمتیں درج ذیل نتیجہ کی مدد سے باآسانی حاصل کی جا سکتی ہیں۔اس نتیجہ کے تحت تکمل کی قیمت صرف ابتدائی اور اختتام نقطوں پر منحصر ہو گی نا کہ منتقلی کی راہ پر۔

\ابتدا{مسئلہ}\شناخت{مسئلہ_سمتی_تکمل_بنیادی_مسئلہ}\موٹا{لکیری تکملات کا بنیادی مسئلہ}\\
\begin{enumerate}[1.]
\item
فرض کریں فضا میں کھلے  تعلق دار خطہ \عددی{D} میں سمتی میدان  \عددی{\kvec{F}=M\ai+N\aj+P\ak} کے اجزاء  استمراری ہیں۔تب صرف اور صرف اس صورت جب  \عددی{D} میں تمام نقاط \عددی{A} اور \عددی{B} کے لئے تکمل \عددی{\int_A^B\kvec{F}\cdot\dif\kvec{r}} کی قیمت، \عددی{D} کے اندر رہتے ہوئے \عددی{A} اور \عددی{B} کے بیچ تمام راہ سے آزاد ہو،  ایسا قابل تفرق تفاعل \عددی{f} موجود ہو گا جو درج ذیل پر پورا اترتا ہو۔
\begin{align*}
\kvec{F}=\nabla f=\frac{\partial f}{\partial x}\ai+\frac{\partial f}{\partial y}\aj+\frac{\partial f}{\partial z}\ak
\end{align*}
\item
اگر تکمل کی قیمت \عددی{A} اور \عددی{B} کے بیچ راہ سے آزاد ہو تب تکمل کی قیمت درج ذیل ہو گی۔
\begin{align*}
\int_A^B\kvec{F}\cdot\dif\kvec{r}=f(B)-f(A)
\end{align*}
\end{enumerate}
\انتہا{مسئلہ}
%========= 
\ابتدا{ثبوت}\موٹا{کہ \عددی{\kvec{F}=\nabla f} سے مراد تکمل کی قیمت کا راہ سے آزاد ہونا ہے}\\
فرض کریں \عددی{D} میں \عددی{A} اور \عددی{B} دو نقطے ہیں اور \عددی{D} میں \عددی{A} سے \عددی{B} تک
 ہموار راہ درج ذیل ہے۔
\[C:\quad \kvec{r}(t)=g(t)\ai+h(t)\aj+k(t)\ak,\quad a\le t\le b\]
 منحنی کی ہمراہ \عددی{t} کے لحاظ سے  \عددی{C} قابل تفرق ہے اور درج ذیل ہو گا۔
\begin{align}
\frac{\dif f}{\dif t}&=\frac{\partial f}{\partial x}\frac{\dif x}{\dif t}+\frac{\partial f}{\partial y}\frac{\dif y}{\dif t}+\frac{\partial f}{\partial z}\frac{\dif z}{\dif t}&&\text{\RL{زنجیری قاعدہ}}\nonumber\\
&=\nabla f\cdot \big(\frac{\dif x}{\dif t}\ai+\frac{\dif y}{\dif t}\aj+\frac{\dif z}{\dif t}\ak\big)=\nabla f\cdot \frac{\dif\kvec{r}}{\dif t}=\kvec{F}\cdot\frac{\dif\kvec{r}}{\dif t}\label{مساوات_سمتی_تکمل_ثبوت_بقائی}
\end{align}
یوں درج ذیل ہو گا۔
\begin{align*}
\int_C\kvec{F}\cdot\dif\kvec{r}&=\int_{t=a}^{t=b}\kvec{F}\cdot\frac{\dif \kvec{r}}{\dif t}\dif t=\int_a^b\frac{\dif f}{\dif t}\dif t&&\text{\RL{مساوات \حوالہ{مساوات_سمتی_تکمل_ثبوت_بقائی}}}\\
&=f(g(t),h(t),k(t))\big]_a^b=f(B)-f(A)
\end{align*}
اس طرح تکمل کام کی قیمت \عددی{A} اور \عددی{B} پر \عددی{f} کی قیمتوں پر منحصر ہو گی نا کہ ان کے بیچ راہ پر۔ یوں مسئلہ کے دوسرا  جزو  کے ساتھ ساتھ پہلا مضمر جزو بھی ثابت ہوتا ہے۔ ہم الٹ مضمر  کا زیادہ پیچیدہ ثبوت پیش نہیں کرتے ہیں۔
\انتہا{ثبوت}
%==================

\ابتدا{مثال}
نقاط \عددی{(-1,3,9)} اور \عددی{(1,6,-4)} کے بیچ ہموار منحنی \عددی{C} پر چلتے  ہوئے درج ذیل بقائی میدان کا کم تلاش کریں۔
\begin{align*}
\kvec{F}=yz\ai+xz\aj+xy\ak=\nabla(xyz)
\end{align*}

حل:\quad
\عددی{f(x,y,z)=xyz} لیتے ہوئے درج ذیل ہو گا۔
\begin{align*}
\int_A^B\kvec{F}\cdot\dif\kvec{r}&=\int_A^B\nabla f\cdot \dif \kvec{r}&&\kvec{F}=\nabla f\\
&=f(B)-f(A)&&\text{\RL{مسئلہ \حوالہ{مسئلہ_سمتی_تکمل_بنیادی_مسئلہ} کا جزو دوم}}\\
&=\left. xyz\right\vert_{(1,6,-4)}-\left. xyz\right\vert_{(-1,3,9)}\\
&=(1)(6)(-4)-(-1)(3)(9)\\
&=-24+27=3
\end{align*}
\انتہا{مثال}
%=====================
\ابتدا{مسئلہ}\شناخت{مسئلہ_سمتی_تکمل_بند_راہ_صفر}
درج ذیل فقرے  معادل ہیں۔
\begin{enumerate}[a.]
\item
خطہ \عددی{D} میں ہر بند راہ پر \عددی{\int\kvec{F}\cdot\dif\kvec{r}=0} ہے۔
\item
خطہ \عددی{D} پر میدان \عددی{\kvec{F}} بقائی ہے۔
\end{enumerate}
\انتہا{مسئلہ}
%=============
\ابتدا{ثبوت}\موٹا{جزو-ا}\\
ہم دکھانا چاہتے ہیں کہ \عددی{D} میں کسی بھی دو نقاط \عددی{A} اور \عددی{B} کے بیچ کسی بھی  دو راہوں \عددی{C_1} اور \عددی{C_2} پر \عددی{\kvec{F}\cdot\dif\kvec{r}} کے تکملات کی قیمتیں ایک دوسرے جیسی ہوں گی۔ہم شکل میں \عددی{C_2} کا رخ الٹ کر کے \عددی{B} سے \عددی{A} تک راہ کو \عددی{-C_2} لکھتے ہیں۔راہ \عددی{C_1} اور راہ \عددی{-C_2} مل کر بند راہ \عددی{C} دیتے ہیں۔اب درج ذیل ہو گا۔
\begin{align*}
\int_{C_1}\kvec{F}\cdot\dif\kvec{r}-\int_{C_2}\kvec{F}\cdot\dif\kvec{r}=\int_{C_1}\kvec{F}\cdot\dif\kvec{r}+\int_{-C_2}\kvec{F}\cdot\dif\kvec{r}=\int_{C}\kvec{F}\cdot\dif\kvec{r}=0
\end{align*}
یوں \عددی{C_1} اور \عددی{C_2} پر تکمل کی قیمتیں ایک دوسرے جیسی  ہیں۔
\انتہا{ثبوت}
%========================
\ابتدا{ثبوت}\موٹا{جزو-ب}\\
ہم دکھانا چاہتے ہیں کہ ہر بند راہ \عددی{C} پر \عددی{\kvec{F}\cdot\dif\kvec{r}} کا تکمل صفر ہے۔ ہم \عددی{C} پر کسی دو نقطوں \عددی{A} اور \عددی{B} کا انتخاب کر کے \عددی{C} کو دو ٹکڑوں میں تقسیم کرتے ہیں: نقطہ \عددی{A} سے \عددی{B} تک ٹکڑے کو  \عددی{C_1} اور نقطہ \عددی{B} سے \عددی{A} تک ٹکڑے کو \عددی{C_2} لیتے ہوئے خلاف گھڑی تکمل درج ذیل ہو گا۔ 
\begin{align*}
\oint_C\kvec{F}\cdot\dif\kvec{r}&=\int_{C_1}\kvec{F}\cdot\dif\kvec{r}+\int_{C_2}\kvec{F}\cdot\dif\kvec{r}=\int_A^B\kvec{F}\cdot\dif\kvec{r}-\int_B^A\kvec{F}\cdot\dif\kvec{r}=0
\end{align*}
\انتہا{ثبوت}
%====================
مسئلہ \حوالہ{مسئلہ_سمتی_تکمل_بنیادی_مسئلہ} اور مسئلہ \حوالہ{مسئلہ_سمتی_تکمل_بند_راہ_صفر} کے نتائج کو  یکجا  کرتے ہیں۔
\begin{align*}
\oint_C\kvec{F}\cdot\dif\kvec{r}=0\,\text{\RL{\(D\)\, میں کسی بھی بند راہ پر}}\quad \Leftrightarrow \quad \text{\RL{\(D\)\, پر \, \(\kvec{F}\)\, بقائی ہے}}\quad \Leftrightarrow\quad   \kvec{F}=\nabla f \,\text{\RL{\(D\)\, پر}}
\end{align*}

یہ جانتے ہوئے کہ  بقائی میدان میں لکیری تکملات کا حل کتنا آسان ہے، دو سوالات پیدا ہوتے ہیں:
\begin{enumerate}[1.]
\item
ہمیں کیسے جان سکتے ہیں کہ  میدان \عددی{\kvec{F}} بقائی ہے؟
\item
بقائی میدان \عددی{\kvec{F}} کا مطابقتی  مخفی قوہ تفاعل \عددی{f} کیسے دریافت کیا جا سکتا ہے (جہاں \عددی{\kvec{F}=\nabla f} ہو گا)۔
\end{enumerate}

\جزوحصہء{بقائی میدان کا مخفی قوہ تفاعل کا حصول}
بقائی میدان کا پرکھ درج ذیل ہے۔
\ابتدا{پرکھ}\موٹا{بقائی میدان کا اجزائی پرکھ}\\
میدان \عددی{\kvec{F}=M(x,y,z)\ai+N(x,y,z)\aj+P(x,y,z)\ak} جس کے اجزائی  تفاعل کے استمراری یک رتبی جزوی تفرقات پائے جاتے ہوں صرف اور صرف اس صورت بقائی ہو گا جب درج ذیل مطمئن  ہوں۔
\begin{align}\label{مساوات_سمتی_تکمل_بقائی_جزوی_پرکھ}
\frac{\partial P}{\partial y}=\frac{\partial N}{\partial z},\quad \frac{\partial M}{\partial z}=\frac{\partial P}{\partial x}, \quad \text{اور}\quad \frac{\partial N}{\partial x}=\frac{\partial M}{\partial y}
\end{align}
\انتہا{پرکھ}
%=====================
\ابتدا{ثبوت پرکھ}
ہم دکھاتے ہیں کہ بقائی \عددی{\kvec{F}} کے لئے  مساوات \حوالہ{مساوات_سمتی_تکمل_بقائی_جزوی_پرکھ} ہر صورت مطمئن ہو گا۔ایسا مخفی قوہ تفاعل \عددی{f} پایا جائے گا جو درج ذیل کو مطمئن کرے گا۔
\begin{align*}
\kvec{F}=M\ai+N\aj+&P\ak=\frac{\partial f}{\partial x}\ai+\frac{\partial f}{\partial y}\aj+\frac{\partial f}{\partial z}\ak
\end{align*}
یوں درج ذیل ہو گا۔
\begin{align*}
\frac{\partial P}{\partial y}&=\frac{\partial}{\partial y}\big(\frac{\partial f}{\partial z}\big)=\frac{\partial^{\,2}f}{\partial y\partial z}\\
&=\frac{\partial^{\,2}f}{\partial z\partial y}&&\text{\small{\begin{minipage}{3.25cm}\RL{استمراری ہونے کی بدولت مدغم جزوی تفرقات ایک دوسرے کے برابر ہوں گے}  \end{minipage}}}\\
&=\frac{\partial}{\partial z}\big(\frac{\partial f}{\partial y}\big)=\frac{\partial N}{\partial z}
\end{align*}
مساوات \حوالہ{مساوات_سمتی_تکمل_بقائی_جزوی_پرکھ} کے باقی دو اجزاء بھی اسی طرح ثابت کیے جا سکتے ہیں۔

ثبوت کا دوسرا حصہ، جس سے مراد لیا جا سکتا ہے کہ مساوات \حوالہ{مساوات_سمتی_تکمل_بقائی_جزوی_پرکھ} کہتی ہے کہ \عددی{\kvec{F}} بقائی ہو گا، مسئلہ سٹوکس کا نتیجہ ہے۔
\انتہا{ثبوت پرکھ}
%=======================

یہ جانتے ہوئے کہ \عددی{\kvec{F}} بقائی ہے، ہم عموماً مخفی قوہ تفاعل \عددی{f} دریافت کرنا چاہیں گے جو مساوات \عددی{\nabla f=\kvec{F}} یعنی
\begin{align*}
\frac{\partial f}{\partial x}\ai+\frac{\partial f}{\partial y}\aj+\frac{\partial f}{\partial z}\ak=M\ai+N\aj+P\ak
\end{align*}
کو \عددی{f} کے لئے حل کرنے سے حاصل ہو گا۔ ہم درج ذیل تین مساوات کا تکمل لے کر ایسا کرتے ہیں۔
\begin{align*}
\frac{\partial f}{\partial x}=M,\quad \frac{\partial f}{\partial y}=N,\quad \frac{\partial f}{\partial z}=P
\end{align*}
%
\ابتدا{مثال}
دکھائیں کہ \عددی{\kvec{F}=(e^x\cos y+yz)\ai+(xz-e^x\sin y)\aj+(xy+z)\ak} بقائی ہے اور اس کا مخفی قوہ تفاعل \عددی{f} تلاش کریں۔

حل:\quad
ہم مساوات \حوالہ{مساوات_سمتی_تکمل_بقائی_جزوی_پرکھ} میں دی گئی پرکھ کا اطلاق 
\begin{align*}
M=e^x\cos y+yz,\quad N=xz-e^x\sin y,\quad P=xy+z
\end{align*}
پر کر کے درج ذیل حاصل کرتے ہیں۔
\begin{align*}
\frac{\partial P}{\partial y}=x=\frac{\partial N}{\partial z},\quad \frac{\partial M}{\partial z}=y=\frac{\partial P}{\partial x},\quad \frac{\partial N}{\partial x}=z-e^x\sin y=\frac{\partial M}{\partial y}
\end{align*}
یہ مساوات مل کر ہمیں بتاتے ہیں کہ  ایسا \عددی{f} پایا جاتا ہے جو \عددی{\nabla f=\kvec{F}} کو مطمئن کرتا ہے۔

ہم درج ذیل مساواتوں کے تکملات سے \عددی{f} کو تلاش کرتے ہیں۔
\begin{align}\label{مساوات_سمتی_تکمل_مثال_بقائی}
\frac{\partial f}{\partial x}=e^x\cos y+yz,\quad \frac{\partial f}{\partial y}=xz-e^x\sin y,\quad \frac{\partial f}{\partial z}=xy+z
\end{align}
ہم بائیں سے شروع کرتے ہوئے \عددی{y} اور \عددی{z} کو مستقل تصور کر کے  پہلی مساوات کا تکمل \عددی{x} کے لحاظ سے  لیتے ہیں:
\begin{align}\label{مساوات_سمتی_تکمل_بقائی_مثال_الف}
f(x,y,z)=e^x\cos y+xyz+g(y,z)
\end{align}
ہم نے تکمل کے مستقل کو \عددی{g(y,z)} لکھا ہے چونکہ اس کی قیمت \عددی{y} اور \عددی{z} کے ساتھ تبدیل ہو سکتی ہے۔اس مساوات سے  ہم \عددی{\tfrac{\partial f}{\partial y}} تلاش کر کے مساوات \حوالہ{مساوات_سمتی_تکمل_مثال_بقائی} میں دی گئی  \عددی{\tfrac{\partial f}{\partial y}} کے برابر پر کرتے ہیں:
 \begin{align*}
-e^x\sin y+xz+\frac{\partial g}{\partial y}=xz-e^x\sin y
\end{align*}
یوں \عددی{\tfrac{\partial g}{\partial y}=0} ہو گا لہٰذا \عددی{g} کی قیمت صرف \عددی{z} پر منحصر ہو گی۔اس طرح مساوات \حوالہ{مساوات_سمتی_تکمل_بقائی_مثال_الف} درج ذیل روپ اختیار کرے گی۔
\begin{align*}
f(x,y,z)&=e^x\cos y+xyz+h(z)
\end{align*}
اس مساوات سے ہم \عددی{\tfrac{\partial f}{\partial z}} معلوم کر کے مساوات \حوالہ{مساوات_سمتی_تکمل_مثال_بقائی} میں دی گئی  \عددی{\tfrac{\partial f}{\partial z}} کے برابر پر کرتے ہیں:
\begin{align*}
xy+\frac{\dif h}{\dif z}&=xy+z\\
\frac{\dif h}{\dif z}&=z\quad \text{یعنی}
\end{align*}
متغیر \عددی{z} کے ساتھ تکمل لیتے ہیں:
\begin{align*}
h(z)==\frac{z^2}{2}+C
\end{align*}
اس طرح درج ذیل ہو گا۔
\begin{align*}
f(x,y,z)=e^x\cos y+xyz+\frac{z^2}{2}+C
\end{align*}
یوں  مستقل \عددی{C} کی لامتناہی ممکنہ منفرد قیمتیں منتخب کر کے  \عددی{\kvec{F}} کے لامتناہی تعداد کے مخفی قوہ تفاعل حاصل ہوں گے۔ 
\انتہا{مثال}
%======================
\ابتدا{مثال}
دکھائیں کہ \عددی{\kvec{F}=(2x-3)\ai-z\aj+(\cos z)\ak} غیر بقائی ہے۔

حل:\quad
ہم مساوات \حوالہ{مساوات_سمتی_تکمل_بقائی_جزوی_پرکھ} میں دی گئی پرکھ استعمال کر کے 
\begin{align*}
\frac{\partial P}{\partial y}=\frac{\partial}{\partial y}(\cos z)=0,\quad \frac{\partial N}{\partial z}=\frac{\partial}{\partial z}(-z)=-1
\end{align*}
حاصل کرتے ہیں۔یہ قیمتیں ایک دوسرے سے مختلف ہیں لہٰذا \عددی{\kvec{F}} غیر بقائی ہو گا۔ پرکھ کے باقی اجزاء کو دیکھنے کی ضرورت نہیں ہے۔
\انتہا{مثال}
%====================
\جزوحصہء{قطعی تفرقی روپ}

