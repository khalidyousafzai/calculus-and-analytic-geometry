\جزوحصہء{سوالات}
\موٹا{یک رتبی جزوی تفرق کی تلاش}\\
سوال \حوالہ{ سوال_کثیرالمتغیر_یک_رتبی_جزوی_الف}  تا سوال  \حوالہ{سوال_کثیرالمتغیر_یک_رتبی_جزوی_ب} میں \عددی{\tfrac{\partial f}{\partial x}} اور \عددی{\tfrac{\partial f}{\partial y}} تلاش کریں۔

\ابتدا{سوال}\شناخت{ سوال_کثیرالمتغیر_یک_رتبی_جزوی_الف}
$f(x,y)=2x^2-3y-4$
\انتہا{سوال}
%===================
\ابتدا{سوال}
$f(x,y)=x^2-xy+y^2$
\انتہا{سوال}
%====================
\ابتدا{سوال}
$f(x,y)=(x^2-1)(y+2)$
\انتہا{سوال}
%====================
\ابتدا{سوال}
$f(x,y)=5xy-7x^2-y^2+3x-6y+2$
\انتہا{سوال}
%====================
\ابتدا{سوال}
$f(x,y)=(xy-1)^2$
\انتہا{سوال}
%====================
\ابتدا{سوال}
$f(x,y)=(2x-3y)^3$
\انتہا{سوال}
%====================
\ابتدا{سوال}
$f(x,y)=\sqrt{x^2+y^2}$
\انتہا{سوال}
%====================
\ابتدا{سوال}
$f(x,y)=(x^3+y/2)^{2/3}$
\انتہا{سوال}
%====================
\ابتدا{سوال}
$f(x,y)=\frac{1}{x+y}$
\انتہا{سوال}
%====================
\ابتدا{سوال}
$f(x,y)=\frac{x}{x^2+y^2}$
\انتہا{سوال}
%====================
\ابتدا{سوال}
$f(x,y)=\frac{x+y}{xy-1}$
\انتہا{سوال}
%====================
\ابتدا{سوال}
$f(x,y)=\tan^{-1}\frac{y}{x}$
\انتہا{سوال}
%====================
\ابتدا{سوال}
$f(x,y)=e^{x+y+1}$
\انتہا{سوال}
%====================
\ابتدا{سوال}
$f(x,y)=e^{-x}\sin(x+y)$
\انتہا{سوال}
%====================
\ابتدا{سوال}
$f(x,y)=\ln(x+y)$
\انتہا{سوال}
%====================
\ابتدا{سوال}
$f(x,y)=e^{xy}\ln y$
\انتہا{سوال}
%====================
\ابتدا{سوال}
$f(x,y)=\sin^2(x-3y)$
\انتہا{سوال}
%====================
\ابتدا{سوال}
$f(x,y)=\cos^2(3x-y^2)$
\انتہا{سوال}
%====================
\ابتدا{سوال}
$f(x,y)=x^y$
\انتہا{سوال}
%====================
\ابتدا{سوال}
$f(x,y)=\log_y x$
\انتہا{سوال}
%====================
\ابتدا{سوال}
$f(x,y)=\int_x^yg(t)\dif t\quad \text{\RL{تمام \عددی{t} کے لئے \عددی{g} استمراری ہے}}$
\انتہا{سوال}
%====================
\ابتدا{سوال}\شناخت{سوال_کثیرالمتغیر_یک_رتبی_جزوی_ب}
$f(x,y)=\sum\limits_{n=0}^{\infty}(xy)^n\quad (\abs{xy}<1)$
\انتہا{سوال}
%====================

سوال \حوالہ{سوال_کثیرالمتغیر_یک_رتبی_تین_متغیر_تفاعل_الف} تا سوال \حوالہ{سوال_کثیرالمتغیر_یک_رتبی_تین_متغیر_تفاعل_ب} میں \عددی{f_x}، \عددی{f_y}  اور \عددی{f_z} تلاش کریں۔

\ابتدا{سوال}\شناخت{سوال_کثیرالمتغیر_یک_رتبی_تین_متغیر_تفاعل_الف}
$f(x,y,z)=1+xy^2-2z^2$
\انتہا{سوال}
%===================
\ابتدا{سوال}
$f(z,y,z)=xy+yz+xz$
\انتہا{سوال}
%=====================
\ابتدا{سوال}
$f(z,y,z)=x-\sqrt{y^2+z^2}$
\انتہا{سوال}
%=====================
\ابتدا{سوال}
$f(z,y,z)=(x^2+y^2+z^2)^{-1/2}$
\انتہا{سوال}
%=====================
\ابتدا{سوال}
$f(z,y,z)=\sin^{-1}(xyz)$
\انتہا{سوال}
%=====================
\ابتدا{سوال}
$f(z,y,z)=\sec^{-1}(x+yz)$
\انتہا{سوال}
%=====================
\ابتدا{سوال}
$f(z,y,z)=\ln(x+2y+3z)$
\انتہا{سوال}
%=====================
\ابتدا{سوال}
$f(z,y,z)=yz\ln(xy)$
\انتہا{سوال}
%=====================
\ابتدا{سوال}
$f(z,y,z)=e^{-(x^2+y^2+z^2)}$
\انتہا{سوال}
%=====================
\ابتدا{سوال}
$f(z,y,z)=e^{-xyz}$
\انتہا{سوال}
%=====================
\ابتدا{سوال}
$f(z,y,z)=\tanh(x+2y+3z)$
\انتہا{سوال}
%=====================
\ابتدا{سوال}\شناخت{سوال_کثیرالمتغیر_یک_رتبی_تین_متغیر_تفاعل_ب}
$f(z,y,z)=\sinh(xy-z^2)$
\انتہا{سوال}
%=====================

سوال \حوالہ{سوال_کثیر_المتغیر_ہر_متغیر_جزوی_الف} تا سوال \حوالہ{سوال_کثیر_المتغیر_ہر_متغیر_جزوی_ب} میں ہر متغیر کے لحاظ سے تفاعل کا جزوی تفرق تلاش کریں۔

\ابتدا{سوال}\شناخت{سوال_کثیر_المتغیر_ہر_متغیر_جزوی_الف}
$f(t,\alpha)=\cos(2\pi t-\alpha)$
\انتہا{سوال}
%=================
\ابتدا{سوال}
$g(u,v)=v^2e^{2u/v}$
\انتہا{سوال}
%=================
\ابتدا{سوال}
$h(\rho,\phi,\theta)=\rho\sin\phi\cos\theta$
\انتہا{سوال}
%=================
\ابتدا{سوال}
$g(r,\theta,z)=r(1-\cos\theta)-z$
\انتہا{سوال}
%=================
\ابتدا{سوال}\ترچھا{قلب کا کام}\\
$W(P,H,\delta,v,g)=PV+\frac{H\delta v^2}{2g}$
\انتہا{سوال}
%=================
\ابتدا{سوال}\شناخت{سوال_کثیر_المتغیر_ہر_متغیر_جزوی_ب}
$A(c,h,k,m,q)=\frac{km}{q}+cm+\frac{hq}{2}$
\انتہا{سوال}
%=================

\موٹا{دو رتبی جزوی تفرق کا حصول}\\
سوال \حوالہ{سوال_کثیرالمتغیر_تمام_جزوی_تلاش_الف} تا سوال \حوالہ{سوال_کثیرالمتغیر_تمام_جزوی_تلاش_ب} میں تفاعل کے تمام دو رتبی جزوی تفرقات تلاش کریں۔

\ابتدا{سوال}\شناخت{سوال_کثیرالمتغیر_تمام_جزوی_تلاش_الف}
$f(x,y)=x+y+xy$
\انتہا{سوال}
%=================
\ابتدا{سوال}
$f(x,y)=\sin xy$
\انتہا{سوال}
%=================
\ابتدا{سوال}
$g(x,y)=x^2y+\cos y+y\sin x$
\انتہا{سوال}
%=================
\ابتدا{سوال}
$h(x,y)=xe^y+y+1$
\انتہا{سوال}
%=================
\ابتدا{سوال}
$r(x,y)=\ln(x,y)$
\انتہا{سوال}
%=================
\ابتدا{سوال}\شناخت{سوال_کثیرالمتغیر_تمام_جزوی_تلاش_ب}
$s(x,y)=\tan^{-1}\frac{y}{x}$
\انتہا{سوال}
%=================

\موٹا{مدغم جزوی تفرقات}\\
سوال \حوالہ{سوال_کثیرالمتغیر_تصدیق_مدغم_الف} تا سوال \حوالہ{سوال_کثیرالمتغیر_تصدیق_مدغم_ب}   میں \عددی{w_{xy}=w_{yx}} کی   تصدیق کریں۔

\ابتدا{سوال}\شناخت{سوال_کثیرالمتغیر_تصدیق_مدغم_الف}
$w=\ln(2x+3y)$
\انتہا{سوال}
%===============
\ابتدا{سوال}
$w=e^x+x\ln y+y]ln x$
\انتہا{سوال}
%==================
\ابتدا{سوال}
$w=xy^2+x^2y^3+x^3y^4$
\انتہا{سوال}
%==================
\ابتدا{سوال}\شناخت{سوال_کثیرالمتغیر_تصدیق_مدغم_ب}
$w=x\sin y+y\sin x+xy$
\انتہا{سوال}
%==================
\ابتدا{سوال}
بغیر قلم اٹھائے   بتائیں کہ درج ذیل میں \عددی{x} کے لحاظ سے پہلے اور \عددی{y} کے  لحاظ سے بعد میں یا اس کے الٹ حل کرتے ہوئے  \عددی{f_{xy}} زیادہ جلدی حاصل ہو گا۔
\begin{enumerate}[a.]
\item
$f(x,y)=x\sin y+e^y$
\item
$f(x,y)=\frac{1}{x}$
\item
$f(x,y)=y+\frac{x}{y}$
\item
$f(x,y)=y+x^2y+4y^3-\ln(y^2+1)$
\item
$f(x,y)=x^2+5xy+\sin x+7e^x$
\item
$f(x,y)=x\ln xy$
\end{enumerate}
\انتہا{سوال}
%===============
\ابتدا{سوال}
درج ذیل میں تمام کا پانچ رتبی جزوی تفرق \عددی{\tfrac{\partial^{\,5}f}{\partial x^2\partial y^3}} صفر کے برابر ہے۔ اس کی تصدیق کرنے کی خاطر آپ کس متغیر کے لحاظ سے پہلے جزوی تفرق لیں گے؟ بغیر کچھ لکھے جواب دینے کی کوشش کریں۔
\begin{enumerate}[a.]
\item
$f(x,y)=y^2x^4e^x+2$
\item
$f(x,y)=y^2+y(\sin x-x^4)$
\item
$f(x,y)=x^2+5xy+\sin x+7e^x$
\item
$f(x,y)=xe^{y^2/2}$
\end{enumerate}
\انتہا{سوال}
%==============

\موٹا{جزوی تفرق کی تعریف کا استعمال}\\
سوال \حوالہ{سوال_کثیرالمتغیر_تعریف_حد_جزوی_تفرق_الف} اور سوال \حوالہ{سوال_کثیرالمتغیر_تعریف_حد_جزوی_تفرق_ب} میں جزوی تفرق کی تعریف بذریعہ حد استعمال کرتے ہوئے دیے گئے نقطہ پر تفاعل کا جزوی تفرق حاصل کریں۔

\ابتدا{سوال}\شناخت{سوال_کثیرالمتغیر_تعریف_حد_جزوی_تفرق_الف}
$f(x,y)=1-x+y-3x^2y,\quad \frac{\partial f}{\partial x},\,\frac{\partial f}{\partial y},\quad (1,2)$
\انتہا{سوال}
%==================
\ابتدا{سوال}\شناخت{سوال_کثیرالمتغیر_تعریف_حد_جزوی_تفرق_ب}
$f(x,y)=4+2x-3y-xy^2,\quad \frac{\partial f}{\partial x},\,\frac{\partial f}{\partial y},\quad (-2,1)$
\انتہا{سوال}
%=================
\ابتدا{سوال}
فرض کریں \عددی{w=f(x,y,z)} تین غیر تابع متغیرات کا تفاعل ہے۔ نقطہ \عددی{(x_0,y_0,z_0)} پر جزوی تفرق  \عددی{\tfrac{\partial f}{\partial z}} کی باضابطہ تعریف  لکھیں کریں۔ اس تعریف کو استعمال کرتے ہوئے \عددی{(1,2,3)} پر \عددی{f(x,y,z)=x^2yz^2} کا \عددی{\tfrac{\partial f}{\partial z}} تلاش کریں۔
\انتہا{سوال}
%===================
\ابتدا{سوال}
فرض کریں \عددی{w=f(x,y,z)} تین غیر تابع متغیرات کا تفاعل ہے۔ نقطہ \عددی{(x_0,y_0,z_0)} پر جزوی تفرق  \عددی{\tfrac{\partial f}{\partial y}} کی باضابطہ تعریف  لکھیں کریں۔ اس تعریف کو استعمال کرتے ہوئے \عددی{(-1,0,3)} پر \عددی{f(x,y,z)=-2xy^2+yz^2} کا \عددی{\tfrac{\partial f}{\partial z}} تلاش کریں۔
\انتہا{سوال}
%===============
\موٹا{خفی جزوی تفرقات}\\
