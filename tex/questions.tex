\حصہ{رقبات،  معیار اثر، اور  مراکز کمیت}
اس حصہ میں دوہرا تکملات استعمال کرتے ہوئے  مستوی میں محدود  خطوں کے رقبات اور ان خطوں  پر باریک چادروں کی کمیت، معیار اثر، مرکز کمیت، اور  \اصطلاح{حرکت دواری}\فرہنگ{حرکت!دواری}\فرہنگ{دوار}\حاشیہب{gyration}\فرہنگ{gyration} کے رداس معلوم کرنا دکھایا جائے گا۔ ان کا حساب باب \حوالہ{باب_تکمل_کا_استعمال}  کے حساب کی طرح ہو گا لیکن اب ہم زیادہ قسم کے اشکال کے لئے حساب کر پائیں گے۔

\جزوحصہء{مستوی میں محدود خطوں کے رقبات} 
گزشتہ حصہ میں خطہ \عددی{R} پر   دوہرا تکمل کی تعریف میں \عددی{f(x,y)=1} لینے سے جزوی مجموعات  کی تخفیف شدہ صورت
\begin{align}
J_n=\sum_{k=1}^{n}f(x_k,y_k)\Delta S_k=\sum_{k=1}^{n}\Delta S_k
\end{align}
حاصل ہو گی۔یہ  تخمینی طور پر \عددی{R} کا رقبہ ہو گا۔ جوں جوں  شکل \حوالہ{شکل_بالکثرت_خانہ_بندی_پہلے} میں \عددی{\Delta x} اور \عددی{\Delta y} صفر کے قریب تر ہوتے جاتے ہیں توں توں \عددی{R} کے زیادہ سے زیادہ حصہ کو  تمام  \عددی{\Delta S_k} مل کر  کو ڈھانپتے ہیں، اور ہم \عددی{R} کی رقبہ کی تعریف درج ذیل لیتے ہیں۔
\begin{align}
\text{رقبہ}=\lim_{n\to\infty}\sum_{k=1}^n \Delta S_k=\iint\limits_R\dif S
\end{align}

\ابتدا{تعریف}
بند محدود خطہ \عددی{R} کا رقبہ درج ذیل ہو گا۔
\begin{align}\label{مساوات_بالکثرت_رقبہ_تعریف}
S=\iint\limits_R \dif S
\end{align}
\انتہا{تعریف}
%============

اس باب  کے دیگر تعریفات کی  طرح،     رقبے کی یک متغیری  تعریف   کے  لحاظ سے، جو ہم پہلے پیش کر چکے ہیں،    موجودہ تعریف زیادہ  اقسام کے خطوں پر قابل اطلاق  ہو  گی، لیکن،  جن خطوں پر دونوں تعریفات قابل اطلاق ہوں، وہاں موجودہ تعریف گزشتہ تعریف کے عین موافق  ہو گی۔

مساوات \حوالہ{مساوات_بالکثرت_رقبہ_تعریف} میں دی گئی تکمل کی قیمت کے حصول میں  ہم \عددی{R} پر \عددی{f(x,y)=1} لیتے ہیں۔

\ابتدا{مثال}\شناخت{مثال_بالکثرت_رقبہ_بیچ_دو_تفاعل}
ربع اول میں \عددی{y=x} اور \عددی{y=x^2} کے بیچ محیط   رقبہ تلاش کریں۔

حل:\quad
ہم اس خطہ کا خاکہ (شکل \حوالہ{شکل_مثال_بالکثرت_رقبہ_بیچ_دو_تفاعل})  بنا کر رقبہ تلاش کرتے ہیں۔
\begin{align*}
S=\int_0^1\int_{x^2}^x\dif y\dif x=\int_0^1\big[y\big]_{x^2}^x\dif x=\int_0^1 (x-x^2)\dif x=\big[\frac{x^2}{2}-\frac{x^3}{3}\big]_0^1=\frac{1}{6}
\end{align*}
\انتہا{مثال}
%==================
\begin{figure}
\centering
\begin{minipage}{0.45\textwidth}
\centering
\begin{tikzpicture}[font=\small]
\pgfmathsetmacro{\a}{3.75}
\pgfmathsetmacro{\b}{2.5}
\draw[-latex](0,0)--++(4.5,0)node[right]{$x$};
\draw[-latex](0,0)--++(0,3.25)node[right]{$y$};
\draw[fill=lgray](0.4+\a*0.07,0.25+\b*0.97)
\foreach \x/\y in{0.07/0.97,0.8/0.97,0.8/0.8,0.9/0.8,0.9/0.7,0.97/0.7,0.97/0.2,0.8/0.2,0.8/0.1,0.6/0.1,
0.6/0.02,0.2/0.02,0.2/0.1,0.07/0.1,0.07/0.3,0/0.3,0/0.8,0.07/0.8,0.07/0.97} {--(0.4+\a*\x,0.25+\b*\y)};
\draw(0.4,0)node[below]{$a$}--++(0,0.1)  (0.4,0)++(\a,0)node[below]{$b$}--++(0,0.1); 
\draw(0,0.25)node[left]{$c$}--++(0.1,0)  (0,0.25)++(0,\b)node[left]{$d$}--++(0.1,0); 
\foreach \x in {0,0.07,0.125,0.2,0.3,0.45,0.6,0.8,0.9,0.97} {\draw(0.4+\a*\x,0.125)--++(0,\b+0.25);}
\foreach \y in {0.02,0.1,0.2,0.3,0.5,0.7,0.8,0.97}{\draw(0.25,0.25+\b*\y)--++(\a+0.25,0);}
\draw[fill=gray](0.4+0.45*\a,0.25+0.3*\b) rectangle (0.4+0.6*\a,0.25+0.5*\b);
\draw[stealth-stealth](0.4+0.45*\a,-0.1)--(0.4+0.6*\a,-0.1)node[pos=0.5,below]{$\Delta x_k$};
\draw[stealth-stealth](-0.1,0.25+0.3*\b)--(-0.1,0.25+0.5*\b)node[pos=0.5,left]{$\Delta y_k$};
\draw(0.4+0.475*\a,0.25+0.4*\b)--++(-0.3,0.3)node[above]{$\Delta S_k$};
\draw(0.4+0.55*\a,0.25+0.45*\b)node[circ]{}--++(0.3,0.2)node[above right]{$(x_k,y_k)$};
\draw[thick](0.3,1/2*\b) to [out=-90,in=170](1/3*\a,0.25) to [out=-10,in=-170](2/3*\a,0.265) to [out=10,in=-90] (\a+0.45,1/2*\b) to [out=90,in=-10](\a-0.5,\b+0.25) to [out=170,in=5](1/4*\a,\b+0.25) to [out=-175,in=90](0.3,1/2*\b);
\draw(0.4+\a*0.54,0.25+\b*0.87)node[]{$R$};
\end{tikzpicture}
\caption{
ایک خطہ کے رقبے کی تلاش میں پہلا قدم خطے کی اندرون کی خانہ بندی ہے۔
}
\label{شکل_بالکثرت_خانہ_بندی_پہلے}
\end{minipage}\hfill
\begin{minipage}{0.45\textwidth}
\centering
\begin{tikzpicture}[font=\small,declare function={f(\x)=\x;g(\x)=\x^2;}]
\pgfmathsetmacro{\a}{0}
\pgfmathsetmacro{\b}{1.1}
\pgfmathsetmacro{\c}{0.5}
\begin{axis}[clip=false,small,axis lines=middle,enlargelimits=true, xlabel={$x$}, ylabel={$y$}, zlabel={$z$}, xtick={1},ytick={1},ztick={\empty},xlabel style={anchor=north}, ylabel style={anchor=west},zlabel style={anchor=south}]
\addplot[name path=kf,domain=\a:\b]{f(x)}node[pos=0.8,left]{$y=x$};
\addplot[name path=kg,domain=\a:\b]{g(x)}node[pos=0.65,below right]{$y=x^2$};
\addplot[]coordinates {(1,1)}node[right,yshift=-1ex]{$(1,1)$};
\addplot[]coordinates{(\c,0)(\c,{\c*\c})};
\addplot[-latex]coordinates{(\c,\c)(\c,0.75)};
\addplot[thick]coordinates{(\c,{\c*\c})(\c,\c)}node[pos=0,pin={[right]-10:{$y=x^2$}}]{}node[pos=1,pin={135:{$y=x$}}]{};
\addplot[lgray]fill between[of={kf and kg},soft clip={domain=0:1}];
\end{axis}
\end{tikzpicture}
\caption{قطع مکافی اور لکیر کے بیچ رقبہ (مثال \حوالہ{مثال_بالکثرت_رقبہ_بیچ_دو_تفاعل})۔}
\label{شکل_مثال_بالکثرت_رقبہ_بیچ_دو_تفاعل}
\end{minipage}
\end{figure}

\ابتدا{مثال}\شناخت{مثال_بالکثرت_قطع_مکافی_لکیر_بیچ}
قطع مکافی \عددی{y=x^2} اور لکیر \عددی{y=x+2} کے بیچ محیط رقبہ تلاش کریں۔

حل:\quad
اگر ہم پہلے \عددی{x} کے لحاظ سے تکمل لیں تب ہمیں اس خطہ کو \عددی{R_1} اور \عددی{R_2} میں تقسیم کر کے  درج ذیل دو علیحدہ علیحدہ تکملات کی  ضرورت پیش آئے گی (شکل \حوالہ{شکل_مثال_بالکثرت_قطع_مکافی_لکیر_بیچ}-ا)۔
\begin{align*}
S=\iint\limits_{R_1}\dif S+\iint\limits_{R_2}\dif S=\int_0^1\int_{-\sqrt{y}}^{\sqrt{y}}\dif x\dif y+\int_1^4\int_{y-2}^{\sqrt{y}}\dif x\dif y
\end{align*}
اس کے برعکس تکمل کی ترتیب الٹ کرنے سے صرف   ایک تکمل
\begin{align*}
S=\int_{-1}^2\int_{x^2}^{x+2}\dif y\dif x
\end{align*}
 کی ضرورت پیش آئے گی (شکل \حوالہ{شکل_مثال_بالکثرت_قطع_مکافی_لکیر_بیچ}-ب)۔ہم   اسی سے  رقبہ تلاش کرتے ہیں۔
\begin{align*}
S=\int_{-1}^2\big[y\big]_{x^2}^{x+2}\dif x=\int_{-1}^2(x+2-x^2)\dif x=\big[\frac{x^2}{2}+2x-\frac{x^3}{3}\big]_{-1}^2=\frac{9}{2}
\end{align*}

\انتہا{مثال}
%=================
\begin{figure}
\centering
\begin{subfigure}{0.45\textwidth}
\centering
\begin{tikzpicture}[font=\small,declare function={f(\x)=\x+2;g(\x)=\x^2;}]
\pgfmathsetmacro{\a}{-1}
\pgfmathsetmacro{\b}{2.1}
\pgfmathsetmacro{\c}{0.5}
\begin{axis}[clip=false,small,axis lines=middle,enlargelimits=true, xlabel={$x$}, ylabel={$y$}, zlabel={$z$}, xtick={2},ytick={4},ztick={\empty},xlabel style={anchor=north}, ylabel style={anchor=west},zlabel style={anchor=south}]
\addplot[name path=kf,domain=\a:\b]{f(x)}node[pos=0.85,left,yshift=1ex]{$y=x+2$};
\addplot[name path=kg,domain=\a:\b]{g(x)}node[above]{$y=x^2$}node[pos=0,left]{$(-1,1)$};
\addplot[]coordinates {(2,4)}node[right,yshift=-1ex]{$(2,4)$};
\addplot[lgray,opacity=0.5]fill between[of={kf and kg},soft clip={domain=-1:2}];
\addplot[draw=none,name path=kx]coordinates{(-1,1)(1,1)}node[pos=0.25,below]{$R_1$}node[pos=0.75,above]{$R_2$};
\addplot[gray,opacity=0.5]fill between[of={kx and kg},soft clip={domain=-1:1}];
\addplot[-latex]coordinates {(-1,0.4)(1,0.4)};
\addplot[-latex]coordinates {(0.25,3)(2,3)};
\addplot[]coordinates{(0.7,0.8)}node[pin={[right]10:{$\int_0^1\int_{-\sqrt{y}}^{\sqrt{y}}\dif x\dif y$}}]{};
\addplot[]coordinates{(1.25,2.2)}node[pin={[right]-10:{$\int_1^4\int_{y-2}^{\sqrt{y}}\dif x\dif y$}}]{};
\end{axis}
\end{tikzpicture}
\caption{}
\end{subfigure}\hfill
\begin{subfigure}{0.45\textwidth}
\centering
\begin{tikzpicture}[font=\small,declare function={f(\x)=\x+2;g(\x)=\x^2;}]
\pgfmathsetmacro{\a}{-1}
\pgfmathsetmacro{\b}{2.1}
\pgfmathsetmacro{\c}{0.5}
\begin{axis}[clip=false,small,axis lines=middle,enlargelimits=true, xlabel={$x$}, ylabel={$y$}, zlabel={$z$}, xtick={2},ytick={4},ztick={\empty},xlabel style={anchor=north}, ylabel style={anchor=west},zlabel style={anchor=south}]
\addplot[name path=kf,domain=\a:\b]{f(x)}node[pos=0.8,left,yshift=1ex]{$y=x+2$};
\addplot[name path=kg,domain=\a:\b]{g(x)}node[above]{$y=x^2$}node[pos=0,left]{$(-1,1)$};
\addplot[]coordinates {(2,4)}node[right,yshift=-1ex]{$(2,4)$};
\addplot[lgray,opacity=0.5]fill between[of={kf and kg},soft clip={domain=-1:2}];
\addplot[-latex]coordinates {(0.5,-0.25)(0.5,3)};
\addplot[]coordinates{(1.25,2.2)}node[pin={[right]-10:{$\int_{-1}^{2}\int_{x^2}^{x+2}\dif y\dif x$}}]{};
\end{axis}
\end{tikzpicture}
\caption{}
\end{subfigure}
\caption{(ا) اگر ہم پہلے \عددی{x} کے لحاظ سے تکمل لیں تب رقبے کے حصول کے لئے دو  تکملات کا مجموعہ درکار ہو گا۔ (ب) البتہ پہلے \عددی{y} کے لحاظ سے تکمل لیتے ہوئے  صرف ایک تکمل سے حاصل ہو گا۔}
\label{شکل_مثال_بالکثرت_قطع_مکافی_لکیر_بیچ}
\end{figure}

\جزوحصہء{اوسط قیمت}
بند وقفہ پر  قابل تکمل واحد متغیر تفاعل   کی اوسط قیمت اس وقفہ پر تفاعل کا تکمل تقسیم  لمبائی وقفہ ہو گی۔ بند اور محدود خطہ پر، جس کا رقبہ قابل ناپ ہو،    معین قابل تکمل دو متغیر تفاعل کی اوسط قیمت اس خطہ پر تفاعل کا تکمل تقسیم خطہ کا رقبہ ہو گی۔ اگر خطہ \عددی{R} اور تفاعل \عددی{f} ہوں تب درج ذیل  ہو گا۔
\begin{align}
\text{\RL{\عددی{R} پر \عددی{f} کی\موٹا{ اوسط قیمت}}}=\frac{1}{\text{\RL{\عددی{R} کا رقبہ}}}\iint\limits_{R}f\dif S
\end{align}
اگر  خطہ  \عددی{R}  پر باریک (پتلی) چادر کی  کثافت رقبہ  \عددی{f} ہو تب \عددی{R} پر \عددی{f} کے دوہرا تکمل کو \عددی{R} کے رقبہ سے تقسیم کرنے سے اس چادر کی اوسط کثافت حاصل ہو گی جس کی اکائی  کمیت فی اکائی رقبہ  ہو گی۔ اگر نقطہ \عددی{(x,y)} سے مقررہ نقطہ \عددی{N} تک فاصلہ \عددی{f(x,y)} ہو تب \عددی{R} پر \عددی{f} کی اوسط قیمت، \عددی{N} سے \عددی{R} کے نقاط کا اوسط فاصلہ ہو گا۔ 

\ابتدا{مثال}
مستطیل \عددی{R:\,0\le x\le \pi,\,0\le y\le 1} پر \عددی{f(x,y)=x\cos xy} کی اوسط قیمت تلاش کریں۔

حل:\quad
خطہ \عددی{R} پر \عددی{f} کا تکمل 
\begin{align*}
\int_0^{\pi}\int_0^1 x\cos xy\dif x\dif y&=\int_0^{\pi}\big[\sin xy\big]_{y=0}^{y=1}\dif x\\
&=\int_0^{\pi}(\sin x-0)\dif x=-\cos x\big]_{0}^{\pi}1+1=2
\end{align*}
ہو گا جبکہ مستطیل \عددی{R}  کا رقبہ \عددی{\pi} ہے۔یوں \عددی{R} پر \عددی{f} کی اوسط قیمت \عددی{\tfrac{2}{\pi}} ہو گی۔
\انتہا{مثال}
%===========================

\جزوحصہء{مراکز کمیت کے  معیار اثر اول اور دوم}
باریک چادروں کی کمیت اور معیار اثر تلاش کرنے کے لئے ہم  باب \حوالہ{باب_تکمل_کا_استعمال} کے کلیات کی طرح کلیات استعمال کرتے ہیں۔ فرق صرف اتنا ہے کہ دوہرا تکمل کی بنا اب ہم زیادہ  اشکال اور  کثافتی تفاعل کو  عمل میں لا  سکتے ہیں۔   جدول میں ان کلیات درج ذیل ہیں۔

\موٹا{مستوی \عددی{xy} میں باریک چادر کی \اصطلاح{کمیت}\فرہنگ{کمیت} ،   \اصطلاح{معیار اثر اول}\فرہنگ{معیار اثر!اول}\حاشیہب{first moment}\فرہنگ{moment!first}،\اصطلاح{ معیار اثر دوم}\فرہنگ{معیار اثر!دوم}\حاشیہب{second moment}\فرہنگ{moment!second} اور\اصطلاح{ رداس دوار}\فرہنگ{دوار!رداس}\فرہنگ{رداس!دوار}\حاشیہب{radius of gyration}\فرہنگ{gyration!radius}  کے کلیات}\\
\begin{description}
\item{کثافت:}\quad
$\delta(x,y)$
\item{کمیت:}\quad
$M=\iint \delta(x,y)\dif S$
\item{معیار اثر اول:}\quad
$M_x=\iint y\delta(x,y)\dif S,\quad M_y=\iint x\delta(x,y)\dif S$
\item{مرکز کمیت:}\quad
$\bar{x}=\frac{M_y}{M},\quad \bar{y}=\frac{M_x}{M}$
\item{معیار اثر دوم (جمودی معیار اثر):}\quad
\begin{align*}
I_x&=\iint y^2\delta(x,y)\dif S&&\text{\RL{بلحاظ محور \عددی{x}}}\\
I_y&=\int x^2\delta(x,y)\dif S&&\text{\RL{بلحاظ محور \عددی{y}}}\\
I_L&=\iint r^2(x,y)\delta(x,y)\dif S,\quad \text{\small\RL{{(جہاں \عددی{L} سے \عددی{(x,y)} کا فاصلہ \عددی{r(x,y)} ہے)}}}&&\text{\RL{بلحاظ  خط  \عددی{L}}}\\
I_0&=\int (x^2+y^2)\delta(x,y)\dif S=I_x+I_y&&\text{\RL{(قطبی معیار اثر) بلحاظ مبدا}}
\end{align*}
\item{رداس دوار:}\quad
\begin{align*}
R_x&=\sqrt{\frac{I_x}{M}}&&\text{\RL{بلحاظ محور \عددی{x}}}\\
R_y&=\sqrt{\frac{I_y}{M}}&&\text{\RL{بلحاظ محور \عددی{y}}}\\
R_0&=\sqrt{\frac{I_0}{M}}&&\text{\RL{بلحاظ مبدا}}\\
\end{align*}
\end{description}
ان کلیات کا استعمال مثالوں کی مدد سے سمجھایا جائے گا۔

معیار اثر اول \عددی{M_x} اور   \عددی{M_y} اور معیار اثر دوم  (جمودی معیار اثر) \عددی{I_x} اور \عددی{I_y}  میں  ریاضیاتی فرق یہ ہے کہ معیار اثر دور  "بیرم کے بازوؤں" کے فاصلوں،      \عددی{x} اور \عددی{y}،      کا مربع لیتا ہے۔ 

معیار اثر \عددی{I_0} کو \اصطلاح{قطبی معیار اثر}\فرہنگ{معیار اثر!قطبی}\حاشیہب{polar moment}\فرہنگ{moment!polar} بھی کہتے ہیں۔ کمیتی کثافت \عددی{\delta(x,y)}  (کمیت فی اکائی رقبہ)   ضرب \عددی{x^2+y^2}، جو  نمائندہ نقطہ  \عددی{(x,y)}  سے مبدا تک فاصلہ ہے،  کا تکمل قطبی معیار اثر کہلاتا ہے۔ چونکہ \عددی{I_0=I_x+I_y}  ہے لہٰذا ان میں سے کسی دو کے حصول کے بعد تیسرے کو اس تعلق سے اخذ کیا جا سکتا ہے۔ (معیار اثر \عددی{I_0}  بعض اوقات \عددی{I_z} لکھا جاتا  اور بلحاظ محور \عددی{z} معیار اثر کہلاتا ہے۔ تب تماثل \عددی{I_z=I_x+I_y}  \اصطلاح{مسئلہ عمودی محور}\فرہنگ{مسئلہ!عمودی محور}\حاشیہب{Perpendicular Axis Theorem}\فرہنگ{theorem!perpendicular axis} کہلاتا ہے۔)

\اصطلاح{رداس دوار}  \عددی{R_x} کی تعریف درج ذیل مساوات  ہے۔
\begin{align*}
I_x=MR_x^2
\end{align*}
رداس دوار ہمیں بتاتا ہے کہ محور \عددی{x}  کتنا دور   پوری چادر کی کمیت منجمد کرتے ہوئے  وہی \عددی{I_x} حاصل ہو گا۔ رداس دوار استعمال کرتے ہوئے ہم معیار اثر کو کمیت اور لمبائی کی صورت میں لکھ پاتے ہیں۔ رداس \عددی{R_y} اور \عددی{R_0} کی تعریفات بھی اسی طرح ہیں:
\begin{align*}
I_y=MR_y^2,\quad I_0=MR_0^2
\end{align*}
ہم ان تعریفی مساوات کے جذر سے \عددی{R_x}، \عددی{R_y} اور \عددی{R_0} کے    کلیات لکھتے ہیں۔

ہمیں معیار اثر میں کیا  دلچسپی ہے؟ ایک جسم کا  پہلا معیار ا اثر  ہمیں ثقلی میدان میں اس جسم کے  توازن  اور مختلف محوروں کے لحاظ  سے اس کی  قوت مروڑ کے بارے میں معلومات فراہم کرتا ہے۔  اب اگر یہ جسم  گھومتا ہوا دھرا ہو تب ہمیں اس   میں ذخیرہ توانائی جاننے میں زیادہ دلچسپی ہو گی تا کہ ہم جان سکیں کہ اس کو روکنے کے لئے یا اس کو کسی خاص زاویاتی رفتار تک پہنچانے میں کتنی توانائی درکار ہو گی۔ایسی صورت میں معیار اثر دوم  استعمال ہو گا۔

اس دھرا کو متعدد چھوٹی کمیتوں \عددی{\Delta m_k} میں  تقسیم کریں  اور  گھومنے کے محور سے \عددی{k} ویں کمیتی ٹکڑے کے  فاصلہ کو  \عددی{r_k} سے ظاہر کریں۔ اگر دھرا کی زاویاتی سمتی  رفتار \عددی{\omega=\tfrac{\dif\theta}{\dif t}}  ریڈیئن فی سیکنڈ ہو، تب اس ٹکڑے کا کمیتی مرکز اپنے مدار میں  خطی رفتار
\begin{align*}
v_k=\frac{\dif}{\dif t}(r_k\theta)=r_k\frac{\dif \theta}{\dif t}=r_k\omega
\end{align*}
سے  حرکت کرے گا۔اس ٹکڑے کی حرکی توانائی تخمیناً
\begin{align}
\frac{1}{2}\Delta m_kv_k^2=\frac{1}{2}\Delta m_k(r_k\omega)^2=\frac{1}{2}\omega^2r_k^2\Delta m_k
\end{align}
ہو گی۔دھرا کی حرکی توانائی تخمیناً
\begin{align}
\sum \frac{1}{2}\omega^2r_k^2\Delta m_k
\end{align}
ہو گی۔دھرا کو زیادہ سے زیادہ ٹکڑوں میں تقسیم کرنے سے اس مجموعہ کی قیمت ایک حد تک پہنچتی ہے جسے تکمل
\begin{align}\label{مساوات_بالکثرت_حرکی_توانائی_تعریف_الف}
\text{\RL{دھرا کی حرکی توانائی}}=\int\frac{1}{2}\omega^2r^2\dif m=\frac{1}{2}\omega^2\int r^2\dif m
\end{align}
لکھا جا سکتا ہے۔ جزو
\begin{align}
I=\int r^2\dif m
\end{align}
درحقیقت گھومنے کے محور کے لحاظ  سے دھرے کا جمودی  معیار اثر ہے جس کو استعمال کرتے ہوئے مساوات \حوالہ{مساوات_بالکثرت_حرکی_توانائی_تعریف_الف} درج ذیل صورت اختیار کرتی ہے۔
\begin{align}
\text{\RL{دھرا کی حرکی توانائی}}=\frac{1}{2}I\omega^2
\end{align}

ایک دھرا،  جس کا جمودی معیار اثر \عددی{I} ہو،  کو \عددی{\omega} زاویاتی سمتی رفتار  تک پہنچانے  کے لئے \عددی{\tfrac{1}{2}I\omega^2} حرکی  توانائی درکار ہو گی اور اس رفتار پر چلتے ہوئے دھرا کو روکنے کے لئے  ہمیں دھرا سے اتنی ہی  حرکی توانائی   نکالنی ہو گی۔ کمیت \عددی{m} کی گاڑی کو سمتی رفتار \عددی{v} تک پہنچانے کے لئے اس کو \عددی{\tfrac{1}{2}mv^2}  حرکی توانائی درکار ہو گی اور اس کو روکنے کے لئے  اس  گاڑی سے اتنی ہی حرکی  توانائی نکالنی ہو گی۔ دھرے کا جمودی معیار اثر  گاڑی کی کمیت کا مماثل ہے۔ گاڑی کی رفتار تیز یا کم کرنے  کو   گاڑی  کی کمیت مشکل بناتی ہے۔اسی طرح دھرے کی زاویاتی رفتار تیز یا کم کرنے  کو  دھرے کا جمودی معیار اثر مشکل بناتا ہے۔ جمودی معیار اثر کمیت کے علاوہ کمیت کی تقسیم پر بھی منحصر ہوتا ہے۔ 


