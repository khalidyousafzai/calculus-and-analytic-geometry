\حصہ{جدول تکمل اور کمپیوٹر}
جیسا آپ جانتے ہیں تکمل کے بنیادی طریقے بدل اور تکمل بالحصص ہیں۔ان طریقوں سے  ہم انجانے تکمل کو جانے پہچانے تکمل میں بدلتے ہیں جس کو ہم حل کرنا جانتے ہیں یا جس کو جدول سے دیکھا جا سکتا ہے۔ ان جدول میں تکمل کہاں سے آتے ہیں؟ جدول میں تکمل بھی بدل یا تکمل بالحصص سے حاصل کیے گئے ہوتے ہیں۔ ہم ان تمام کو خود حاصل کر سکتے ہیں لیکن جدول کی موجودگی ہمیں بار بار ایک ہی طرز کے تکمل حاصل کرنے سے چھٹکارا فراہم کرتے ہیں۔ جب کوئی تکمل جدول میں پایا جاتا ہو یا الجبرا، تکونیات، بدل اور احصاء کی استعمال سے اس کو جدول میں درج کسی تکمل کی صورت میں لانا ممکن ہو، تب ہم تکمل کا حل جدول سے پڑھ سکتے ہیں۔ اس حصہ کی مثالوں اور سوالات میں جدول میں درج کلیات کو اخذ کرنا اور ان کا استعمال سکھایا جائے گا۔ یہاں استعمال پر زور دیا جائے گا۔ کتاب کے آخر میں کلیات کو مستقل \عددی{a,b,c,m,n}  وغیرہ کی صورت میں لکھ گیا ہے۔ یہ مستقل عموماً حقیقی اعداد ہوں گے جو غیر عدد صحیح ہو سکتے ہیں۔ جہاں ان مستقل پر شرائط مسلط ہو، وہاں اس کا ذکر کیا گیا ہے۔ مثال کے طور پر کلیہ \عددی{5} میں \عددی{n\ne -1} ہونا ضروری ہے جبکہ کلیہ \عددی{11} میں \عددی{n\ne -2} ہونا ضروری ہے۔ 

ان کلیات میں  مستقل وہ قیمت اختیار نہیں کر سکتے ہیں جن کی بنا صفر سے تقسیم کرنا پڑے یا منفی اعداد کا جفت جذر لینا پڑے۔ مثال کے طور پر کلیہ \عددی{8} میں \عددی{a\ne 0} ہو گا جبکہ کلیہ \عددی{13}-الف صرف اس صورت قابل استعمال ہو گا جب \عددی{b} منفی ہو۔ 

بہت سارے غیر قطعی تکمل کو کمپیوٹر کی مدد سے بھی حل کیا جا سکتا ہے جہاں تکمل کو کسی خاص صورت میں لکھنے کی ضرورت پیش نہیں آتی ہے۔ کمپیوٹر الجبرا پر اس حصہ کے آخر میں غور کیا جائے گا۔

\جزوحصہء{جدول کی مدد سے تکمل}
\ابتدا{مثال}
تکمل \عددی{\int x(2x+5)^{-1}\dif x} حل کریں۔

حل:\quad
ہم کلیہ \عددی{8} استعمال کرتے ہیں (نا کہ کلیہ \عددی{7} جہاں \عددی{n\ne -1} ہونا ضروری ہے۔)
\begin{align*}
\int x(ax+b)^{-1}\dif x=\frac{x}{a}-\frac{b}{a^2}\ln\abs{ax+b}+C
\end{align*}
یوں \عددی{a=2} اور \عددی{b=5} کی صورت میں درج ذیل ہو گا۔
\begin{align*}
\int x(2x+5)^{-1}\dif x=\frac{x}{2}-\frac{5}{4}\ln\abs{2x+5}+C
\end{align*}
\انتہا{مثال}
%===============
\ابتدا{مثال}
تکمل \عددی{\int\frac{\dif x}{x\sqrt{2x+4}}} حل کریں۔

حل:\quad
ہم کلیہ \عددی{13}-ب استعمال کرتے ہیں۔
\begin{align*}
\int\frac{\dif x}{x\sqrt{ax+b}}&=\frac{1}{\sqrt{b}}\ln\abs{\frac{\sqrt{ax+b}-\sqrt{b}}{\sqrt{ax+b}+\sqrt{b}}}+C&&\text{\RL{اگر $b>0$ ہو}}
\end{align*}
یوں \عددی{a=2} اور \عددی{b=4} کی صورت میں درج ذیل ہو گا۔
\begin{align*}
\int\frac{\dif x}{x\sqrt{2x+4}}&=\frac{1}{\sqrt{4}}\ln\abs{\frac{\sqrt{2x+4}-\sqrt{4}}{\sqrt{2x+4}+\sqrt{4}}}+C\\
&=\frac{1}{2}\ln\abs{\frac{\sqrt{2x+4}-2}{\sqrt{2x+4}+2}}+C
\end{align*}
کلیہ \عددی{13}-الف یہاں قابل استعمال نہیں ہو گا چونکہ اس میں \عددی{b<0} ضروری ہے، البتہ اگلی مثال میں یہ کارآمد ہو گا۔
\انتہا{مثال}
%====================
\ابتدا{مثال}\شناخت{مثال_طریقہ_تکمل_استعمال_کلیات_تیسرا}
تکمل \عددی{\int\frac{\dif x}{x\sqrt{2x-4}}} حل کریں۔

حل:\quad
ہم کلیہ \عددی{13}-الف استعمال کرتے ہیں۔
\begin{align*}
\int\frac{\dif x}{x\sqrt{ax-b}}=\frac{2}{\sqrt{b}}\tan^{-1}\sqrt{\frac{ax-b}{b}}+C
\end{align*}
یوں \عددی{a=2} اور \عددی{b=4} لیتے ہوئے درج ذیل ہو گا۔
\begin{align*}
\int \frac{\dif x}{x\sqrt{2x-4}}=\frac{2}{\sqrt{4}}\tan^{-1}\sqrt{\frac{2x-4}{4}}+C=\tan^{-1}\sqrt{\frac{x-2}{2}}+C
\end{align*}
\انتہا{مثال}
%=============
\ابتدا{مثال}
تکمل \عددی{\int \frac{\dif x}{x^2\sqrt{2x-4}}} حل کریں۔

حل:\quad
ہم کلیہ \عددی{15} سے شروع کرتے ہیں۔
\begin{align*}
\int\frac{\dif x}{x^2\sqrt{ax+b}}=-\frac{\sqrt{ax+b}}{bx}-\frac{a}{2b}\int\frac{\dif x}{x\sqrt{ax+b}}+C
\end{align*}
یوں \عددی{a=2} اور \عددی{b=-4} لیتے ہوئے
\begin{align*}
\int\frac{\dif x}{x^2\sqrt{2x-4}}=-\frac{\sqrt{2x-4}}{-4x}+\frac{2}{2\cdot 4}\int\frac{\dif x}{x\sqrt{2x-4}}+C
\end{align*}
ملتا ہے۔ اب ہم کلیہ \عددی{13}-الف استعمال کرتے ہوئے دائیں ہاتھ تکمل حل کرتے ہیں (مثال \حوالہ{مثال_طریقہ_تکمل_استعمال_کلیات_تیسرا} سے رجوع کریں):
\begin{align*}
\int\frac{\dif x}{x^2\sqrt{2x-4}}=\frac{\sqrt{2x-4}}{4x}+\frac{1}{4}\tan^{-1}\sqrt{\frac{x-2}{2}}+C
\end{align*}
\انتہا{مثال}
%=================
\ابتدا{مثال}
تکمل \عددی{\int x\sin^{-1}x\dif x} حل کریں۔

حل:\quad
ہم کلیہ \عددی{99} استعمال کرتے ہیں۔
\begin{align*}
\int x^n\sin^{-1}ax\dif x=\frac{x^{n+1}}{n+1}\sin^{-1}ax-\frac{a}{n+1}\int\frac{x^{n+1}\dif x}{\sqrt{1-a^2x^2}},\quad n\ne -1
\end{align*}
یوں \عددی{n=1} اور \عددی{a=1} لے کر
\begin{align*}
\int x\sin^{-1}x\dif x=\frac{x^2}{2}\sin^{-1}x-\frac{1}{2}\int\frac{x^2\dif x}{\sqrt{1-x^2}}
\end{align*}
ہو گا۔ دائیں ہاتھ تکمل جدول میں کلیہ \عددی{33} ہے:
\begin{align*}
\int \frac{x^2}{\sqrt{a^2-x^2}}\dif x=\frac{a^2}{2}\sin^{-1}\frac{x}{a}-\frac{1}{2}x\sqrt{a^2-x^2}+C
\end{align*}
اب \عددی{a=1} کے لئے درج ذیل ہو گا۔
\begin{align*}
\int \frac{x^2}{\sqrt{1-x^2}}\dif x=\frac{1}{2}\sin^{-1}x-\frac{1}{2}x\sqrt{1-x^2}+C
\end{align*}
یوں مجموعی حل درج ذیل ہو گا۔
\begin{align*}
\int x\sin^{-1}x\dif x&=\frac{x^2}{2}\sin^{-1}x-\frac{1}{2}\big(\frac{1}{2}\sin^{-1}x-\frac{1}{2}x\sqrt{1-x^2}\big)+C'\\
&=\big(\frac{x^2}{2}-\frac{1}{4}\big)\sin^{-1}x+\frac{1}{4}x\sqrt{1-x^2}+C'
\end{align*}
\انتہا{مثال}
%=======================

\جزوحصہء{کلیات تخفیف}
دہراتے ہوئے تکمل بالحصص میں درج ذیل صورت کے کلیات مددگار ثابت ہوتے ہیں جنہیں \اصطلاح{کلیات تخفیف}\فرہنگ{کلیات!تخفیف}\فرہنگ{تخفیف!کلیات}\حاشیہب{reduction formulae}\فرہنگ{reduction formulae} کہتے ہیں۔
\begin{align}
\int \tan^n x\dif x&=\frac{1}{n-1}\tan^{n-1}x-\int \tan^{n-2}x\dif x\label{مساوات_طریقہ_کلیات_تخفیف_الف}\\
\int(\ln x)^n\dif x&=x(\ln x)^n-n\int(\ln x)^{n-1}\label{مساوات_طریقہ_کلیات_تخفیف_ب}\\
\int\sin^nx\cos^mx\dif x&=-\frac{\sin^{n-1}x\cos^{m+1}x}{m+n}\dif x\nonumber\\
&\phantom{=}+\frac{n-1}{m+n}\int\sin^{n-2}x\cos^mx\dif x,\quad (n\ne -m)\label{مساوات_طریقہ_کلیات_تخفیف_پ}
\end{align}
کلیات تخفیف کسی تفاعل کے طاقت کے تکمل کو اسی طرز کے تکمل جس میں تفاعل کی طاقت کم ہو سے تبدیل کرتا ہے۔ طاقت کی تخفیف کی بنا انہیں کلیات تخفیف کہتے ہیں۔ کلیات تخفیف بار بار استعمال کرتے ہوئے آخر کار تکمل میں تفاعل کی طاقت اتنی کم ہو جاتی ہے کہ تکمل با آسانی حل ہوتا ہے۔

\ابتدا{مثال}
تکمل \عددی{\int \tan^5x\dif x} حل کریں۔

حل:\quad
ہم \عددی{n=5} لیتے ہوئے کلیہ \عددی{1} استعمال کرتے ہیں۔
\begin{align*}
\int\tan^5x\dif x=\frac{1}{4}\tan^4x-\int\tan^3x\dif x
\end{align*}
ہم \عددی{n=3} لیتے ہوئے  کلیہ \عددی{1} دوبارہ استعمال کرتے ہیں۔
\begin{align*}
\int\tan^3x\dif x=\frac{1}{2}\tan^2x-\int \tan x\dif x=\frac{1}{2}\tan^2x+\ln\abs{\cos x}+C
\end{align*}
یوں مکمل نتیجہ درج ذیل ہو گا۔
\begin{align*}
\int^5x \dif x=\frac{1}{4}\tan^4x-\frac{1}{2}\tan^2x-\ln\abs{\cos x}+C
\end{align*}
\انتہا{مثال}
%==========
کلیات تخفیف کو تکمل بالحصص سے حاصل کیا جاتا ہے۔

\ابتدا{مثال}\ترچھا{کلیات تخفیف کا حصول}\\
کسی بھی مثبت عدد صحیح \عددی{n} کے لئے درج ذیل کی تصدیق کریں۔
\begin{align*}
\int (\ln x)^n\dif x=x(\ln x)^n-n\int (\ln x)^{n-1}\dif x
\end{align*}
حل:\quad
ہم تکمل بالحصص کے کلیہ
\begin{align*}
\int u\dif v=uv-\int v\dif u
\end{align*}
میں
\begin{align*}
u=(\ln x)^n,\quad \dif u=n(\ln x)^{n-1}\frac{\dif x}{x},\quad \dif v=\dif x,\quad v=x
\end{align*}
لے کر درج ذیل حاصل کرتے ہیں۔
\begin{align*}
\int (\ln x)^n\dif x=x(\ln x)^n-n\int (\ln x)^{n-1}\dif x
\end{align*}
\انتہا{مثال}
%===================
بعض اوقات دو کلیات تخفیف کا استعمال ضروری ہو گا۔

\ابتدا{مثال}
تکمل \عددی{\int\sin^2x\cos^3x\dif x} حل کریں۔

حل الف:\quad
ہم کلیہ \عددی{3} میں \عددی{n=2} اور \عددی{m=3} لے کر درج ذیل حاصل کرتے ہیں۔
\begin{align*}
\int \sin^2x\cos^3x\dif x&=-\frac{\sin x\cos^4x}{2+3}+\frac{1}{2+3}\int\sin^0x\cos^3x\dif x\\
&=-\frac{\sin x\cos^4x}{5}+\frac{1}{5}\int\cos^3x\dif x
\end{align*}
ہم باقی تکمل کو کلیہ \عددی{61} سے حل کر سکتے ہیں۔
\begin{align*}
\int\cos^nax\dif x=\frac{\cos^{n-1}ax\sin ax}{na}+\frac{n-1}{n}\int\cos^{n-2}ax\dif x
\end{align*}
یوں \عددی{n=3} اور \عددی{a=1} لیتے ہوئے
 \begin{align*}
\int\cos^3x\dif x&=\frac{\cos^2x\sin x}{3}+\frac{2}{3}\int\cos x\dif x\\
&=\frac{\cos^2x\sin x}{3}+\frac{2}{3}\sin x+C
\end{align*}
حاصل ہو گا۔مجموعی نتیجہ درج ذیل ہو گا۔
\begin{align*}
\int\sin^2x\cos^3x\dif x &=-\frac{\sin x\cos^4x}{5}+\frac{1}{5}\big(\frac{\cos^2x\sin x}{3}+\frac{2}{3}\sin x+C\big)\\
&=-\frac{\sin x\cos^4 x}{5}+\frac{\cos^2x\sin x}{15}+\frac{2}{15}\sin x+C'
\end{align*}

حل ب:\quad
مساوات \حوالہ{مساوات_طریقہ_کلیات_تخفیف_پ} جدول میں کلیہ \عددی{68} ہے ، مگر ہم کلیہ \عددی{69} بھی استعمال کر سکتے ہیں جس میں \عددی{a=1} لیتے ہوئے
\begin{align*}
\int \sin^nx\cos^mx\dif x=\frac{\sin^{n+1}x\cos^{m-1}x}{m+n}+\frac{m-1}{m+n}\int \sin^nx\cos^{m-2}x\dif x
\end{align*}
لکھا جائے گا۔ اب \عددی{n=2} اور \عددی{m=3} لیتے ہوئے درج ذیل ملتا ہے۔
\begin{align*}
\int\sin^2x\cos^3x\dif x&=\frac{\sin^3x\cos^2x}{5}+\frac{2}{5}\int\sin^2x\cos x\dif x\\
&=\frac{\sin^3x\cos^2x}{5}+\frac{2}{5}\big(\frac{\sin^3x}{3}\big)+C\\
&=\frac{\sin^3x\cos^2x}{5}+\frac{2}{15}\sin^3x+C
\end{align*} 
آپ نے دیکھا کہ کلیہ \عددی{69} زیادہ جلدی نتیجہ فراہم کرتا ہے۔ عموماً قبل از وقت یہ جاننا ممکن نہیں ہوتا ہے کہ کونسا کلیہ زیادہ جلدی نتیجہ دیگا گا۔ اس پر وقت ضائع نہ کریں۔ جو بھی کلیہ قابل استعمال نظر آئے، اس کو فوراً استعمال کریں۔

آپ نے یہ بھی دیکھا ہو گا کہ کلیہ \عددی{68} اور کلیہ \عددی{69} کے نتائج مختلف نظر آتے ہیں۔ تکونیاتی تکمل میں عموماً ایسا ہی ہو گا۔ آپ فکر نہ کریں چونکہ ایسے نتائج درحقیقت بالکل ایک دوسرے جیسے ہوں گے۔
\انتہا{مثال}
%=============== 

\جزوحصہء{غیر بنیادی تکمل}
وہ الٹ تفرق جنہیں بنیادی تفاعل (وہ تفاعل جن پر اب تک غور کیا گیا) کی صورت میں لکھنا ممکن نہ ہو \اصطلاح{غیر بنیادی}\فرہنگ{تکمل!غیر بنیادی}\فرہنگ{غیر بنیادی تکمل}\حاشیہب{nonelementary}\فرہنگ{nonelementary} تکمل کہلاتے ہیں۔ غیر بنیادی تکمل کا حل لامتناہی سلسلہ  یا اعدادی تراکیب سے حاصل ہو گا۔ اعدادی تراکیب سے حل ہونے والے تکمل میں تفاعل خلل
\begin{align*}
\erf(x)=\frac{2}{\pi}\int_0^x e^{-t^2}\dif t
\end{align*}
اور درج ذیل قسم کے تکمل شامل ہیں جو انجینئری اور طبیعیات میں پائے جاتے ہیں۔
\begin{align*}
\int\sin x^2\dif x,\quad \int \sqrt{1+x^4}\dif x
\end{align*}
ان کے علاوہ
\begin{align*}
\int\frac{e^x}{x}\dif x,\quad &\int e^{(e^x)}\dif x,\quad \int\frac{1}{\ln x}\dif x,\quad \int \ln(\ln x)\dif x,\quad \int\frac{\sin x}{x}\dif x\\
&\int \sqrt{1-k^2\sin^2x}\dif x,\quad 0<k<1
\end{align*}
بھی غیر بنیادی تکمل ہیں جو بظاہر سادہ نظر آتے ہیں۔ انہیں حل کرنے کی کشش کر کے دیکھیں۔ یہ ثابت کیا جا سکتا ہے کہ مختلف بنیادی تفاعل کو کسی طرح بھی آپس میں جوڑ کر غیر بنیادی تکمل کا حل نہیں لکھا جا سکتا ہے۔وہ تکمل جن میں بدل پر کر کے غیر بنیادی تکمل میں تبدیل کرنا ممکن ہو کے لئے بھی یہی کچھ درست ہو گا۔ چونکہ یہ تمام متکمل استمراری ہیں لہٰذا ان کا الٹ تفرق ضرور پایا جائے گا، لیکن یہ الٹ تفرق غیر بنیادی ہوں گے۔ 

اس باب میں آپ کو کہیں پر بھی غیر بنیادی تکمل حل کرنے کو نہیں کہا جائے گا البتہ حقیقی دنیا میں آپ کو ان سے واسطہ ضرور پڑے گا۔

\حصہء{سوالات}
\موٹا{جدول تکمل کا استعمال}\\
کتاب کے آخر میں دیا گیا جدول تکمل استعمال کرتے ہوئے سوال \حوالہ{سوال_طریقہ_جدول_الف} تا سوال \حوالہ{سوال_طریقہ_جدول_ب} حل کریں۔

\ابتدا{سوال}\شناخت{سوال_طریقہ_جدول_الف}
$\int\frac{\dif x}{x\sqrt{x-3}}$
\انتہا{سوال}
%======================
\ابتدا{سوال}
$\int\frac{\dif x}{x\sqrt{x+4}}$
\انتہا{سوال}
%======================
\ابتدا{سوال}
$\int\frac{x\dif x}{\sqrt{x-2}}$
\انتہا{سوال}
%======================
\ابتدا{سوال}
$\int\frac{x\dif x}{(2x+3)^{3/2}}$
\انتہا{سوال}
%======================
\ابتدا{سوال}
$\int x\sqrt{2x-3}\dif x$
\انتہا{سوال}
%======================
\ابتدا{سوال}
$\int x(7x+5)^{3/2}\dif x$
\انتہا{سوال}
%======================
\ابتدا{سوال}
$\int \frac{\sqrt{9-4x}}{x^2}\dif x$
\انتہا{سوال}
%======================
\ابتدا{سوال}
$\int\frac{\dif x}{x^2\sqrt{4x-9}}$
\انتہا{سوال}
%======================
\ابتدا{سوال}
$\int x\sqrt{4x-x^2}\dif x$
\انتہا{سوال}
%======================
\ابتدا{سوال}
$\int\frac{\sqrt{x-x^2}}{x}\dif x$
\انتہا{سوال}
%======================
\ابتدا{سوال}
$\int\frac{\dif x}{x\sqrt{7+x^2}}$
\انتہا{سوال}
%======================
\ابتدا{سوال}
$\int\frac{\dif x}{x\sqrt{7-x^2}}$
\انتہا{سوال}
%======================
\ابتدا{سوال}
$\int\frac{\sqrt{4-x^2}}{x}\dif x$
\انتہا{سوال}
%======================
\ابتدا{سوال}
$\int\frac{\sqrt{x^2-4}}{x}\dif x$
\انتہا{سوال}
%======================
\ابتدا{سوال}
$\int\sqrt{25-p^2}\dif x$
\انتہا{سوال}
%======================
\ابتدا{سوال}
$\int q^2\sqrt{25-q^2}\dif x$
\انتہا{سوال}
%======================
\ابتدا{سوال}
$\int \frac{r^2}{\sqrt{4-r^2}}\dif r$
\انتہا{سوال}
%======================
\ابتدا{سوال}
$\int \frac{\dif s}{\sqrt{s^2-2}}$
\انتہا{سوال}
%======================
\ابتدا{سوال}
$\int\frac{\dif \theta}{5+4\sin 2\theta}$
\انتہا{سوال}
%======================
\ابتدا{سوال}
$\int \frac{\dif \theta}{4+5\sin 2\theta}$
\انتہا{سوال}
%======================
\ابتدا{سوال}
$\int e^{2t}\cos 3t \dif t$
\انتہا{سوال}
%======================
\ابتدا{سوال}
$\int e^{-3t}\sin 4t\dif t$
\انتہا{سوال}
%======================
\ابتدا{سوال}
$\int x \cos^{-1} x\dif x$
\انتہا{سوال}
%======================
\ابتدا{سوال}
$\int x\sin^{-1}x\dif x$
\انتہا{سوال}
%======================
\ابتدا{سوال}
$\int\frac{\dif s}{(9-s62)^2}$
\انتہا{سوال}
%======================
\ابتدا{سوال}
$\int \frac{\dif \theta}{(2-\theta^2)^2}$
\انتہا{سوال}
%======================
\ابتدا{سوال}
$\int \frac{\sqrt{4x+9}}{x^2}\dif x$
\انتہا{سوال}
%======================
\ابتدا{سوال}
$\int\frac{\sqrt{9x-4}}{x^2}\dif x$
\انتہا{سوال}
%======================
\ابتدا{سوال}
$\int\frac{\sqrt{3t-4}}{t}\dif t$
\انتہا{سوال}
%======================
\ابتدا{سوال}
$\int\frac{\sqrt{3t+9}}{t}\dif t$
\انتہا{سوال}
%======================
\ابتدا{سوال}
$\int x^2\tan^{-1}x\dif x$
\انتہا{سوال}
%======================
\ابتدا{سوال}
$\int \frac{\tan^{-1}x}{x^2}\dif x$
\انتہا{سوال}
%======================
\ابتدا{سوال}
$\int \sin 3x\cos 2x\dif x$
\انتہا{سوال}
%======================
\ابتدا{سوال}
$\int \sin 2x\cos 3x\dif x$
\انتہا{سوال}
%======================
\ابتدا{سوال}
$\int 8\sin 4t\sin\frac{t}{2}\dif t$
\انتہا{سوال}
%======================
\ابتدا{سوال}
$\int \sin\frac{t}{3}\sin \frac{t}{6}\dif t$
\انتہا{سوال}
%======================
\ابتدا{سوال}
$\int \frac{\theta}{3}\cos\frac{\theta}{4}\dif \theta$
\انتہا{سوال}
%======================
\ابتدا{سوال}\شناخت{سوال_طریقہ_جدول_ب}
$\int \cos\frac{\theta}{2}\cos 7\theta \dif \theta$
\انتہا{سوال}
%======================

\موٹا{بدل اور جدول}\\
سوال \حوالہ{سوال_طریقہ_بدل_جدول_الف} تا سوال \حوالہ{سوال_طریقہ_بدل_جدول_ب} میں بدل استعمال کر کے ایسا تکمل حاصل کریں جو جدول میں پایا جاتا ہو۔ اس نئے تکمل کو جدول کی مدد سے حل کریں۔ 

\ابتدا{سوال}\شناخت{سوال_طریقہ_بدل_جدول_الف}
$\int\frac{x^3+x+1}{(x^2+1)^2}\dif x$
\انتہا{سوال}
%===================
\ابتدا{سوال}
$\int\frac{x^2+6x}{(x^2+3)^2}\dif x$
\انتہا{سوال}
%===================
\ابتدا{سوال}
$\int\sin^{-1}\sqrt{x}\dif x$
\انتہا{سوال}
%===================
\ابتدا{سوال}
$\int\frac{\cos^{-1}\sqrt{x}}{\sqrt{x}}\dif x$
\انتہا{سوال}
%===================
\ابتدا{سوال}
$\int\frac{\sqrt{x}}{\sqrt{1-x}}\dif x$
\انتہا{سوال}
%===================
\ابتدا{سوال}
$\int\frac{\sqrt{2-x}}{\sqrt{x}}\dif x$
\انتہا{سوال}
%===================
\ابتدا{سوال}
$\int \cot t\sqrt{1-\sin^2t}\dif t,\quad 0<t<\frac{\pi}{2}$
\انتہا{سوال}
%===================
\ابتدا{سوال}
$\int\frac{\dif t}{\tan t\sqrt{4-\sin^2t}}$
\انتہا{سوال}
%===================
\ابتدا{سوال}
$\int\frac{\dif y}{y\sqrt{3+(\ln y)^2}}$
\انتہا{سوال}
%===================
\ابتدا{سوال}
$\int\frac{\cos\theta\dif\theta}{\sqrt{5+\sin^2\theta}}$
\انتہا{سوال}
%===================
\ابتدا{سوال}
$\int\frac{3\dif r}{\sqrt{9r^2-1}}$
\انتہا{سوال}
%===================
\ابتدا{سوال}
$\int\frac{3\dif y}{\sqrt{1+9y^2}}$
\انتہا{سوال}
%===================
\ابتدا{سوال}
$\int\cos^{-1}\sqrt{x}\dif x$
\انتہا{سوال}
%===================
\ابتدا{سوال}\شناخت{سوال_طریقہ_بدل_جدول_ب}
$\int\tan^{-1}\sqrt{y}\dif y$
\انتہا{سوال}
%===================

\موٹا{کلیات تخفیف کا استعمال}\\
سوال \حوالہ{سوال_طریقہ_کلیات_تخفیف_الف} تا سوال \حوالہ{سوال_طریقہ_کلیات_تخفیف_ب} کو کلیات تخفیف کی مدد سے حل کریں۔

\ابتدا{سوال}\شناخت{سوال_طریقہ_کلیات_تخفیف_الف}
$\int\sin^52x\dif x$
\انتہا{سوال}
%======================
\ابتدا{سوال}
$\int\sin^5\frac{\theta}{2}\dif\theta$
\انتہا{سوال}
%======================
\ابتدا{سوال}
$\int 8\cos^42\pi t\dif t$
\انتہا{سوال}
%======================
\ابتدا{سوال}
$\int 3\cos^53y\dif y$
\انتہا{سوال}
%======================
\ابتدا{سوال}
$\int \sin^22\theta\cos^32\theta\dif\theta$
\انتہا{سوال}
%======================
\ابتدا{سوال}
$\int 9\sin^3\theta\cos^{3/2}\theta\dif \theta$
\انتہا{سوال}
%======================
\ابتدا{سوال}
$\int 2\sin^2t\sec^4t\dif t$
\انتہا{سوال}
%======================
\ابتدا{سوال}
$\int\csc^2y\cos^5y\dif y$
\انتہا{سوال}
%======================
\ابتدا{سوال}
$\int 4\tan^32x\dif x$
\انتہا{سوال}
%======================
\ابتدا{سوال}
$\int \tan^4(x/2)\dif x$
\انتہا{سوال}
%======================
\ابتدا{سوال}
$\int 8\cot^4t\dif t$
\انتہا{سوال}
%======================
\ابتدا{سوال}
$\int 4\cot^32t\dif t$
\انتہا{سوال}
%======================
\ابتدا{سوال}
$\int 2\sec^3\pi x\dif x$
\انتہا{سوال}
%======================
\ابتدا{سوال}
$\int\frac{1}{2}\csc^3\frac{x}{2}\dif x$
\انتہا{سوال}
%======================
\ابتدا{سوال}
$\int 3\sec^43x\dif x$
\انتہا{سوال}
%================
\ابتدا{سوال}
$\csc^4\frac{\theta}{3}\dif \theta$
\انتہا{سوال}
%===========================
\ابتدا{سوال}
$\int\csc^5x\dif x$
\انتہا{سوال}
%======================
\ابتدا{سوال}
$\int\sec^5x\dif x$
\انتہا{سوال}
%======================
\ابتدا{سوال}
$\int 16x^3(\ln x)^2\dif x$
\انتہا{سوال}
%======================
\ابتدا{سوال}\شناخت{سوال_طریقہ_کلیات_تخفیف_ب}
$\int (\ln x)^3\dif x$
\انتہا{سوال}
%======================

\موٹا{قوت نما ضرب $x$ کے طاقت}\\
سوال\حوالہ{سوال_طریقہ_کلیات_جدول_کی_مدد_الف} تا سوال \حوالہ{سوال_طریقہ_کلیات_جدول_کی_مدد_ب} کو جدول کے کلیات \عددی{103} تا \عددی{106} کی مدد سے حل کریں۔

\ابتدا{سوال}\شناخت{سوال_طریقہ_کلیات_جدول_کی_مدد_الف}
$\int xe^{3x}\dif x$
\انتہا{سوال}
%======================
\ابتدا{سوال}
$\int xe^{-2x}\dif x$
\انتہا{سوال}
%======================
\ابتدا{سوال}
$\int x^3e^{x/2}\dif x$
\انتہا{سوال}
%======================
\ابتدا{سوال}
$\int x^2e^{\pi x}\dif x$
\انتہا{سوال}
%======================
\ابتدا{سوال}
$\int x^22^x\dif x$
\انتہا{سوال}
%======================
\ابتدا{سوال}
$\int x^22^{-x}\dif x$
\انتہا{سوال}
%======================
\ابتدا{سوال}
$\int x\pi^x\dif x$
\انتہا{سوال}
%======================
\ابتدا{سوال}\شناخت{سوال_طریقہ_کلیات_جدول_کی_مدد_ب}
$\int x2^{\sqrt{2}x}\dif x$
\انتہا{سوال}
%======================
\موٹا{بدل اور کلیات تخفیف}\\
سوال \حوالہ{سوال_طریقہ_بدل_تخفیف_حل_الف} تا سوال \حوالہ{سوال_طریقہ_بدل_تخفیف_حل_ب}  میں بدل (ممکنہ تکونیاتی) کے بعد کلیات تخفیف استعمال کرتے ہوئے تکمل حل کریں۔

\ابتدا{سوال}\شناخت{سوال_طریقہ_بدل_تخفیف_حل_الف}
$\int e^t\sec^3(e^t-1)\dif t$
\انتہا{سوال}
%======================
\ابتدا{سوال}
$\int \frac{\csc^3\sqrt{\theta}}{\sqrt{\theta}}\dif \theta$
\انتہا{سوال}
%======================
\ابتدا{سوال}
$\int_0^12\sqrt{x^2+1}\dif x$
\انتہا{سوال}
%======================
\ابتدا{سوال}
$\int_0^{\sqrt{3}/2}\frac{\dif y}{(1-y^2)^{5/2}}$
\انتہا{سوال}
%======================
\ابتدا{سوال}
$\int_1^2\frac{(r^2-1)^{3/2}}{r}\dif r$
\انتہا{سوال}
%======================
\ابتدا{سوال}\شناخت{سوال_طریقہ_بدل_تخفیف_حل_ب}
$\int_0^{1/\sqrt{3}}\frac{\dif t}{(t^2+1)^{7/2}}$
\انتہا{سوال}
%======================
\موٹا{ہذلولی تفاعل}\\
سوال \حوالہ{سوال_طریقہ_جدول_تکمل_الف} تا سوال \حوالہ{سوال_طریقہ_جدول_تکمل_ب} کو جدول تکمل کی مدد سے حل کریں۔

\ابتدا{سوال}\شناخت{سوال_طریقہ_جدول_تکمل_الف}
$\int\frac{1}{8}\sinh^53x\dif x$
\انتہا{سوال}
%=======================
\ابتدا{سوال}
$\int\frac{\cosh^4\sqrt{x}}{\sqrt{x}}\dif x$
\انتہا{سوال}
%=======================
\ابتدا{سوال}
$\int x62\cosh 3x\dif x$
\انتہا{سوال}
%=======================
\ابتدا{سوال}
$\int x\sinh 5x \dif x$
\انتہا{سوال}
%=======================
\ابتدا{سوال}
$\int \sech^7x\tanh x\dif x$
\انتہا{سوال}
%=======================
\ابتدا{سوال}\شناخت{سوال_طریقہ_جدول_تکمل_ب}
$\int \csch^32x\coth 2x \dif x$
\انتہا{سوال}
%=======================
\موٹا{نظریہ اور مثالیں}\\
سوال \حوالہ{سوال_طریقہ_کلیات_اخذ_الف} تا سوال \حوالہ{سوال_طریقہ_کلیات_اخذ_ب} میں کتاب کے آخر میں جدول تکمل کے کلیات اخذ کرنے کو کہا گیا ہے۔

\ابتدا{سوال}\شناخت{سوال_طریقہ_کلیات_اخذ_الف}
بدل \عددی{u=ax+b} پر کرتے ہوئے کلیہ \عددی{9} اخذ کرتے ہوئے درج ذیل تکمل حل کریں۔
\begin{align*}
\int\frac{x}{(ax+b)^2}\dif x
\end{align*}
\انتہا{سوال}
%=======================
\ابتدا{سوال}
تکونیاتی بدل پر کرتے ہوئے کلیہ \عددی{17} اخذ کرتے ہوئے درج ذیل تکمل حل کریں۔
\begin{align*}
\int\frac{\dif x}{(a^2+x^2)^2}
\end{align*}
\انتہا{سوال}
%=======================
\ابتدا{سوال}
تکونیاتی بدل  پر کرتے ہوئے کلیہ \عددی{29} اخذ کرتے ہوئے درج ذیل تکمل حل کریں۔
\begin{align*}
\int\sqrt{a^2-x^2}\dif x
\end{align*}
\انتہا{سوال}
%=======================
\ابتدا{سوال}
تکونیاتی بدل  پر کرتے ہوئے کلیہ \عددی{46} اخذ کرتے ہوئے درج ذیل تکمل حل کریں۔
\begin{align*}
\int\frac{\dif x}{x^2\sqrt{x^2-a^2}}
\end{align*}
\انتہا{سوال}
%=======================
\ابتدا{سوال}
درج ذیل کو تکمل بالحصص کی مدد سے حل کرتے ہوئے کلیہ \عددی{80} اخذ کریں۔
\begin{align*}
\int x^n \sin ax \dif x
\end{align*}
\انتہا{سوال}
%=======================
\ابتدا{سوال}
تکونیاتی بدل  پر کرتے ہوئے کلیہ \عددی{110} اخذ کرتے ہوئے درج ذیل تکمل حل کریں۔
\begin{align*}
\int\ x^n (\ln ax)^m\dif x
\end{align*}
\انتہا{سوال}
%=======================
\ابتدا{سوال}
تکونیاتی بدل  پر کرتے ہوئے کلیہ \عددی{99} اخذ کرتے ہوئے درج ذیل تکمل حل کریں۔
\begin{align*}
\int x^n\sin^{-1}ax\dif x
\end{align*}
\انتہا{سوال}
%=======================
\ابتدا{سوال}\شناخت{سوال_طریقہ_کلیات_اخذ_ب}
تکونیاتی بدل  پر کرتے ہوئے کلیہ \عددی{101} اخذ کرتے ہوئے درج ذیل تکمل حل کریں۔
\begin{align*}
\int x^n\tan^{-1} ax\dif x
\end{align*}
\انتہا{سوال}
%=======================
\ابتدا{سوال}
منحنی \عددی{y=\sqrt{x^2+2}\, 0\le x\le \sqrt{2}} کو محور \عددی{x} کے گرد گھما کر سطح طواف پیدا کیا جاتا ہے۔ اس سطح کا رقبہ تلاش کریں۔
\انتہا{سوال}
%=====================
\ابتدا{سوال}
منحنی \عددی{y=x^2,\, 0\le x\le \tfrac{\sqrt{3}}{2}} کی لمبائی تلاش کریں۔
\انتہا{سوال}
%===================
\ابتدا{سوال}
ربع اول میں لکیر \عددی{x=3} اور منحنی \عددی{y=\tfrac{1}{\sqrt{x+1}}} ایک خطہ گھیرتے ہیں۔ اس خطے کا وسطانی مرکز تلاش کریں۔ 
\انتہا{سوال}
%================
\ابتدا{سوال}
ربع اول میں لکیر \عددی{x=3} اور منحنی \عددی{y=\tfrac{36}{2x+3}} کے بیچ مستقل کثافت \عددی{\delta=1} کی چادر پائی جاتی ہے۔ محور \عددی{y} کے لحاظ سے اس خطے کا معیار اثر تلاش کریں۔
\انتہا{سوال}
%=====================
\ابتدا{سوال}
محور \عددی{x} کے گرد منحنی \عددی{y=x^2,\, -1\le x\le 1} گھما کر سطح طواف پیدا کیا جاتا ہے۔جدول تکمل اور کیلکولیٹر کی مدد سے اس کا رقبہ \عددی{2} اعشاریہ درستگی تک تلاش کریں۔
\انتہا{سوال}
%=====================
\ابتدا{سوال}\شناخت{سوال_طریقہ_حوض_تیل}
ایک افقی دائری حوض کا رداس \عددی{r} سنٹی میٹر اور لمبائی \عددی{L} سنٹی میٹر ہے۔ اس میں تیل کی گہرائی \عددی{d} سنٹی میٹر ہے (شکل \حوالہ{شکل_سوال_طریقہ_حوض_تیل})۔ (ا) دکھائیں کہ تیل کا حجم درج ذیل ہے۔
\begin{align*}
H=2L\int_{-r}^{-r+d}\sqrt{r^2-y^2}\dif y
\end{align*}
(ب) اس تکمل کو حل کریں۔
\انتہا{سوال}
%===================
\begin{figure}
\centering
\begin{tikzpicture}
\pgfmathsetmacro{\r}{1}
\pgfmathsetmacro{\L}{4}
\draw[-latex](0,-\r-0.25)--(0,\r+0.5)node[above]{$y$};
\draw([shift={(90:1/4*\r cm and \r cm)}]0,0) arc (90:270:1/4*\r cm and \r cm);
\draw(\L,0) circle (1/4*\r cm and \r cm);
\draw(0,-\r)node[left]{$-r$}--++(\L,0);
\draw(0,\r)node[left]{$r$}--++(\L,0);
\draw[stealth-stealth](0,-\r-0.2)--++(\L,0)node[pos=0.5,below]{$L$};
\draw(\L,-\r-0.1)--++(0,-0.2);
\draw[stealth-stealth](\L+0.4,-\r)--++(0,3/4*\r)node[pos=0.5,right]{$d$};
\end{tikzpicture}
\caption{تیل کا حوض (سوال \حوالہ{سوال_طریقہ_حوض_تیل})}
\label{شکل_سوال_طریقہ_حوض_تیل}
\end{figure}

\ابتدا{سوال}
کسی بھی \عددی{a} اور \عددی{b} کے لئے \عددی{\int_a^b\sqrt{x-x^2}\dif x} کی زیادہ سے زیادہ قیمت کیا ممکن ہے؟ اپنے جواب کی وجہ پیش کریں۔
\انتہا{سوال}
%==============
\ابتدا{سوال}
کسی بھی \عددی{a} اور \عددی{b} کے لئے \عددی{\int_a^b x\sqrt{x-x^2}\dif x} کی زیادہ سے زیادہ قیمت کیا ممکن ہے؟ اپنے جواب کی وجہ پیش کریں۔
\انتہا{سوال}
%==============
\موٹا{کمپیوٹر کا استعمال}\\
 کمپیوٹر میں الجبرا کے کئی پروگرام پائے جاتے ہیں۔ان میں سے ایک پروگرام \اصطلاح{میکسما}\فرہنگ{میکسما}\حاشیہب{maxima}\فرہنگ{maxima} کہلاتا ہے جو نہایت طاقتور پروگرام ہے۔ متغیر \عددی{x} کے تفاعل \عددی{f(x)} کا غیر قطعی تکمل حاصل کرنے کے لئے میکسما میں \عددی{integrate(f,x)} لکھنا ہو گا۔اسی طرح \عددی{a} تا \عددی{b} قطعی تکمل کے لئے \عددی{integrate(f,x,a,b)} لکھنا ہو گا۔ میکسما یا الجبرا کے کسی دوسرے پروگرام کو سیکھ کسےر اس کی مدد  سوال \حوالہ{سوال_طریقہ_کمپیوٹر_تکمل_الف} اور سوال \حوالہ{سوال_طریقہ_کمپیوٹر_تکمل_ب} کو حل کریں۔

\ابتدا{سوال}\شناخت{سوال_طریقہ_کمپیوٹر_تکمل_الف}
\begin{multicols}{3}
\begin{enumerate}[a.]
\item
$\int x\ln x\dif x$
\item
$\int x^2\ln x\dif x$
\item
$\int x^3\ln x\dif x$
\item
آپ کو کیا نقش نظر آتا ہے؟ تکمل \عددی{\int x^4\ln x\dif x} کے کلیہ کی پیش گوئی کریں۔ کمپیوٹر سے اس کی تصدیق کریں۔
\item
تکمل \عددی{\int x^n\ln x\dif x,\, n\ge 1} کا کلیہ کیا ہو گا؟ کمپیوٹر سے اس کی تصدیق کریں۔
\end{enumerate}
\end{multicols}
\انتہا{سوال}
%=====================
\ابتدا{سوال}\شناخت{سوال_طریقہ_کمپیوٹر_تکمل_ب}
\begin{multicols}{3}
\begin{enumerate}[a.]
\item
$\int \frac{\ln x}{x^2}\dif x$
\item
$\int \frac{\ln x}{x^3}\dif x$
\item
$\int \frac{\ln x}{x^4}\dif x$
\item
آپ کو کیا نقش نظر آتا ہے؟ تکمل \عددی{\int \frac{\ln x}{x^2}\dif x} کے کلیہ کی پیش گوئی کریں۔ کمپیوٹر سے اس کی تصدیق کریں۔
\item
تکمل \عددی{\int\tfrac{\ln x}{x^n}\dif x,\, n\ge 2} کا کلیہ کیا ہو گا؟ کمپیوٹر سے اس کی تصدیق کریں۔
\end{enumerate}
\end{multicols}
\انتہا{سوال}
%=====================
\ابتدا{سوال}
(ا) درج ذیل تکمل کو کمپیوٹر کی مدد سے حل کریں، جہاں \عددی{n} اختیاری مستقل ہے۔
\begin{align*}
\int_0^{\pi/2}\frac{\sin^n x}{\sin^n x+\cos^n x}\dif x
\end{align*}
کیا آپ کا کمپیوٹر پروگرام اس کو حل کر پاتا ہے؟   (ب) اختیاری مستقل \عددی{n=1,2,3,5,7} لیتے ہوئے تکمل کی قیمت تلاش کریں۔ نتائج کی پیچیدگی پر تبصرہ کریں۔ (ج) اب \عددی{x=\tfrac{\pi}{2}-u} پر کر کے نئے اور پرانے تکمل کا مجموعہ لیں۔ اب درج ذیل تکمل کی  قیمت کیا ہے؟
\begin{align*}
\int_0^{\pi/2}\frac{\sin^nx}{\sin^nx+\cos^nx}\dif x
\end{align*}
آپ دیکھ سکتے ہیں کہ معمولی سی ریاضیاتی عمل سے تکمل کتنا آسان ہو سکتا ہے۔
\انتہا{سوال}
%=====================

\حصہ{غیر مناسب تکمل}
