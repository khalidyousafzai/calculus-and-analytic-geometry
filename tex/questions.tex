\حصہ{فضا میں خطوط اور مستوی}
اس حصہ میں غیر سمتی ضرب اور سمتی ضرب استعمال کرتے ہوئے فضا میں خطوط، قطعات اور مستوی کے مساوات لکھنا سکھایا جائے گا۔

\جزوحصہء{فضا میں خطوط اور قطعات} 
فرض کریں فضا میں نقطہ \عددی{N_0(x_0,y_0,z_0)} سے گزرتا اور سمتیہ \عددی{\kvec{v}=A\ai+B\aj+C\ak} کے متوازی ایک خط \عددی{L} ہے۔ تب \عددی{L} ان تمام نقطوں \عددی{N(x,y,z)} کا سلسلہ ہو گا جن کے لئے \عددی{\krightharpoonup{N_0N}} سمتیہ \عددی{\kvec{v}} کے متوازی ہو۔ یعنی \عددی{L} پر \عددی{N} صرف اور صرف اس صورت پایا جائے گا جب \عددی{\krightharpoonup{N_0N}} سمتیہ \عددی{\kvec{v}} کا غیر سمتی مضرب ہو۔ 

\موٹا{سمتیہ \عددی{\kvec{v}} کا متوازی خط جو نقطہ \عددی{N_0(x_0,y_0,z_0)} سے گزرتا ہو کی مساوات درج ذیل ہو گی۔}
\begin{align}\label{مساوات_سمتیہ_خط_کی_مساوات_الف}
\krightharpoonup{N_0N}=t\kvec{v},\quad -\infty<t<\infty
\end{align}
مساوات \حوالہ{مساوات_سمتیہ_خط_کی_مساوات_الف} کے دونوں اطراف مطابقتی اجزاء کو ایک دوسرے کے برابر لکھتے ہوئے تین غیر سمتی مساوات حاصل ہوں گے جن میں مقدار معلوم \عددی{t} پایا جائے گا:
\begin{align*}
(x-x_0)\ai+(y-y_0)\aj+(z-z_0)\ak&=t(A\ai+B\aj+C\ak)&&\text{\RL{\small{اتساع  مساوات\حوالہ{مساوات_سمتیہ_خط_کی_مساوات_الف}}}}\\
x-x_0=tA,\quad y-y_0=tB,&\quad z-z_0=tC&&\text{\RL{\small{مطابقتی اجزاء}}}
\end{align*}
ان مساوات سے وقفہ \عددی{-\infty<t<\infty} پر سمتیہ \عددی{\kvec{v}} کے متوازی نقطہ \عددی{N_0(x_0,y_0,z_0)} سے گزرتے خط کی درج ذیل معیاری مقدار معلوم مساوات حاصل ہوتی ہے:
\begin{align}\label{مساوات_سمتیہ_خط_کی_مساوات_ب}
x=x_0+tA,\quad y=y_0+tB,\quad z=z_0+tC,\quad -\infty<t<\infty
\end{align}

\ابتدا{مثال}
سمتیہ \عددی{\kvec{v}=2\ai+4\aj-2\ak} کے متوازی خط جو نقطہ \عددی{(-2,0,4)} سے گزرتا ہو کی مقدار معلوم مساوات تلاش کریں۔

حل:\quad
دی گئی معلومات کو مساوات \حوالہ{مساوات_سمتیہ_خط_کی_مساوات_ب} میں پر کر کے خط کی مقدار معلوم مساوات حاصل کرتے ہیں۔
\begin{align*}
x=-2+2t,\quad y=4t,\quad z=4-2t
\end{align*}

\انتہا{مثال}
%=====================
\ابتدا{مثال}
نقطہ \عددی{N(-3,2,-3)} اور \عددی{Q(1,-1,4)} سے گزرتے ہوئے خط کی مقدار معلوم مساوات تلاش کریں۔

حل:\quad
ان نقطوں کے بیچ خط کا متوازی سمتیہ
\begin{align*}
\krightharpoonup{NQ}=(1-(-3))\ai+(-1-2)\aj+(4-(-3))\ak=4\ai-3\aj+7\ak
\end{align*}
ہے جس کو مساوات \حوالہ{مساوات_سمتیہ_خط_کی_مساوات_ب} میں \عددی{(x_0,y_0,z_0)=(-3,2,-3)}کے ساتھ لیتے ہوئے مقدار معلوم مساوات حاصل کرتے ہیں۔
\begin{align*}
x=-3+4t,\quad y=2-3t,\quad z=-3+7t
\end{align*}
آپ دیکھ سکتے ہیں کہ \عددی{t=0} پر \عددی{x=-3+4(0)=-3}، \عددی{y=2-3(0)=2} اور \عددی{z=-3+7(0)=-3} یعنی ابتدائی نقطہ \عددی{(-3,2,-3)} حاصل ہوتا ہے۔ہم نقطہ \عددی{Q(1,-1,4)} کو بھی ابتدائی نقطہ منتخب کر سکتے ہیں۔ایسا کرنے سے درج ذیل مساوات حاصل ہو گا۔
\begin{align*}
x=1+4t,\quad y=-1-3t,\quad z=4+7t
\end{align*}
اب \عددی{t=0} پر \عددی{x=1}، \عددی{y=-1} اور \عددی{z=4} یعنی ابتدائی نقطہ \عددی{(1,-1,4)} حاصل ہوتا ہے۔درج بالا دونوں مساوات درست ہیں۔ ان کے ابتدائی نقطے مختلف ہیں۔ 
\انتہا{مثال}
%===============
\ابتدا{مثال}
نقطہ \عددی{} اور \عددی{} کے بیچ قطع کی مقدار معلوم مساوات تلاش کریں۔
\انتہا{مثال}
%====================
