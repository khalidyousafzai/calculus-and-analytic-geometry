\حصہ{ترتیب کا حد تلاش کرنے کے مسئلے}
حد پر غور کرتے وقت ہر مرتبہ ارتکاز کو تعریف سے ثابت کرنا مشکل کام ہے۔ خوش قسمتی سے تین مسائل اس عمل سے، کم و بیش ہر  زیادہ تر، چھٹکارا دیتے ہیں۔ پہلا مسئلہ درج ذیل ہے جو حصہ \حوالہ{حصہ_حد_قواعد} میں مسئلہ \حوالہ{مسئلہ_حد_قواعد-الف} کی ایک قسم ہے۔

\ابتدا{مسئلہ}\شناخت{مسئلہ_ترتیب_قواعد_حد_ب}
فرض کریں \عددی{\{a_n\}} اور \عددی{\{b_n\}} حقیقی اعداد کے ترتیب ہیں اور \عددی{A} اور \عددی{B} حقیقی اعداد ہیں۔ اگر \عددی{\lim_{n\to\infty}a_n=A} اور \عددی{   \lim_{n\to\infty}b_n=B} ہوں تب درج ذیل قواعد درست ہوں گے۔
\begin{description}
\item{قاعدہ مجموعہ:}\quad
$\lim\limits_{n\to \infty}(a_n+b_n)=A+B$
\item{قاعدہ فرق:}\quad
$\lim\limits_{n\to\infty}(a_n-b_n)=A-B$
\item{قاعدہ ضرب:}\quad
$\lim\limits_{n\to\infty}(a_n\cdot b_n)=A\cdot B$
\item{قاعدہ ضرب مستقل:}\quad
$\lim\limits_{n\to\infty}(k\cdot b_n)=k\cdot B$\quad
(جہاں \عددی{k} عدد ہے)
\item{قاعدہ حاصل تقسیم:}\quad
$\lim\limits_{n\to\infty}\frac{a_n}{b_n}=\frac{A}{B}$\quad
اگر \عددی{B\ne 0} ہو۔
\end{description}
\انتہا{مسئلہ}
%==========================

\ابتدا{مثال}
ہم مسئلہ \حوالہ{مسئلہ_ترتیب_قواعد_حد_ب} کے ساتھ گزشتہ حصے کی مثال \حوالہ{مثال_ترتیب_تعریف_کی_پرکھ} ملا کر درج ذیل حاصل کرتے ہیں۔
\begin{align*}
\lim_{n\to\infty}\big(-\frac{1}{n}\big)&=1\cdot\lim_{n\to\infty}\frac{1}{n}=-1\cdot0=0\\
\lim_{n\to\infty}\big(\frac{n-1}{n}\big)&=\lim_{n\to\infty}\big(1-\frac{1}{n}\big)=\lim_{n\to\infty} 1-\lim_{n\to\infty}\frac{1}{n}=1-0=1\\
\lim_{n\to\infty}\frac{5}{n^2}&=5\cdot\lim_{n\to\infty}\frac{1}{n}\cdot\lim_{n\to\infty}\frac{1}{n}=5\cdot 0\cdot 0=0\\
\lim_{n\to\infty}\frac{4-7n^6}{n^6+3}&=\lim_{n\to\infty}\frac{\tfrac{4}{n^6}-7}{1+\tfrac{3}{n^6}}=\frac{0-7}{1+0}=-7
\end{align*}
\انتہا{مثال}
%========================

مسئلہ \حوالہ{مسئلہ_ترتیب_قواعد_حد_ب} کے تحت منفرج ترتیب \عددی{\{a_n\}} کو ہر غیر صفر عدد سے ضرب دینے سے منفرج ترتیب ہی حاصل ہوتی ہے۔ مثال کے طور پر اگر اس کے برعکس کسی عدد \عددی{c\ne 0} کے لئے \عددی{\{ca_n\}}  مرتکز ہو تب  مسئلہ \حوالہ{مسئلہ_ترتیب_قواعد_حد_ب} میں قاعدہ ضرب مستقل میں \عددی{k=\tfrac{1}{c}} لیتے ہوئے ہم دیکھتے ہیں درج ذیل ترتیب مرتکز ہو گی۔
\begin{align*}
\left\{\frac{1}{c}\cdot ca_n\right\}=\{a_n\}
\end{align*}
یوں \عددی{\{ca_n\}} صرف اس صورت مرتکز ہو گی جب \عددی{a_n} مرتکز ہو۔ اگر \عددی{\{a_n\}} مرتکز نہ ہو تب \عددی{\{ca_n\}} مرتکز نہیں ہو سکتی ہے۔

اگلا مسئلہ حصہ \حوالہ{حصہ_حد_قواعد} میں مسئلہ بیچ (مسئلہ \حوالہ{مسئلہ_حد_بیچ}) کی ترتیب پر قابل لاگو قسم ہے۔

\ابتدا{مسئلہ}\شناخت{مسئلہ_ترتیب_قواعد_حد_ج}\موٹا{ترتیب کے لئے مسئلہ بیچ}\\
فرض کریں \عددی{\{a_n\}}، \عددی{\{b_n\}} اور \عددی{\{c_n\}} حقیقی اعداد کی ترتیبات ہیں۔ اگر کسی اشاریہ \عددی{N} کے بعد تمام \عددی{n} کے لئے
\begin{align*}
a_n\le b_n\le c_n
\end{align*}
 ہو اور اگر \عددی{\lim_{n\to\infty}a_n=\lim_{n\to\infty}c_n=L} ہو تب \عددی{\lim_{n\to\infty}b_n=L} ہو گا۔
\انتہا{مسئلہ}
%=====================

اگر \عددی{\abs{b_n}\le c_n} ہو اور \عددی{c_n\to 0} ہو تب چونکہ \عددی{-c_n\le b_n\le c_n} ہو گا لہٰذا مسئلہ \حوالہ{مسئلہ_ترتیب_قواعد_حد_ج} کے تحت \عددی{b_n\to 0} ہو گا۔اس حقیقت کو اگلی مثال میں استعمال کیا جائے گا۔

\ابتدا{مثال}
چونکہ \عددی{\tfrac{1}{n}\to0} ہوتا ہے لہٰذا ہم جانتے ہیں کہ درج ذیل ہوں گے۔
\begin{align*}
\text{\RL{(الف)}}&& \frac{\cos n}{n}&\to 0&& \big(\abs{\frac{\cos n}{n}}=\frac{\abs{\cos n}}{n}\le \frac{1}{n}\big)\\
\text{\RL{(ب)}}&&\frac{1}{2^n}&\to 0&&\big(\frac{1}{2^n}\le \frac{1}{n}\big)\\
\text{\RL{(ج)}}&& (-1)^n\frac{1}{n}&\to 0&&\big(\abs{(-1)^n\frac{1}{n}}\le \frac{1}{n}\big)
\end{align*}
\انتہا{مثال}
%=========================

ایک مسئلہ جو کہتا ہے کہ استمراری تفاعل کی مرتکز ترتیب پر اطلاق سے مرتکز ترتیب ملتی ہے مسئلہ \حوالہ{مسئلہ_ترتیب_قواعد_حد_ب} اور مسئلہ \حوالہ{مسئلہ_ترتیب_قواعد_حد_ج} کو وسعت دیتا ہے۔ ہم اس مسئلے کو بغیر ثبوت کے پیش کرتے ہیں۔

\ابتدا{مسئلہ}\شناخت{مسئلہ_ترتیب_قواعد_حد_د}\موٹا{استمراری تفاعل مسئلہ برائے ترتیبات}\\
فرض کریں \عددی{\{a_n\}} حقیقی اعداد کی ترتیب ہے۔ اگر \عددی{a_n\to L} ہو اور \عددی{f} ایسا تفاعل ہو جو \عددی{L} پر استمراری اور تمام \عددی{a_n} پر معین ہو تب \عددی{f(a_n)\to f(L)} ہو گا۔
\انتہا{مسئلہ}
%=========================

\ابتدا{مثال}
دکھائیں کہ \عددی{\sqrt{\tfrac{n+1}{n}}\to 1} ہو گا۔

حل:\quad
ہم جانتے ہیں کہ \عددی{\tfrac{n+1}{n}\to 1} ہے۔ مسئلہ \حوالہ{مسئلہ_ترتیب_قواعد_حد_د} میں \عددی{f(x)=\sqrt{x}} اور \عددی{L=1} لینے سے درج ذیل حاصل ہو گا۔ 
\begin{align*}
\sqrt{\tfrac{n+1}{n}}\to \sqrt{1}=1
\end{align*}
\انتہا{مثال}
%=========================

\جزوحصہء{فنیات}
ترتیب \عددی{\{2^{1/n}\}}:
\quad
کیلکولیٹر میں \عددی{2} لکھ کر بار بار جذر لینے سے کیا حاصل ہو گا؟ آپ دیکھیں گے کہ  جوابات  ایسی ترتیب دیتے ہیں جو \عددی{1} کو مرتکز ہے۔ یہ ترتیب درج ذیل ہے۔ کیلکولیٹر استعمال کر کے اس ترتیب کو خود حاصل کریں۔
\begin{align*}
\begin{array}{rr}
n&2^{1/n}\\
\toprule
2&\num{1.414213562}\\
4&\num{1.189207115}\\
8&\num{1.090507733}\\
64&\num{1.010889286}\\
256&\num{1.002711275}\\
1024&\num{1.000677131}\\
16384&\num{1.000042307}\\
\end{array}
\end{align*}

درج بالا جدول میں کیا ہو رہا ہے؟ ترتیب \عددی{\{\tfrac{1}{n}\}} عدد \عددی{0} کو مرتکز ہے۔ مسئلہ \حوالہ{مسئلہ_ترتیب_قواعد_حد_د} میں \عددی{a_n=\tfrac{1}{n}}، \عددی{f(x)=2^x} اور \عددی{L=0} لینے سے  ہم دیکھتے ہیں کہ \عددی{2^{1/n}=f(\tfrac{1}{n})\to f(L)=2^0=1} ہو گا۔ چونکہ \عددی{2} کے یک بعد دیگرے  جذر، ترتیب \عددی{\{2^{1/n}\}} کی ذیلی ترتیب \عددی{2^{1/2},2^{1/4},2^{1/8},\cdots} دیتے ہیں لہٰذا یہ جذر بھی  \عددی{1} کو مرتکز ہو گا (شکل \حوالہ{شکل_ترتیب_حد_ترتیب})۔
\begin{figure}
\centering
\begin{tikzpicture}[declare function={f(\x)=2^(\x);}]
\begin{axis}[small, axis lines=middle,ymin=0,xtick={0.33,0.5,1},xticklabels={$\tfrac{1}{3}$,$\tfrac{1}{2}$,$1$},ytick={1,2},xlabel={$x$},ylabel={$y$},xlabel style={at={(current axis.right of origin)},anchor=west},ylabel style={at={(current axis.above origin)},anchor=south}]
\addplot[domain=-0.25:1.3]{f(x)}node[pos=0.9,left]{$y=2^x$};
\draw (0.33,{f(0.33)})node[circ]{}node[below,yshift={-0.5ex}]{$(\tfrac{1}{3},2^{1/3})$};
\draw (0.5,{f(0.5)})node[circ]{}node[above,xshift={-0.75ex}]{$(\tfrac{1}{2},2^{1/2})$};
\draw (1,{f(1)})node[circ]{}node[below right]{$(1,2)$};
\end{axis}
\end{tikzpicture}
\caption{
جیسے جیسے \عددی{n\to \infty} ہوتا ہے ویسے ویسے \عددی{\tfrac{1}{n}\to 0} اور \عددی{2^{1/n}\to 2^0} ہوتے ہیں۔ 
}
\label{شکل_ترتیب_حد_ترتیب}
\end{figure}

\جزوحصہء{قاعدہ لھوپیٹال کا استعمال}
اگلا مسئلہ ہمیں قاعدہ لھوپیٹال کی مدد سے چند ترتیبات کے حد تلاش کرنے کے قابل بناتا ہے۔

\ابتدا{مسئلہ}\شناخت{مسئلہ_ترتیب_قواعد_حد_ہ}
فرض کریں کہ تفاعل \عددی{f(x)} تمام \عددی{x\ge n_0} کے لئے معین ہے اور \عددی{\{a_n\}} حقیقی اعداد کی ایک ایسی ترتیب ہے کہ تمام \عددی{n\ge n_0} کے لئے \عددی{a_n=f(n)} ہے۔ ایسی صورت میں درج ذیل ہو گا۔
\begin{align*}
\lim_{x\to\infty}f(x)=L\quad \implies\quad \lim_{n\to \infty}a_n=L
\end{align*}  
\انتہا{مسئلہ}
%========================
\ابتدا{ثبوت}
فرض کریں کہ \عددی{\lim{x\to \infty}f(x)=L} ہے۔ تب ہر مثبت عدد \عددی{\epsilon} کے لئے ایسا عدد \عددی{M} پاتا ہے کہ تمام \عددی{x} کے لئے درج ذیل ہو۔
\begin{align*}
x>M\quad\implies \quad \abs{f(x)-L}<\epsilon
\end{align*}
فرض کریں عدد صحیح \عددی{N} عدد \عددی{M} سے بڑا جبکہ \عددی{n_0} کے برابر یا اس سے بڑا ہے۔ تب درج ذیل ہو گا۔
\begin{align*}
n>N\quad &\implies \quad a_n=f(n)\\
\abs{a_n-L}&=\abs{f(n)-L}<\epsilon
\end{align*}
\انتہا{ثبوت}
%====================

\ابتدا{مثال}\شناخت{مثال_ترتیب_قاعدہ_لھوپیٹال_الف}
دکھائیں \عددی{\lim_{n\to\infty}\tfrac{\ln n}{n}=0}

حل:\quad
تفاعل \عددی{\tfrac{\ln x}{x}} تمام \عددی{x\ge 1} کے لئے معین ہے اور مثبت عدد صحیح کے لئے اس ترتیب سے اتفاق کرتا ہے۔ یوں \عددی{\lim_{n\to\infty}\tfrac{\ln n}{n}} مسئلہ \حوالہ{مسئلہ_ترتیب_قواعد_حد_ہ} کے تحت \عددی{\lim_{x\to \infty}\tfrac{\ln x}{x}} (اگر موجود ہو) کے ساتھ اتفاق کرے گا۔ قاعدہ لھوپیٹال کی ایک استعمال سے درج ذیل حاصل ہو گا۔
\begin{align*}
\lim_{x\to \infty}\frac{\ln x}{x}=\lim_{x\to\infty}\frac{1/x}{1}=\frac{0}{1}=0
\end{align*}  
یوں \عددی{\lim_{n\to \infty}\tfrac{\ln n}{n}=0} ہو گا۔
\انتہا{مثال}
%=====================
قاعدہ لھوپیٹال کی مدد سے ترتیب کا حد تلاش کرتے ہوئے ہم \عددی{n} کو استمراری حقیقی متغیر تصور کر کے اس کو \عددی{n} کے لحاظ سے تفرق کرتے ہیں۔اس طرح ہمیں \عددی{\{a_n\}} کا کلیہ دوبارہ لکھنے کی ضرورت پیش نہیں آتی ہے، جیسا ہمیں مثال \حوالہ{مثال_ترتیب_قاعدہ_لھوپیٹال_الف} میں کرنا پڑا۔

\ابتدا{مثال}
حد \عددی{\lim_{n\to\infty}\tfrac{2^n}{5n}} تلاش کریں۔

حل:\quad
قاعدہ لھوپیٹال استعمال کرتے ہیں۔
\begin{align*}
\lim_{n\to\infty}\frac{2^n}{5n}=\lim_{n\to \infty}\frac{2^n\cdot \ln 2}{5}=\infty
\end{align*}
\انتہا{مثال}
%=====================

\جزوحصہء{عموماً پائے جانے والے حد}
جدول \حوالہ{جدول_ترتیب_عمومی_حد} میں عموماً پائے جانے والے حد دیے گئے ہیں جہاں کلیہ \عددی{3} تا \عددی{6} میں \عددی{n\to\infty} کرتے ہوئے \عددی{x} مستقل رہتا ہے۔ پہلا حد مثال \حوالہ{مثال_ترتیب_قاعدہ_لھوپیٹال_الف} سے ہے۔ اگلے دو حد تلاش کرنے کے لئے لوگارتھم لے کر مسئلہ \حوالہ{مسئلہ_ترتیب_قواعد_حد_د} استعمال کریں۔ باقی ثبوت ضمیمہ میں دیے گئے ہیں۔
\begin{table}
\caption{عموماً پائے جانے والے حد}
\label{جدول_ترتیب_عمومی_حد}
\renewcommand{\arraystretch}{2}
\centering
\begin{tabular}{RL}
\toprule
\text{شمار}&\multicolumn{1}{C}{\text{حد}}\\
\midrule
1&\lim_{n\to\infty}\frac{\ln n}{n}=0\\
2&\lim_{n\to\infty}\sqrt[n]{n}=1\\
3&\lim_{n\to\infty}x^{1/n}=1\quad (x>0)\\
4&\lim_{n\to\infty} x^n=0\quad (\abs{x}<1)\\
5&\lim_{n\to\infty} \big(1+\frac{x}{n}\big)^n=e^x\\
6&\lim_{n\to\infty} \frac{x^n}{n!}=0\\
\bottomrule
\end{tabular}
\end{table}

\ابتدا{مثال}
ایک ترتیب کا \عددی{n} واں جزو \عددی{a_n=(\tfrac{n+1}{n-1})^n} ہے۔ کیا یہ ترتیب مرتکز ہے؟ اگر ترتیب مرتکز ہو تب اس کا حد تلاش کریں۔

حل:\quad
حد کی تلاش نا قابل معلوم قیمت \عددی{1^{\infty}} دیتی ہے۔ ہم \عددی{a_n} کا قدرتی لوگارتھم لے کر \عددی{\infty\cdot 0} حاصل کرتے ہیں لہٰذا  قاعدہ لھوپیٹال استعمال کیا جا سکتا ہے۔
\begin{align*}
\ln a_n&=\ln\big(\frac{n+1}{n-1}\big)^n\\
&=n\ln \big(\frac{n+1}{n-1}\big)
\end{align*}
یوں
\begin{align*}
\lim_{n\to\infty} \ln a_n&=\lim_{n\to\infty} n\ln\big(\frac{n+1}{n-1}\big)&& \infty\cdot 0\\
&=\lim_{n\to\infty}\frac{\ln \big(\frac{n+1}{n-1}\big)}{1/n}&&\frac{0}{0}\\
&=\lim_{n\to\infty}\frac{-2/(n^2-1)}{-1/n^2}&&\text{\RL{قاعدہ لھوپیٹال}}\\
&=\lim_{n\to\infty}\frac{2n^2}{n^2-1}=2
\end{align*}
ہو گا۔چونکہ \عددی{\ln a_n\to 2} ہے اور \عددی{f(x)=e^x} استمراری ہے لہٰذا مسئلہ \حوالہ{مسئلہ_ترتیب_قواعد_حد_د} کے تحت \عددی{a_n=e^{\ln a_n}\to e^2} ہو گا۔ ترتیب \عددی{\{a_n\}} عدد \عددی{e^2} پر مرتکز ہے۔
\انتہا{مثال}
%==========================

\جزوحصہء{جذر تلاش کرنے کی ترکیب پکاغ}
درج ذیل مساوات
\begin{align}\label{مساوات_ترتیب_پکاغ_الف}
f(x)=0
\end{align}
سے مراد 
\begin{align}\label{مساوات_ترتیب_پکاغ_ب}
g(x)=f(x)+x=x
\end{align}
کا حل لیا جا سکتا ہے جہاں دونوں اطراف \عددی{x} جمع کیا گیا ہے۔ اس معمولی تبدیلی کی بنا اس مساوات کو کمپیوٹر پر  \اصطلاح{ترکیب پکاغ}\فرہنگ{ترکیب!پکاغ}\حاشیہب{Picard's method}\فرہنگ{method!Picard's}  سے حل\حاشیہد{فرانسیسی ریاضی دان شاغل مل پکاغ [1856-1941]} کرنا ممکن ہو جاتا ہے۔

اگر \عددی{g} کے دائرہ کار میں \عددی{g} کا سعت بھی شامل ہو تب ہم دائرہ کار میں نقطہ \عددی{x_0} سے شروع کر کے \عددی{g} سے یک بعد دیگرے  درج ذیل حاصل کرتے ہیں۔
\begin{align}\label{مساوات_ترتیب_پکاغ_پ}
x_1=g(x_0),\quad x_2=g(x_1),\quad x_3=g(x_2),\quad, \cdots
\end{align} 
سادہ پابندیاں، جنہیں جلد پیش کیا جائے گا، لاگو کرتے ہوئے کلیہ توالی \عددی{x_{n+1}=g(x_n)} سے حاصل ترتیب ایک ایسے نقطہ \عددی{x} پر مرتکز ہو گی جس پر \عددی{g(x)=x} ہو گا۔ چونکہ اس نقطہ پر
\begin{align}\label{مساوات_ترتیب_پکاغ_ت}
f(x)=g(x)-x=x-x=0
\end{align}
ہو گا لہٰذا یہ نقطہ مساوات \عددی{f(x)=0} کا حل ہو گا۔

وہ نقطہ جس پر \عددی{g(x)=x} ہو \عددی{g} کا \اصطلاح{مقررہ نقطہ}\فرہنگ{مقررہ نقطہ}\حاشیہب{fixed point}\فرہنگ{fixed point} کہلاتا ہے۔ہم مساوات \حوالہ{مساوات_ترتیب_پکاغ_ت} سے دیکھتے ہیں کہ \عددی{g} کے مقررہ نقطے \عددی{f} کے جذر ہیں۔

\ابتدا{مثال}\موٹا{ترکیب کی پرکھ}

\انتہا{مثال}
%=======================
