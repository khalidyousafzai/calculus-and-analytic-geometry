\حصہ{ رخی تفرقات، سمتیہ ڈھلوان، اور مماسی سطحیں}
ہم حصہ \حوالہ{حصہ_کثیرالمتغیر_زنجیری_قاعدہ} سے جانتے ہیں کہ قابل تفرق تفاعل \عددی{f(x,y)}  منحنی \عددی{x=g(t),\,y=h(t)} کے ساتھ ساتھ چلتے ہوئے \عددی{t}  کے لحاظ سے شرح تبدیلی درج ذیل ہو گی۔
\begin{align*}
\frac{\dif f}{\dif t}=\frac{\partial f}{\partial x}\frac{\dif x}{\dif t}+\frac{\partial f}{\partial y}\frac{\dif y}{\dif t}
\end{align*} 
نقطہ \عددی{N_0(x_0,y_0)=N_0(g(t_0),h(t_0))}  پر یہ مساوات بڑھتی \عددی{t} کے لحاظ سے \عددی{f} کی شرح تبدیلی دیتی ہے،  جو دیگر  چیزوں کے ساتھ  منحنی پر چلنے کے رخ پر بھی منحصر ہے۔ یہ  مشاہدہ   اس صورت خصوصاً     اہم  ہو  گا  جب یہ    منحنی ایک سیدھی  لکیر ہو  اور نقطہ  \عددی{N_0} سے منحنی پر اکائی سمتیہ \عددی{\kvec{u}} کے رخ  چلتے ہوئے  مقدار معلوم لمبائی قوس   \عددی{t}  ہو۔چونکہ تب  \عددی{\kvec{u}} کے رخ \عددی{f} کے دائرہ کار میں  فاصلہ کے لحاظ  سے   \عددی{f} کی شرح تبدیلی  \عددی{\tfrac{\dif f}{\dif t}} ہو گی۔ ہم    \عددی{\kvec{u}}  تبدیل کرتے ہوئے نقطہ \عددی{N_0} پر،     فاصلہ کے لحاظ سے \عددی{f} کی مختلف رخ میں  شرح تبدیلی    دریافت کر سکتے ہیں۔   ان  \اصطلاح{رخی تفرقات}\فرہنگ{تفرق!رخی}\حاشیہب{directional derivative}\فرہنگ{derivative!directional} کی  سائنس،  انجینئری اور ریاضیات   میں کارآمد  تشریحات  کی جاتی ہیں۔ اس حصہ میں ان کی قیمت دریافت کرنے کا کلیہ اخذ کیا جائے گا جس کے بعد فضا میں سطحوں کی مماسی سطحیں اور عمودی سطحیں تلاش کی جائیں گی۔

\جزوحصہء{مستوی میں رخی تفرقات}
فرض کریں  مستوی \عددی{xy} میں پورے خطہ \عددی{R} میں  تفاعل \عددی{f(x,y)} معین ہے، \عددی{N_0(x_0,y_0)} خطہ \عددی{R} میں ایک نقطہ ہے، اور \عددی{\kvec{u}=u_1\ai+u_2\aj} ایک اکائی سمتیہ ہے۔  تب \عددی{\kvec{u}} کے متوازی نقطہ \عددی{N_0}  سے گزرتے  خط کی مقدار معلوم مساواتیں درج ذیل ہوں گی۔
\begin{align*}
x=x_0+su_1,\quad y=y_0+su_2
\end{align*}
 اکائی سمتیہ \عددی{\kvec{u}} کے رخ  نقطہ \عددی{N_0} سے فاصلہ  کو مقدار معلوم \عددی{s}سے ظاہر کیا جاتا ہے۔ ہم نقطہ \عددی{N_0} پر \عددی{\kvec{u}} کے رخ \عددی{f} کی شرح تبدیلی   \عددی{\tfrac{\dif f}{\dif s}}  سے حاصل کرتے ہیں۔

\ابتدا{تعریف}
نقطہ \عددی{N_0(x_0,y_0)} پر اکائی سمتیہ \عددی{\kvec{u}=u_1\ai+u_2\aj} کے رخ \عددی{f} کا تفرق درج ذیل عدد ہو گا
\begin{align}\label{مساوات_کثیرالمتغیر_رخی_تفرق_تعریف}
\big(\frac{\dif f}{\dif s}\big)_{\kvec{u},N_0}=\lim_{s\to 0}\frac{f(x_0+su_1,y_0+su_2)-f(x_0,y_0)}{s}
\end{align}
 بشرطیکہ یہ حد موجود ہو۔
\انتہا{تعریف}
%==============

رخی تفرق کو درج ذیل سے بھی ظاہر کیا جاتا ہے۔
\begin{align*}
&(D_{\kvec{u}}f)_{N_0}&&\text{\RL{\عددی{N_0} پر \عددی{\kvec{u}} کے رخ \عددی{f} کا تفرق}}
\end{align*}

\ابتدا{مثال}
نقطہ \عددی{N_0(1,2)} پر اکائی سمتیہ \عددی{\kvec{u}=\tfrac{1}{\sqrt{2}}\ai+\tfrac{1}{\sqrt{2}}\aj} کے رخ درج ذیل کا تفرق تلاش کریں۔
\begin{align*}
f(x,y)=x^2+xy
\end{align*}
حل:\quad
\begin{align*}
\big(\frac{\dif f}{\dif s}\big)_{\kvec{u},N_0}&=\lim_{s\to 0}\frac{f(x_0+su_1,y_0+su_2)-f(x_0,y_0)}{s}&&\text{\RL{مساوات \حوالہ{مساوات_کثیرالمتغیر_رخی_تفرق_تعریف}}}\\
&=\lim_{s\to 0}\frac{f\big(1+s\cdot \frac{1}{\sqrt{2}},2+s\cdot\frac{1}{\sqrt{2}}\big)-f(1,2)}{s}\\
&=\lim_{s\to 0}\frac{\big(1+\frac{s}{\sqrt{2}}\big)^2+\big(1+\frac{s}{\sqrt{2}}\big)\big(2+\frac{s}{\sqrt{2}}\big)-(1^2+1\cdot 2)}{s}\\
&=\lim_{s\to 0}\frac{\big(1+\frac{2s}{\sqrt{2}}+\frac{s^2}{2}\big)+\big(2+\frac{3s}{\sqrt{2}}+\frac{s^2}{2}\big)-3}{s}\\
&=\lim_{s\to 0}\frac{\frac{5s}{\sqrt{2}}+s^2}{s}=\lim_{s\to 0}\big(\frac{5}{\sqrt{2}}+s\big)=\big(\frac{5}{\sqrt{2}}+0\big)=\frac{5}{\sqrt{2}}
\end{align*}
نقطہ \عددی{N_0(1,2)} پر \عددی{\kvec{u}=\tfrac{1}{\sqrt{2}}\ai+\tfrac{1}{\sqrt{2}}\aj} کے رخ \عددی{f(x,y)=x^2+xy} کی تبدیلی کی شرح \عددی{\tfrac{5}{\sqrt{2}}} ہے۔
\انتہا{مثال}
%========================

\جزوحصہء{رخی تفرق کی جیومیٹریائی  تشریح}
مساوات  \عددی{z=f(x,y)} فضا میں ایک سطح \عددی{S}  کو ظاہر کرتی ہے۔ اگر \عددی{z_0=f(x_0,y_0)} ہو تب نقطہ \عددی{N_0(x_0,y_0,z_0)}  سطح \عددی{S}  پر واقع ہو گا۔ اکائی سمتیہ \عددی{\kvec{u}} کے متوازی انتصابی مستوی،   جو \عددی{N(x,y)} اور \عددی{N_0(x_0,y_0)} سے گزرتا  ہو،  \عددی{S} کو منحنی \عددی{C} میں قطع کرے گا۔اکائی سمتیہ \عددی{\kvec{u}} کے رخ \عددی{f} کی شرح تبدیلی \عددی{N} پر \عددی{C} کے مماس کی ڈھلوان ہو گی۔

دھیان رہے کہ جب \عددی{\kvec{u}=\ai} ہو، \عددی{N_0} پر  رخی تفرق \عددی{\tfrac{\partial f}{\partial x}} ہو گا جس کی قیمت \عددی{(x_0,y_0)} پر حاصل کی جائے گی۔ اسی طرح  جب   \عددی{\kvec{u}=\aj} ہو، \عددی{N_0} پر  رخی تفرق \عددی{\tfrac{\partial f}{\partial y}} ہو گا جس کی قیمت \عددی{(x_0,y_0)} پر حاصل کی جائے گی۔ رخی تفرق ان دو جزوی تفرقات کو عمومی بناتا ہے۔ ہم اب \عددی{\ai} اور \عددی{\aj} کے علاوہ کسی بھی رخ  \عددی{\kvec{u}}،    تفاعل \عددی{f} کی تبدیلی شرح  جان سکتے ہیں۔

\جزوحصہء{حساب}
جیسا آپ جانتے ہیں، تفرق کی تعریف بطور حد سے کسی بھی تفرق کا حصول اتنا آسان نہیں ہوتا ہے۔   رخی تفرق   کی تعریف سے بھی رخی تفرق کا حصول مشکل کام ہے۔ آئیں  رخی تفرق کا زیادہ  آسان کلیہ اخذ کریں۔  ہم  خط
\begin{align}\label{مساوات_کثیرالمتغیر_خط_مقدار_معلوم_الف}
x=x_0+su_1,\quad y=y_0+su_2
\end{align}
سے شروع کرتے ہیں جو نقطہ \عددی{N_0(x_0,y_0)} سے گزرتے خط کی مقدار معلوم مساوات ہے جس میں  اکائی سمتیہ \عددی{\kvec{u}=u_1\ai+u_2\aj} کے رخ بڑھتا ہوا  \عددی{s}  مقدار معلوم لمبائی قوس  ہے۔ تب درج ذیل ہو گا۔
\begin{align}
\big(\frac{\dif f}{\dif s}\big)_{\kvec{u},N_0}&=\big(\frac{\partial f}{\partial x}\big)_{N_0}\frac{\dif x}{\dif s}+\big(\frac{\partial f}{\partial y}\big)_{N_0}\frac{\dif y}{\dif s}&&\text{\RL{زنجیری قاعدہ}}\nonumber\\
&=\big(\frac{\partial f}{\partial x}\big)_{N_0}\cdot u_1+\big(\frac{\partial f}{\partial y}\big)_{N_0}\cdot u_2&&\text{\RL{مساوات \حوالہ{مساوات_کثیرالمتغیر_خط_مقدار_معلوم_الف} سے \عددی{\tfrac{\dif x}{\dif s}=u_1} اور \عددی{\tfrac{\dif y}{\dif s}=u_2}}}\nonumber\\
&=\underbrace{\big[\big(\frac{\partial f}{\partial x}\big)_{N_0}\ai+\big(\frac{\partial f}{\partial y}\big)_{N_0}\cdot \aj\big]}_{\text{\RL{\عددی{N_0} پر \عددی{f} کی ڈھلوان}}}\cdot\underbrace{\big[u_1\ai+u_2\aj\big]}_{\text{\RL{\عددی{\kvec{u}} کا رخ}}}\label{مساوات_کثیرالمتغیر_خط_مقدار_معلوم_ب}
\end{align}

\ابتدا{تعریف}
نقطہ \عددی{N_0(x_0,y_0)} پر \عددی{f(x,y)} کا سمتیہ ڈھلوان (ڈھلوان)  درج ذیل سمتیہ ہو گا
\begin{align*}
\nabla f=\frac{\partial f}{\partial x}\ai+\frac{\partial f}{\partial y}\aj
\end{align*}
جس کی قیمت \عددی{N_0} پر \عددی{f} کے رخی تفرق سے حاصل کی جائے گی۔
\انتہا{تعریف}
%===================

سمتی  ڈھلوان \عددی{\nabla f} کو "ڈھلوان \عددی{f}"  پڑھتے  ہیں۔ علامت \عددی{\nabla}  یونانی حرف "نیبلا" ہے۔

مساوات \حوالہ{مساوات_کثیرالمتغیر_خط_مقدار_معلوم_ب} کہتی ہے کہ \عددی{N_0} پر \عددی{\kvec{u}} کے رخ \عددی{f} کا تفرق  \عددی{N_0} پر \عددی{f}  کی ڈھلوان اور \عددی{\kvec{u}} کا حاصل ضرب ہو گا۔

\ابتدا{مسئلہ}
اگر \عددی{N_0(x_0,y_0)} پر \عددی{f(x,y)} کے جزوی تفرقات معین ہوں تب \عددی{N_0} پر \عددی{\kvec{u}} کے رخ \عددی{f} کا تفرق،   \عددی{N_0} پر \عددی{f} کی ڈھلوان اور \عددی{\kvec{u}} کا  غیر سمتی  ضرب    ہو گا:
\begin{align}
\big(\frac{\dif f}{\dif s}\big)_{\kvec{u},N_0}=\big(\nabla\big)_{N_0}\cdot \kvec{u}
\end{align}
\انتہا{مسئلہ}
%===============

\ابتدا{مثال}
نقطہ \عددی{(2,0)} پر \عددی{\kvec{A}=3\ai-4\aj} رخ \عددی{f(x,y)=xe^y+\cos(xy)} کا رخی تفرق تلاش کریں۔

حل:\quad
ہم \عددی{\kvec{A}} کی لمبائی سے \عددی{\kvec{A}} کو تقسیم کرتے ہوئے \عددی{\kvec{A}} کے رخ اکائی سمتیہ تلاش کرتے ہیں۔
\begin{align*}
\kvec{u}=\frac{\kvec{A}}{\abs{\kvec{A}}}=\frac{\kvec{A}}{5}=\frac{3}{5}\ai--\frac{4}{5}\aj
\end{align*}
نقطہ \عددی{(2,0)} پر \عددی{f} کے جزوی تفرقات
\begin{align*}
f_x(2,0)&=(e^y-y\sin(xy))_{(2,0)}=e^0-0=1\\
f_y(2,0)&=(xe^y-x\sin(xy))_{(2,0)}=2e^0-2\cdot 0=2
\end{align*}
ہوں گے لہٰذا نقطہ \عددی{(2,0)} پر \عددی{f} کی ڈھلوان 
\begin{align*}
\left.\nabla f\right\vert_{(2,0)}=f_x(2,0)\ai+f_y(2,0)\aj=\ai+2\aj
\end{align*}
ہو گی۔نقطہ \عددی{(2,0)} پر \عددی{\kvec{A}} رخ \عددی{f} کا تفرق درج ذیل ہو گا۔
\begin{align*}
\left.\big(D_{\kvec{u}}f\big)\right\vert_{(2,0)}&=\left.\nabla f\right\vert_{(2,0)}\cdot \kvec{u}\\
&=(\ai+2\aj)\cdot\big(\frac{3}{5}\ai-\frac{4}{5}\aj\big)=\frac{3}{5}-\frac{8}{5}=-1
\end{align*}
\انتہا{مثال}
%====================

\جزوحصہء{رخی تفرقات  کے خواص}
رخی تفرق کے کلیہ
\begin{align*}
D_{\kvec{u}}f=\nabla f\cdot\kvec{u}=\abs{\nabla f}\abs{\kvec{u}}\cos\theta=\abs{\nabla f}\cos\theta
\end{align*}
کی قیمت  تلاش  کرنے سے درج ذیل خواص   دریافت ہوتے ہیں:

\موٹا{رخی تفرق \عددی{D_{\kvec{u}}f=\nabla f\cdot\kvec{u}=\abs{\nabla f}\cos\theta} کے خواص}\\
\begin{enumerate}[1.]
\item
تفاعل \عددی{f} اس صورت تیز ترین بڑھتا ہے جب \عددی{\cos\theta=1} ہو، یعنی جب \عددی{\kvec{u}} اور \عددی{\nabla f} ایک ہی رخ ہوں۔اس طرح، اپنے دائرہ کار میں، نقطہ \عددی{N} پر  سمتیہ ڈھلوان \عددی{\nabla f} کے رخ،  \عددی{f}  تیز ترین بڑھتا ہے۔ اس رخ تفرق درج ذیل ہو گا۔
\begin{align*}
D_{\kvec{u}}f=\abs{\nabla f}\cos(0)=\abs{\nabla f}
\end{align*}
\item
اسی طرح  \عددی{-\nabla f} کے رخ  \عددی{f}  تیز ترین  گھٹتا ہے۔اس رخ تفرق \عددی{D_{\kvec{u}}f=\abs{\nabla f}\cos(\pi)=-\abs{\nabla f}} ہو گا۔
\item
ڈھلوان کے عمودی  کسی بھی رخ \عددی{\kvec{u}}  کوئی تبدیلی نہیں پائی جائے گی۔ایسے رخ \عددی{\theta} کی قیمت \عددی{\tfrac{\pi}{2}} ہو گی  لہٰذا درج ذیل ہو گا:
\begin{align*}
D_{\kvec{u}}f=\abs{\nabla f}\cos(\pi/2)=\abs{\nabla f}\cdot 0 =0
\end{align*}
\end{enumerate}

جیسا ہم دیکھیں گے، یہ خواص تین بعدی فضا میں بھی کارآمد  ہوں گی۔

\ابتدا{مثال}
نقطہ \عددی{(1,1)} پر  وہ رخ تلاش کریں جس رخ  تفاعل \عددی{f(x,y)=\tfrac{x^2}{2}+\tfrac{y^2}{2}}        (ا) تیز ترین بڑھتا ہو، (ب)  تیز ترین گھٹتا ہو، (ج)  میں کوئی تبدیلی رونما نہیں ہوتی ہو۔

حل:\quad
(ا) یہ تفاعل نقطہ \عددی{(1,1)} پر  \عددی{\nabla f} کے   رخ تیز ترین بڑھے گا ۔اس نقطہ پر   ڈھلوان
\begin{align*}
(\nabla f)_{(1,1)}=(x\ai+y\aj)_{(1,1)}=\ai+\aj
\end{align*}
ہے  لہٰذا تیز ترین بڑھنے کا رخ   درج ذیل اکائی سمتیہ دیگا۔
\begin{align*}
 \kvec{u}=\frac{\ai+\aj}{\abs{\ai+\aj}}=\frac{1}{\sqrt{2}}\ai+\frac{1}{\sqrt{2}}\aj
\end{align*}
(ب) یہ تفاعل \عددی{-\nabla f}  رخ تیز ترین گھٹے گا۔ یہ رخ درج ذیل ہو گا۔
\begin{align*}
-\kvec{u}=-\frac{1}{\sqrt{2}}\ai-\frac{1}{\sqrt{2}}\aj
\end{align*}
(ج) نقطہ \عددی{(1,1)} پر صفر تبدیلی کا رخ  \عددی{\nabla f}   کو عمودی ہو گا۔یہ رخ درج ذیل ہوں گے۔
\begin{align*}
\kvec{n}=-\frac{1}{\sqrt{2}}\ai+\frac{1}{\sqrt{2}}\aj,\quad -\kvec{n}=\frac{1}{\sqrt{2}}\ai-\frac{1}{\sqrt{2}}\aj
\end{align*}
\انتہا{مثال}
%==================

\جزوحصہء{ہم قد منحنی کے ڈھلوان اور مماس}
اگر ہموار منحنی \عددی{\kvec{r}=g(t)\ai+h(t)\aj}  پر قابل تفرق تفاعل \عددی{f(x,y)}  کی قیمت  مستقل ہو (جس کی بنا یہ منحنی،  \عددی{f} کی ہم قد منحنی ہو گی)،  تب  \عددی{f(g(t),h(t))=0} ہو گا۔دونوں اطراف کا \عددی{t}  کے لحاظ سے تفرق درج ذیل مساوات دیگا:
\begin{gather}
\begin{aligned}\label{مساوات_کثیرالمتغیر_ڈھلوان_مماس}
\frac{\dif}{\dif t}f(g(t),h(t))&=\frac{\dif}{\dif t}(c)\\
\frac{\partial f}{\partial x}\frac{\dif g}{\dif t}+\frac{\partial f}{\partial y}\frac{\dif h}{\dif t}&=0&&\text{\RL{زنجیری قاعدہ}}\\
\underbrace{\big(\frac{\partial f}{\partial x}\ai+\frac{\partial f}{\partial y}\aj\big)}_{\nabla f}\cdot\underbrace{\big(\frac{\dif g}{\dif t}\ai+\frac{\dif h}{\dif t}\aj\big)}_{\tfrac{\dif \kvec{r}}{\dif t}}&=0
\end{aligned}
\end{gather}
مساوات \حوالہ{مساوات_کثیرالمتغیر_ڈھلوان_مماس} کہتی ہے  کہ مماسی سمتیہ \عددی{\tfrac{\dif\kvec{r}}{\dif t}} کو  \عددی{\nabla f} عمودی ہو گا، لہٰذا  \عددی{\nabla f}   نقطہ \عددی{(x_0,y_0)} پر  منحنی کو عمودی ہو گا۔

تفاعل \عددی{f(x,y)} کے دائرہ کار میں ہر نقطہ \عددی{(x_0,y_0)} پر \عددی{f} کی ڈھلوان نقطہ \عددی{(x_0,y_0)} پر ہم قد منحنی کو عمودی ہو گی۔

ہم اس  مشاہدہ   کی بنا  ہم قد منحنیات  کی مماسات   کی مساواتیں  دریافت کر سکتے ہیں۔یہ ڈھلوان کو عمودی خطوط ہوں گے۔ نقطہ \عددی{N_0(x_0,y_0)} سے گزرتا ہوا،  سمتیہ \عددی{\kvec{N}=A\ai+B\aj} کو عمودی خط  کی مساوات درج ذیل ہو گی۔
\begin{align*}
A(x-x_0)+B(y-y_0)=0
\end{align*}
اگر \عددی{\kvec{N}} ڈھلوان \عددی{(\nabla f)_{(x_0,y_0)}=f_x(x_0,y_0)\ai+f_y(x_0,y_0)\aj} ہو  تب اس مساوات کی صورت درج ذیل ہو گی۔
\begin{align}\label{مساوات_کثیرالمتغیر_ہم_قد_کی_خط_مماس}
f_x(x_0,y_0)(x-x_0)+f_y(x_0,y_0)(y-y_0)=0
\end{align}

\ابتدا{مثال}
نقطہ \عددی{(-2,1)} پر درج ذیل ترخیم کے مماس کی مساوات تلاش کریں۔
\begin{align*}
\frac{x^2}{4}+y^2=2
\end{align*}
حل:\quad
یہ ترخیم درج ذیل تفاعل کی ہم قد منحنی ہے۔
\begin{align*}
f(x,y)=\frac{x^2}{4}+y^2
\end{align*}
نقطہ \عددی{(-2,1)} پر  \عددی{f} کی ڈھلوان 
\begin{align*}
\left.\nabla f\right\vert_{(-2,1)}=\big(\frac{x}{2}\ai+2y\aj\big)_{(-2,1)}=-\ai+2\aj
\end{align*}
ہو گی لہٰذا مماسی خط کی مساوات درج ذیل ہو گی۔
\begin{align*}
(-1)(x+2)+(2)(y-1)&=0&&\text{\RL{مساوات \حوالہ{مساوات_کثیرالمتغیر_ہم_قد_کی_خط_مماس}}}\\
x-2y&=-4
\end{align*}
\انتہا{مثال}
%==================

\جزوحصہء{تین متغیرات کا تفاعل}
ہم دو متغیرات کلیات کے ساتھ جزو \عددی{z} شامل کر کے تین متغیرات کلیات حاصل کرتے ہیں۔فضا میں قابل تفرق تفاعل \عددی{f(x,y,z)} اور اکائی سمتیہ \عددی{\kvec{u}=u_1\ai+u_2\aj+u_3\ak}  کے لئے ہم
\begin{align*}
\nabla f&=\frac{\partial f}{\partial x}\ai+\frac{\partial f}{\partial y}\aj+\frac{\partial f}{\partial z}\ak\\
D_{\kvec{u}}f&=\nabla f\cdot \kvec{u}=\frac{\partial f}{\partial x}u_1+\frac{\partial f}{\partial y}u_2+\frac{\partial f}{\partial z}u_3&&\text{اور}
\end{align*}
لکھیں گے۔ رخی تفرق اب بھی
\begin{align*}
D_{\kvec{u}}f=\nabla f\cdot \kvec{u}=\abs{\nabla f}\abs{\kvec{u}}\cos\theta=\abs{\nabla f}\cos\theta
\end{align*}
ہو گا لہٰذا دو متغیرات کے خواص  (جن کا ہم ذکر کر چکے ہیں)، تین متغیرات کے تفاعل کے لئے بھی کارآمد ہوں گے۔کسی بھی نقطہ پر  \عددی{\nabla f} رخ تفاعل تیز ترین بڑھتا ہے اور \عددی{-\nabla f} رخ تیز ترین گھٹتا ہے، جبکہ \عددی{\nabla f} کے عمودی کسی بھی رخ، تفرق صفر ہو گا۔

\ابتدا{مثال}
(ا) نقطہ \عددی{(N_0(1,1,0))} پر \عددی{\kvec{A}=2\ai-3\aj+6\ak} رخ \عددی{f(x,y,z)=x^3-xy^2-z} کا تفرق تلاش کریں۔ (ب) نقطہ \عددی{N_0} پر \عددی{f} کس رخ تیز ترین بڑھتا ہے؟ اس رخ شرح تبدیلی کیا ہو گی؟

حل:\quad
(ا) سمتیہ \عددی{\kvec{A}}  کو اس کی لمبائی سے تقسیم کر کے  اس کے  رخ اکائی سمتیہ \عددی{\kvec{u}} تلاش کرتے ہیں۔
\begin{align*}
\abs{\kvec{A}}&=\sqrt{(2)^2+(-3)^2+(6)^2}=\sqrt{49}=7\\
\kvec{u}&=\frac{\kvec{A}}{\abs{\kvec{A}}}=\frac{2}{7}\ai-\frac{3}{7}\aj+\frac{6}{7}\ak
\end{align*}  
نقطہ \عددی{N_0} پر جزوی تفرقات
\begin{align*}
f_x&=\left. 3x^2-y^2\right\vert_{(1,1,0)}=2,\quad f_y=\left. -2xy\right\vert_{(1,1,0)}=-2,\quad f_z=\left.-1\right\vert_{(1,1,0)}=-1
\end{align*}
ہوں گے لہٰذا \عددی{N_0} پر \عددی{f} کی ڈھلوان درج ذیل ہو گی۔
\begin{align*}
\left.\nabla f\right\vert_{(1,1,0)}=2\ai-2\aj-\ak
\end{align*}
نقطہ \عددی{N_0} پر \عددی{\kvec{A}} کے رخ \عددی{f} کا تفرق درج ذیل ہو گا۔
\begin{align*}
\left.D_{\kvec{u}}f\right\vert_{(1,1,0)}&=\left.\nabla f\right\vert_{(1,1,0)}\cdot\kvec{u}=(2\ai-2\aj-\ak)\cdot\big(\frac{2}{7}\ai-\frac{3}{7}\aj+\frac{6}{7}\ak\big)\\
&=\frac{4}{7}+\frac{6}{7}-\frac{6}{7}=\frac{4}{7}
\end{align*}
(ب) تفاعل تیز ترین \عددی{\nabla f=2\ai-2\aj-\ak} رخ بڑھتا ہے اور \عددی{-\nabla f=-2\ai+2\aj+\ak} رخ تیز ترین گھٹتا ہے۔ ان رخ تبدیلی  کی شرح بالترتیب درج ذیل ہوں گی۔
\begin{align*}
\abs{\nabla f}&=\sqrt{(2)^2+(-2)^2+(-1)^2}=\sqrt{9}=3\\
-\abs{\nabla f}&=-3
\end{align*}
\انتہا{مثال}
%================

\جزوحصہء{مماسی مستوی  اور عمودی خطوط کی مساواتیں}
اگر قابل تفرق تفاعل \عددی{f}  کی ہم قد منحنی \عددی{f(x,y,z)=c}  پر \عددی{\kvec{r}=g(t)\ai+h(t)\aj+k(t)\ak} ایک ہموار منحنی ہو تب \عددی{f(g(t),h(t),k(t))=0} ہو گا۔ دونوں اطراف کا \عددی{t} کے لحاظ سے تفرق درج ذیل  دیگا:
\begin{align}
\frac{\dif}{\dif t}f(g(t),h(t),k(t))&=\frac{\dif}{\dif t}(c)\nonumber\\
\frac{\partial f}{\partial x}\frac{\dif g}{\dif t}+\frac{\partial f}{\partial y}\frac{\dif h}{\dif t}+\frac{\partial f}{\partial z}\frac{\dif k}{\dif t}&=0&&\text{\RL{زنجیری قاعدہ}}\nonumber\\
\underbrace{\big(\frac{\partial f}{\partial x}\ai+\frac{\partial f}{\partial y}\aj+\frac{\partial f}{\partial z}\ak\big)}_{\nabla f}\cdot\underbrace{\big(\frac{\dif g}{\dif t}\ai+\frac{\dif h}{\dif t}\aj+\frac{\dif k}{\dif t}\ak\big)}_{\tfrac{\dif \kvec{r}}{\dif t}}&=0\label{مساوات_کثیرالمتغیر_مماسی_مستوی_عمودی_خط}
\end{align}
منحنی  کے ساتھ ساتھ  ہر نقطہ  پر \عددی{\nabla f}،  منحنی کی سمتیہ  رفتار   کو عمودی ہو گا۔

آئیں اب نقطہ \عددی{N_0}  سے گزرتی منحنی تک اپنے آپ کو محدود رکھتے ہیں۔نقطہ \عددی{N_0} پر تمام سمتیات  رفتار،  \عددی{N_0} پر \عددی{\nabla f} کو عمودی ہوں گے لہٰذا منحنی  کے تمام مماسی خط  اس مستوی میں پائے جائیں گے جو \عددی{N_0} پر \عددی{\nabla f} کو عمودی ہو۔ اس مستوی کو ہم \عددی{N_0} پر سطح کا مماسی مستوی کہتے ہیں۔ نقطہ \عددی{N_0} سے گزرتا ہوا ایسا خط جو  اس مستوی کو عمودی ہو، \عددی{N_0} پر سطح کا عمودی خط ہو گا۔

\ابتدا{تعریف}
نقطہ \عددی{N_0(x_0,y_0,z_0)}  پر  \عددی{\left.\nabla f\right\vert_{N_0}} کا  عمودی مستوی،   نقطہ \عددی{N_0}   پر  ہم قد منحنی \عددی{f(x,y,z)=c}  کا مماسی مستوی  ہو گا۔

نقظہ \عددی{N_0} پر \عددی{\left.\nabla f\right\vert_{N_0}} کا متوازی خط ،  نقطہ \عددی{N_0} پر سطح کا عمودی خط ہو گا۔
\انتہا{تعریف}
%==================
 یوں حصہ \حوالہ{حصہ_سمتیہ_فضا_خطوط_مستوی}  کے تحت، مماسی مستوی اور عمودی خط کی بالترتیب مساوات درج ذیل ہوں گی۔
\begin{align}
f_x(N_0)(x-x_0)+f_y(N_0)(y-y_0)+f_z(N_0)(z-z_0)=0\\
x=x_0+f_x(N_0)t,\quad y=y_0+f_y(N_0)t,\quad z=z_0+f_z(N_0)t
\end{align}

\ابتدا{مثال}
نقطہ \عددی{N_0(1,2,4)} پر درج ذیل کا مماسی مستوی اور عمودی خط دریافت کریں۔
\begin{align*}
f(x,y,z)&=x^2+y^2+z-9=0&&\text{\RL{دائری قطع مکافی}}
\end{align*}
حل:\quad
نقطہ \عددی{N_0} پر \عددی{f} کی ڈھلوان کو عمودی سطح، نقطہ \عددی{N_0} پر مستوی   ہو گا۔ ڈھلوان
\begin{align*}
\left.\nabla f\right\vert_{N_0}=(2x\ai+2y\aj+\ak)_{(1,2,4)}=2\ai+4\aj+\ak
\end{align*}
ہے لہٰذا مستوی درج ذیل ہو گا۔
\begin{align*}
2x+4y+z=14\quad \text{یعنی}\quad 2(x-1)+4(y-2)+(z-4)=0
\end{align*}
نقطہ \عددی{N_0} پر سطح کا عمودی خط درج ذیل ہو گا۔
\begin{align*}
x=1+2t,\quad y=2+4t,\quad z=4+t
\end{align*}
\انتہا{مثال}
%======================
\ابتدا{مثال}
بیلنی سطح
\begin{align*}
f(x,y,z)=x^2+y^2-2=0
\end{align*}
اور مستوی
\begin{align*}
g(x,y,z)=x+z-4=0
\end{align*}
ایک ترخیم \عددی{T} میں ملتے ہیں۔ نقطہ \عددی{N_0(1,1,3)} پر \عددی{T} کے مماسی خط کی مقدار معلوم مساوات تلاش کریں۔

حل:\quad
نقطہ \عددی{N_0} پر مماسی خط \عددی{\nabla f} اور \عددی{\nabla g} دونوں کو  عمودی  لہٰذا \عددی{\kvec{v}=\nabla f\times \nabla g} کو متوازی ہو گا۔ نقطہ \عددی{N_0} کے محدد اور  \عددی{\kvec{v}} کے اجزاء ہمیں  مماسی  خط کی مساوات دیتے ہیں۔ ہمارے پاس درج ذیل ہے۔
\begin{align*}
\nabla f_{(1,1,3)}&=(2x\ai+2y\aj)_{(1,1,3)}=2\ai+2\aj\\
\nabla g_{(1,1,3)}&=(\ai+\ak)_{(1,1,3)}=\ai+\ak\\
\kvec{v}&=(2\ai+2\aj)\times (\ai+\ak)=\begin{vmatrix}\ai&\aj&\ak\\ 2&2&0\\ 1&0&1  \end{vmatrix}=2\ai-2\aj-2\ak
\end{align*}
مماسی خط درج ذیل ہو گا۔
\begin{align*}
x=1+2t,\quad y=1-2t,\quad z=3-2t
\end{align*}
\انتہا{مثال}
%================

\جزوحصہء{سطح \عددی{z=f(x,y)} کا مماسی مستوی}
نقطہ \عددی{N_0(x_0,y_0,z_0)} پر     سطح \عددی{z=f(x,y)}  کے مماسی مستوی کی مساوات تلاش کریں۔ اس نقطہ پر  \عددی{z_0=f(x_0,y_0)}  ہو گا۔ ہم دیکھتے ہیں کہ مساوات \عددی{z=f(x,y)}  کو \عددی{f(x,y)-z=0} لکھا  جا سکتا ہے۔ یوں سطح \عددی{z=f(x,y)}  درحقیقت تفاعل \عددی{F(x,y,z)=f(x,y)-z} کا صفر ہم قد مستوی ہو گا۔تفاعل \عددی{F}  کے جزوی تفرقات
\begin{align*}
F_x&=\frac{\partial}{\partial x}(f(x,y)-z)=f_x-0=f_x\\
F_y&=\frac{\partial}{\partial y}(f(x,y)-z)=f_y-0=f_y\\
F_z&=\frac{\partial}{\partial z}(f(x,y)-z)=0-1=-1
\end{align*}
ہوں گے۔ نقطہ \عددی{N_0} پر مماسی مستوی کا کلیہ
\begin{align*}
F_x(N_0)(x-x_0)+F_y(N_0)(y-y_0)+F_z(N_0)(z-z_0)=0
\end{align*}
یوں درج ذیل صورت اختیار کرے گا۔
\begin{align}\label{مساوات_کثیرالمتغیر_مماسی_مستوی}
f_x(x_0,y_0)(x-x_0)+f_y(x_0,y_0)(y-y_0)+(z-z_0)
\end{align}
%====================
\ابتدا{مثال}
نقطہ \عددی{(0,0,0)} پر سطح \عددی{z=x\cos y-ye^x} کا مماسی مستوی تلاش کریں۔

حل:\quad
ہم  جزوی تفرقات معلوم کر کے مساوات \حوالہ{مساوات_کثیرالمتغیر_مماسی_مستوی} استعمال کریں گے:
\begin{align*}
f_x(0,0)&=(\cos y-ye^x)_{(0,0)}=1-0\cdot 1=1\\
f_y(0,0)&=(-x\sin y-e^x)_{(0,0)}=0-1=-1
\end{align*}
یوں مماسی مستوی  
\begin{align*}
1\cdot(x-0)-1\cdot(y-0)-(z-0)&=0&&\text{\RL{مساوات \حوالہ{مساوات_کثیرالمتغیر_مماسی_مستوی}}}
\end{align*}
یعنی درج ذیل ہو گا۔
\begin{align*}
x-y-z=0
\end{align*}
\انتہا{مثال}
%======================

\جزوحصہء{بڑھوتری اور فاصلہ}
