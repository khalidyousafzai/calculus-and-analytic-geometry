\حصہ{سمتی میدان، کام، دورانیت، اور بہاو}
ان طبیعی  مظہر  کے مطالعہ کے دوران، جنہیں سمتیات سے ظاہر کیا جاتا ہے، ہم بند راہ پر تکملات کی بجائے سمتی میدان میں راہ پر تکملات استعمال کرتے ہیں۔متغیر قوت کے خلاف ایک مقام سے دوسری مقام کسی جسم کو منتقل کرنے (جیسا قوت ثقل کے خلاف خلاء میں سواری بھیجنے) کے لئے درکار کام  یا سمتی میدان میں ایک جسم کو کسی راہ پر حرکت دینے (جیسا مسرع کسی ذرے کی توانائی بڑھاتا ہے)  کے لئے درکار کام   اس طرح کے تکملات سے حاصل کیا جاتا ہے۔  ہم منحنیات کے پار سیال کی بہاو کی شرح بھی لکیری تکملات سے حاصل کرتے ہیں۔

\جزوحصہء{سمتی میدان}
مستوی یا فضا میں دائرہ کار پر  \اصطلاح{سمتی میدان}\فرہنگ{میدان!سمتی}\حاشیہب{vector field}\فرہنگ{field!vector} سے مراد ایسا تفاعل ہے جو دائرہ کار کے ہر نقطہ کو ایک سمتیہ مختص کرتا ہو۔سہ ابعادی سمتیات کے میدان کا ایک کلیہ درج ذیل ہو سکتا ہے۔
\begin{align*}
\kvec{F}(x,y,z)=M(x,y,z)\ai+N(x,y,z)\aj+P(x,y,z)\ak
\end{align*}
اگر جزوی تفاعل \عددی{M}، \عددی{N}، \عددی{P} استمراری ہوں تب یہ میدان استمراری ہو گا، اگر \عددی{M}، \عددی{N}، \عددی{P} قابل تفرق ہوں تب یہ میدان قابل تفرق ہو گا، وغیرہ وغیرہ۔ دو ابعادی سمتیات کے میدان کا ایک کلیہ درج ذیل ہو سکتا ہے۔
\begin{align*}
\kvec{F}(x,y)=M(x,y)\ai+N(x,y)\aj
\end{align*}
گول انداز کے گزرگاہ کے مستوی میں  گزرگاہ کے ہر نقطہ کے ساتھ گول انداز کی سمتی رفتار کا سمتیہ منسلک کرنے سے گزرگاہ  کے ساتھ ساتھ دو ابعادی میدان حاصل ہو گا۔  غیر سمتی تفاعل کے ہم-قد سطح کے ہر نقطہ کے ساتھ  تفاعل کا سمتیہ ڈھلوان منسلک کرنے سے سطح پر سہ ابعادی میدان حاصل ہو گا۔   متحرک سیال کے ہر نقطہ کے ساتھ سمتی رفتار کا سمتیہ منسلک کرنے سے فضا میں اس خطہ پر سہ ابعادی  میدان حاصل ہو گا۔ ان میدان کے علاوہ دیگر میدان بھی شکل میں دکھائے گئے ہیں جہاں چند میدانوں کے کلیات بھی دیے گئے ہیں۔
 

وہ میدان ترسیم کرنے کے لئے جن کے کلیات معلوم ہوں، ہم دائرہ کار میں چند نقطے منتخب کر کے ان کے ساتھ منسلک سمتیات  کا خاکہ بناتے ہیں۔ دھیان رہے کہ روایتی طور پر اس نقطہ پر، جہاں سمتی تفاعل کی قیمت حاصل کی گئی ہو،  سمتیہ ظاہر کرنے والی تیر دار لکیر  کی دم  رکھی جاتی ہے نا کہ سر۔ تعین گر سمتیات (باب \حوالہ{سمتی_قیمت_تفاعل_اور_فضا_میں_حرکت}) میں ایسا نہیں کیا گیا بلکہ تعین گر سمتیات کو ظاہر کرنے والی تیر دار لکیر کی دم کو مبدا پر رکھا گیا جبکہ اس کے سر کا سیارہ یا گول انداز کے مقام پر رکھا گیا۔
