\حصہ{تکملات بالکثرت میں بدل}
اس حصہ میں بارہا تکمل کی قیمت کا حصول بذریعہ بدل سکھایا جائے گا۔ واحد تکمل کی طرح یہاں بھی پیچیدہ تکمل کو سادہ تکمل سے بدلہ  جاتا  ہے۔ بدل سے متکمل  یا تکمل کی حدوں یا ان دونوں  کی سادہ  روپ استعمال کی جاتی ہے۔


\جزوحصہء{دوہرا تکملات میں میں بدل} 


ہم قطبی محدد د کی بدل کا استعمال حصہ \حوالہ{حصہ_بالکثرت_دوہرا_تکملات_قطبی_روپ} میں  دیکھ چکے ہیں جو دہرا تکملات کی بدل، جس میں متغیرات کی تبدیلی کو خطے کی تبدیلی تصور کیا جاتا ہے، کی ایک  مخصوص شکل ہے۔ 


فرض کریں مستوی \عددی{uv} کے خطہ \عددی{G} کو ایک ایک مطابقت کے ساتھ مساوات
\[x=g(u,v),\quad y=h(u,v)\]
 کے ذریعہ مستوی \عددی{xy} کے خطہ \عددی{R} میں بدلا جاتا ہے۔ ہم \عددی{R} کو اس بدل میں \عددی{G} کا \اصطلاح{عکس}\فرہنگ{عکس}\حاشیہب{image}\فرہنگ{image} اور \عددی{G} کو \عددی{R} کا  \اصطلاح{قبل عکس}\فرہنگ{عکس!قبل}\حاشیہب{preimage}\فرہنگ{image!pre} کہتے ہیں۔خطہ \عددی{R} کسی بھی تفاعل \عددی{f(x,y)} کو خطہ \عددی{G} میں معین تفاعل \عددی{f(g(u,v),h(u,v))} بھی تصور کیا جا سکتا ہے۔ خطہ \عددی{R} میں \عددی{f(x,y)} کے تکمل کا خطہ \عددی{G} میں \عددی{f(g(u,v),h(u,v))} کے تکمل کے ساتھ کیا تعلق ہوگا؟


اس کا جواب: اگر \عددی{g}، \عددی{h} اور \عددی{f} کے جزوی تفرقات استمراری ہوں اور \عددی{J(u,v)} (جس پر جلد تبصرہ کیا جائے گا) صرف  تنہا نقطوں پر صفر ہو (اگر صفر ہو بھی)  تب درج ذیل ہو گا۔ 
\begin{align}\label{مساوات_بالکثرت_یعقوبی_بدل}
\iint\limits_{R} f(x,y)\dif x\dif y=\iint\limits_G f(g(u,v),h(u,v))\abs{J(u,v)}\dif u\dif v
\end{align}
مذکورہ بالا مساوات میں \عددی{J(u,v)}، جو \ترچھا{ یعقوبی} کہلاتا ہے، کی مطلق قیمت استعمال کی گئی۔


\ابتدا{تعریف}
  \اصطلاح{یعقوبی مقطع}\فرہنگ{یہ ریاضی دان کارل   گستاف یعقوب یعقوبی کے نام سے منسوب ہے۔} یا محددی بدل \عددی{x=g(u,v)}، \عددی{y=h(u,v)} کے \اصطلاح{یعقوبی}\فرہنگ{یعقوبی}\حاشیہب{Jacobian}\فرہنگ{Jacobian} سے مراد درج ذیل ہے:
\begin{align}\label{مساوات_بالکثرت_یعقوبی_تعریف}
J(u,v)=\begin{vmatrix}
\frac{\partial x}{\partial u}&\frac{\partial x}{\partial v}\\
\frac{\partial y}{\partial u}&\frac{\partial y}{\partial v}
\end{vmatrix}=
\frac{\partial x}{\partial u}\frac{\partial y}{\partial v}-\frac{\partial y}{\partial u}\frac{\partial x}{\partial v}
\end{align}
\انتہا{تعریف}


یعقوبی کو 
\begin{align*}
J(u,v)=\frac{x,y}{u,v}
\end{align*}
سے بھی ظاہر کیا جاتا ہے جو ہمیں یاد دلاتا ہے کہ \عددی{x} اور \عددی{y} کی جزوی تفرقات سے یعقوبی (مساوات \حوالہ{مساوات_بالکثرت_یعقوبی_تعریف})  حاصل ہوتا ہے۔مساوات \حوالہ{مساوات_بالکثرت_یعقوبی_بدل} کی استخراج آپ کو اعلٰی احصاء کے نصاب میں ملے گی جس کو یہاں پیش نہیں کیا جائے گا۔






قطبی محدد میں میں \عددی{u} اور \عددی{v} کی جگہ \عددی{r} اور \عددی{\theta} ہوں گے لہٰذا \عددی{x=r\cos\theta} اور \عددی{y=r\sin\theta} لیتے ہوئے یعقوبی
\begin{align*}
J(r,\theta)=\begin{vmatrix}
\frac{\partial x}{\partial r}&\frac{\partial x}{\partial \theta}\\
\frac{\partial y}{\partial r}&\frac{\partial y}{\partial \theta}
\end{vmatrix}=\begin{vmatrix}
\cos\theta&-r\sin\theta\\
\sin\theta&r\cos\theta
\end{vmatrix}=
r(\cos^2\theta+\sin^2\theta)=r
\end{align*}
ہو گا اور مساوات \حوالہ{مساوات_بالکثرت_یعقوبی_بدل} درج ذیل صورت اختیار کرے گی جو حصہ  \حوالہ{حصہ_بالکثرت_دوہرا_تکملات_قطبی_روپ}کی مساوات \حوالہ{مساوات_بالکثرت_قطبی_محدد_میں_رقبہ_کی_عمومی} ہے۔
\begin{gather}
\begin{aligned}
\iint\limits_R f(x,y)\dif x\dif y&=\iint\limits_G f(r\cos\theta,r\sin\theta)\abs{r}\dif r \dif \theta\\
&=\iint\limits_G f(r\cos\theta,r\sin\theta)r\dif r\dif\theta&&\text{\RL{اگر \عددی{r\ge 0} ہو}}
\end{aligned}
\end{gather}
