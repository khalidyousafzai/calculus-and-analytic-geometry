\حصہ{تکمل بالحصص}
تکمل بالحصص کی ترکیب سے تکمل
\begin{align}
\int f(x)g(x)\dif x
\end{align}
جس میں \عددی{f} بار بار قابل تفرق اور \عددی{g} بار بار قابل تکمل ہو کو کی سادہ روپ حاصل کی جا سکتی ہے۔ درج ذیل تکمل
\begin{align*}
\int xe^x\dif x
\end{align*}
اس قسم کا ایک تکمل ہے جہاں \عددی{f(x)=x}  دو بار تفرق کے بعد صفر ہو جاتا ہے  جبکہ \عددی{g(x)=e^x} کا تکمل بار بار لیا جا سکتا ہے۔ تکمل بالحصص کی ترکیب درج ذیل قسم کے تکمل پر بھی قابل اطلاق ہے
\begin{align*}
\int e^x\sin x\dif x
\end{align*}  
جس میں ہر دو بار تفرق اور ہر دو بار تکمل کے بعد وہی \عددی{f} اور \عددی{g} دوبارہ حاصل ہوتے ہیں۔

اس حصہ میں تکمل بالحصص پر غور کیا جائے گا اور اس کا استعمال سکھایا جائے گا۔

\جزوحصہء{تکمل بالحصص کا کلیہ}
\اصطلاح{تکمل بالحصص}\فرہنگ{تکمل!بالحصص}\حاشیہب{integration by parts}\فرہنگ{integration!by parts} کا کلیہ قاعدہ ضرب
\begin{align*}
\frac{\dif}{\dif x}(uv)=u\frac{\dif v}{\dif x}+v\frac{\dif u}{\dif x}
\end{align*}
سے حاصل ہوتا ہے جس کو تفریقی روپ
\begin{align*}
\dif(uv)=u\dif v+v\dif u
\end{align*}
یا
\begin{align*}
u\dif v=\dif(uv)-v\dif u
\end{align*}
میں لکھ کر تکمل  لینے سے درج ذیل کلیہ اخذ ہوتا ہے۔
\begin{align}
\int u\dif v&=uv-\int v\dif u&&\text{\RL{کلیہ تکمل بالحصص}}
\end{align}

تکمل بالحصص کا کلیہ ایک تکمل، \عددی{\int u\dif v}، کو دوسرے تکمل، \عددی{\int v\dif u}، کی صورت میں بیان ہے۔ \عددی{u} اور \عددی{v} کی صحیح انتخاب سے دوسرا تکمل حل کرنا زیادہ آسان ہو گا۔ یہی اس کلیہ کی اہمیت اسی حقیقت کی بنا ہے۔ جب ہمیں کسی تکمل کو حل کرنے میں ناکامی ہو، ہم اس کو دوسرے تکمل میں تبدیل کرتے ہوئے توقع کرتے ہیں کہ ہم اس کو حل کر پائیں گے۔

قطعی تکمل کے لئے اس کا مساوی کلیہ درج ذیل ہے۔
\begin{align}
\int_{v_1}^{v_2}u\dif v=(u_2v_2-u_1v_1)-\int_{u_1}^{u_2}v\dif u
\end{align}  
