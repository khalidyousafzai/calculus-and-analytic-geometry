\حصہ{سمتی میدان، کام، دائری بہاو، اور بہاو}
ان طبیعی  مظہر  کے مطالعہ کے دوران، جنہیں سمتیات سے ظاہر کیا جاتا ہے، بند راہ پر تکملات کی بجائے سمتی میدان میں راہ پر تکملات استعمال کیے جاتے ہیں۔متغیر قوت کے خلاف ایک مقام سے دوسری مقام کسی جسم کو منتقل کرنے (جیسا قوت ثقل کے خلاف خلاء میں سواری بھیجنے)   یا سمتی میدان میں ایک جسم کو کسی راہ پر حرکت دینے (جیسا مسرع کسی ذرے کی توانائی بڑھاتا ہو)  کے لئے درکار کام   اس طرح کے تکملات سے حاصل کیے  جاتے ہیں۔  منحنیات عبور کرتا ہوا سیال کے بہاو کی شرح بھی لکیری تکملات سے حاصل کی جاتی ہے۔

\جزوحصہء{سمتی میدان}
مستوی یا فضا میں دائرہ کار پر  \اصطلاح{سمتی میدان}\فرہنگ{میدان!سمتی}\حاشیہب{vector field}\فرہنگ{field!vector} سے مراد ایسا تفاعل ہے جو دائرہ کار کے ہر نقطہ کو ایک سمتیہ مختص کرتا ہو۔سہ ابعادی سمتیات کے میدان کا ایک کلیہ درج ذیل ہو سکتا ہے۔
\begin{align*}
\kvec{F}(x,y,z)=M(x,y,z)\ai+N(x,y,z)\aj+P(x,y,z)\ak
\end{align*}
استمراری جزوی تفاعل \عددی{M}، \عددی{N}، \عددی{P} کی صورت میں یہ میدان استمراری ہو گا، قابل تفرق \عددی{M}، \عددی{N}، \عددی{P} کی صورت میں یہ میدان قابل تفرق ہو گا، وغیرہ وغیرہ۔ دو ابعادی سمتیات کے میدان کا ایک کلیہ درج ذیل ہو سکتا ہے۔
\begin{align*}
\kvec{F}(x,y)=M(x,y)\ai+N(x,y)\aj
\end{align*}
گول انداز کی گزرگاہ کے مستوی میں  گزرگاہ کے ہر نقطہ کے ساتھ گول انداز کا سمتی رفتاری سمتیہ منسلک کرنے سے گزرگاہ  کی ہمراہ دو ابعادی میدان حاصل ہو گا۔  غیر سمتی تفاعل کے ہم قد سطح کے ہر نقطہ کے ساتھ  تفاعل کا سمتیہ ڈھلوان منسلک کرنے سے سطح پر سہ ابعادی میدان حاصل ہو گا۔   متحرک سیال کے ہر نقطہ کے ساتھ سمتی رفتاری سمتیہ منسلک کرنے سے فضا میں اس خطہ پر سہ ابعادی  میدان حاصل ہو گا۔ بشمول ان  کے چند  میدان  شکل میں دکھائے گئے ہیں جہاں کچھ میدانوں کے کلیات بھی دیے گئے ہیں۔
 

وہ میدان ترسیم کرنے کے لئے جن کے کلیات معلوم ہوں، ہم دائرہ کار میں چند نقطے منتخب کر کے ان نقطوں پر نقطوں کے ساتھ منسلک سمتیات  کا خاکہ بناتے ہیں۔ دھیان رہے کہ روایتی طور پر اس نقطہ پر، جہاں سمتی تفاعل کی قیمت حاصل کی گئی ہو،  سمتیہ ظاہر کرنے والی تیر دار لکیر  کی دم  رکھی جاتی ہے نا کہ سر۔ تعین گر سمتیات (باب \حوالہ{سمتی_قیمت_تفاعل_اور_فضا_میں_حرکت}) کے لئے ایسا نہیں کیا جاتا ہے بلکہ تعین گر سمتیات کو ظاہر کرنے والی تیر دار لکیر کی دم کو مبدا پر رکھا جاتا ہے جبکہ اس کا سر  سیارہ یا گول انداز کے مقام پر رکھا جاتا ہے۔

\جزوحصہء{میدان ڈھلوان}
\ابتدا{تعریف}
قابل تفرق تفاعل \عددی{f(x,y,z)} کے \اصطلاح{میدان ڈھلوان}\فرہنگ{میدان!ڈھلوان}\حاشیہب{gradient field}\فرہنگ{gradient!field} سے مراد سمتیات ڈھلوان
\begin{align*}
\nabla f=\frac{\partial f}{\partial x}\ai+\frac{\partial f}{\partial y}\aj+\frac{\partial f}{\partial z}\ak
\end{align*}
 کا میدان ہے۔
\انتہا{تعریف}
%=======================
\ابتدا{مثال}
تفاعل \عددی{f(x,y,z)=xyz} کا میدان ڈھلوان تلاش کریں۔

حل:\quad
تفاعل \عددی{f} کے میدان ڈھلوان سے مراد میدان \عددی{\kvec{F}=\nabla f=yz\ai+xz\aj+xy\ak} ہے۔
\انتہا{مثال}
%=====================

 ہم  دیکھیں گے کہ انجینئری، ریاضیات، طبیعیات، وغیرہ میں میدان ڈھلوان  خصوصی اہمیت رکھتے ہیں۔

\جزوحصہء{فضا میں منحنی کی ہمراہ قوت کا کام}
فرض کریں  فضا کے ایک خطہ  میں سمتی میدان \عددی{\kvec{F}=M(x,y,z)\ai+N(x,y,z)\aj+P(x,y,z)\ak}  ایک قوت کو ظاہر کرتا ہے (یہ قوت ثقل  یا کسی قسم کی برقناطیسی قوت ہو سکتی ہے) جبکہ اس خطہ میں درج ذیل ایک ہموار منحنی ہے۔
\begin{align*}
\kvec{r}(t)=g(t)\ai+h(t)\aj+k(t)\ak,\quad a\le t\le b
\end{align*}
ایسی صورت میں منحنی پر \عددی{\kvec{F}\cdot\kvec{T}}،  اکائی مماسی سمتیہ کے رخ \عددی{\kvec{F}} کا غیر سمتی جزو، کے تکمل کو \عددی{a} تا \عددی{b} \عددی{\kvec{F}} کا کام کہتے ہیں۔

\ابتدا{تعریف}
ہموار منحنی \عددی{\kvec{r}(t)=g(t)\ai+h(t)\aj+k(t)\ak} پر \عددی{a} تا \عددی{b} قوت
 \عددی{\kvec{F}=M(x,y,z)\ai+N(x,y,z)\aj+P(x,y,z)\ak} کا کام \عددی{W} درج ذیل ہو گا۔
\begin{align}\label{مساوات_سمتی_میدان_کام_کی_تعریف}
W=\int_{t=1}^{t=b}\kvec{F}\cdot \kvec{T}\dif s
\end{align}
\انتہا{تعریف}
%=======================

 مثبت محور \عددی{x} رخ  مقدار \عددی{F(x)} کی استمراری قوت کے کام کا کلیہ  \عددی{W=\int_a^bF(x)\dif x} حصہ \حوالہ{حصہ_تکمل_استعمال_کام} میں اخذ کیا گیا۔مساوات \حوالہ{مساوات_سمتی_میدان_کام_کی_تعریف} کے حصول کے دلائل وہی ہیں۔ ہم منحنی کو چھوٹے قطعات میں تقسیم کر کے، ہر قطع پر کام کو تخمیناً مستقل قوت ضرب فاصلہ لے کر،  نتائج کے مجموعہ کو منحنی پر کام کی تخمینی قیمت حاصل کرتے ہیں، اور قطعات کی تعداد زیادہ سے زیادہ کر کے ہر قطع کی لمبائی کم سے کم کرتے ہوئے، تخمینی مجموعات کی تحدیدی قیمت کو کام کی تعریف لیتے ہیں۔ یہ جاننے کے لئے کہ تحدیدی تکمل کی قیمت کیا ہو گی، ہم قطع \عددی{ِ=[a,b]} کی خانہ بندی عمومی طرح  کر کے ہر ذیلی قطع \عددی{[t_k,t_{k+1}]} پر نقطہ \عددی{c_k} منتخب کرتے ہیں۔ قطع \عددی{I} کی خانہ بندی  منحنی کی خانہ بندی تعین (پیدا) کرتی ہے، جہاں تعین گر سمتیہ \عددی{\kvec{r}} کا سر نقطہ \عددی{N_k} پر ہو گا اور ذیلی قطع \عددی{N_kN_{k+1}} کی لمبائی \عددی{\Delta s_k} ہو گی۔ اگر منحنی پر \عددی{t=c_k} کے مطابقتی نقطہ  پر \عددی{\kvec{F}} کی قیمت \عددی{\kvec{F}_k} اور منحنی کا مماسی سمتیہ \عددی{\kvec{T}_k} ہو تب \عددی{t=c_k}  پر \عددی{\kvec{T}} کے رخ \عددی{\kvec{F}} کا غیر سمتی جزو \عددی{\kvec{F}_k\cdot\kvec{T}_k} ہو گا۔ منحنی کے قطع \عددی{N_kN_{k+1}} کی ہمراہ \عددی{\kvec{F}} کا کام تخمیناً درج ذیل ہو گا۔
\begin{align*}
(\text{\RL{حرکت کے رخ قوت کا جزو}})\times (\text{\RL{طے فاصلہ}})=\kvec{F}_k\cdot\kvec{T}_k\Delta s_k
\end{align*}
منحنی کی ہمراہ  \عددی{t=a} تا \عددی{t=b} کام تخمیناً درج ذیل ہو گا۔
\begin{align*}
\sum_{k=1}^{n}\kvec{F}_k\cdot\kvec{T}_k\Delta s_k
\end{align*}
جیسا جیسا \عددی{[a,b]} کے خانہ بندی کا معیار صفر کے قریب سے قریب ہوتا  ہے، ویسے ویسے منحنی کی پیدا کردہ خانہ بندی کا معیار بھی صفر کے قریب سے قریب ہوتا ہے اور مجموعہ درج ذیل لکیری تکمل کو پہنچتا ہے۔
\begin{align*}
\int_{t=a}^{t=b}\kvec{F}\cdot\kvec{T}\dif s
\end{align*} 
اس تکمل سے حاصل عدد کی علامت، \عددی{t} بڑھانے سے حاصل  پر چلنے کے، رخ پر منحصر ہو گی۔ منحنی پر چلنے کا رخ الٹ کرنے سے \عددی{\kvec{T}} کا رخ الٹ ہو گا جس کی بنا \عددی{\kvec{F}\cdot\kvec{T}} اور تکمل کی علامت الٹ ہو گی۔ 

\جزوحصہء{علامتیت اور قیمت کا حصول}
تکمل کام (مساوات \حوالہ{مساوات_سمتی_میدان_کام_کی_تعریف}) کو لکھنے کے چھ طریقے درج ذیل ہیں۔
\begin{gather}
\begin{aligned}\label{مساوات_سمتی_تکمل_متعدد_روپ_کام}
W&=\int_{t=a}^{t=b}\kvec{F}\cdot\kvec{T}\dif s&&\text{\RL{تعریف}}\\
&=\int_{t=a}^{t=b}\kvec{F}\cdot\dif \kvec{r}&&\text{\RL{مختصر تفریقی روپ}}\\
&=\int_{t=a}^{t=b}\kvec{F}\cdot\frac{\dif \kvec{r}}{\dif t}\dif t&&
\text{\RL{پھیلا کر \عددی{\dif t}  شامل کیا گیا ہے}}\\
&=\int_{t=a}^{t=b}\big(M\frac{\dif g}{\dif t}+N\frac{\dif h}{\dif t}+P\frac{\dif k}{\dif t}\big)\dif t&&\text{\RL{مقدار معلوم \عددی{t} اور سمتی رفتار \عددی{\frac{\dif\kvec{r}}{\dif t}} اجاگر کیے گئے}}\\
&=\int_{t=a}^{t=b}\big(M\frac{\dif x}{\dif t}+N\frac{\dif y}{\dif t}+P\frac{\dif z}{\dif t}\big)\dif t&&\text{\RL{جزوی تفاعل اجاگر کیے گئے}}\\
&=\int_{t=a}^{t=b} M\dif x+N\dif y+P\dif z&&\text{\RL{\عددی{\dif t} منسوخ کر کے عمومی روپ حاصل کی گئی}}
\end{aligned}
\end{gather}

مساوات \حوالہ{مساوات_سمتی_تکمل_متعدد_روپ_کام} کے کلیات کی  قیمتوں کا حصول،  بظاہر مختلف روپ  کے باوجود،  ایک ہی طرح کیا جاتا ہے۔ 

\موٹا{تکمل کام کی قیمت کا حصول}\\
تکمل کام کی قیمت حاصل کرنے کے اقدام درج ذیل ہیں۔
\begin{enumerate}[1.]
\item
منحنی پر \عددی{\kvec{F}} کی قیمت مقدار معلوم \عددی{t} کے تفاعل کی روپ میں لکھیں۔
\item
تفرق \عددی{\frac{\dif \kvec{r}}{\dif t}} تلاش کریں۔
\item
\عددی{\kvec{F}} اور \عددی{\frac{\dif\kvec{r}}{\dif t}} کا غیر سمتی ضرب لیں۔
\item
\عددی{t=a} سے \عددی{t=b} تک تکمل کریں۔
\end{enumerate}

%==================
\ابتدا{مثال}
منحنی \عددی{\kvec{r}(t)=t\ai+t^2\aj+t^3\ak,\,0\le t\le 1} کی ہمراہ  \عددی{(0,0,0)} سے  \عددی{(1,1,1)}  تک \عددی{\kvec{F}=(y-x^2)\ai+(z-y^2)\aj+(x-z^2)\ak} کا کام  تلاش کریں۔

حل:\\
پہلا قدم:\quad
منحنی پر \عددی{\kvec{F}} کی قیمت کا حصول۔
\begin{align*}
\kvec{F}&=(y-x^2)\ai+(z-y^2)\aj+(x-z^2)\ak\\
&=(\underbrace{t^2-t^2}_0)\ai+(t^3-t^4)\aj+(t-t^6)\ak
\end{align*}
دوسرا قدم:\quad
\عددی{\frac{\dif \kvec{r}}{\dif t}} حاصل کرتے ہیں۔
\begin{align*}
\frac{\dif \kvec{r}}{\dif t}&=\frac{\dif}{\dif t}(t\ai+t^2\aj+t^3\ak)=\ai+2t\aj+3t^2\ak
\end{align*}
تیسرا قدم:\quad
\عددی{\kvec{F}} اور \عددی{\frac{\dif\kvec{r}}{\dif t}} کا غیر سمتی ضرب۔
\begin{align*}
\kvec{F}\cdot\frac{\dif\kvec{r}}{\dif t}&=[(t^3-t^4)\aj+(t-t^6)\ak]\cdot(\ai+2t\aj+3t^2\ak)\\
&=(t^3-t^4)(2t)+(t-t^6)(3t^2)=2t^4-2t^5+3t^3-3t^8
\end{align*}
چوتھا قدم:\quad
\عددی{t=0} تا \عددی{t=1} تکمل۔
\begin{align*}
\text{کام}=W&=\int_0^1(2t^4-2t^5+3t^3-3t^8)\dif t\\
&=\big[\frac{2}{5}t^5-\frac{2}{6}t^6+\frac{3}{4}t^4-\frac{3}{9}t^9\big]_0^1=\frac{29}{60}
\end{align*}
\انتہا{مثال}
%============

\جزوحصہء{تکمل ہمراہ بہاو اور دائری بہاو}
فرض کریں \عددی{\kvec{F}=M\ai+N\aj+P\ak} میدان قوت کی بجائے  فضا کے ایک خطہ (مثلاً پانی سے چلنے والے جنریٹر کا چرخی خانہ   یا سمندری طاس) میں بہتا ہوا سیال کے سمتی رفتاری میدان  کو ظاہر کرتا ہے ۔ ایسی صورت میں منحنی کی ہمراہ 
 \عددی{\kvec{F}\cdot \kvec{T}} کا تکمل، منحنی کی ہمراہ سیال کا بہاو دے گا۔

\ابتدا{تعریف}
استمراری سمتی رفتاری میدان کے دائرہ کار میں ہموار منحنی \عددی{\kvec{r}(t)=g(t)\ai+h(t)\aj+k(t)\ak,\, a\le t\le b} پر \عددی{a} سے \عددی{b} تک \عددی{\kvec{F}\cdot\kvec{T}} کا تکمل،    \عددی{t=a} سے \عددی{t=b} تک  منحنی کا  ہمراہ بہاو دے گا:
\begin{align}
\text{\RL{ہمراہ بہاو}}=\int_a^b \kvec{F}\cdot\kvec{T}\dif s
\end{align}
اس تکمل کو \اصطلاح{تکمل ہمراہ بہاو}\فرہنگ{تکمل!ہمراہ بہاو}\حاشیہب{flow integral}\فرہنگ{flow!integral} کہتے ہیں۔ بند منحنی کی صورت میں اس بہاو کو  منحنی کے گرد منحنی کی ہمراہ \اصطلاح{دائری بہاو}\فرہنگ{دائری بہاو}\حاشیہب{circulation}\فرہنگ{circulation} کہتے ہیں۔
\انتہا{تعریف}
%========================

تکمل ہمراہ بہاو کی قیمت  بھی تکمل کام کی قیمت کی طرح حاصل کی جاتی ہے۔

\ابتدا{مثال}
ایک سیال کا  سمتی رفتاری میدان \عددی{\kvec{F}=x\ai+z\aj+y\ak} ہے۔ درج ذیل پیچدار منحنی کے ساتھ ساتھ اس کی ہمراہ بہاو تلاش کریں۔
\[\kvec{r}(t)=(\cos t)\ai+(\sin t)\aj+t\ak,\, 0\le t\le \tfrac{\pi}{2}\]

حل:\\
پہلا قدم:\quad
منحنی پر \عددی{\kvec{F}} کی قیمت تلاش کرتے ہیں۔
\begin{align*}
\kvec{F}=x\ai+z\aj+y\ak=(\cos t)\ai+(\sin t)\ai+t\ak
\end{align*}
دوسرا قدم:\quad
\عددی{\tfrac{\dif \kvec{r}}{\dif t}} تلاش کرتے ہیں۔
\begin{align*}
\frac{\dif \kvec{r}}{\dif t}=(-\sin t)\ai+(\cos t)\aj+\ak
\end{align*}
تیسرا قدم:\quad
غیر سمتی ضرب \عددی{\kvec{F}\cdot \tfrac{\dif\kvec{r}}{\dif t}} تلاش کرتے ہیں۔
\begin{align*}
\kvec{F}\cdot\frac{\dif \kvec{r}}{\dif t}&=(\cos t)(-\sin t)+(t)(\cos t)+(\sin t)(1)\\
&=-\sin t\cos t+t\cos t+\sin t
\end{align*}
چوتھا قدم:\quad
\عددی{t=a} تا \عددی{t-b} تکمل لیتے ہیں۔
\begin{align*}
\text{بہاو}&=\int_{t=a}^{t=b} \kvec{F}\cdot \frac{\dif \kvec{r}}{\dif t}\dif t=\int_0^{\pi/2}(-\sin t\cos t+t\cos t+\sin t)\dif t\\
&=\big[\frac{\cos^2t}{2} +\sin t\big]_0^{\pi/2}=\big(0+\frac{\pi}{2}\big)-\big(\frac{1}{2}+0\big)=\frac{\pi}{2}-\frac{1}{2}
\end{align*}
\انتہا{مثال}
%==================== 
\ابتدا{مثال}
میدان \عددی{\kvec{F}=(x-y)\ai+x\aj} کا دائرہ \عددی{\kvec{r}(t)=(\cos t)\ai+(\sin t)\aj,\, 0\le t\le 2\pi} کی ہمراہ دائرہ کے گرد دائری بہاو تلاش کریں۔

حل:\quad
\begin{enumerate}[1.]
\item
دائرہ پر \عددی{\kvec{F}=(x-y)\ai+x\aj=(\cos t-\sin t)\ai+(\cos t)\aj} ہو گا۔
\item
\(\frac{\dif \kvec{r}}{\dif t}=(-\sin t)\ai+(\cos t)\aj\)
\item
\(\kvec{F}\cdot \frac{\dif \kvec{r}}{\dif t}=-\sin t\cos t+\underbrace{\sin^2t+\cos^2t}_1\)
\item
دائری بہاو درج ذیل ہو گا۔
\begin{align*}
\text{\RL{دائری بہاو}}&=\int_0^{2\pi}\kvec{F}\cdot \frac{\dif \kvec{r}}{\dif t}\dif t=\int_0^{2\pi}(1-\sin t\cos t)\dif t\\
&=\big[t-\frac{\sin^2t}{2}\big]_0^{2\pi}=2\pi
\end{align*}
\end{enumerate}
\انتہا{مثال}
%==================

\جزوحصہء{مستوی منحنی کو عبور کرتا ہوا  بہاو}
مستوی \عددی{xy} میں ہموار بند منحنی \عددی{C} میں محیط خطہ سے سیال کے اخراج و دخول کی شرح  \عددی{C} پر  \عددی{\kvec{F}\cdot\kvec{n}} ( منحنی کے بیرونی  عمودی اکائی سمتیہ  کے رخ رفتاری میدان کے غیر سمتی جزو)  کے لکیری تکمل سے حاصل ہو گی۔ اس تکمل کی قیمت کو  \عددی{C} عبور کرتا ہوا \عددی{\kvec{F}} کا بہاو (یا \اصطلاح{نفاذ}\فرہنگ{نفاذ}) کہتے ہیں۔ برقی یا مقناطیسی میدان \عددی{\kvec{F}} کی صورت میں بھی اس تکمل کی قیمت کو \عددی{C} عبور کرتا ہوا بہاو یا نفاذ کہیں گے اگرچہ ان میں کوئی بہتا ہوا سیال نہیں پایا جاتا ہے۔

\ابتدا{تعریف}
استمراری سمتی میدان \عددی{\kvec{F}=M(x,y)\ai+N(x,y)\aj} کے دائرہ کار میں ہموار مسطح بند منحنی \عددی{C} جس کا بیرونی رخ عمودی اکائی سمتیہ \عددی{\kvec{n}} ہو کی صورت میں \عددی{C} کو عبور کرتا ہوا \عددی{\kvec{F}} کا \اصطلاح{بہاو}\فرہنگ{بہاو}\حاشیہب{flux}\فرہنگ{flux} درج ذیل تکمل دے گا۔
\begin{align}\label{مساوات_سمتی_تکمل_عبور_کرتا_بہاو_تعریف}
\text{\RL{\(C\) عبور کرتا ہوا \(\kvec{F}\) کا بہاو}}=\int_C \kvec{F}\cdot \kvec{n}\dif s
\end{align}
\انتہا{تعریف}

منحنی کو عبور کرتے ہوئے بہاو (نفاذ) اور دائری بہاو میں فرق سے واقف ہونا ضروری ہے۔ منحنی \عددی{C} کو عبور کرتا ہوا \عددی{\kvec{F}} کے بہاو سے مراد منحنی  پر \عددی{\kvec{F}\cdot\kvec{n}} (منحنی کے بیرونی عمود کے رخ \عددی{\kvec{F}} کے غیر سمتی جزو) کا تکمل ہے۔ بند منحنی \عددی{C} کے گرد \عددی{\kvec{F}} کے دائری بہاو سے مراد منحنی پر \عددی{\kvec{F}\cdot\kvec{T}} (منحنی کے اکائی مماس کے رخ \عددی{\kvec{F}} کے غیر سمتی جزو) کا تکمل ہے۔ منحنی عبور کرتا ہوا بہاو سے مراد \عددی{\kvec{F}} کے عمودی جزو کا تکمل جبکہ دائری بہاو سے مراد \عددی{\kvec{F}} کے مماسی جزو کا تکمل ہے۔ روز مرہ زندگی میں منحنی کو عبور کرتے ہوئے بہاو یعنی نفاذ کو مختصراً \اصطلاح{بہاو} کہا جاتا ہے۔

مساوات \حوالہ{مساوات_سمتی_تکمل_عبور_کرتا_بہاو_تعریف} کے تکمل کی قیمت معلوم کرنے کی خاطر ہم مقدار معلوم روپ 
\begin{align*}
x=g(t),\quad y=h(t),\quad a\le t\le b
\end{align*}
سے ابتدا کرتے ہیں۔یوں \عددی{t} کی قیمت \عددی{a} سے  \عددی{b} تک بڑھانے سے منحنی پر ایک سر سے دوسرے سر تک ٹھیک ایک بار چلا جائے گا۔ منحنی کے اکائی مماسی سمتیہ \عددی{\kvec{T}} اور کارتیسی محددی نظام کے اکائی سمتیہ \عددی{\kvec{k}} کا سمتی ضرب منحنی کا بیرونی اکائی عمودی سمتیہ \عددی{\kvec{n}} دے گا۔ سوال پیدا ہوتا ہے کہ ایسے دو سمتی ضرب \عددی{\kvec{T}\times\kvec{k}} اور \عددی{\kvec{k}\times\kvec{T}} پائے جاتے ہیں۔ان میں کونسا بیرونی اکائی سمتیہ دے گا؟ مقدار معلوم \عددی{t} بڑھانے سے \عددی{C} پر چلنے کے رخ پر اس کا جواب منحصر ہو گا۔ اگر منحنی پر حرکت گھڑی وار ہو تب \عددی{\kvec{k}\times\kvec{T}} بیرونی اکائی سمتیہ دے گا جبکہ خلاف گھڑی حرکت کی صورت میں \عددی{\kvec{T}\times\kvec{k}} بیرونی اکائی سمتیہ دے گا۔ عموماً خلاف گھڑی حرکت کے لئے کلیات اخذ کیے جاتے ہیں۔ یوں  \عددی{\kvec{n}=\kvec{T}\times\kvec{k}} ہو گا۔اگرچہ مساوات \حوالہ{مساوات_سمتی_تکمل_عبور_کرتا_بہاو_تعریف} میں دیے گئے تکمل کی قیمت منحنی پر چلنے کے رخ پر منحصر نہیں ہے، ہم خلاف گھڑی حرکت تصور کرتے ہوئے   مساوات \حوالہ{مساوات_سمتی_تکمل_عبور_کرتا_بہاو_تعریف} کو حل کرنے کے کلیات اخذ کرتے ہیں۔ 

ارکان کی صورت میں درج ذیل ہو گا۔
\begin{align*}
\kvec{n}=\kvec{T}\times\kvec{k}=\big(\frac{\dif x}{\dif s}\ai+\frac{\dif y}{\dif s}\aj\big)\times \kvec{k}=\frac{\dif y}{\dif s}\ai-\frac{\dif x}{\dif s}\aj
\end{align*}
اگر \عددی{\kvec{F}=M(x,y)\ai+N(x,y)\aj} ہو تب
\begin{align*}
\kvec{F}\cdot \kvec{n}=M(x,y)\frac{\dif y}{\dif s}-N(x,y)\frac{\dif x}{\dif s}
\end{align*}
لہٰذا
\begin{align*}
\int_C\kvec{F}\cdot\kvec{n}\dif s=\int_C\big(M\frac{\dif y}{\dif s}-N\frac{\dif x}{\dif s}\big)\dif s
\end{align*}
ہو گا جس سے درج ذیل حاصل ہوتا ہے۔
\begin{align}\label{مساوات_سمتی_تکمل_نفاذ}
\int_C\kvec{F}\cdot\kvec{n}\dif s=\oint_C M\dif y-N\dif x
\end{align}
 یاد رہے کہ \عددی{C} پر خلاف گھڑی چلتے ہوئے  بند تکمل \عددی{\oint} کی قیمت حاصل کی جائے گی۔ اس تکمل کے حصول کی خاطر ہم \عددی{M}، \عددی{\dif y}، \عددی{N} اور \عددی{\dif x} کو مقدار معلوم \عددی{t} کی روپ میں لکھ کر \عددی{t=a} تا \عددی{t=b} تکمل لیتے ہیں۔ہمیں بہاو (نفاذ) تلاش کرنے کے لئے  \عددی{\kvec{n}} یا \عددی{\dif s} جاننے کی ضرورت پیش نہیں آتی ہے۔ منحنی \عددی{C} پر خلاف گھڑی ٹھیک ایک بار چلتے ہوئے  کوئی بھی ہموار مقدار معلوم روپ \عددی{x=g(t)}، \عددی{y=h(t)}، جہاں \عددی{a\le t\le b} ہو، استعمال کیا جا سکتا ہے۔

\ابتدا{مثال}
مستوی \عددی{xy} میں دائرہ \عددی{x^2+y^2=1} کو پار کرتا ہوا \عددی{\kvec{F}=(x-y)\ai+x\aj} کا بہاو (نفاذ) تلاش کریں۔

حل:\quad
مقدار معلوم روپ \عددی{\kvec{r}(t)=(\cos t)\ai+(\sin t)\aj,\, 0\le t\le 2\pi} دائرے پر ٹھیک ایک بار چلتا ہے لہٰذا مساوات \حوالہ{مساوات_سمتی_تکمل_نفاذ} حل کرنے کی خاطر درج ذیل لیے جا سکتے ہیں۔
\begin{align*}
M&=x-y=\cos t-\sin t,&&\dif y=\dif\,(\sin t)=\cos t\dif t\\
N&=x=\cos t,&&\dif x=\dif\,(\cos t)=-\sin t\dif t
\end{align*}
یوں درج ذیل ہو گا۔
\begin{align*}
\text{\RL{}}&=\int_C M\dif y-N\dif x=\int_0^{2\pi}(\cos^2t-\sin t\cos t+\cos t\sin t)\dif t&& \text{\RL{مساوات \حوالہ{مساوات_سمتی_تکمل_نفاذ}}}\\
&=\int_0^{2\pi}\cos^2 t\dif t=\int_0^{2\pi}\frac{1+\cos 2t}{2}\dif t=\big[\frac{t}{2}+\frac{\sin 2t}{4}\big]_0^{2\pi}=\pi
\end{align*}
دائرہ کو عبور کرتا ہوا \عددی{\kvec{F}} کا بہاو \عددی{\pi} ہے۔چونکہ نتیجہ مثبت ہے لہٰذا دائرے سے کل بہاو کا رخ باہر کو ہو گا۔ دائرے میں دخول کی صورت میں نتیجہ منفی ہوتا۔
\انتہا{مثال}
%===========

\جزوحصہء{سوالات}
\موٹا{سمتی میدان اور میدان ڈھلوان}\\
سوال \حوالہ{سوال_سمتی_تکمل_میدان_ڈھلوان_الف} تا سوال \حوالہ{سوال_سمتی_تکمل_میدان_ڈھلوان_ب} میں تفاعل کے میدان ڈھلوان تلاش کریں۔

\ابتدا{سوال}\شناخت{سوال_سمتی_تکمل_میدان_ڈھلوان_الف}
\(f(x,y,z)=(x^2+y^2+z^2)^{-1/2}\)
\انتہا{سوال}
%==================
\ابتدا{سوال}
\(f(x,y,z)=\ln \sqrt{x^2+y^2+z^2}\)
\انتہا{سوال}
%==================
\ابتدا{سوال}
\(g(x,y,z)=e^z-\ln \sqrt{x^2+y^2}\)
\انتہا{سوال}
%==================
\ابتدا{سوال}\شناخت{سوال_سمتی_تکمل_میدان_ڈھلوان_ب}
\(g(x,y,z)=xy+yz+xz\)
\انتہا{سوال}
%==================
\ابتدا{سوال}
مستوی میں میدان کا  ایسا کلیہ \عددی{\kvec{F}=M(x,y)\ai+N(x,y)\aj} پیش کریں کہ \عددی{\kvec{F}} کی مقدار مبدا سے \عددی{(x,y)} تک فاصلہ کے مربع  کا بالعکس متناسب  ہو اور \عددی{\kvec{F}} کا رخ  مبدا کے رخ ہو۔ (یہ میدان مبدا پر غیر معین ہے۔) 
\انتہا{سوال}
%=====================
\ابتدا{سوال}
مستوی میں میدان کا  ایسا کلیہ \عددی{\kvec{F}=M(x,y)\ai+N(x,y)\aj} پیش کریں کہ \عددی{(0,0)} پر  \عددی{\kvec{F}=\kvec{0}} ہو جبکہ کسی دوسرے نقطہ \عددی{(a,b)} پر \عددی{\abs{\kvec{F}}=\sqrt{a^2+b^2}} ہو اور   \عددی{\kvec{F}} دائرہ \عددی{x^2+y^2=a^2+b^2} کو  گھڑی وار مماسی ہو۔
\انتہا{سوال}
%=====================
\موٹا{کام}\\
سوال \حوالہ{سوال_سمتی_تکمل_کام_الف} تا سوال \حوالہ{سوال_سمتی_تکمل_کام_ب} میں  \عددی{(0,0,0)} تا \عددی{(1,1,1)} قوت \عددی{\kvec{F}} کا کام درج ذیل راہوں پر تلاش کریں۔
\begin{enumerate}[a.]
\item
سیدھی لکیر \عددی{C_1:\, \kvec{r}(t)=t\ai+t\aj+t\ak,\, 0\le t\le 1}
\item
قوسی راہ \عددی{C_2:\,\kvec{r}(t)=t\ai+t^2\aj+t^4\ak,\, 0\le t\le 1}
\item
راہ \عددی{C_3\cup C_4} جو \عددی{(0,0,0)} تا \عددی{(1,1,0)} اور \عددی{(1,1,0)} تا \عددی{(1,1,1)} خطی قطعات پر مشتمل ہے۔
\end{enumerate}

\ابتدا{سوال}\شناخت{سوال_سمتی_تکمل_کام_الف}
\(\kvec{F}=3y\ai+2x\aj+4z\ak\)
\انتہا{سوال}
%========================
\ابتدا{سوال}
\(\kvec{F}=\tfrac{1}{1+x^2}\aj\)
\انتہا{سوال}
%========================
\ابتدا{سوال}
\(\kvec{F}=\sqrt{z}\ai-2x\aj+\sqrt{y}\ak\)
\انتہا{سوال}
%========================
\ابتدا{سوال}
\(\kvec{F}=xy\ai+yz\aj+xz\ak\)
\انتہا{سوال}
%========================
\ابتدا{سوال}
\(\kvec{F}=(3x^2-3x)\ai+3z\aj+\ak\)
\انتہا{سوال}
%========================
\ابتدا{سوال}\شناخت{سوال_سمتی_تکمل_کام_ب}
\(\kvec{F}=(y+z)\ai+(z+x)\aj+(x+y)\ak\)
\انتہا{سوال}
%========================
سوال \حوالہ{سوال_سمتی_تکمل_بڑھتا_رخ_کام_الف} تا سوال \حوالہ{سوال_سمتی_تکمل_بڑھتا_رخ_کام_ب} میں بڑھتے \عددی{t} رخ \عددی{\kvec{F}} کا کام تلاش کریں۔

\ابتدا{سوال}\شناخت{سوال_سمتی_تکمل_بڑھتا_رخ_کام_الف}
\(\kvec{F}=xy\ai+y\aj-yz\ak,\, \kvec{r}(t)=t\ai+t^2\aj+t\ak,\, 0\le t\le 1\)
\انتہا{سوال}
%=====================
\ابتدا{سوال}
\(\kvec{F}=2y\ai+3x\aj+(x+y)\ak,\)\\
\( \kvec{r}(t)=(\cos t)\ai+(\sin t)\aj+\tfrac{t}{6}\ak,\, 0\le t\le 2\pi\)
\انتہا{سوال}
%=====================
\ابتدا{سوال}
\(\kvec{F}=z\ai+x\aj+y\ak,\, \kvec{r}(t)=(\sin t)\ai+(\cos t)\aj+t\ak,\, 0\le t\le 2\pi\)
\انتہا{سوال}
%=====================
\ابتدا{سوال}\شناخت{سوال_سمتی_تکمل_بڑھتا_رخ_کام_ب}
\(\kvec{F}=6z\ai+y^2\aj+12x\ak,\, \kvec{r}(t)=(\sin t)\ai+(\cos t)\aj+\tfrac{t}{6}\ak,\, 0\le t\le 2\pi\)
\انتہا{سوال}
%=====================
\موٹا{لکیری تکمل اور مستوی میں سمتی میدان}\\
\ابتدا{سوال}
نقطہ \عددی{(-1,1)} سے \عددی{(2,4)} تک منحنی \عددی{y=x^2} پر \عددی{\int_Cxy\dif x+(x+y)\dif y} کی قیمت تلاش کریں۔
\انتہا{سوال}
%=========================
\ابتدا{سوال}
ایک مثلث جس کے راس \عددی{(0,0)}، \عددی{(1,0)} اور \عددی{(0,1)} ہیں پر  \عددی{\int_C(x-y)\dif x+(x+y)\dif y} کی قیمت خلاف گھڑی چلتے ہوئے  تلاش کریں۔
\انتہا{سوال}
%=====================
\ابتدا{سوال}
نقطہ \عددی{(4,2)} سے \عددی{(1,-1)} تک منحنی \عددی{x=y^2} پر چلتے ہوئے سمتی میدان \عددی{\kvec{F}=x^2\ai-y\aj} کے لئے \عددی{\int_C\kvec{F}\cdot\kvec{T}\dif s} کی قیمت تلاش کریں۔ 
\انتہا{سوال}
%=====================
\ابتدا{سوال}
نقطہ \عددی{(1,0)} سے \عددی{(0,1)} تک اکائی دائرہ \عددی{x^2+y^2=1} پر چلتے ہوئے سمتی میدان \عددی{\kvec{F}=y\ai-x\aj} کے لئے \عددی{\int_C\kvec{F}\cdot\dif \kvec{\kvec{r}}} کی قیمت خلاف گھڑی چلتے ہوئے تلاش کریں۔ 
\انتہا{سوال}
%=====================
