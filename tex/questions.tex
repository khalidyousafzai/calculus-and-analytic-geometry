\جزوحصہء{سوالات}
\موٹا{ترسیم کی پہچان}\\
سوال \حوالہ{سوال_مخروط_ہم_پلہ_الف} تا سوال \حوالہ{سوال_مخروط_ہم_پلہ_الف} میں دیے قطع مکافی کا ہم پلہ درج ذیل میں تلاش کریں۔
\begin{align*}
x^2=2y,\quad x^2=-6y,\quad y^2=8x,\quad y^2=-4x
\end{align*}
اس کے بعد قطع مکافی کے ماسکہ اور ناظمہ دریافت کریں۔

\ابتدا{سوال}\شناخت{سوال_مخروط_ہم_پلہ_الف}
شکل \حوالہ{شکل_مخروط_ترخیم_بائیں_دائیں}-ا
\انتہا{سوال}
%===================
\ابتدا{سوال}
شکل \حوالہ{شکل_مخروط_ترخیم_بائیں_دائیں}-ب
\انتہا{سوال}
%===================
\ابتدا{سوال}
شکل \حوالہ{شکل_مخروط_ترخیم_بائیں_دائیں}-ج
\انتہا{سوال}
%===================
\ابتدا{سوال}\شناخت{سوال_مخروط_ہم_پلہ_ب}
شکل \حوالہ{شکل_مخروط_ترخیم_بائیں_دائیں}-د
\انتہا{سوال}
%===================
\begin{figure}
\centering
\begin{subfigure}{0.45\textwidth}
\centering
\begin{tikzpicture}[declare function={f(\x)=2*sqrt(\x);}]
\begin{axis}[small,axis lines=middle,xlabel={$x$},ylabel={$y$},xlabel style={at={(current axis.right of origin)},anchor=west},ylabel style={at={(current axis.above origin)},anchor=south},xtick={\empty},ytick={\empty},enlargelimits]
\addplot[domain=0:0.2]{f(x)};
\addplot[domain=0.2:2]{f(x)};
\addplot[domain=0:0.2]{-f(x)};
\addplot[domain=0.2:2]{-f(x)};
\end{axis}
\end{tikzpicture}
\caption{}
\end{subfigure}\hfill
\begin{subfigure}{0.45\textwidth}
\centering
\begin{tikzpicture}[declare function={f(\x)=2*sqrt(-\x);}]
\begin{axis}[small,axis lines=middle,xlabel={$x$},ylabel={$y$},xlabel style={at={(current axis.right of origin)},anchor=west},ylabel style={at={(current axis.above origin)},anchor=south},xtick={\empty},ytick={\empty},enlargelimits=true]
\addplot[domain=0:-0.2]{f(x)};
\addplot[domain=-0.2:-2]{f(x)};
\addplot[domain=0:-0.2]{-f(x)};
\addplot[domain=-0.2:-2]{-f(x)};
\end{axis}
\end{tikzpicture}
\caption{}
\end{subfigure}
\begin{subfigure}{0.45\textwidth}
\centering
\begin{tikzpicture}[declare function={f(\x)=-1/4*\x^2;}]
\begin{axis}[small,axis lines=middle,xlabel={$x$},ylabel={$y$},xlabel style={at={(current axis.right of origin)},anchor=west},ylabel style={at={(current axis.above origin)},anchor=south},xtick={\empty},ytick={\empty},enlargelimits]
\addplot[domain=0:0.2]{f(x)};
\addplot[domain=0.2:2]{f(x)};
\addplot[domain=0:0.2](-x,{f(x)});
\addplot[domain=0.2:2](-x,{f(x)});
\end{axis}
\end{tikzpicture}
\caption{}
\end{subfigure}\hfill
\begin{subfigure}{0.45\textwidth}
\centering
\begin{tikzpicture}[declare function={f(\x)=1/4*\x^2;}]
\begin{axis}[small,axis lines=middle,xlabel={$x$},ylabel={$y$},xlabel style={at={(current axis.right of origin)},anchor=west},ylabel style={at={(current axis.above origin)},anchor=south},xtick={\empty},ytick={\empty},enlargelimits=true]
\addplot[domain=0:0.2]{f(x)};
\addplot[domain=0.2:2]{f(x)};
\addplot[domain=0:0.2](-x,{f(x)});
\addplot[domain=0.2:2](-x,{f(x)});
\end{axis}
\end{tikzpicture}
\caption{}
\end{subfigure}
\caption{ترسیم برائے سوال \حوالہ{سوال_مخروط_ہم_پلہ_الف} تا سوال \حوالہ{سوال_مخروط_ہم_پلہ_ب}}
\label{شکل_مخروط_ترخیم_بائیں_دائیں}
\end{figure}

سوال \حوالہ{سوال_مخروط_ہم_پلہ_تلاش_الف} تا سوال \حوالہ{سوال_مخروط_ہم_پلہ_تلاش_ب} میں دیے مخروط کا درج ذیل میں ہم پلہ مساوات تلاش کریں۔
\begin{align*}
\frac{x^2}{4}+\frac{y^2}{9}=1,\quad \frac{x^2}{2}+y^2=1,\quad \frac{y^2}{4}-x^2=1,\quad \frac{x^2}{4}-\frac{y^2}{9}=1
\end{align*}
دیے گئے مخروط کا ماسکہ اور راس تلاش کریں۔ اگر قطع زائد دیا گیا ہو تب اس کے متقارب بھی دریافت کریں۔

\ابتدا{سوال}\شناخت{سوال_مخروط_ہم_پلہ_تلاش_الف}
ترسیم شکل \حوالہ{شکل_مخروط_کئی_الف}-ا میں دیا گیا ہے
\انتہا{سوال}
%===================
\ابتدا{سوال}
ترسیم شکل \حوالہ{شکل_مخروط_کئی_الف}-ب میں دیا گیا ہے
\انتہا{سوال}
%========================
\ابتدا{سوال}
ترسیم شکل \حوالہ{شکل_مخروط_کئی_الف}-ج میں دیا گیا ہے
\انتہا{سوال}
%========================
\ابتدا{سوال}\شناخت{سوال_مخروط_ہم_پلہ_تلاش_ب}
ترسیم شکل \حوالہ{شکل_مخروط_کئی_الف}-د میں دیا گیا ہے
\انتہا{سوال}
%========================
\begin{figure}
\centering
\begin{subfigure}{0.45\textwidth}
\centering
\begin{tikzpicture}[declare function={f(\x)=sqrt(5)/2*sqrt(\x^2-4);}]
\begin{axis}[small,axis lines=middle,xlabel={$x$},ylabel={$y$},xlabel style={at={(current axis.right of origin)},anchor=west},ylabel style={at={(current axis.above origin)},anchor=south},xtick={\empty},ytick={\empty},enlargelimits]
\addplot[domain=2:2.2]{f(x)};
\addplot[domain=2.2:6]{f(x)};
\addplot[domain=2:2.2]{-f(x)};
\addplot[domain=2.2:6]{-f(x)};
\addplot[domain=2:2.2](-x,{f(x)});
\addplot[domain=2.2:6](-x,{f(x)});
\addplot[domain=2:2.2](-x,{-f(x)});
\addplot[domain=2.2:6](-x,{-f(x)});
\end{axis}
\end{tikzpicture}
\caption{}
\end{subfigure}\hfill
\begin{subfigure}{0.45\textwidth}
\centering
\begin{tikzpicture}[declare function={f(\x)=4/3*sqrt(9-\x^2);}]
\begin{axis}[axis equal,small,axis lines=middle,xlabel={$x$},ylabel={$y$},xlabel style={at={(current axis.right of origin)},anchor=west},ylabel style={at={(current axis.above origin)},anchor=south},xtick={\empty},ytick={\empty},enlargelimits]
\addplot[domain=-3:-2.8]{f(x)};
\addplot[domain=-2.8:2.8]{f(x)};
\addplot[domain=2.8:3]{f(x)};
\addplot[domain=-3:-2.8]{-f(x)};
\addplot[domain=-2.8:2.8]{-f(x)};
\addplot[domain=2.8:3]{-f(x)};
\end{axis}
\end{tikzpicture}
\caption{}
\end{subfigure}
\begin{subfigure}{0.45\textwidth}
\centering
\begin{tikzpicture}[declare function={f(\x)=3/4*sqrt(16-\x^2);}]
\begin{axis}[axis equal,small,axis lines=middle,xlabel={$x$},ylabel={$y$},xlabel style={at={(current axis.right of origin)},anchor=west},ylabel style={at={(current axis.above origin)},anchor=south},xtick={\empty},ytick={\empty},enlargelimits]
\addplot[domain=-4:-3.8]{f(x)};
\addplot[domain=-3.8:3.8]{f(x)};
\addplot[domain=3.8:4]{f(x)};
\addplot[domain=-4:-3.8]{-f(x)};
\addplot[domain=-3.8:3.8]{-f(x)};
\addplot[domain=3.8:4]{-f(x)};
\end{axis}
\end{tikzpicture}
\caption{}
\end{subfigure}\hfill
\begin{subfigure}{0.45\textwidth}
\centering
\begin{tikzpicture}[declare function={f(\x)=2/sqrt(5)*sqrt(5+\x^2);}]
\begin{axis}[axis equal,small,axis lines=middle,xlabel={$x$},ylabel={$y$},xlabel style={at={(current axis.right of origin)},anchor=west},ylabel style={at={(current axis.above origin)},anchor=south},xtick={\empty},ytick={\empty},enlargelimits]
\addplot[domain=-4:4]{f(x)};
\addplot[domain=-4:4]{-f(x)};
\end{axis}
\end{tikzpicture}
\caption{}
\end{subfigure}
\caption{ترسیمات برائے سوال \حوالہ{سوال_مخروط_ہم_پلہ_تلاش_الف} تا سوال \حوالہ{سوال_مخروط_ہم_پلہ_تلاش_ب}}
\label{شکل_مخروط_کئی_الف}
\end{figure}

\موٹا{قطع مکافی}\\
سوال \حوالہ{سوال_مخروط_قطع_مکافی_دیا_الف} تا سوال \حوالہ{سوال_مخروط_قطع_مکافی_دیا_ب} میں دیے گئے قطع مکافی کا ماسکہ اور ناظمہ تلاش کرنے کے بعد  اس کو ترسیم کریں۔ ماسکہ اور ناظمہ کو بھی ترسیم میں شامل کریں۔

\ابتدا{سوال}\شناخت{سوال_مخروط_قطع_مکافی_دیا_الف}
$y^2=12x$
\انتہا{سوال}
%=====================
\ابتدا{سوال}
$x^2=6y$
\انتہا{سوال}
%=====================
\ابتدا{سوال}
$x^2=-8y$
\انتہا{سوال}
%=====================
\ابتدا{سوال}
$y^2=-2x$
\انتہا{سوال}
%=====================
\ابتدا{سوال}
$y=4x^2$
\انتہا{سوال}
%=====================
\ابتدا{سوال}
$y=-8x^2$
\انتہا{سوال}
%=====================
\ابتدا{سوال}
$x=-3y^2$
\انتہا{سوال}
%=====================
\ابتدا{سوال}\شناخت{سوال_مخروط_قطع_مکافی_دیا_ب}
$x=2y^2$
\انتہا{سوال}
%=====================

\موٹا{ترخیم}\\
سوال \حوالہ{سوال_مخروط_ترخیم_دیا_الف} تا سوال \حوالہ{سوال_مخروط_ترخیم_دیا_ب} میں دیے گئے ترخیم کی مساوات کو معیاری روپ میں لکھ کر ترسیم کر کے ترسیم پر ماسکہ دکھائیں۔ 

\ابتدا{سوال}\شناخت{سوال_مخروط_ترخیم_دیا_الف}
$16x^2+25y^2=400$
\انتہا{سوال}
%===================
\ابتدا{سوال}
$7x^2+16y^2=112$
\انتہا{سوال}
%=====================
\ابتدا{سوال}
$2x^2+y^2=2$
\انتہا{سوال}
%=====================
\ابتدا{سوال}
$2x^2+y^2=4$
\انتہا{سوال}
%=====================
\ابتدا{سوال}
$3x^2+2y^2=6$
\انتہا{سوال}
%=====================
\ابتدا{سوال}
$9x^2+10y^2=90$
\انتہا{سوال}
%=====================
\ابتدا{سوال}
$6x^2+9y^2=54$
\انتہا{سوال}
%=====================
\ابتدا{سوال}\شناخت{سوال_مخروط_ترخیم_دیا_ب}
$169x^2+25y^2=4225$
\انتہا{سوال}
%=====================
