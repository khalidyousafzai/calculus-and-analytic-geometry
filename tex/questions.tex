\جزوحصہء{سوالات}
\موٹا{حد کی قیمت کی تلاش}\\
سوال \حوالہ{سوال_کثیرالمتغیر_حد_قیمت_تلاش_الف} تا سوال \حوالہ{سوال_کثیرالمتغیر_حد_قیمت_تلاش_ب} میں حد کی قیمت تلاش کریں۔

\ابتدا{سوال}\شناخت{سوال_کثیرالمتغیر_حد_قیمت_تلاش_الف}
$\lim\limits_{(x,y)\to(0,0)}\frac{23x^2-y^2+5}{x^2+y^2+2}$
\انتہا{سوال}
%==================
\ابتدا{سوال}
$\lim\limits_{(x,y)\to(0,4)}\frac{x}{\sqrt{y}}$
\انتہا{سوال}
%===================
\ابتدا{سوال}
$\lim\limits_{(x,y)\to(3,4)}\sqrt{x^2+y^2-1}$
\انتہا{سوال}
%===================
\ابتدا{سوال}
$\lim\limits_{(x,y)\to(2,-3)}\big(\frac{1}{x}+\frac{1}{y}\big)^2$
\انتہا{سوال}
%===================
\ابتدا{سوال}
$\lim\limits_{(x,y)\to(0,\pi/2)}\sec x\tan y$
\انتہا{سوال}
%===================
\ابتدا{سوال}
$\lim\limits_{(x,y)\to(0,0)}\cos\frac{x^2+y^3}{x+y+1}$
\انتہا{سوال}
%===================
\ابتدا{سوال}
$\lim\limits_{(x,y)\to(0,\ln 2)}e^{x-y}$
\انتہا{سوال}
%===================
\ابتدا{سوال}
$\lim\limits_{(x,y)\to(1,1)}\ln\abs{1+x^2y^2}$
\انتہا{سوال}
%===================
\ابتدا{سوال}
$\lim\limits_{(x,y)\to(0,0)} \frac{e^y\sin x}{x}$
\انتہا{سوال}
%===================
\ابتدا{سوال}
$\lim\limits_{(x,y)\to(1,1)}\cos\sqrt[3]{\abs{xy}-1}$
\انتہا{سوال}
%===================
\ابتدا{سوال}
$\lim\limits_{(x,y)\to(1,0)}\frac{x\sin y}{x^2+1}$
\انتہا{سوال}
%===================
\ابتدا{سوال}\شناخت{سوال_کثیرالمتغیر_حد_قیمت_تلاش_ب}
$\lim\limits_{(x,y)\to(\pi/2,0)}\frac{\cos y+1}{y-\sin x}$
\انتہا{سوال}
%===================

\موٹا{حاصل تقسیم کے حد}\\
حاصل تقسیم کو ترتیب دیتے ہوئے سوال \حوالہ{سوال_کثیرالمتغیر_ترتیب_حد_تلاش_الف} تا سوال \حوالہ{سوال_کثیرالمتغیر_ترتیب_حد_تلاش_ب} میں حد تلاش کریں۔

\ابتدا{سوال}\شناخت{سوال_کثیرالمتغیر_ترتیب_حد_تلاش_الف}
$\lim\limits_{(x,y)\to(1,1)}\frac{x^2-2xy+y^2}{x-y}$
\انتہا{سوال}
%=================
\ابتدا{سوال}
$\lim\limits_{\substack{(x,y)\to(1,1)\\ x\ne y}}\frac{x^2-y^2}{x-y}$
\انتہا{سوال}
%=================
\ابتدا{سوال}
$\lim\limits_{\substack{(x,y)\to(1,1)}\\x\ne 1}\frac{xy-y-2x+2}{x-1}$
\انتہا{سوال}
%=================
\ابتدا{سوال}
$\lim\limits_{\substack{(x,y)\to(2,-4)\\y\ne -4,\, x\ne x^2}}\frac{y+4}{x^2y-xy+4x^2-4x}$
\انتہا{سوال}
%=================
\ابتدا{سوال}
$\lim\limits_{\substack{(x,y)\to(0,0)\\x\ne y}}\frac{x-y+2\sqrt{x}-2\sqrt{y}}{\sqrt{x}-\sqrt{y}}$
\انتہا{سوال}
%=================
\ابتدا{سوال}
$\lim\limits_{\substack{(x,y)\to(2,2)\\x+y\ne 4}}\frac{x+y-4}{\sqrt{x+y}-2}$
\انتہا{سوال}
%=================
\ابتدا{سوال}
$\lim\limits_{\substack{(x,y)\to(2,0)\\2x-y\ne 4}}\frac{\sqrt{2x-y}-2}{2x-y-4}$
\انتہا{سوال}
%=================
\ابتدا{سوال}\شناخت{سوال_کثیرالمتغیر_ترتیب_حد_تلاش_ب}
$\lim\limits_{\substack{(x,y)\to(4,3)\\x\ne y+1}}\frac{\sqrt{x}-\sqrt{y+1}}{x-y-1}$
\انتہا{سوال}
%=================

\موٹا{تین متغیرات کے تفاعل کا حد}\\
سوال \حوالہ{سوال_کثیرالمتغیر_تین_متغیر_حد_الف} تا سوال \حوالہ{سوال_کثیرالمتغیر_تین_متغیر_حد_ب} میں حد تلاش کریں۔

\ابتدا{سوال}\شناخت{سوال_کثیرالمتغیر_تین_متغیر_حد_الف}
$\lim\limits_{N\to (1,3,4)}\big(\frac{1}{x}+\frac{1}{y}+\frac{1}{z}\big)$
\انتہا{سوال}
%=====================
\ابتدا{سوال}
$\lim\limits_{N\to(1,-1,-1)}\frac{2xy+yz}{x^2+z^2}$
\انتہا{سوال}
%================
\ابتدا{سوال}
$\lim\limits_{N\to(3,3,0)}(\sin^2x+\cos^2y+\sec^2z)$
\انتہا{سوال}
%======================
\ابتدا{سوال}
$\lim\limits_{N\to(-1/4,\pi/2,2)}\tan^{-1}xyz$
\انتہا{سوال}
%======================
\ابتدا{سوال}
$\lim\limits_{N\to(\pi,0,3)}ze^{-2y}\cos 2x$
\انتہا{سوال}
%======================
\ابتدا{سوال}\شناخت{سوال_کثیرالمتغیر_تین_متغیر_حد_ب}
$\lim\limits_{N\to(0,-2,0)}\ln\sqrt{x^2+y^2+z^2}$
\انتہا{سوال}
%======================

\موٹا{مستوی میں استمرار}\\
سوال \حوالہ{سوال_کثیرالمتغیر_استمراری_سطح_الف} تا سوال \حوالہ{سوال_کثیرالمتغیر_استمراری_سطح_ب} میں  کس نقطہ \عددی{(x,y)} پر مستوی میں تفاعل استمراری ہیں؟

\ابتدا{سوال}\شناخت{سوال_کثیرالمتغیر_استمراری_سطح_الف}
(ا)
$f(x,y)=\sin(x+y)$\quad
(ب)
$f(x,y)=\ln(x^2+y^2)$
\انتہا{سوال}
%==================
\ابتدا{سوال}
(ا)
$f(x,y)=\frac{x+y}{x-y}$\quad
(ب)
$f(x,y)=\frac{y}{x^2+1}$
\انتہا{سوال}
%=================
\ابتدا{سوال}
(ا)
$g(x,y)=\sin\frac{1}{xy}$\quad
(ب)
$g(x,y)=\frac{x+y}{2+\cos x}$
\انتہا{سوال}
%=================
\ابتدا{سوال}\شناخت{سوال_کثیرالمتغیر_استمراری_سطح_ب}
(ا)
$g(x,y)=\frac{x^2+y^2}{x^2-3x+2}$\quad
(ب)
$g(x,y)=\frac{1}{x^2-y}$
\انتہا{سوال}
%=================

\موٹا{فضا میں استمرار}\\
سوال \حوالہ{سوال_کثیرالمتغیر_استمراری_فضا_الف} تا سوال \حوالہ{سوال_کثیرالمتغیر_استمراری_فضا_ب} میں  کس نقطہ \عددی{(x,y,z)} پر فضا  میں تفاعل استمراری ہیں؟

\ابتدا{سوال}\شناخت{سوال_کثیرالمتغیر_استمراری_فضا_الف}
(ا)
$f(x,y,z)=x^2+y^2-2z^2$\quad
(ب)
$f(x,y,z)=\sqrt{x^2+y^2-1}$
\انتہا{سوال}
%=================
\ابتدا{سوال}
(ا)
$f(x,y,z)=\ln xyz$\quad
(ب)
$f(x,y,z)=e^{x+y}\cos z$
\انتہا{سوال}
%=====================
\ابتدا{سوال}
(ا)
$h(x,y,z)=xy\sin\frac{1}{z}$\quad
(ب)
$h(x,y,z)=\frac{1}{x^2+y^2-1}$
\انتہا{سوال}
%=====================
\ابتدا{سوال}\شناخت{سوال_کثیرالمتغیر_استمراری_فضا_ب}
(ا)
$h(x,y,z)=\frac{1}{\abs{y}+\abs{z}}$\quad
(ب)
$h(x,y,z)=\frac{1}{\abs{xy}+\abs{z}}$
\انتہا{سوال}
%=====================

\موٹا{نقطہ پر حد غیر موجود}\\
نقطہ تک   مختلف راہ پر پہنچتے ہوئے سوال \حوالہ{سوال_کثیرالمتغیر_غیر_موجود_حد_الف} تا سوال \حوالہ{سوال_کثیرالمتغیر_غیر_موجود_حد_ب} میں دکھائیں کہ \عددی{(x,y)\to(0,0)} کرتے ہوئے تفاعل کا کوئی حد نہیں پایا جاتا ہے۔

\ابتدا{سوال}\شناخت{سوال_کثیرالمتغیر_غیر_موجود_حد_الف}
$f(x,y)=-\frac{x}{\sqrt{x^2+y^2}}$
\انتہا{سوال}
%===================
\ابتدا{سوال}
$h(x,y)=\frac{x^4}{x^4+y^2}$
\انتہا{سوال}
%=================
\ابتدا{سوال}
$h(x,y)=\frac{x^4-y^2}{x^4+y^2}$
\انتہا{سوال}
%================
\ابتدا{سوال}
$f(x,y)=\frac{xy}{\abs{xy}}$
\انتہا{سوال}
%=================
\ابتدا{سوال}
$g(x,y)=\frac{x-y}{x+y}$
\انتہا{سوال}
%=================
\ابتدا{سوال}
$g(x,y)=\frac{x+y}{x-y}$
\انتہا{سوال}
%=================
\ابتدا{سوال}
$h(x,y)=\frac{x^2+y}{y}$
\انتہا{سوال}
%=================
\ابتدا{سوال}\شناخت{سوال_کثیرالمتغیر_غیر_موجود_حد_ب}
$h(x,y)=\frac{x^2}{x^2-y}$
\انتہا{سوال}
%=================
\موٹا{نظریہ اور مثالیں}\\
\ابتدا{سوال}
کیا  \عددی{\lim_{(x,y)\to(x_0,y_0)}f(x,y)=L} کی صورت میں \عددی{(x_0,y_0)} کا معین ہونا لازمی ہے؟ اپنے جواب کی وجہ پیش کریں۔
\انتہا{سوال}
%=====================
\ابتدا{سوال}
اگر \عددی{f(x_0,y_0)=3} ہو تب درج ذیل کے بارے میں  (ا)   \عددی{(x_0,y_0)} پر استمراری \عددی{f} کی صورت میں،
\begin{align*}
\lim_{(x,y)\to(x_0,y_0)} f(x,y)
\end{align*}
(ب)  \عددی{(x_0,y_0)} پر غیر استمراری \عددی{f} کی صورت میں کیا کہا جا  سکتا ہے۔ اپنے جواب کہ وجہ پیش کریں۔
\انتہا{سوال}
%===============
دو متغیرات کے تفاعل کا مسئلہ بیچ  کہتا ہے کہ اگر ایک قرص، جس کا  مرکز  \عددی{(x_0,y_0)} ہو، کے اندر تمام \عددی{(x,y)\ne (x_0,y_0)} پر \عددی{g(x,y)\le f(x,y)\le h(x,y)} ہو،  اور \عددی{(x,y)\to (x_0,y_0)} کرتے ہوئے  \عددی{g} اور \عددی{h} دونوں کا  حد  متناہی اور \عددی{L}  تب
\begin{align*}
\lim_{(x,y)\to(x_0,y_0)}f(x,y)=L
\end{align*}
ہو گا۔سوال \حوالہ{سوال_کثیرالمتغیر_مسئلہ_بیچ_الف} تا سوال \حوالہ{سوال_کثیرالمتغیر_مسئلہ_بیچ_ب} میں  اس نتیجہ  کا سہارا لیتے ہوئے جواب دیں۔ 

\ابتدا{سوال}\شناخت{سوال_کثیرالمتغیر_مسئلہ_بیچ_الف}
کیا 
\begin{align*}
1-\frac{x^2y^2}{3}<\frac{\tan^{-1}xy}{xy}<1
\end{align*}
جانتے ہوئے آپ
\begin{align*}
\lim\limits_{(x,y)\to(0,0)}\frac{\tan^{-1}xy}{xy}
\end{align*}
کے بارے میں کچھ کہہ سکتے ہیں؟ اپنے جواب کی وجہ پیش کریں۔
\انتہا{سوال}
%============
\ابتدا{سوال}
کیا
\begin{align*}
2\abs{xy}-\frac{x^2y^2}{6}<4-4\cos\sqrt{\abs{xy}}<2\abs{xy}
\end{align*}
جانتے ہوئے
\begin{align*}
\lim\limits_{(x,y)\to(0,0)}\frac{4-4\cos\sqrt{\abs{xy}}}{\abs{xy}}
\end{align*}
کے بارے میں کچھ کہا جا سکتا ہے؟ اپنے جواب کی وجہ پیش کریں۔
\انتہا{سوال}
%===============
\ابتدا{سوال}
کیا \عددی{\abs{\sin(1/y)}\le 1} جانتے ہوئے
\begin{align*}
\lim\limits_{(x,y)\to(0,0)}y\sin\frac{1}{x}
\end{align*}
کے بارے میں کچھ کہا جا سکتا ہے؟ اپنے جواب کی وجہ پیش کریں۔
\انتہا{سوال}
%==============
\ابتدا{سوال}
کیا \عددی{\abs{\cos(1/y)}\le 1} جانتے ہوئے
\begin{align*}
\lim\limits_{(x,y)\to(0,0)}x\cos\frac{1}{y}
\end{align*}
کے بارے میں کچھ کہا جا سکتا ہے؟ اپنے جواب کی وجہ پیش کریں۔
\انتہا{سوال}
%==============
\ابتدا{سوال}
(ا) مثال \حوالہ{مثال_کثیرالمتغیر_حد_یکتا_ضروری} کو دوبارہ پڑھیں۔اب درج ذیل کلیہ میں \عددی{m=\tan\theta} پر کر کے اس کی سادہ صورت حاصل کرتے ہوئے دکھائیں کہ \عددی{f} کی قیمت  لکیر کے زاویہ میلان  پر منحصر ہے۔
\begin{align*}
\left.f(x,y)\right\vert_{y=mx}=\frac{2m}{1+m^2}
\end{align*}
(ب) جزو-ا میں حاصل کلیہ استعمال کرتے ہوئے دکھائیں کہ لکیر \عددی{y=mx} پر چلتے ہوئے \عددی{(x,y)\to (0,0)} کرنے سے \عددی{f} کے  حد کی قیمت \عددی{-1} تا \عددی{1} ہو سکتی ہے جو قریب پہنچنے کی راہ کے  زاویہ پر منحصر ہو گی۔  
\انتہا{سوال}
%=================
\ابتدا{سوال}\شناخت{سوال_کثیرالمتغیر_مسئلہ_بیچ_ب}
\عددی{f(0,0)} کی ایسی تعریف پیش کریں جو درج ذیل کو مبدا پر بھی استمراری بناتا ہو۔
\begin{align*}
f(x,y)=xy\frac{x^2-y^2}{x^2+y^2}
\end{align*}
\انتہا{سوال}
%==============

\موٹا{قطبی محدد میں تبادلہ}\\
