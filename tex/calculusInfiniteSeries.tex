\باب{لامتناہی تسلسل}
اس باب میں ہم ایک حیران کن کلیہ اخذ کرتے ہیں جس کی مدد سے بہت سارے تفاعل کو "لامتناہی کثیر رکنی" کی صورت میں لکھنا ممکن ہو گا اور ساتھ ہی کثیر رکنی کے  ارکان حذف  کر کے کثیر رکنی کو متناہی بنانے سے پیدا خلل بھی جان پائیں گے۔ ان تسلسل کو طاقتی تسلسل کہتے ہیں۔ قابل تفرق تفاعل کو تخمینی طور پر کثیر رکنی سے ظاہر کرنے میں مدد دینے کے علاوہ طاقتی تسلسل دیگر مواقع پر بھی کار آمد ثابت ہوتے ہیں۔ غیر بنیادی تکمل کی قیمت کے حصول کے علاوہ حراری توانائی کی منتقلی، ارتعاش، کیمیائی نفوذ اور ترسیل اشارات کے تفرقی مساوات  کے حل میں یہ موثر کردار ادا کرتے ہیں۔ آپ یہاں وہ ان تفاعل کے بارے میں سیکھ پائیں گے جو سائن اور انجینئری میں بہت زیادہ استعمال ہوتے ہیں۔

\حصہ{اعداد کی ترتیب کی حد}
غیر رسمی طور پر ترتیب سے مراد مرتب چیزوں کا سلسلہ ہے۔ اس باب میں ہمیں اعداد کی ترتیب سے غرض ہو گا۔ ترکیب نیوٹن سے حاصل اعداد کی ترتیب   \عددی{x_0,x_1,\cdots,x_n,\cdots} یا (ہلگے ون کوچ کے) برفانی روئی کے کثیر الاضلاع کی ترتیب \عددی{c_1,c_2,\cdots,c_n,\cdots} ہم دیکھ چکے ہیں۔ ان ترتیبوں کی حد پائی جاتی ہے، البتہ بہت سارے اہم ترتیبوں کے حد نہیں پائے جاتے ہیں۔

\جزوحصہء{تعریف اور علامتیت}
ہم \عددی{3} کے  ہر عدد صحیح مضرب کو ایک مقام مختص کر کے ایک فہرست بنا سکتے ہیں:
\begin{align*}
\begin{array}{rcccccc}
\text{\RL{دائرہ کار}} &1&2&3&\cdots&n&\cdots\\
\text{\RL{سعت}} &3&6&9&&3n&
\end{array}
\end{align*}
پہلا عدد \عددی{3}، دوسرا \عددی{6}، تیسرا \عددی{9}، وغیرہ، وغیرہ ہیں۔ مختص کرنے کا عمل ایک تفاعل ہے جو \عددی{n} ویں مقام کو \عددی{3n} مختص کرتا ہے۔ ترتیب کی بناوٹ کا بنیادی تصور یہی ہے۔ ایک تفاعل ہمیں بتاتا ہے کہ کس مقام پر کونسا عدد ہو گا۔

\ابتدا{تعریف}
ایک تفاعل جس کا دائرہ کار کسی عدد صحیح \عددی{n_0} کے برابر یا اس سے بڑے عدد صحیح پر مشتمل اعداد کا سلسلہ ہو \اصطلاح{لامتناہی ترتیب}\فرہنگ{ترتیب!لامتناہی}\حاشیہب{infinite sequence}\فرہنگ{sequence!infinite} (یا \اصطلاح{ترتیب}\فرہنگ{ترتیب}\حاشیہب{sequence}\فرہنگ{sequence})  کہلاتا ہے۔
\انتہا{تعریف}
%================

عموماً \عددی{n_0=1} ہوتا ہے اور ترتیب کا دائرہ کار مثبت اعداد صحیح پر مشتمل ہو گا۔ البتہ بعض اوقات ہم تسلسل کو کسی دوسرے عدد صحیح سے شروع کرنا چاہتے ہیں۔ ترکیب نیوٹن میں ہم \عددی{n_0=0} لیتے ہیں۔ اگر ہم \عددی{n} اضلاع پر مشتمل کثیر الاضلاع کی ترتیب کی بات کریں تب ہم \عددی{n_0=3} منتخب کرنا چاہیں گے۔

ترتیب کی تعریف کسی بھی تفاعل کی طرح کی جاتی ہے، مثلاً:
\begin{align*}
a(n)=\sqrt{n},\quad a(n)=(-1)^{n+1}\frac{1}{n},\quad a(n)=\frac{n-1}{n}
\end{align*}

یہ ظاہر کرنے کی خاطر کہ دائرہ کار عدد صحیح ہے، ہم حرف \عددی{n} استعمال کرتے ہیں نا کہ دیگر غیر تابع متغیر کے لئے مستعمل حروف \عددی{x}، \عددی{y}، \عددی{z}، \عددی{t}، وغیرہ۔  مذکورہ بالا کی طرح تعریفی قاعدہ میں کلیات عموماً مثبت عدد صحیح سے زیادہ بڑے دائرہ کار کے لئے درست ہوتے ہیں۔ جیسا ہم دیکھیں گے یہ بعض اوقات سود مند ثابت ہوتا ہے۔ 

عدد \عددی{a(n)} ترتیب کا \اصطلاح{\عددی{n} واں جزو} یا اشاریہ \عددی{n} والا جزو ہو گا۔ اگر \عددی{a(n)=\tfrac{n-1}{n}} ہو تب درج ذیل ہو گا۔
\begin{align*}
\begin{array}{ccccc}
\text{\RL{پہلا جزو}}&\text{\RL{دوسرا جزو}}&\text{\RL{تیسرا جزو}}&&\text{\RL{$n$ واں جزو}}\\
\toprule
a(1)=0&a(2)=\frac{1}{2}&a(3)=\frac{2}{3}&\cdots&a(n)=\frac{n-1}{n}
\end{array}
\end{align*}
اشاریہ علامت استعمال کرتے ہوئے ہم \عددی{a(n)} کو \عددی{a_n} لکھتے ہیں۔ اشاریہ علامتی روپ میں یہی ترتیب درج ذیل لکھی جائے گی۔
\begin{align*}
\begin{array}{ccccc}
\text{\RL{پہلا جزو}}&\text{\RL{دوسرا جزو}}&\text{\RL{تیسرا جزو}}&&\text{\RL{$n$ واں جزو}}\\
\toprule
a_1=0&a_2=\frac{1}{2}&a_3=\frac{2}{3}&\cdots&a_n=\frac{n-1}{n}
\end{array}
\end{align*}
ترتیب پر تبصرہ کرتے ہوئے ہم عموماً \عددی{n} ویں جزو کے کلیہ کے ساتھ ساتھ چند ابتدائی اجزاء  لکھتے ہیں۔

\ابتدا{مثال}\شناخت{مثال_تسلسل_متعدد_ترتیبات}
\begin{align*}
\renewcommand{\arraystretch}{2}
\begin{array}{lll}
\text{\RL{اس کے لئے ہم درج ذیل لکھتے ہیں}}&\phantom{kkkk}&\text{\RL{جس ترتیب کا تعریفی کلیہ درج ذیل ہو}}\\
\toprule
1,\sqrt{2},\sqrt{3},\sqrt{4},\cdots,\sqrt{n},\cdots&&a_n=\sqrt{n}\\
1,\frac{1}{2},\frac{1}{3},\cdots,\frac{1}{n},\cdots&&a_n=\frac{1}{n}\\
1,-\frac{1}{2},\frac{1}{3},-\frac{1}{4},\cdots (-1)^{n+1}\frac{1}{n},\cdots&&a_n=(-1)^{n+1}\frac{1}{n}\\
0,\frac{1}{2},\frac{2}{3},\frac{3}{4},\cdots,\frac{n-1}{n},\cdots&&a_n=\frac{n-1}{n}\\
0,-\frac{1}{2},\frac{2}{3},-\frac{3}{4},\cdots,(-1)^{n+1}\big(\frac{n-1}{n}\big),\cdots&&a_n=(-1)^{n+1}\big(\frac{n-1}{n}\big)\\
3,3,3,\cdots,3,\cdots&&a_n=3
\end{array}
\end{align*}
ان تمام ترتیبوں کو دو مختلف انداز میں شکل \حوالہ{شکل_تسلسل_مرتکز_منفرج_الف} تا شکل \حوالہ{شکل_تسلسل_مرتکز_منفرج_ث} میں دکھایا گیا ہے۔
\انتہا{مثال}
%=================== 

\جزوحصہء{علامتیت}
جس ترتیب کا \عددی{n} واں جزو \عددی{a_n} ہو اس ترتیب کو ہم \عددی{\{a_n\}} سے ظاہر کرتے ہیں جو  ترتیب \عددی{a} اشاریہ \عددی{n} پڑھا جاتا ہے۔ مثال \حوالہ{مثال_تسلسل_متعدد_ترتیبات} میں دوسری ترتیب \عددی{\{\tfrac{1}{n}\}} ہے جو ترتیب ایک بٹہ تین پڑھا جاتا ہے۔ آخری ترتیب \عددی{\{3\}} ہے جو مستقل ترتیب \عددی{3} کہلائے گی۔

\begin{figure}
\centering
\begin{subfigure}{0.45\textwidth}
\centering
\begin{tikzpicture}[font=\small,xscale=2]
\draw[-latex](-0.25,0)--(2.5,0);
\foreach \n in {1,2,3,4,5}{\pgfmathsetmacro{\k}{sqrt(\n)} \draw(\k,0)node[circ]{}node[above]{$a_{\n}$};}
\foreach \x in {0,1,2}{\draw(\x,0)++(0,-0.15)node[below]{$\x$}--++(0,0.3);}
\draw(1.25,-0.5)node[below]{$a_n=\sqrt{n}$};
\end{tikzpicture}
\end{subfigure}\hfill
\begin{subfigure}{0.45\textwidth}
\centering
\begin{tikzpicture}[font=\small,xscale=1]
\draw[-latex](-0.25,0)--(5.5,0)node[right]{$n$};
\draw[-latex](0,-0.2)--(0,3.5)node[above]{$a_n$};
\foreach \n in {1,2,3,4,5}{\pgfmathsetmacro{\k}{sqrt(\n)} \draw(\n,\k)node[circ]{}node[above]{$(\n,\sqrt{\n})$};}
\foreach \x in {1,2,3,4,5}{\draw(\x,0)node[below]{$\x$}--++(0,0.2);}
\foreach \y in {1,2,3,}{\draw(0,\y)node[left]{$\y$}--++(0.2,0);}
\draw(3,3)node[above]{منفرج};
\end{tikzpicture}
\end{subfigure}
\caption{جزو $a_n$ آخر کار ہر عدد صحیح سے بڑھتا ہے لہٰذا ترتیب $\{a_n\}$ منفرج ہے۔}
\label{شکل_تسلسل_مرتکز_منفرج_الف}
\end{figure}
%%%%%%%%%%%%%%%%
\begin{figure}
\centering
\begin{subfigure}{0.45\textwidth}
\centering
\begin{tikzpicture}[font=\small,xscale=4]
\draw[-latex](-0.125,0)--(1.125,0);
\foreach \n in {1,2,3,4}{\pgfmathsetmacro{\k}{1/\n} \draw(\k,0)node[circ]{}node[above]{$a_{\n}$};}
\foreach \x in {0,1}{\draw(\x,0)++(0,-0.15)node[below]{$\x$}--++(0,0.3);}
\draw(0.5,-0.5)node[below]{$a_n=\frac{1}{n}$};
\end{tikzpicture}
\end{subfigure}\hfill
\begin{subfigure}{0.45\textwidth}
\centering
\begin{tikzpicture}[font=\small,xscale=1]
\draw[-latex](-0.25,0)--(5.5,0)node[right]{$n$};
\draw[-latex](0,-0.2)--(0,1.5)node[above]{$a_n$};
\foreach \n in {1,2,3,4,5}{\pgfmathsetmacro{\k}{1/\n} \draw(\n,\k)node[circ]{}node[above]{$(\n,\tfrac{1}{\n})$};}
\foreach \x in {1,2,3,4,5}{\draw(\x,0)node[below]{$\x$}--++(0,0.2);}
\foreach \y in {1}{\draw(0,\y)node[left]{$\y$}--++(0.12,0);}
\draw(3,1)node[above]{\RL{$0$ پر مرتکز}};
\end{tikzpicture}
\end{subfigure}
\caption{جزو $a_n=\tfrac{1}{n}$  بتدریج $n$ بڑھنے سے گھٹتے ہوئے $0$ کے قریب پہنچتے ہیں لہٰذا ترتیب $\{a_n\}$ صفر کو مرتکز ہے۔}
\label{شکل_تسلسل_مرتکز_منفرج_ب}
\end{figure}
%%%%%%%%%%%%%%%%%%%%%%%
\begin{figure}
\centering
\begin{subfigure}{0.45\textwidth}
\centering
\begin{tikzpicture}[font=\small,xscale=2]
\draw[-latex](-1.125,0)--(1.125,0);
\foreach \n in {1,2,3,4,5}{\pgfmathsetmacro{\k}{(-1)^(\n+1)/\n} \draw(\k,0)node[circ]{}node[above]{$a_{\n}$};}
\foreach \x in {0,1,-1}{\draw(\x,0)++(0,-0.15)node[below]{$\x$}--++(0,0.3);}
\draw(0,-0.5)node[below]{$a_n=\frac{(-1)^{n+1}}{n}$};
\end{tikzpicture}
\end{subfigure}\hfill
\begin{subfigure}{0.45\textwidth}
\centering
\begin{tikzpicture}[font=\small,xscale=1]
\draw[-latex](-0.25,0)--(5.5,0)node[right]{$n$};
\draw[-latex](0,-0.2)--(0,1.5)node[above]{$a_n$};
\foreach \n in {1,3,5}{\pgfmathsetmacro{\k}{(-1)^(\n+1)/\n} \draw(\n,\k)node[circ]{}node[above]{$(\n,\frac{1}{\n})$};}
\foreach \n in {2,4}{\pgfmathsetmacro{\k}{(-1)^(\n+1)/\n} \draw(\n,\k)node[circ]{}node[below]{$(\n,-\frac{1}{\n})$};}
\foreach \x in {1,2,3,4,5}{\draw(\x,0)--++(0,0.2);}
\foreach \y in {1}{\draw(0,\y)node[left]{$\y$}--++(0.12,0);}
\draw(3,1)node[above]{\RL{$0$ پر مرتکز}};
\end{tikzpicture}
\end{subfigure}
\caption{جزو $\tfrac{(-1)^{n+1}}{n}$ کی علامت ہر مرتبہ تبدیل ہوتی ہے لیکن اس کی قیمت  $0$ پر مرتکز ہے۔}
\label{شکل_تسلسل_مرتکز_منفرج_پ}
\end{figure}
%%%%%%%%%%%%%%%%%%%%%%%
\begin{figure}
\centering
\begin{subfigure}{0.45\textwidth}
\centering
\begin{tikzpicture}[font=\small,xscale=4]
\draw[-latex](-0.125,0)--(1.125,0);
\foreach \n in {1,2,3,4}{\pgfmathsetmacro{\k}{(\n-1)/\n} \draw(\k,0)node[circ]{}node[above]{$a_{\n}$};}
\foreach \x in {0,1}{\draw(\x,0)++(0,-0.15)node[below]{$\x$}--++(0,0.3);}
\draw(0.75,-0.5)node[below]{$a_n=\frac{n-1}{n}$};
\end{tikzpicture}
\end{subfigure}\hfill
\begin{subfigure}{0.45\textwidth}
\centering
\begin{tikzpicture}[font=\small,xscale=1]
\draw[-latex](-0.25,0)--(5.5,0)node[right]{$n$};
\draw[-latex](0,-0.2)--(0,1.5)node[above]{$a_n$};
\draw[gray](0,1)--(5.5,1);
\foreach \n/\nn in {1}{\draw(1,0)node[circ]{}node[above]{$(1,0)$};}
\foreach \n/\nn in {2/1,3/2,4/3,5/4}{\pgfmathsetmacro{\k}{(\n-1)/\n}; \draw(\n,\k)node[circ]{}node[above]{$(\n,\frac{\nn}{\n})$};}
\foreach \x in {1,2,3,4,5}{\draw(\x,0)--++(0,0.2);}
\foreach \y in {1}{\draw(0,\y)node[left]{$\y$}--++(0.12,0);}
\draw(1,1)node[above]{\RL{$1$ پر مرتکز}};
\end{tikzpicture}
\end{subfigure}
\caption{جیسے جیسے $n$ بڑھتا ہے جزو $a_n=\tfrac{n-1}{n}$ بتدریج $1$ تک پہنچتا ہے لہٰذا ترتیب $\{a_n\}$ مرتکز ہے $1$ پر۔}
\label{شکل_تسلسل_مرتکز_منفرج_ت}
\end{figure}
%%%%%%%%%%%%%%%%%%%%%%%%%%%
\begin{figure}
\centering
\begin{subfigure}{0.45\textwidth}
\centering
\begin{tikzpicture}[font=\small,xscale=2]
\draw[-latex](-1.125,0)--(1.125,0);
\foreach \n in {1,2,3,4,5}{\pgfmathsetmacro{\k}{(-1)^(\n+1)*(\n-1)/\n} \draw(\k,0)node[circ]{}node[above]{$a_{\n}$};}
\foreach \x in {-1,0,1}{\draw(\x,0)++(0,-0.15)node[below]{$\x$}--++(0,0.3);}
\draw(0,-0.5)node[below]{$a_n=(-1)^{n+1}\big(\frac{n-1}{n}\big)$};
\end{tikzpicture}
\end{subfigure}\hfill
\begin{subfigure}{0.45\textwidth}
\centering
\begin{tikzpicture}[font=\small,xscale=1]
\draw[-latex](-0.25,0)--(5.5,0)node[right]{$n$};
\draw[-latex](0,-1.25)--(0,1.25)node[above]{$a_n$};
\draw[gray](0,1)node[left,black]{$1$}--(5.5,1)  (0,-1)node[left,black]{$-1$}--(5.5,-1);
\foreach \n/\nn in {1}{\draw(1,0)node[circ]{}node[above]{$(1,0)$};}
\foreach \n/\nn in {3/2,5/4}{\pgfmathsetmacro{\k}{(-1)^(\n+1)*(\n-1)/\n}; \draw(\n,\k)node[circ]{}node[above]{$(\n,\tfrac{\nn}{\n})$};}
\foreach \n/\nn in {2/1,4/3}{\pgfmathsetmacro{\k}{(-1)^(\n+1)*(\n-1)/\n}; \draw(\n,\k)node[circ]{}node[below]{$(\n,-\tfrac{\nn}{\n})$};}
\foreach \x in {1,2,3,4,5}{\draw(\x,0)--++(0,0.2);}
\foreach \y in {3}{\draw(0,\y)node[left]{$\y$}--++(0.12,0);}
\draw(4,1)node[above]{\RL{منفرج}};
\end{tikzpicture}
\end{subfigure}
\caption{جزو $a_n=(-1)^{n+1}[\tfrac{n-1}{n}]$ کی علامت ہر قدم پر تبدیل ہوتی ہے۔ مثبت اجزاء $1$ کو پہنچتے ہیں جبکہ منفی اجزاء $-1$ کو پہنچتے ہیں لہٰذا ترتیب $\{a_n\}$ منفرج ہے۔}
\label{شکل_تسلسل_مرتکز_منفرج_ٹ}
\end{figure}
%%%%%%%%%%%%%%%%%%%%%%%
\begin{figure}
\centering
\begin{subfigure}{0.45\textwidth}
\centering
\begin{tikzpicture}[font=\small,xscale=0.75]
\draw[-latex](-0.25,0)--(5.5,0);
\foreach \n in {3}{\pgfmathsetmacro{\k}{\n} \draw(\k,0)node[circ]{}node[above]{$a_n$};}
\foreach \x in {0,1,2,3,4,5}{\draw(\x,0)++(0,-0.15)node[below]{$\x$}--++(0,0.3);}
\draw(2.5,-0.5)node[below]{$a_n=3$};
\end{tikzpicture}
\end{subfigure}\hfill
\begin{subfigure}{0.45\textwidth}
\centering
\begin{tikzpicture}[font=\small,xscale=1,yscale=0.3]
\draw[-latex](-0.25,0)--(5.5,0)node[right]{$n$};
\draw[-latex](0,-0.2)--(0,4)node[above]{$a_n$};
\foreach \n in {1,2,3,4,5}{\pgfmathsetmacro{\k}{3}; \draw(\n,\k)node[circ]{};}
\foreach \x in {1,2,3,4,5}{\draw(\x,0)--++(0,0.2);}
\foreach \y in {3}{\draw(0,\y)node[left]{$\y$}--++(0.12,0);}
\draw(4.5,1)node[above]{\RL{$3$ پر مرتکز}};
\end{tikzpicture}
\end{subfigure}
\caption{مستقل اجزاء $a_n=3$ کی قیمت $3$ ہی رہتی ہے لہٰذا ترتیب $\{a_n\}$ کی قیمت $3$ پر مرتکز ہے۔}
\label{شکل_تسلسل_مرتکز_منفرج_ث}
\end{figure}
%%%%%%%%%%%%%%%%%%

\جزوحصہء{ارتکاز اور انفراج}

