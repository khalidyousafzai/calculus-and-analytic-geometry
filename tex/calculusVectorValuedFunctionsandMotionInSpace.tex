\باب{سمتی  قیمت تفاعل اور فضا میں حرکت}

\موٹا{سر سری جائزہ}\quad
جب کوئی جسم فضا میں حرکت کرتا ہو،   مساوات   \عددی{x=f(t)}، \عددی{y=g(t)} اور \عددی{z=h(t)}  جو اس جسم کے محدد  کو بطور وقت کا تفاعل  دیتی ہیں،  اس جسم کی راہ اور حرکت کی مقدار معلوم مساوات ہوں گی۔ سمتیہ   علامتیت  کی مدد سے ہم انہیں ایک مساوات \عددی{\kvec{r}(t)=f(t)\ai+g(t)\aj+h(t)\ak} کی صورت میں لکھ سکتے ہیں جو اس جسم کا مقام بطور وقت کا سمتی تفاعل دیتی ہے۔

اس باب میں  ہم احصاء استعمال کرتے ہوئے حرکت پذیر اجسام کی راہ، سمتی رفتار اور اسراع پر غور کریں گے۔ ہم  گولا،  سیارہ  اور مصنوعی سیارہ کی راہ اور حرکت کے عمومی سوالات کے جوابات جان سکے گے۔آخر حصہ میں ہم نیوٹن کے قوانین اور    تجاذب کی مدد سے سیاروں کی مدار کے  قوانین کپلر دریافت کریں گے۔ 


\حصہ{سمتی قیمت تفاعل اور فضائی منحنیات}
فضا میں متحرک ذرہ    کی حرکت جاننے کی خاطر ہم  مبدا سے  اس ذرہ  تک سمتیہ \عددی{\kvec{r}} لے کر \عددی{\kvec{r}} میں تبدیلی پر غور کرتے ہیں (شکل \حوالہ{شکل_سمتی_تفاعل_تعین_گر_سمتیہ})۔اگر اس ذرہ  کے محدد مقام  وقت کے ساتھ دو بار قابل تفرق ہوں،  تب  \عددی{\kvec{r}} بھی ایسا ہو گا، اور ہم کسی بھی لمحہ پر وقت کے لحاظ سے  \عددی{\kvec{r}}   کے  تفرق لے کر اس ذرہ  کی سمتی رفتار اور اسراع جان سکتے ہیں۔اگر ہمیں اس ذرہ  کی سمتیہ  سمتی رفتار یا سمتیہ  اسراع   بطور  وقت کے استمراری تفاعل معلوم ہو اور ہمیں ذرے کی ابتدائی  مقام اور سمتیہ رفتار کے بارے میں معقول معلومات ہو، تب ہم تکمل کی مدد سے، وقت کا تفاعل  \عددی{\kvec{r}} جان سکتے ہیں۔
\begin{figure}
\centering
\begin{minipage}{0.45\textwidth}
\centering
\pgfmathsetmacro{\a}{-0.25/360}
\pgfmathsetmacro{\ka}{1}
\pgfmathsetmacro{\b}{0.5/360}
\pgfmathsetmacro{\kb}{0.5}
\pgfmathsetmacro{\c}{1/360}
\pgfmathsetmacro{\d}{150}
\begin{tikzpicture}[declare function={fx(\r,\t)=(\ka+\a*\t+\r*cos(\t));fy(\r,\t)=(\kb+\b*\t+\r*sin(\t));fz(\r,\t)=\c*\t;}]
\begin{axis}[clip=false,small,axis lines =center,view/h=135,xlabel={$x$},ylabel={$y$},zlabel={$z$},xlabel style={anchor=east},ylabel style={anchor=west},zlabel style={anchor=west},xtick={\empty},ytick={\empty},ztick={\empty},enlargelimits=true,xmin=0,ymin=0,
axis x line=none,axis y line=none,axis z line=none]
\addplot3[smooth,domain y=0:460,variable=\r,variable y=\t,samples y=50]({fx(1,t)},{fy(1,t)},{fz(1,t)});
\addplot3[-latex]coordinates {(0,0,0)({fx(1,\d)},{fy(1,\d)},{fz(1,\d)})}node[pos=0.5,yshift=1ex]{$\kvec{r}$}node[right]{$N(f(t),g(t),h(t))$}node[pos=0,left,yshift=1ex]{$O$};
\addplot3[-latex]coordinates {(0,0,0)(2,0,0)}node[left]{$x$};
\addplot3[-latex]coordinates {(0,0,0)(0,2,0)}node[right]{$y$};
\addplot3[-latex]coordinates {(0,0,0)(0,0,1)}node[left]{$z$};
\end{axis}
\end{tikzpicture}
\caption{فضا میں متحرک ذرہ کا تعین گر سمتیہ \عددی{\kvec{r}=\krightharpoonup{ON}} متغیر \عددی{t} کا تفاعل  ہو گا۔}
\label{شکل_سمتی_تفاعل_تعین_گر_سمتیہ}
\end{minipage}\hfill
\begin{minipage}{0.45\textwidth}
\centering
\begin{tikzpicture}[font=\small,declare function={fx(\ra,\t)=\ra*cos(\t);fy(\rb,\t)=\rb*sin(\t);fz(\rc,\t)=\rc*\t;}]
\pgfmathsetmacro{\h}{2}
\pgfmathsetmacro{\ra}{2}
\pgfmathsetmacro{\rb}{0.25*\ra}
\pgfmathsetmacro{\rc}{\h/470}
\pgfmathsetmacro{\ta}{-110}
\pgfmathsetmacro{\tb}{0}
\pgfmathsetmacro{\tc}{180}
\pgfmathsetmacro{\td}{360}
\pgfmathsetmacro{\te}{-20}
\pgfmathsetmacro{\tf}{360+\ta}
\draw[-latex](0,0)--++(-145:2)node[left]{$x$};
\draw[-latex](0,0)--++(0:2.5)node[right]{$y$};
\draw[-latex](0,0)--++(0,3)node[left]{$z$};
\draw[thick,-stealth]([shift={(\ta:0.75cm and 0.185cm)}]0,0) arc (\ta:-20:0.75cm and 0.185 cm)node[pos=0.5,fill=white]{$t$};
 \draw[-latex](0,0,0)node[left,yshift=1ex]{$O$}--({fx(\ra,\te)},{fz(\rc,\te-\ta)+fy(\rb,\te)})node[circ]{}node[pos=0.4,above]{$\kvec{r}$};
 \draw[dashed,thick](0,0,0)--({fx(\ra,\te)},{fy(\rb,\te)})--({fx(\ra,\te)},{fz(\rc,\te-\ta)+fy(\rb,\te)});
\draw[dashed,domain=40:180,variable=\t]plot ({fx(\ra,\t)},{fy(\rb,\t)});
\draw[domain=180:360,variable=\t]plot ({fx(\ra,\t)},{fy(\rb,\t)});
\draw[domain=0:360,variable=\t]plot ({fx(\ra,\t)},{\h+fy(\rb,\t)});
\draw  ({fx(\ra,300)},{fy(\rb,300)})node[below]{$x^2+y^2=1$};
\draw(-\ra,0)--++(0,\h);
\draw(\ra,0)--++(0,\h);
\draw[domain=\ta:\tb,variable=\t]plot ({fx(\ra,\t)},{fz(\rc,\t-\ta)+fy(\rb,\t)});
\draw[dashed,domain=\tb:\tc,variable=\t]plot ({fx(\ra,\t)},{fz(\rc,\t-\ta)+fy(\rb,\t)});
\draw[domain=\tc:\td,variable=\t]plot ({fx(\ra,\t)},{fz(\rc,\t-\ta)+fy(\rb,\t)});
\draw  ({fx(\ra,\ta)},{fz(\rc,\ta-\ta)+fy(\rb,\ta)})node[circ]{}node[below,xshift=1ex]{$t=0$}node[left,xshift=-2ex,yshift=-1ex]{$(1,0,0)$};
\draw  ({fx(\ra,\tb)},{fz(\rc,\tb-\ta)+fy(\rb,\tb)})node[circ]{}node[right]{$t=\frac{\pi}{2}$};
\draw  ({fx(\ra,\tf)},{fz(\rc,\tf-\ta)+fy(\rb,\tf)})node[circ]{}node[below]{$t=2\pi$};
\end{tikzpicture}
\caption{پیچدار منحنی \عددی{\kvec{r}(t)=(\cos t)\ai+(\sin t)\aj} کا بالائی نصف حصہ}
\label{شکل_سمتی_تفاعل_پیچدار}
\end{minipage}
\end{figure}
\جزوحصہء{تعریف}
جب وقفہ \عددی{I} کے دوران ایک ذرہ فضا میں حرکت  کرتا ہو، ہم اس ذرہ کے محدد جو وقت کے تفاعل ہو گے کی تعریف درج ذیل کرتے ہیں۔
\begin{align}\label{مساوات_سمتی_تفاعل_مقدار_معلوم_راہ}
x=f(t),\quad y=g(t),\quad z=h(t),\quad t\in I
\end{align}
نقاط \عددی{(x,y,z)=(f(t),g(t),h(t)),\,t\in I}  فضا میں وہ  \اصطلاح{منحنی} دیتے ہیں جنہیں ہم اس ذرے کی \اصطلاح{راہ}\فرہنگ{راہ}\حاشیہب{path}\فرہنگ{path} کہتے ہیں۔ مساوات \حوالہ{مساوات_سمتی_تفاعل_مقدار_معلوم_راہ} اس منحنی  کی \اصطلاح{مقدار معلوم روپ }\فرہنگ{مقدار معلوم!روپ}  ہے۔ مبدا سے ذرے  کے \اصطلاح{ مقام}  \عددی{N(f(t),g(t),h(t))}  تک لمحہ \عددی{t} پر   سمتیہ 
\begin{align*}
\kvec{r}(t)=\krightharpoonup{ON}=f(t)\ai+g(t)\aj+h(t)\ak
\end{align*} 
اس ذرے کا\اصطلاح{   تعین گر سمتیہ}\فرہنگ{تعین گر!سمتیہ}\حاشیہب{position vector}\فرہنگ{vector!position}  ہے۔تفاعل \عددی{f}، \عددی{g} اور \عددی{h}  تعین گر سمتیہ کے\اصطلاح{ اجزاء}  ہیں۔
 ذرے کی راہ سے مراد وقفہ \عددی{t} کے دوران   \عددی{\kvec{r}}  کی  پیداکردہ منحنی ہے۔


مساوات \حوالہ{مساوات_سمتی_تفاعل_مقدار_معلوم_راہ}  سمتیہ \عددی{\kvec{r}} کی تعریف وقفہ \عددی{I} پر  حقیقی متغیر \عددی{t} کی صورت میں دیتی ہے۔ زیادہ عمومی طور پر  دائرہ کار،   سلسلہ \عددی{D}،  پر  \اصطلاح{سمتی تفاعل}\فرہنگ{سمتی!تفاعل}\حاشیہب{vector function}\فرہنگ{vector!function} یا  \اصطلاح{سمتی  قیمت تفاعل}\فرہنگ{سمتی قیمت تفاعل}\حاشیہب{vector-valued function}\فرہنگ{vector-valued!function}   سے مراد وہ قاعدہ ہو گا   جو \عددی{D} کے  ہر رکن کو فضا میں ایک سمتیہ مختص کرتا ہو۔موجودہ استعمال میں دائرہ کار حقیقی اعداد  کے وقفوں    پر مشتمل ہوں  گے۔ بعد کے ایک باب میں دائرہ کار، مستوی یا فضا میں خطوں پر مشتمل ہوں گے   جہاں ہم  سمتی تفاعل کو سمتی میدان  کہیں گے۔

ہم حقیقی قیمت تفاعل کو \اصطلاح{غیر سمتی تفاعل}\فرہنگ{غیر سمتی! تفاعل}\حاشیہب{scalar functions}\فرہنگ{scalar!functions} کہتے ہیں تا کہ ان میں اور سمتی تفاعل میں فرق کرنا ممکن ہو۔  سمتیہ \عددی{\kvec{r}} کے اجزاء \عددی{t}  کے غیر سمتی تفاعل ہیں۔سمتی تفاعل کی تعریف  اس کے  ارکان تفاعل کی صورت میں دیتے وقت ہم فرض کرتے ہیں کہ   سمتی تفاعل کا دائرہ کار ہی  ارکان کے دائرہ کار   ہیں۔

\ابتدا{مثال}\ترچھا{پیچ دار تفاعل}\\
تمام حقیقی متغیر  \عددی{t} کے لئے سمتی تفاعل
\begin{align*}
\kvec{r}(t)=(\cos t)\ai+(\sin t)\aj+t\ak
\end{align*}
معین ہے اور  \عددی{\kvec{r}}   دائری نلکی \عددی{x^2+y^2=1} کے گرد  لپٹ کر چلتا ہے (شکل \حوالہ{شکل_سمتی_تفاعل_پیچدار})۔  سمتی تفاعل \عددی{\kvec{r}} کے \عددی{\ai} اور \عددی{\aj} اجزاء  جو \عددی{\kvec{r}} کے سر  کے  \عددی{x} اور \عددی{y} محدد ہیں   دائری نلکی  کی مساوات
\begin{align*}
x^2+y^2=(\cos t)^2+(\sin t)^2=1
\end{align*}
کو مطمئن کرتے ہیں لہٰذا \عددی{\kvec{r}} اس نلکی پر پایا جاتا ہے۔ متغیر \عددی{t} بڑھنے  \عددی{\ak}  جزو بڑھتا ہے  جس کی بنا   منحنی   اوپر بلند ہو گی۔ نلکی کے گرد ایک دائرہ \عددی{t=2\pi}  پر مکمل ہو گا۔  درج ذیل مساوات  پیچ دار تفاعل  کی مقدار معلوم  مساوات ہے، جہاں وقفہ \عددی{-\infty\le t\le \infty} ہے۔
\begin{align*}
x=\cos t,\quad y=\sin t,\quad z=t
\end{align*}
\انتہا{مثال}
%==============

\جزوحصہء{حد اور استمرار}
ہم  سمتی  قیمت تفاعل کے حد کی تعریف  حقیقی قیمت تفاعل کے حد  کی طرح کرتے ہیں۔

\ابتدا{تعریف}
فرض کریں \عددی{\kvec{r}=f(t)\ai+g(t)\aj+h(t)\ak} ایک سمتی تفاعل اور \عددی{\kvec{L}} ایک سمتیہ ہے۔ اگر  ہر عدد \عددی{\epsilon >0} کے لئے   ایک ایسا مطابقتی  عدد \عددی{\delta>0} پایا جاتا ہو کہ تمام \عددی{t} کے لئے
\begin{align*}
0<\abs{t-t_0}<\delta\quad \implies \quad \abs{\kvec{r}(t)-\kvec{L}}<\epsilon
\end{align*}
ہو تب ہم کہتے ہیں کہ جب \عددی{t}  کی قیمت \عددی{t_0} کے قریب تر ہو تب  \عددی{\kvec{r}} کا  \اصطلاح{حد}\فرہنگ{حد}\حاشیہب{limit}\فرہنگ{limit} \عددی{\kvec{L}} ہو گا جس کو درج ذیل لکھا جاتا ہے۔
\begin{align*}
\lim_{t\to t_0}\kvec{r}(t)=\kvec{L}
\end{align*}
\انتہا{تعریف}
%================

اگر \عددی{\kvec{L}=L_1\ai+L_2\aj+L_3\ak} ہو تب \عددی{\lim_{t\to t_0}\kvec{r}(t)=\kvec{L}} ٹھیک اس صورت ہو گا جب درج ذیل ہو۔
\begin{align*}
\lim_{t\to t_0}f(t)=L_1,\quad \lim_{t\to t_0}g(t)=L_2,\quad \lim_{t\to t_0}h(t)=L_3
\end{align*}
درج ذیل مساوات سمتی تفاعل کا حد تلاش کرنے کی  عملی ترکیب دیتی ہے۔
\begin{align}
\lim_{t\to t_0}\kvec{r}(t)=\big(\lim_{t\to t_0}f(t)\big)\ai+\big(\lim_{t\to t_0}g(t)\big)\aj+\big(\lim_{t\to t_0}h(t)\big)\ak
\end{align}


\ابتدا{مثال}
اگر \عددی{\kvec{r}(t)=(\cos t)\ai+(\sin t)\aj+t\ak} ہو تب درج ذیل ہو گا۔
\begin{align*}
\lim_{t\to \tfrac{\pi}{4}}&=\big(\lim_{t\to \tfrac{\pi}{4}}\cos t\big)\ai+\big(\lim_{t\to \tfrac{\pi}{4}}\sin t\big)\aj+\big(\lim_{t\to \tfrac{\pi}{4}}t\big)\ak\\
&=\frac{\sqrt{2}}{2}\ai+\frac{\sqrt{2}}{2}\aj+\frac{\pi}{4}\ak
\end{align*}
\انتہا{مثال}
%==============

ہم سمتی تفاعل کی استمرار کی تعریف  حقیقی قیمت تفاعل کی استمرار کی تعریف کی طرح کرتے ہیں۔

\ابتدا{تعریف}
اگر \عددی{\kvec{r}(t)} کے دائرہ کار میں نقطہ \عددی{t_0} پر   \عددی{\lim_{t\to t_0}\kvec{r}(t)=\kvec{r}(t_0)}  ہو تب  \عددی{\kvec{r}(t)}  اس\اصطلاح{ نقطہ پر استمراری}\فرہنگ{استمرار!نقطہ پر}\حاشیہب{continuous at a point}\فرہنگ{continuity!at a point} ہو گا۔ اگر  اپنے پورے دائرہ کار میں ہر نقطہ پر  \عددی{\kvec{r}(t)}  استمراری ہو تب یہ تفاعل \اصطلاح{استمراری}\فرہنگ{استمراری}\حاشیہب{continuous}\فرہنگ{continuous} ہو گا۔
\انتہا{تعریف}
%===================


چونکہ حد کو اجزاء کی صورت میں لکھا جا سکتا ہے لہٰذا  سمتی تفاعل کو استمرار کے لئے پرکھنے کی خاطر ہم اس  کے اجزاء  پر نظر ڈالتے ہیں۔

\موٹا{ایک نقطہ پر ارکان  کے  استمرار کا  پرکھ }\\
نقطہ \عددی{t_0} پر سمتی  تفاعل \عددی{\kvec{r}(t)=f(t)\ai+g(t)\aj+h(t)\ak}  اس صورت استمراری ہو گا جب  \عددی{t_0} پر \عددی{f}، \عددی{g} اور \عددی{h} استمراری ہوں۔

\ابتدا{مثال}
(ا) درج ذیل تفاعل اس لئے استمراری  ہے کہ \عددی{\cos t}، \عددی{\sin t} اور \عددی{t} استمراری ہیں۔
\begin{align*}
\kvec{r}(t)=(\cos t)\ai+(\sin t)\aj+t \ak
\end{align*}
(ب) درج ذیل  تفاعل ہر عدد صحیح پر  عدم استمراری ہے۔
\begin{align*}
\kvec{r}(t)=(\cos t)\ai+(\sin t)\aj+\abs{t} \ak
\end{align*}
\انتہا{مثال}
%========================

\جزوحصہء{تفرقات اور حرکت}
فرض کریں فضا میں ایک متحرک  ذرہ جو  ایک منحنی پر چل رہا ہو کا    تعین گر سمتیہ \عددی{\kvec{r}(t)=f(t)\ai+g(t)\aj+h(t)\ak} ہو     جہاں \عددی{f}، \عددی{g} اور \عددی{h} متغیر \عددی{t} کے قابل تفرق تفاعل ہیں۔ایسی  صورت میں لمحات  \عددی{t} اور \عددی{t+\Delta t}  پر اس ذرے کے مقام  میں فرق 
\begin{align*}
\Delta \kvec{r}(t)=\kvec{r}(t+\Delta t)-\kvec{r}(t)
\end{align*}
ہو گا جس کو اجزاء کی صورت میں درج ذیل لکھا جا سکتا ہے (شکل \حوالہ{شکل_سمتی_تفاعل_فضا_میں_حرکت})۔
\begin{align*}
\Delta \kvec{r}&=\kvec{r}(t+\Delta t)-\kvec{r}(t)\\
&=[f(t+\Delta t)\ai+g(t+\Delta t)\aj+h(t+\Delta t)\ak]-[f(t)\ai+g(t)\aj+h(t)\ak]\\
&=[f(t+\Delta t)-f(t)]\ai+[g(t+\Delta t)-g(t)]\aj+[h(t+\Delta t)-h(t)]\ak
\end{align*}
اب اگر \عددی{\Delta t} صفر کے قریب  ہونے کی کوشش کرے تب تین  اقدام بیکوقت ہوتے نظر آئیں گے۔اول،  منحنی پر چلتے ہوئے \عددی{Q} نقطہ \عددی{N} تک پہنچے گا۔ دوسرا،  سیکنٹ خط \عددی{NQ}  نقطہ \عددی{N} پر منحنی کے تحدیدی   مماسی مقام پر پہنچے گا۔ تیسرا، حاصل تقسیم \عددی{\tfrac{\Delta \kvec{r}}{\Delta t}} درج ذیل حد  تک پہنچے گا۔
\begin{align*}
\lim_{\Delta t\to 0}\frac{\Delta \kvec{r}}{\Delta t}&=\big[\lim_{\Delta t\to 0}\frac{f(t+\Delta t)-f(t)}{\Delta t}\big]\ai+\big[\lim_{\Delta t\to 0}\frac{g(t+\Delta t)-g(t)}{\Delta t}\big]\aj\\
&\quad +\big[\lim_{\Delta t\to 0}\frac{h(t+\Delta t)-h(t)}{\Delta t}\big]\ak\\
&=\big[\frac{\dif f}{\dif t}\big]\ai+\big[\frac{\dif g}{\dif t}\big]\aj+\big[\frac{\dif h}{\dif t}\big]\ak
\end{align*}
یوں ماضی  کے تجربات ہمیں  درج ذیل تعریف تک پہنچاتے  ہیں۔

\begin{figure}
\centering
\begin{minipage}{0.45\textwidth}
\centering
\begin{tikzpicture}
\draw[->-=0.95](0,0) to [out=20,in=-110]coordinate[pos=0.3](ka)coordinate[pos=0.7](kb)node[pos=0.95,right]{\RL{بڑھتے \عددی{t} کا رخ}}++(2,2);
\draw[-latex](-1,1)node[left]{$O$}--(ka)node[pos=0.5,below]{$\kvec{r}(t)$}node[below]{$N$};
\draw[-latex](-1,1)--(kb)node[pos=0.5,above]{$\kvec{r}(t+\Delta t)$}node[right]{$Q$};
\draw[-latex](ka)--(kb)node[pos=0.3,right,xshift=1ex]{$\Delta \kvec{r}$\,\, \RL{سمتیہ ہٹاو}};
\end{tikzpicture}
\caption{لمحات \عددی{t} اور \عددی{t+\Delta t} کے بیچ ایک ذرے کا ہٹاو \عددی{\krightharpoonup{NQ}=\Delta \kvec{r}} ہو گا۔نیا سمتی مجموعہ \عددی{\kvec{r}(t)+\Delta \kvec{r}}، ذرے کا نیا مقام \عددی{\kvec{r}(t+\Delta t)} دے گا۔}
\label{شکل_سمتی_تفاعل_فضا_میں_حرکت}
\end{minipage}\hfill
\begin{minipage}{0.45\textwidth}
\centering
\begin{tikzpicture}
\draw(0,0)node[circ]{}to [out=70,in=-135]node[circ]{}node[pos=0.5,above left]{$C_1$} ++(1,1)to [out=-110,in=120]node[circ]{}node[pos=0.5,below left]{$C_2$}++(1,-0.75) to [out=30,in=-100]node[circ]{}node[pos=0.5,above left]{$C_3$}++(1,1)--++(1,-1)node[circ]{}node[pos=0.6,left]{$C_4$} to [out=45,in=60]node[circ]{}node[pos=0.5,above]{$C_5$}++(1,-0.5)node[circ]{};
\end{tikzpicture}
\caption{پانچ ہموار منحنیات کو ساتھ ساتھ جوڑ کر ٹکڑوں میں ہموار منحنی حاصل کی گئی ہے۔}
\label{شکل_سمتی_تفاعل_ٹکڑوں_میں_ہموار}
\end{minipage}
\end{figure}

\ابتدا{تعریف}
نقطہ \عددی{t_0} پر سمتی تفاعل \عددی{\kvec{r}(t)=f(t)\ai+g(t)\aj+h(t)\ak}  اس صورت قابل تفرق ہو گا جب \عددی{t_0} پر \عددی{f}، \عددی{g} اور \عددی{h} قابل تفرق ہوں۔ اس طرح اگر \عددی{\kvec{r}} اپنے دائرہ کار میں ہر نقطہ پر قابل تفرق ہو تب \عددی{\kvec{r}} قابل تفرق ہو گا۔ کسی بھی نقطہ \عددی{t}  پر جہاں \عددی{\kvec{r}} قابل تفرق ہو، اس کا تفرق درج ذیل سمتیہ ہو گا۔
\begin{align*}
\frac{\dif \kvec{r}}{\dif t}=\lim_{\Delta t\to 0}\frac{\kvec{r}(t+\Delta t)-\kvec{r}(t)}{\Delta t}=\frac{\dif f}{\dif t}\ai+\frac{\dif g}{\dif t}\aj+\frac{\dif h}{\dif t}\ak
\end{align*}

اگر \عددی{\tfrac{\dif \kvec{r}}{\dif t}} استمراری  اور کبھی بھی  \عددی{\kvec{0}} نہ ہو، یعنی جب \عددی{f}، \عددی{g} اور \عددی{h} کے استمراری  پہلے تفرق پائے جاتے ہوں اور جو بیکوقت \عددی{0}  نہ ہوں،    تب   جس منحنی پر  \عددی{\kvec{r}} چلتا  ہو وہ   \اصطلاح{ہموار}\فرہنگ{ہموار}\حاشیہب{smooth}\فرہنگ{smooth} ہو گی۔
\انتہا{تعریف}
%==========

سمتیہ \عددی{\tfrac{\dif \kvec{r}}{\dif t}} جب \عددی{\kvec{0}} سے مختلف ہو، یہ منحنی کا مماسی سمتیہ ہو گا۔ نقطہ \عددی{(f(t_0),g(t_0),h(t_0))} پر ایک منحنی کے \اصطلاح{مماسی خط} سے مراد  وہ خط ہے جو اس نقطہ سے گزرتا ہو اور جو \عددی{t_0} پر \عددی{\tfrac{\dif \kvec{r}}{\dif t}}  کے متوازی ہو۔  ہم ہموار منحنی    پر   \عددی{\tfrac{\dif \kvec{r}}{\dif t}\ne \kvec{0}} کی شرط    رکھ    کر اس بات کو یقینی بناتے ہیں کہ ہر نقطہ پر منحنی کا مماس استمراری طور پر مڑے گا۔ ایک ہموار منحنی پر   سخت  موڑ نہیں پایا جاتا ہے اور نا ہی اس پر کوئی کنگرہ پایا جاتا ہے۔

ایک منحنی  جو متناہی تعداد کی  ہموار  منحنیات  (بغیر خالی فاصلہ چھوڑے، ساتھ ساتھ )    ملا کر حاصل  کی گئی ہو  \اصطلاح{ٹکڑوں میں  ہموار}\فرہنگ{ہموار!ٹکڑوں میں}\حاشیہب{piecewise smooth}\فرہنگ{smooth!piecewise}   کہلاتی ہے (شکل \حوالہ{شکل_سمتی_تفاعل_ٹکڑوں_میں_ہموار})۔

شکل \حوالہ{شکل_سمتی_تفاعل_ٹکڑوں_میں_ہموار}  پر ایک بار دوبارہ نظر ڈالیں۔ ہم نے اس  شکل کو مثبت \عددی{\Delta t} کے لئے بنایا لہٰذا \عددی{\Delta \kvec{r}}   آگے چلنے  کی طرف    اشارہ کرے گا۔سمتیہ \عددی{\tfrac{\Delta \kvec{r}}{\Delta t}} (جسے دکھایا نہیں گیا ہے اور)  جس کا وہی رخ ہے جو \عددی{\Delta \kvec{r}} کا ہے  بھی آگے کی رخ اشارہ کرے گا۔ اگر \عددی{\Delta t} منفی ہوتا تب \عددی{\Delta \kvec{r}}  چلنے کے مخالف  رخ اشارہ کرے گا  البتہ حاصل تقسیم \عددی{\tfrac{\Delta \kvec{r}}{\Delta t}}  جو \عددی{\Delta \kvec{r}} کا منفی غیر سمتی مضرب  ہے اب بھی چلنے کے رخ اشارہ کرے  گا۔ ہم نے دیکھا کہ \عددی{\Delta \kvec{r}}  جس رخ بھی اشارہ کرتا ہو، \عددی{\tfrac{\Delta \kvec{r}}{\Delta t}}  ہر صورت چلنے کے رخ اشارہ کرتا ہے اور ہم توقع کرتے ہیں کہ سمتیہ \عددی{\tfrac{\dif \kvec{r}}{\dif t}=\lim_{\Delta t\to 0}\tfrac{\Delta \kvec{r}}{\Delta t}} جب \عددی{\kvec{0}} نہ ہو،  بھی ہر صورت  چلنے کے رخ اشارہ کرے گا۔ اس طرح ایک ذرہ کی سمتی رفتار کو ہم \عددی{\tfrac{\dif \kvec{r}}{\dif t}} سے ظاہر کر سکتے ہیں۔ یہ چلنے کی رخ دیتا ہے اور اس کی شرح، وقت کے لحاظ سے مقام  کی تبدیلی دیتا ہے۔ ایک ہموار منحنی  کے  لئے سمتی رفتار کبھی بھی صفر نہیں ہو گا؛   یہ ذرہ نا کبھی رکتا ہے اور نا ہی یہ واپسی اختیار کرتا ہے۔

\ابتدا{تعریف}
اگر فضا میں ہموار منحنی پر  چلتے ہوئے ذرے کا  تعین گر سمتیہ \عددی{\kvec{r}} ہو تب
\begin{align*}
\kvec{v}(t)=\frac{\dif \kvec{r}}{\dif t}
\end{align*}
اس ذرے کی \اصطلاح{سمتی رفتار}\فرہنگ{سمتی رفتار}\حاشیہب{velocity}\فرہنگ{velocity} ہو گی  جو   اس منحنی کو مماسی ہو گی۔ کسی بھی لمحہ \عددی{t} پر،  \عددی{\kvec{v}} کا رخ \اصطلاح{چلنے کا رخ} ہو گا، \عددی{\kvec{v}} کی مقدار اس ذرے کی رفتار ہو گی، اور تفرق \عددی{\kvec{a}=\tfrac{\dif \kvec{v}}{\dif t}}،   جب پایا جاتا ہو،  اس ذرے کی \اصطلاح{اسراع}\فرہنگ{اسراع}\حاشیہب{acceleration}\فرہنگ{acceleration} ہو گی۔  مختصراً  درج ذیل ہو گا۔
\begin{enumerate}[a.]
\item
مقام کا تفرق،  سمتی رفتار ہو گا:
\quad
$\kvec{v}=\frac{\dif \kvec{r}}{\dif t}$
\item
سمتی رفتار کی مقدار،  ذرے کی رفتار ہو گی:
\quad
$\text{رفتار}=\abs{\kvec{v}}$
\item
سمتی رفتار کا تفرق،  اسراع ہو گا:
\quad
$\kvec{a}=\frac{\dif \kvec{v}}{\dif t}=\frac{\dif^{\,2}\kvec{r}}{\dif t^2}$
\item
لمحہ \عددی{t} پر چلنے کا رخ سمتیہ \عددی{\tfrac{\kvec{v}}{\abs{\kvec{v}}}} دیگا۔
\end{enumerate}
\انتہا{تعریف}
%=================

ہم متحرک ذرے کی سمتی رفتار کو اس کی رفتار اور رخ کا حاصل ضرب لکھ سکتے ہیں۔
\begin{align*}
\text{\RL{سمتی رفتار}}=\abs{\kvec{v}}\big(\frac{\kvec{v}}{\abs{\kvec{v}}}\big)=(\text{رفتار})(\text{رخ})
\end{align*}

\ابتدا{مثال}
لمحہ \عددی{t} پر ایک متحرک جسم کا مقام  سمتیہ
\begin{align*}
\kvec{r}(t)=(3\cos t)\ai+(3\sin t)\aj+t^2\ak
\end{align*}
دیتا ہے۔ اس جسم کی رفتار اور رخ لمحہ \عددی{t=2} پر معلوم کریں۔ کس لمحہ پر  (اگر کبھی  ایسا ہو بھی ) اس جسم کی سمتی رفتار اور اسراع آپس میں عمودی  ہوں گے؟

حل:\quad
\begin{align*}
\kvec{r}(t)&=(3\cos t)\ai+(3\sin t)\aj+t^2\ak\\
\kvec{v}&=\frac{\dif \kvec{r}}{\dif t}=-(3\sin t)\ai+(3\cos t)\aj+2t\ak\\
\kvec{a}&=\frac{\dif^{\,2}\kvec{r}}{\dif t^2}=-(3\cos t)\ai-(3\sin t)\aj+2\ak
\end{align*}
لمحہ \عددی{t=2} پر اس جسم کی رفتار اور رخ درج ذیل ہیں۔
\begin{align*}
\abs{\kvec{v}(2)}&=\sqrt{(-3\sin 2)^2+(-3\cos 2)^2+(4)^2}=5&&\text{رفتار}\\
\frac{\kvec{v}(2)}{\abs{\kvec{v}(2)}}&=-\big(\frac{3}{5}\sin 2\big)\ai+\big(\frac{3}{5}\cos 2\big)\aj+\frac{4}{5}\ak&&\text{رخ}
\end{align*}
جس لمحہ پر \عددی{\kvec{v}} اور \عددی{\kvec{a}} ایک دوسرے کے عمودی ہوں اس لمحہ پر \عددی{\kvec{v}\cdot\kvec{a}=0} ہو گا۔یوں
\begin{align*}
\kvec{v}\cdot\kvec{a}=9\sin t\cos t-9\cos t\sin t+4t=4t=0
\end{align*}
سے \عددی{t=0} حاصل ہوتا ہے۔  اس  لمحہ  پر سمتی رفتار اور اسراع ایک دوسرے کے عمودی   ہوں گے۔
\انتہا{مثال}
%======================

\جزوحصہء{قواعد تفرقات}
چونکہ سمتی تفاعل کے تفرقات  جزو در جزو حاصل کرنا ممکن  ہے لہٰذا  سمتی تفاعل کے تفرقات کے قواعد کی نوعیت  غیر سمتی تفاعل کے تفرقات کے قواعد کی طرح ہو گی۔

\موٹا{سمتی تفاعل کے تفرقات کے قواعد}\\
\begin{description}
\item{قاعدہ مستقل تفاعل:}\quad
$\frac{\dif}{\dif t}\kvec{C}=\kvec{0}$\quad
جہاں \عددی{\kvec{C}} ایک مستقل سمتیہ ہے۔

اگر \عددی{\kvec{u}} اور \عددی{\kvec{v}} متغیر \عددی{t} کے  قابل تفرق سمتی تفاعل ہوں اور \عددی{f} متغیر \عددی{t} کا قابل تفرق غیر سمتی تفاعل ہو  تب
\item{قاعدہ غیر سمتی مضرب:}\quad
$\frac{\dif}{\dif t}(c\kvec{u})=c\frac{\dif \kvec{u}}{\dif t}$\quad
جہاں \عددی{c}  مستقل عدد ہے۔

\phantom{\hspace{1.5cm}}$\frac{\dif}{\dif t}(f\kvec{u})=\frac{\dif f}{\dif t}\kvec{u}+f\frac{\dif \kvec{u}}{\dif t}$
\item{قاعدہ مجموعہ:}\quad
$\frac{\dif}{\dif t}(\kvec{u}+\kvec{v})=\frac{\dif \kvec{u}}{\dif t}+\frac{\dif \kvec{v}}{\dif t}$
\item{قاعدہ فرق:}\quad
$\frac{\dif}{\dif t}(\kvec{u}-\kvec{v})=\frac{\dif \kvec{u}}{\dif t}-\frac{\dif \kvec{v}}{\dif t}$
\item{قاعدہ  ضرب نقطہ:}\quad
$\frac{\dif}{\dif t}(\kvec{u}\cdot \kvec{v})=\frac{\dif \kvec{u}}{\dif t}\cdot \kvec{v}+\kvec{u}\cdot\frac{\dif \kvec{v}}{\dif t}$
\item{قاعدہ  ضرب صلیبی:}\quad
$\frac{\dif}{\dif t}(\kvec{u}\times \kvec{v})=\frac{\dif \kvec{u}}{\dif t}\times \kvec{v}+\kvec{u}\times\frac{\dif \kvec{v}}{\dif t}$
\item{قاعدہ  زنجیر:}\quad
$\frac{\dif \kvec{r}}{\dif s}=\frac{\dif \kvec{r}}{\dif t}\frac{\dif t}{\dif s}$\quad
جہاں \عددی{\kvec{r}}  متغیر \عددی{t} کا قابل تفرق تفاعل ہے اور \عددی{t} متغیر \عددی{s} کا قابل تفرق تفرق ہے۔
\end{description}

صلیبی ضرب میں سمتیات کی ترتیب نہایت اہم ہے۔یوں اگر بائیں ہاتھ \عددی{\kvec{u}} کے بعد \عددی{\kvec{v}} آئے، تب دائیں ہاتھ بھی \عددی{\kvec{u}} کے بعد \عددی{\kvec{v}}ہو    گا۔ایسا نہ کرنے سے قیمت کی علامت تبدیل ہو گی۔

ہم قاعدہ  ضرب اور زنجیری قاعدہ کو ثابت کرتے ہیں۔ باقی ثبوت  آپ کو مشق میں پیش کرنے ہوں گے۔

\ابتدا{ثبوت}\ترچھا{قاعدہ ضرب نقطہ}\\
درج ذیل سمتیات فرض کریں۔
\begin{align*}
\kvec{u}&=u_1(t)\ai+u_2(t)\aj+u_3(t)\ak\\
\kvec{v}&=v_1(t)\ai+v_2(t)\aj+v_3(t)\ak
\end{align*} 
تب درج  ذیل ہو گا۔
\begin{align*}
\frac{\dif}{\dif t}(\kvec{u}\cdot \kvec{v})&=\frac{\dif}{\dif t}(u_1v_1+u_2v_2+u_3v_3)\\
&=\underbrace{u_1'v_1+u_2'v_2+u_3'v_3}_{\kvec{u}'\cdot \kvec{v}}+\underbrace{u_1v_1'+u_2v_2'+u_3v_3'}_{\kvec{u}\cdot\kvec{v}'}
\end{align*}
\انتہا{ثبوت}
%===================

\ابتدا{ثبوت}\ترچھا{قاعدہ ضرب صلیبی}\\
ہم غیر سمتی تفاعل کے قاعدہ ضرب کی طرح اس کو ثابت کرتے ہیں۔ تفرق کی تعریف کی رو سے
\begin{align*}
\frac{\dif}{\dif t}(\kvec{u}\times \kvec{v})=\lim_{h\to 0}\frac{\kvec{u}(t+h)\times \kvec{v}(t+h)-\kvec{u}(t)-\kvec{v}(t)}{h}
\end{align*}
ہو گا۔ ہم  شمار کنندہ  کے ساتھ   \عددی{\kvec{u}(t)\times \kvec{v}(t+h)} جمع اور منفی کرتے ہیں تا کہ درج بالا   کو ایسی حاصل تقسیمات  کی صورت میں لکھنا ممکن ہو جن میں  \عددی{\kvec{u}} اور \عددی{\kvec{v}} کے تفرقات    پائے جاتے ہوں۔یوں درج ذیل ہو گا۔
\begin{align*}
\frac{\dif}{\dif t}&(\kvec{u}\times \kvec{v})\\
&=\lim_{h\to 0}\frac{\kvec{u}(t+h)\times \kvec{v}(t+h)-\kvec{u}(t)\times \kvec{v}(t+h)+\kvec{u}(t)\times \kvec{v}(t+h)-\kvec{u}(t)\times \kvec{v}(t)}{h}\\
&=\lim_{h\to 0}\big[\frac{\kvec{u}(t+h)-\kvec{u}(t)}{h}\times \kvec{v}(t+h)+\kvec{u}(t)\times \frac{\kvec{v}(t+h)-\kvec{v}(t)}{h}\big]\\
&=\lim_{h\to 0}\frac{\kvec{u}(t+h)-\kvec{u}(t)}{h}\times \lim_{h\to 0}\kvec{v}(t+h)+\lim_{h\to 0}\kvec{u}(t)\times \lim_{h\to 0}\frac{\kvec{v}(t+h)-\kvec{v}(t)}{h}
\end{align*}
آخری لکیر پر دونوں مساوات اس لئے ٹھیک ہیں کہ دو سمتیات کے سمتی ضرب کا حد،  ان کے حدوں کا سمتی ضرب ہوتا ہے۔  چونکہ \عددی{t} پر \عددی{\kvec{v}} قابل تفرق لہٰذا استمراری ہے، اس لئے   جیسے جیسے \عددی{h} کی قیمت صفر تک پہنچتی ہے ویسے ویسے  \عددی{\kvec{v}(t+h)} کی قیمت \عددی{\kvec{v}(t)} تک پہنچتی ہے۔ ان  دو حاصل تقسیم کی قیمتیں  \عددی{t} پر  \عددی{\tfrac{\dif \kvec{u}}{\dif t}} اور \عددی{\tfrac{\dif \kvec{v}}{\dif t}} تک پہنچتی ہیں۔مختصراً درج ذیل ہو گا۔
\begin{align*}
\frac{\dif}{\dif t}(\kvec{u}\times\kvec{v})=\frac{\dif \kvec{u}}{\dif t}\times \kvec{v}+\kvec{u}\times\frac{\dif \kvec{v}}{\dif t}
\end{align*} 
\انتہا{ثبوت}
%====================

\ابتدا{ثبوت}\ترچھا{زنجیری قاعدہ}\\
فرض کریں \عددی{\kvec{r}(t)=f(t)\ai+g(t)\aj+h(t)\ak} متغیر \عددی{t} کا قابل تفرق سمتی  تفاعل ہے اور \عددی{t} از خود کسی متغیر \عددی{s} کا قابل تفرق غیر سمتی تفاعل ہے۔ تب \عددی{f}، \عددی{g} اور \عددی{h} متغیر \عددی{s} کے قابل تفرق تفاعل ہوں گے  اور حقیقی قیمت تفاعل کے زنجیری قاعدہ کے تحت درج ذیل ہو گا۔
\begin{align*}
\frac{\dif \kvec{r}}{\dif s}&=\frac{\dif f}{\dif s}\ai+\frac{\dif g}{\dif s}\aj+\frac{\dif h}{\dif s}\ak\\
&=\frac{\dif f}{\dif t}\frac{\dif t}{\dif s}\ai+\frac{\dif g}{\dif t}\frac{\dif t}{\dif s}\aj+\frac{\dif h}{\dif t}\frac{\dif t}{\dif s}\ak\\
&=\big(\frac{\dif f}{\dif t}\ai+\frac{\dif g}{\dif t}\aj+\frac{\dif h}{\dif t}\ak\big)\frac{\dif t}{\dif s}\\
&=\frac{\dif \kvec{r}}{\dif t}\frac{\dif t}{\dif s}
\end{align*}
\انتہا{ثبوت}
%===================

\begin{figure}
\centering
\begin{tikzpicture}[font=\small]
\pgfmathsetmacro{\r}{1.5}
\draw(0,0) circle (\r);
\draw(-160:\r) to [out=85,in=-110]coordinate[pos=0.4](ka)coordinate[pos=0.41](kb)(60:\r);
\draw[-latex](0,0)--(ka)node[pos=0.6,below]{$\kvec{r}(t)$}node[circ]{}node[left,yshift=0.5ex]{$N$};
\RightAngle{(0,0)}{(ka)}{(kb)}
\draw[-latex](ka)--++(20:1.5)node[pos=0.75,below right]{$\frac{\dif \kvec{r}}{\dif t}$};
\draw[dashed](0,0)--(-0.5*\r,-0.5*\r);
\draw[dashed](0,0)--++(-10:0.9*\r);
\draw[dashed](0,0)--++(0,\r);
\draw[-latex](-0.5*\r,-0.5*\r)--++(-0.5,-0.5)node[left]{$x$};
\draw[-latex](-10:0.9*\r)--++(-10:0.5)node[right]{$y$};
\draw[-latex](0,\r)--++(0,0.25)node[left]{$z$};
\end{tikzpicture}
\caption{اگر ایک ذرہ ایک کرہ پر یوں حرکت کرتا ہو کہ اس کی مقام \عددی{\kvec{r}} وقت کا قابل تفرق تفاعل ہو، تب \عددی{\kvec{r}\cdot (\tfrac{\dif \kvec{r}}{\dif t})=0} ہو گا۔}
\label{شکل_سمتی_تفاعل_مستقل_لمبائی}
\end{figure}

\جزوحصہء{مستقل لمبائی کے سمتی تفاعل}  
 ایک کرہ   جس کا مرکز مبدا پر ہو، پر  جو جسم حرکت کرتا ہو، اس جسم  کے  تعین گر سمتیہ کی لمبائی اس کرہ کے رداس جتنی ہو گی  (شکل \حوالہ{شکل_سمتی_تفاعل_مستقل_لمبائی})۔اس کا سمتی رفتار سمتیہ  \عددی{\tfrac{\dif \kvec{r}}{\dif t}}،  جو حرکت کی راہ کو مماسی ہو گا، اس کرہ کو مماسی   لہٰذا  \عددی{\kvec{r}}   کو قائمہ ہو گا۔ مستقل لمبائی والے قابل تفرق سمتی تفاعل کے لئے  ہر بار ایسا ہی ہو گا۔ ایسا سمتیہ اور اس کا پہلا تفرق ایک دوسرے کو عمودی  ہوں گے۔ لمبائی مستقل ہونے کی بدولت، سمتیہ میں تبدیلی درحقیقت سمتیہ کے رخ میں تبدیلی ہو گی اور رخ کی یہ تبدیلی سمتی تفاعل کے ساتھ  زاویہ قائمہ پر ہو گی۔

\ابتدا{قانون}
اگر \عددی{\kvec{u}} متغیر \عددی{t} کا قابل تفرق سمتی تفاعل ہو اور اس کی لمبائی اٹل ہو تب درج ذیل ہو گا۔
\begin{align}\label{مساوات_سمتی_تفاعل_مستقل_لمبائی}
\kvec{u}\cdot\frac{\dif \kvec{u}}{\dif t}=0
\end{align}
\انتہا{قانون}
%=================== 


یہ دیکھنے کی خاطر کہ مساوات \حوالہ{مساوات_سمتی_تفاعل_مستقل_لمبائی} کیوں درست ہے ہم فرض کرتے ہیں کہ سمتی تفاعل  \عددی{\kvec{u}} متغیر \عددی{t} کا قابل تفرق تفاعل ہے  اور \عددی{\abs{\kvec{u}}} ایک مستقل  قیمت ہے۔ یوں \عددی{\kvec{u}\cdot\kvec{u}=\abs{\kvec{u}}^2} ایک مستقل ہو گا اور ہم اس مساوات کی دونوں اطراف کا تفرق لے  کر درج ذیل حاصل کرتے ہیں۔
\begin{align*}
\frac{\dif}{\dif t}(\kvec{u}\cdot\kvec{u})&=\frac{\dif}{\dif t}(\text{مستقل})=0\\
\frac{\dif \kvec{u}}{\dif t}\cdot\kvec{u}+\kvec{u}\cdot\frac{\dif \kvec{u}}{\dif t}&=0&&\text{\RL{قاعدہ ضرب نقطہ میں \عددی{\kvec{v}=\kvec{u}} لیتے ہوئے}}\\
2\kvec{u}\cdot\frac{\dif\kvec{u}}{\dif t}&=0&&\text{\RL{ضرب نقطہ قابل تبادل ہے}}\\
\kvec{u}\cdot\frac{\dif \kvec{u}}{\dif t}&=0
\end{align*}

\ابتدا{مثال}
دکھائیں کہ درج ذیل سمتیہ  کی لمبائی مستقل ہے اور  اس سمتیہ  کا تفرق اور \عددی{\kvec{u}} آپس میں عمودی ہیں۔
\begin{align*}
\kvec{u}(t)=(\sin t)\ai+(\cos t)\aj+\sqrt{3}\ak
\end{align*}
حل:\quad
\begin{align*}
\kvec{u}(t)&=(\sin t)\ai+(\cos t)\ak+\sqrt{3}\ak\\
\abs{\kvec{u}(t)}&=\sqrt{(\sin t)^2+(\cos t)^2+(\sqrt{3})^2}=\sqrt{+3}=2\\
\frac{\dif \kvec{u}}{\dif t}&=(\cos t)\ai-(\sin t)\aj\\
\kvec{u}\cdot\frac{\dif \kvec{u}}{\dif t}&=\sin t\cos t-\sin t \cos t=0
\end{align*}
\انتہا{مثال}
%====================

\جزوحصہء{سمتی تفاعل کے تکملات}
اگر وقفہ \عددی{I}کے ہر نقطہ  پر \عددی{\tfrac{\dif \kvec{R}}{\dif t}=\kvec{r}} ہو تب    قابل تفرق سمتی تفاعل \عددی{\kvec{R}(t)}،وقفہ \عددی{I}   پر  سمتی تفاعل \عددی{\kvec{r}(t)} کا الٹ تفرق ہو گا۔ اگر \عددی{I} پر \عددی{\kvec{r}} کا الٹ تفرق \عددی{\kvec{R}} ہو تب ،ایک  وقت میں ایک جزو  کے ساتھ کام کرتے ہوئے،  یہ دکھایا جا سکتا ہے کہ \عددی{I} پر \عددی{\kvec{r}} کے الٹ تفرق  کی صورت \عددی{\kvec{R}+\kvec{C}} ہو گی جہاں \عددی{\kvec{C}} کوئی مستقل سمتیہ ہو گا۔وقفہ \عددی{I} پر \عددی{\kvec{r}} کے الٹ تفرقات کا سلسلہ \عددی{I} پر \عددی{\kvec{r}} کا \اصطلاح{غیر قطعی تکمل}\فرہنگ{تکمل!غیر قطعی}\حاشیہب{indefinite integral}\فرہنگ{integral!indefinite} ہو گا۔ 

\ابتدا{تعریف}
متغیر \عددی{t} کے لحاض سے \عددی{\kvec{r}} کا غیر قطعی تکمل، \عددی{\kvec{r}} کے تمام الٹ تفرقات کا سلسلہ ہو گا، جس کو \عددی{\int\kvec{r}(t)\dif t} سے ظاہر کیا جاتا ہے۔ اگر \عددی{\kvec{r}} کا الٹ تفرق \عددی{\kvec{R}} ہو تب درج ذیل ہو گا۔
\begin{align*}
\int\kvec{r}(t)\dif t&=\kvec{R}(t)+\kvec{C}&&\text{\RL{\عددی{\kvec{C}} مستقل سمتیہ ہے}}
\end{align*}
\انتہا{تعریف}
%=====================

غیر قطعی تکملات کے تمام حسابی  اصول یہاں قابل اطلاق ہوں گے۔

\ابتدا{مثال}
\begin{align*}
\int((\cos t)\ai&+\aj-2t\ak)\dif t=\big(\int \cos t\dif t\big)\ai+\big(\int \dif t\big)\aj-\big(\int 2t\dif t\big)\ak\\
&=(\sin t+C_1)\ai+(t+C_2)\aj-(t^2+C_3)\ak\\
&=(\sin t)\ai+t\aj-t^2\ak+\kvec{C}\quad\quad\quad\kvec{C}=C_1\ai+C_2\aj-C_3\ak
\end{align*}
غیر سمتی تفاعل کے  تکمل کی طرح یہاں بھی، درمیانے  دو  قدم کے بغیر،   آپ  بائیں ہاتھ سے   سیدھا نتیجہ لکھ سکتے ہیں۔
\انتہا{مثال}
%============

سمتی تفاعل کے قطعی تکمل کی تعریف اس کے اجزاء کی صورت میں کی  جاتی ہے۔

\ابتدا{تعریف}
اگر وقفہ \عددی{[a,b]} پر \عددی{\kvec{r}(t)=f(t)\ai+g(t)\aj+h(t)\ak} کے اجزاء قابل تفرق ہوں تب اس وقفہ پر \عددی{\kvec{r}} بھی قابل تفرق ہو گا اور \عددی{a} تا \عددی{b}  سمتی تفاعل \عددی{\kvec{r}} کا قطعی تکمل درج ذیل ہو گا۔
\begin{align*}
\int_a^b\kvec{r}(t)\dif t=\big(\int_a^b f(t)\dif t\big)\ai+\big(\int_a^b g(t)\dif t\big)\aj+\big(\int_a^b h(t)\dif t\big)\ak
\end{align*}
\انتہا{تعریف}
%==========

قطعی تکملات کے تمام حسابی  اصول یہاں قابل اطلاق ہوں گے۔

\ابتدا{مثال}
\begin{align*}
\int_0^{\pi}((\cos t)\ai+\aj-2t\ak)\dif t&=\big(\int_0^{\pi}\cos t\dif t \big)\ai+\big(\int_0^{\pi}\dif t \big)\aj-\big(\int_0^{\pi}2t\dif t \big)\ak\\
&=\big[\sin t\big]_0^{\pi}\ai+\big[t\big]_0^{\pi}\aj-\big[t^2\big]_0^{\pi}\\
&=[0-0]\ai+[\pi-0]\aj-[\pi^2-0^2]\ak\\
&=\pi \aj-\pi^2\ak
\end{align*}
\انتہا{مثال}
%=================
\ابتدا{مثال}\ترچھا{ذرے کے ابتدائی مقام اور ابتدائی سمتی رفتار سے اس کے مقام کا حصول}\\
فضا  میں حرکت کرتے ہوئے ایک  ذرے کا سمتی رفتار
\begin{align*}
\frac{\dif \kvec{r}}{\dif t}=(\cos t)\ai-(\sin t)\aj+\ak
\end{align*}
ہے۔اگر  لمحہ \عددی{t=0} پر اس ذرے کا مقام \عددی{\kvec{r}=2\ai+\ak} ہو تب لمحہ \عددی{t} پر اس کا مقام کیا ہو گا؟

حل:\quad
ہمیں  درج ذیل ابتدائی قیمت مسئلہ حل کرنا ہو گا۔
\begin{align*}
\frac{\dif \kvec{r}}{\dif t}&=(\cos t)\ai-(\sin t)\aj+\ak&&\text{\RL{تفرقی مساوات}}\\
\kvec{r}(0)&=2\ai+\ak&&\text{\RL{ابتدائی معلومات}}
\end{align*} 
دونوں اطراف  کا  \عددی{t} کے لحاض سے تکمل  لے کر
\begin{align*}
\kvec{r}(t)=(\sin t)\ai+(\cos t)\aj+t\ak+\kvec{C}
\end{align*}
حاصل ہوتا ہے۔ہم ابتدائی معلومات استعمال کرتے ہوئے تکمل کا مستقل \عددی{\kvec{C}} معلوم کرتے ہیں۔
\begin{align*}
(\sin 0)\ai+(\cos 0)\aj+(0)\ak+\kvec{C}&=2\ai+\ak&&\kvec{r}(0)=2\ai+\ak\\
\aj+\kvec{C}&=2\ai+\ak\\
\kvec{C}&=2\ai-\aj+\ak
\end{align*}
وقت \عددی{t} کے لحاض سے ذرے کا مقام درج ذیل ہو گا۔
\begin{align*}
\kvec{r}(t)=(\sin t+2)\ai+(\cos t-1)\aj+(t+1)\ak
\end{align*}
حاصل نتیجہ کو پرکھنے کی خاطر  ہم اس سے
\begin{align*}
\frac{\dif \kvec{r}}{\dif t}&=(\cos t+0)\ai+(-\sin t-0)\aj+(1+0)\ak\\
&=(\cos t)\ai-(\sin t)\aj+\ak
\end{align*}
اور
\begin{align*}
\kvec{r}(0)&=(\sin 0+2)\ai+(\cos 0-1)\aj+(0+1)\ak\\
&=2\ai+\ak
\end{align*}
حاصل کرتے ہیں۔
\انتہا{مثال}
%=============
