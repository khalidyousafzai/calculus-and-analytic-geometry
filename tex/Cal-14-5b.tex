%p1099 to end of example-5
%edited. figures missing and external references too if any
%exercise 14.5 (p1103) that should follow is missing. 
\جزوحصہء {سطحی تکملات} 
سطحی رقبے کے  حساب میں جو تصورات استعمال کیے گئے انہیں بروئے  کار لاتے ہوئے سطح پر تفاعل کے تکمل کا حصول سکتے ہیں۔

 فرض کریں ، سطح  \عددی{f(x,y,z)=c}  پر برقی بار  تقسیم  کیا گیا ہے (شکل \حوالہء{14.44})اور \عددی{S}  کے ہر نقطہ پر تفاعل \عددی{g(x,y,z)}  فی اکائی رقبہ بار  (کثافت بار) دیتا ہے۔  ایسی صورت میں  \عددی{S}  پر کل بار کا  حساب تکمل کی صورت میں درج ذیل طریقے  سے لکھا جا سکتا ہے ۔
 
 ہم  \عددی{S} کے نیچے ، زمینی مستوی پر سایہ دار خطہ \عددی{R}  کو چھوٹے چھوٹے مستطیلوں میں بالکل اسی طرح تقسیم کرتے ہیں \عددی{g}سے  \عددی{S}  کا سطحی  رقبہ تلاش کرنے کے لیے کیا  جاتا ہے۔ ہر ایک \عددی{\Delta A_k} کے بالکل اوپر کچھ بلندی پر سطح \عددی{\Delta \sigma_k} پایا جاتا ہے ، جس کو ہم مماسی مستوی پر مستطیلی حصہ  \عددی{\Delta P_k}  سے تخمین کرتے ہیں ۔ 
 
  ہم سطحی رقبے کی تعریف  کی طرز پر  چل رہے ہیں، تاہم یہاں  ایک اضافی قدم لیتے ہیں:  ہم \عددی{(x_k,y_k,z_k)}  پر  \عددی{g}  کی قیمت معلوم کر کے رقبہ  \عددی{\Delta \sigma_k}   پر کل بار کو تخمیناً حاصل  ضرب \عددی{g(x_k,y_k,z_k)\Delta P_k}  سے ظاہر کرتے ہیں ۔اس کی وجہ درج ذیل ہے: خطہ \عددی{R} کو کافی چھوٹے چھوٹے مستطیلوں میں تقسیم کرنے کی صورت میں پورے  \عددی{\Delta \sigma_k}  میں   \عددی{g}  کی قیمت تقریباً مستقل ہوگی ، اور \عددی{\Delta P_k} تقریباً \عددی{\Delta \sigma_k}کے کے برابر ہو گا ۔ یوں  \عددی{S}  پر کل بار تخمیناً درج ذیل مجموعہ کے برابر ہوگا۔
  %eq 7 p1099
  \begin{align}\label{مساوات_سمتی_میدان_کل_بار}
  \text{\RL{کل بار}}=\approx\sum g(x_k,y_k,z_k)\Delta P_k=\sum g(x_k,y_k,z_k)\frac{\Delta A_k}{\abs{\cos \gamma_k}}
  \end{align}
  
   اگر سطح\عددی{S} کو ظاہر کرنے والا تفاعل\عددی{\kvec{F}} اور اس کے  یک رتبی  جزوی تفرقات استمراری ہو ں ، اور  \عددی{S}  پر \عددی{g} استمراری ہو ، تب \عددی{R} کی خانہ بندی ہمیشہ  کی طرح نفیس بنانے سے مساوات  \حوالہ{مساوات_سمتی_میدان_کل_بار}  کے دائیں ہاتھ میں پیش مجموعے درج ذیل حد تک پہنچتے ہیں۔ 
   %eq 8 p 1099
   \begin{align}\label{مساوات_سمتی_میدان_کل_بار_ب}
   \iint_R g(x,y,z)\frac{\dif A}{\abs{\cos \gamma}}=\iint g(x,y,z)\frac{\abs{\nabla f}}{\abs{\nabla f\cdot \kvec{p}}}\dif A
   \end{align}
    یہ حد   سطح\عددی{S} پر \عددی{g} کا تکمل  کہلاتا ہے ، جس کو \عددی{R} پر دہرا تکمل لکھا اور حاصل کیا جاتا ہے ۔ اس تکمل کی قیمت سطح\عددی{S} پر کل بار دے گی ۔
    
    اگر مساوات \حوالہ{مساوات_سمتی_میدان_کل_بار_ب} کا تکمل موجود ہو ، تب آپ توقع   کر سکتے ہیں کہ مساوات  \حوالہ{مساوات_سمتی_میدان_کل_بار_ب} کا کلیہ سطح\عددی{S} پر کسی بھی تفاعل \عددی{g} کے تکمل کی تعریف ہے ۔
    
    \جزوحصہء{تعریفات } 
    اگر سطح\عددی{S}،  جس کی تعریف مساوات\عددی{f(x,y,z)=c}   دیتی ہے ،   کا سایہ \عددی{R} ہو اور  \عددی{S}   کے  تمام نقطوں  پر \عددی{g} استمراری تفاعل ہو،  \موٹا{\عددی{S}  پر \عددی{g} کا تکمل } درج ذیل ہوگا :
    %eq 9 p1099
    \begin{align}\label{مساوات_سمتی_میدان_سطحی_تکمل_تعریف}
    \iint g(x,y,z)\frac{\abs{\nabla f}}{\abs{\nabla f\cdot \kvec{p}}}\dif A
    \end{align}
     جہاں \عددی{\kvec{p}}  سایہ دار خطہ \عددی{R} کا اکائی عمودی سمتیہ ہے اور \عددی{\nabla f\cdot \kvec{p}\ne 0} ہے۔    اس تکمل کو \اصطلاح{ سطحی تکمل }\فرہنگ{تکمل!سطحی}\حاشیہب{surface integral}\فرہنگ{surface!integral}کہتے ہیں ۔
    
    مختلف عملی  استعمال  میں مساوات  \حوالہ{مساوات_سمتی_میدان_سطحی_تکمل_تعریف} کا تکمل مختلف معنی رکھتا ہے ۔ اگر \عددی{g} کی قیمت مستقل  طور پر   \عددی{1}  ہو ،  تکمل \عددی{S} کا رقبہ دے گا۔  اگر \عددی{g} ایک باریک خول  ، جس کی شکل کا نمونہ  \عددی{S}  ہو ، کی کمیتی کثافت ہو   تب  تکمل خول  کی کمیت  دے گا۔
    
\     \جزوحصہء{الجبرائی  خواص : سطحی رقبہ  کی تفریق}
ہم \عددی{
(\abs{\nabla f}/\abs{\nabla f\cdot \kvec{p}})\dif A
}  کو \عددی{\dif \sigma}  لکھ کر مساوات \حوالہ{مساوات_سمتی_میدان_سطحی_تکمل_تعریف}   کا  تکمل مختصراً  (درج ذیل) لکھ سکتے ہیں ۔
%eq 10,  p-1100
\begin{align}
&\text{\RL{\موٹا{سطحی رقبہ کی تفریق اور سطحی تکملات کا تفریقی روپ }}}\nonumber\\
\dif \sigma&=\frac{\abs{\nabla f}}{\abs{\nabla f\cdot \kvec{p}}}\dif A &&\quad\quad\iint_S g\dif \sigma\\
&\text{\small{\RL{سطحی رقبے کی تفریق}}} &&\text{\small{\RL{سطحی تکملات کا تفریقی کلیہ}}}\nonumber
\end{align}

سطحی تکملات   دیگر دہرا تکملات کی طرح رویہ رکھتے ہیں:  دو تفاعلات کا تکمل ان تفاعلات کے انفرادی تکملات کا مجموعہ ہوگا وغیرہ وغیرہ۔  دائرہ کار کی جمع پذیری  خاصیت درج ذیل روپ اختیار کرتی ہے ۔
\begin{align*}
\iint_S g\dif \sigma =\iint_{S_1}g\dif \sigma+\iint_{S_2}g\dif \sigma+\cdots+\iint_{S_n}g\dif \sigma
\end{align*}
 کلیدی  تصور یہ ہے کہ اگر \عددی{S}  کو    ہموار منحنیات ، متناہی تعداد کی غیر منطبق ، ہموار قطعات میں خانہ بند کرتی ہوں  (یعنی اگر \عددی{S}   \اصطلاح{ٹکڑوں میں ہموار}\فرہنگ{ٹکڑوں میں ہموار}\حاشیہب{piecewise smooth}\فرہنگ{piecewise smooth} ہو ) تب قطعات  پر تکملات کا مجموعہ \عددی{S}  پر تکمل کے برابر ہوگا ۔  یوں  ، مکعب کی سطح پر  تفاعل کا تکمل ، مکعب کی چھ  سطحوں پر تکملات کا مجموعہ ہوگا ۔

%------------------
%ex 3, p1100 
\ابتدا{مثال}
 ربع اول سے مستویات \عددی{x=1} ، \عددی{y=1}، اور \عددی{z=1}  ایک مکعب کاٹتی ہیں (شکل \حوالہء{14.45})۔ اس مکعب کی سطح پر \عددی{g(x,y,z)=xyz}کا تکمل لیں ۔
 
  \موٹا{حل:}\quad
  ہم باری باری مکعب کی چھ سطحوں پر  \عددی{ xyz} کا تکمل لے کر تمام نتائج کا مجموعہ لیتے ہیں ۔ محددی مستویات میں پائے جانے والے اطراف میں \عددی{xyz=0} ہوگا، لہٰذا  مکعب کی سطح پر تکمل گھٹ کر درج ذیل صورت اختیار کرتا ہے ۔
  \begin{align*}
  \iint_{\text{\RL{کعبی سطح}}} xyz \,\dif \sigma= \iint_{\text{\RL{سطح الف}}} xyz\,\dif \sigma + \iint_{\text{\RL{سطح ب}}} xyz\,\dif \sigma + \iint_{\text{\RL{سطح ج}}} xyz\,\dif \sigma 
  \end{align*}
مستوی \عددی{xy} میں چوکور خطہ \عددی{
R_{xy}:0\le x\le 1, 0\le y\le 1
}  پر سطح     الف کی مساوات \عددی{f(x,y,z)=z=1} ہے ۔اس سطح اور خطہ کے لیے ذیل ہو گا۔
\begin{align*}
\kvec{p}&=\kvec{k},\quad \nabla f=\kvec{k},\quad \abs{\nabla f}=1, \quad \abs{\nabla f\cdot \kvec{p}}=\abs{\kvec{k}\cdot \kvec{k}}=1,\\
\dif \sigma&=\frac{\abs{\nabla f}}{\abs{\nabla f\cdot \kvec{p}}}\dif A=\frac{1}{1}\dif x\dif y=\dif x\dif y,\\
xyz&=xy(1)=xy
\end{align*}
یوں ذیل ہو گا۔
\begin{align*}
\iint_{\text{\small\RL{سطح الف}}} xyz\,\dif\sigma=\iint_{R_{xy}} xy\dif x\dif y=\int_0^1\int_0^1 xy \dif x\dif y=\int_0^1 \frac{y}{2}\dif y=\frac{1}{4}
\end{align*}
 تشاکلیت کی بنا پر  سطح  ب اور ج پر \عددی{xyz} کے تکمل بھی \عددی{\tfrac{1}{4}} دیں گے۔ یوں ذیل ہو گا۔
 \begin{align*}
 \iint_{\text{\small\RL{کعبی سطح}}} xyz \dif \sigma=\frac{1}{4}+\frac{1}{4}+\frac{1}{4}=\frac{3}{4}
 \end{align*}
\انتہا{مثال}
%-----------------------------------------------
%????KKKK
\جزوحصہء{سمت بندی}
  وہ  ہموار سطح  \عددی{S}،   جس پر    عمودی  اکائی سمتیات کا  ایسا میدان \عددی{\kvec{n}}   متعارف کیا جا سکتا ہو جو مقام کے ساتھ استمراری تبدیل ہو،  \اصطلاح{قابل  سمت  بند  }\فرہنگ{قابل سمت بند}\حاشیہب{orientable}\فرہنگ{orientable} یا \اصطلاح{ دو طرفہ }\فرہنگ{دو طرفہ}\حاشیہب{two-sided} \فرہنگ{two-sided} کہلاتی  ہے ۔ قابل سمت بند سطح کا ہر ذیلی ٹکڑا  قابل سمت بند ہوگا ۔ کرہ اور فضا میں دیگر بند سطحیں( ایسی ہموار سطحیں جو ٹھوس اجسام کو احاطہ کرتی ہوں) قابل  سمت بند ہوں گی۔روایتاً \عددی{\kvec{n}} کا رخ بند سطح  سے  باہر کی طرف  رکھا  جاتا ہے۔
  
    میدان  \عددی{\kvec{n}} منتخب کرنے کے بعد ہم کہتے ہیں  سطح  \اصطلاح{سمت بند }\فرہنگ{سمت بند}\حاشیہب{oriented}\فرہنگ{oriented} بنائی گئی ہے اور سطح بشمول  عمودی میدان   \اصطلاح{سمت بند سطح }\فرہنگ{سمت بند سطح}\حاشیہب{oriented surface}\فرہنگ{oriented surface} کہلاتی ہے۔ کسی بھی نقطے پر سمتیہ  \عددی{\kvec{n}} ، اس نقطے  پر مثبت رخ   کہلاتا ہے (شکل \حوالہء{14.46} )۔ 
    
    شکل \حوالہء{14.47} میں \اصطلاح{ موبیوس پٹی }\فرہنگ{موبیوس پٹی}\حاشیہب{Mobius band}\فرہنگ{Mobius band} دکھائے گئی ہے جو نا قابل سمت بند ہے ۔آپ کسی بھی نقطہ سے آغاز  کر کے استمراری اکائی عمودی میدان بنانا شروع کریں۔  اکائی  عمودی  سمتیہ \عددی{\kvec{n}}  کو سطح  پر   استمراری حرکت دینے سے آپ   ابتدائی نقطے تک یوں پہنچ سکتے ہیں کہ  \عددی{\kvec{n}} کا رخ ابتدائی رخ کا مخولف ہو ۔ اس نقطے پر سمتیہ بیک وقت دو مخولف رخ نہیں ہو سکتا ، اگرچہ  استمراری میدان  کی صورت میں ایسا ہی ہونا چاہیے ۔ یوں ہم اخذ کرتے ہیں کہ ایسا میدان موجود نہیں ہو  گا ۔

\جزوحصہء  {بہاو کا سطحی تکمل }
فرض کریں قابل سمت بند سطح \عددی{S}  پر استمراری سمتی میدان \عددی{\kvec{F}} ہے ، اور  سطح پر منتخب اکائی عمودی میدان  \عددی{\kvec{n}} ہے ۔ہم \عددی{S} پر \عددی{\kvec{F}\cdot\kvec{n}}  کے تکمل  کو   \عددی{S} پر  مثبت رخ کا  بہاو کہتے ہیں ۔یوں  \عددی{\kvec{n}} کے رخ\عددی{\kvec{F}} کے غیر سمتی جزو کا \عددی{S} پر تکمل ،  سطح سے گزرتا بہاو دے گا۔ 

\جزوحصہء{ تعریف}
 قابل سمت بند سطح  \عددی{S}   کو پار کرتا  ، \عددی{\kvec{n}}  کے رخ ، تین ابعادی سمتی میدان\عددی{\kvec{F}} کا  بہاو درج ذیل کلیہ دے گا ۔ 
%eq-11, p1101
\begin{align}\label{مساوات_سمتی_میدان_بہاو_تعریف}
\text{\RL{بہاو}}=\iint_S \kvec{F}\cdot \kvec{n}\dif \sigma
\end{align} 

 یہ تعریف مسطح  منحنی  \عددی{C} کو  پار  کرتے، دو ابعادی میدان\عددی{\kvec{F}} کے بہاو کی تعریف کے عین مطابق ہے ۔مستوی میں بہاو کو  ، منحنی  \عددی{C} کو\عددی{\kvec{F}} کے  عمودی غیر سمتی جزو کا تکمل :
\begin{align*}
\int_C \kvec{F}\cdot \kvec{n}\dif s
\end{align*} 
 
ظاہر کرتا ہے۔

گر\عددی{\kvec{F}} تین ابعادی سیالی بہاو کا سمتی رفتار میدان ہو ، تب \عددی{S}  سے ( گزرتا ) \عددی{\kvec{F}} کا بہاو ،  منتخب مثبت رخ میں \عددی{S}   سے گزرتے  سیال  کی  خولص  شرح دے گا ۔ایسے  بہاو پر تفصیلاً تبصرہ حصہ \حوالہء{14.7}  میں کیا جائے گا۔

اگر \عددی{\kvec{S}}  ہم قد سطح \عددی{g(x,y,z)=c} کا حصہ ہو ، تب  \عددی{\kvec{n}} درج ذیل میں سے  ، جو پسند کا رخ دیتا ہو،  ہو گا۔
%eq-12, p1101
\begin{align}
\kvec{n}=\pm \frac{\nabla g}{\abs{\nabla g}}
\end{align}
مطابقتی بہاو ذیل ہو گا۔
%eq-13, p1101
\begin{align}
\text{\RL{بہاو}}&=\iint_S\kvec{F}\cdot\kvec{n}\dif \sigma \tag{\setlatin مساوات \حوالہ{مساوات_سمتی_میدان_بہاو_تعریف}}\\
&=\iint_R \big(\kvec{F}\cdot \frac{\pm \nabla g}{\abs{\nabla g}}\big)\frac{\abs{\nabla g}}{\abs{\nabla g\cdot \kvec{p}}}\dif  A\nonumber\\
&=\iint_R \kvec{F}\cdot \frac{\pm \nabla g}{\abs{\nabla g\cdot \kvec{p}}}\dif A
\end{align}

%ex-4, p1102
\ابتدا{مثال}
بیلن \عددی{y^2+z^2=1, z\ge 0} سے مستویات \عددی{x=0} اور \عددی{x=1}  سطح \عددی{S} کاٹتی ہیں۔ سطح \عددی{S} سے  باہر رخ \عددی{\kvec{F}=yz\aj+z^2\ak} کا بہاو تلاش کریں۔

 \موٹا{حل:}\quad
 سطح  \عددی{S}  پر باہر روح عمودی میدان  \عددی{\kvec{n}} کو  \عددی{g(x,y,z)=y^2+z^2} کی  ڈھلوان   (ذیل دیکھیں)سے معلوم کیا جا سکتا ہے۔
 \begin{align*}
 \kvec{n}=+\frac{\nabla g}{\abs{\nabla g}}=\frac{2y\aj+2z\ak}{\sqrt{4y^2+4z^2}}=\frac{2yaj+2z\ak}{2\sqrt{1}}=y\aj+z\ak
 \end{align*}
 چونکہ \عددی{\kvec{p}=\ak} ہے لہٰذا درج ذیل بھی ہو گا۔
 \begin{align*}
 \dif \sigma=\frac{\abs{\nabla g}}{\abs{\nabla g\cdot \ak}}\dif A=\frac{2}{\abs{2z}}\dif A=\frac{1}{z}\dif A
 \end{align*}
 ہم \عددی{z\ge 0} کے بنا  مطلق قیمت کی علامت ہٹا سکتے ہیں۔
 
 سطح پر \عددی{\kvec{F}\cdot \kvec{n}} کی قیمت ذیل کلیہ دیگا، جہاں آخری قدم میں \عددی{S} پر \عددی{y^2+z^2=1} ہو گا۔
 \begin{align*}
 \kvec{F}\cdot\kvec{n}&=(yz\aj+z^2\ak)\cdot(y\aj+z\ak)\\
 &=y^2z+z^3=z(y^2+z^2)\\
 &=z
 \end{align*}
 یوں \عددی{S} سے  \عددی{\kvec{F}} کا    اخراجی بہاو ذیل ہو گا۔
 \begin{align*}
 \iint_S\kvec{F}\cdot\kvec{n}\dif \sigma=\iint_S(z)\big(\frac{1}{z}\dif A\big)=\iint_{R_{xy}}\dif A=\text{\RL{کا رقبہ}}\, R_{xy}=2
 \end{align*}
 \انتہا{مثال}
 
\جزوحصہء{ باریک خول کی کمیت اور معیار اثر }
گھریلو برتن ، ڈھول ، اور دیگر باریک خول کو سطحوں  سے ظاہر کیا جا سکتا ہے ۔ان کی کمیت  اور معیار اثر  جدول \حوالہ{جدول_سمتی_میدان_کمیت_اور_معیار_اثر}   میں پیش  کلیات  سے حاصل کی  جا سکتی ہیں۔
%table 14.3, p1103
\begin{table}
\caption{نہایت باریک خول کی کمیت اور معیار اثر}
\label{جدول_سمتی_میدان_کمیت_اور_معیار_اثر}
\centering
\renewcommand{\arraystretch}{1.5}
\begin{tabular}{rL}
\toprule
\موٹا{کمیت}&
M=\iint_S \delta(x,y,z)\dif \sigma\\
&\text{\RL{کثافت (کمیت فی اکائی رقبہ) ہے}}  \delta (x,y,z)\,\text{\RL{پر}}\,(x,y,z)\,\text{\RL{نقطہ}})\\
\midrule
\موٹا{محددی مستویات کے}& 
M_{yz}=\iint_Sx \delta \dif \sigma,\quad\quad M_{xz}=\iint_S y\delta \dif \sigma, \quad\quad M_{xy}=\iint_S z\delta \dif \sigma\\
\موٹا{لحاظ سے اول معیار اثر}&\\
\midrule
\موٹا{مرکز کمیت کے محدد}&
\overline{x}=M_{yz}/M,\quad\quad \overline{y}=M_{xz}/M,\quad\quad \overline{z}=M_{xy}/M\\
\midrule
\موٹا{جمودی معیار اثر}&
I_x=\iint_S (y^2+z^2)\delta \dif \sigma, \quad\quad\quad I_y=\iint_S (x^2+z^2)\delta \dif \sigma\\
&
I_z=\iint_S (x^2+y^2)\delta \dif \sigma, \quad\quad\quad I_L=\iint_S r^2\delta \dif \sigma\\
&
\text{\RL{ہے}}\,r(x,y,z)\,\text{\RL{کا فاصلہ}}\,L\, \text{\RL{سے لکیر}}\,(x,y,z)\,\text{\RL{نقطہ}}\\
\midrule
\موٹا{لکیر  \(L\) کے لحاظ سے }&
R_L=\sqrt{I_L/M}\\
\موٹا{رداس دوار}\\
\bottomrule
\end{tabular}
\end{table}
%---------------------------------------

%ex-5, p1102
\ابتدا{مثال} 
رداس \عددی{a}   اور مستقل کثافت\عددی{\delta}کے باریک نصف کروی خول کا مرکز  کمیت تلاش کریں ۔

\موٹا{حل :}\quad
ہم  خول کو نصف کرہ :
\begin{align*}
f(x,y,z)=x^2+y^2+z^2=a^2,\quad z\ge 0
\end{align*}
سے ظاہر کرتے ہیں ( شکل\حوالہء{14.49} ) ۔ محور   \عددی{z} کے لحاظ سے سطح کی تشاکلی  کی بنا پر \عددی{\overline{x}=\overline{y}=0} ہوگا ۔کلیہ  \عددی{\overline{z}=M_{xy}/M} سے \عددی{\overline{z}} تلاش کرنا باقی ہے ۔

 خول کی کمیت درج ذیل ہے۔
 \begin{align*}
 M=\iint_S \delta \dif \sigma=\delta \iint _S \dif \sigma=(\delta)(\text{\RL{کا رقبہ}}, S)=2\pi a^2\delta
 \end{align*}
\عددی{M_{xy}} کا  تکمل  تلاش کرنے کی خاطر  \عددی{\kvec{p}=\ak} لیتے ہیں ۔یوں،
\begin{align*}
M_{xy}&=\iint_S z\delta \dif \sigma=\delta \iint_R z\frac{a}{z}\dif \sigma=\delta a \iint_R \dif A=\delta a(\pi a^2)=\delta \pi a^3
\end{align*}
لہٰذا ذیل ہو گا۔
\begin{align*}
\overline{z}=\frac{M_{xy}}{M}=\frac{\pi a^3 \delta}{2\pi a^2\delta}=\frac{a}{2}
\end{align*}
خول کا مرکز کمیت نقطہ \عددی{(0,0,a/2)} پر ہو گا۔
\انتہا{مثال}
%-------------------------------
%exercise 14.5 follows
