\حصہ{دیگر شرح تبدیلی}
ٹینکی سے \عددی{\SI{3000}{\liter\per\minute}} پانی کے انعکاس سے ٹینکی میں پانی کی گہرائی کس شرح سے تبدیل ہو گی؟ اس طرح کے سوالات میں ہم اس شرح کو معلوم کرنا چاہتے  ہیں جس کو ہم ناپ نہیں سکتے ہیں۔قابل ناپ شرح استعمال کرتے ہوئے یہ معلومات حاصل کی جاتی ہے۔

\ابتدا{مثال}\شناخت{مثال_تفرق_انعکاس}\ترچھا{انعکاس}\\
\عددی{\SI{3000}{\liter\per\minute}} کی شرح سے انعکاس کی صورت میں ٹینکی میں پانی کی گہرائی کم ہونے کی شرح جاننے کی خاطر ہم رداس \عددی{r} کی ٹینکی لیتے ہیں جس میں پانی کی گہرائی \عددی{h} ہے۔یوں پانی کا حجم \عددی{H=\pi r^2h} ہو گا جہاں حجم کو \عددی{H} سے ظاہر کیا گیا ہے (شکل \حوالہ{شکل_مثال_تفرق_انعکاس})۔اب ہمیں انعکاس
\begin{align*}
\frac{\dif H}{\dif t}=-3000
\end{align*}
بتلایا گیا ہے جہاں \عددی{t} وقت کو ظاہر کرتی ہے اور وقت کے ساتھ حجم کم ہونے کو منفی کی علامت سے ظاہر کیا گیا ہے۔ہمیں
\begin{align*}
\frac{\dif h}{\dif t}
\end{align*}
تلاش کرنا ہے۔ایسا کرنے کی خاطر ہمیں \عددی{H} اور \عددی{h} کا تعلق مساوات کی صورت میں لکھنا ہو گا۔یہ مساوات متغیرات کی اکائیوں پر منحصر ہو گی۔یوں حجم کو لٹر جبکہ رداس اور گہرائی کو میٹر میں رکھتے ہوئے درج ذیل لکھا جا سکتا ہے۔
\begin{align*}
H=1000\pi r^2h
\end{align*}
یاد رہے کہ ایک  مربع میٹر میں \عددی{1000} لٹر ہوتے ہیں۔دونوں اطراف کا وقت کے ساتھ تفرق لیتے ہیں
\begin{align*}
\frac{\dif H}{\dif t}=1000\pi r^2\frac{\dif h}{\dif t}
\end{align*}
جہاں دائیں جانب \عددی{r} مستقل ہے۔اس میں \عددی{\tfrac{\dif H}{\dif t}} کی معلوم قیمت پر کرتے ہوئے نا معلوم شرح \عددی{\tfrac{\dif h}{\dif t}} حاصل کرتے ہیں۔
\begin{align*}
\frac{\dif h}{\dif t}=\frac{-3000}{1000\pi r^2}=-\frac{3}{\pi r^2}
\end{align*}
پانی کی گہرائی \عددی{\tfrac{3}{\pi r^2}} میٹر فی منٹ کی شرح سے کم ہو گی۔آپ دیکھ سکتے ہیں کہ یہ شرح رداس پر منحصر ہے۔ کم رداس کی صورت میں شرح زیادہ اور زیادہ رداس کی صورت میں شرح کم ہو گی۔مثلاً \عددی{r=\SI{1}{\meter}} اور \عددی{r=\SI{10}{\meter}} کی صورت میں شرح درج ذیل ہوں گی۔
\begin{align*}
\frac{\dif h}{\dif t}&=-\frac{3}{\pi}\approx \SI{-0.95}{\meter\per\minute}=\SI{-95}{\centi\meter\per\minute}&&(r=\SI{1}{\meter})\\
\frac{\dif h}{\dif t}&=-\frac{3}{100\pi}\approx \SI{-0.0095}{\meter\per\minute}=\SI{-0.95}{\centi\meter\per\minute}&&(r=\SI{10}{\meter})
\end{align*}
%
\begin{figure}
\centering
\begin{minipage}{0.45\textwidth}
\centering
\begin{tikzpicture}
\draw(0,2.25) circle (1cm and 0.25cm);
\draw([shift={(0:1cm and 0.25cm)}]0,0) arc (0:-180:1cm and 0.25cm);
\draw(-1,0)--(-1,2.25);
\draw(1,0)--(1,2.25);
\draw([shift={(0:1cm and 0.25cm)}]0,1.5) arc (0:-180:1cm and 0.25cm);
\draw[gray]([shift={(0:1cm and 0.25cm)}]0,1.5) arc (0:180:1cm and 0.25cm);
\draw[stealth-stealth] (1.25,0)--(1.25,1.5)node[pos=0.5,right]{$h$};
\draw[-stealth](-1.25,1.5)--(-1.25,1)node[pos=0.5,left]{$\tfrac{\dif h}{\dif t}=?$};
\fill[white](-1.25,0) rectangle++(0.5,0.25);
\draw(-1,0.25)--++(-0.25-0.25,0)--++(0,-0.25-0.25)coordinate(kA);
\draw(-1,0)--++(-0.25,0)--++(0,-0.25)coordinate(kB);
\draw(-1,0) to [out=70,in=-70] (-1,0.25);
\draw(kA) to [out=-20,in=-160] (kB);
\draw[-stealth](kA)++(0.125,-0.1)--++(0,-0.5)node[pos=0.5,right]{$\tfrac{\dif H}{\dif t}=\SI{-3000}{\litre\per\minute}$};
\end{tikzpicture}
\caption{پانی کی ٹینکی (مثال \حوالہ{مثال_تفرق_انعکاس})}
\label{شکل_مثال_تفرق_انعکاس}
\end{minipage}\hfill
\begin{minipage}{0.45\textwidth}
\centering
\begin{tikzpicture}
\pgfmathsetmacro{\ang}{atan(2/3)}
\pgfmathsetmacro{\r}{0.4}
\draw[fill=lgray](0,0) circle (\r);
\draw(-\r,-0.1)--++(0.25,-0.75)coordinate(kA);
\draw(\r,-0.1)--++(-0.25,-0.75);
\draw[fill=lgray](kA)--++(0.3,0)--++(0,-0.15)--++(-0.3,0)--++(0,0.15);
\draw(kA)++(0.15,-0.15)coordinate(kB)--++(0,-2)node[pos=0.5,right]{$y$}--++(-3,0)coordinate(kC)node[pos=0.5,below]{$\SI{100}{\meter}$}node[circ]{}node[left]{\RL{فاصلہ پیما}}--(kB);
\draw[-stealth](kB)++(0.5,0)--++(0,0.25)node[pos=0.5,right]{$\tfrac{\dif y}{\dif t}=?$};
\draw[-stealth]([shift={(0:0.5)}]kC) arc (0:\ang:0.5);
\draw(kC)++(\ang/2:0.8)node[]{$\theta$};
\end{tikzpicture}
\caption{غبارہ (مثال \حوالہ{مثال_تفرق_غبارہ})}
\label{شکل_مثال_تفرق_غبارہ}
\end{minipage}
\end{figure}
\انتہا{مثال}
%=======================
\ابتدا{مثال}\شناخت{مثال_تفرق_غبارہ}\ترچھا{غبارہ کی اڑان}
گرم ہوا کا غبارہ زمین سے سیدھا آسمان کی طرف اٹھتا ہے (شکل \حوالہ{شکل_مثال_تفرق_غبارہ})۔ غبارے کی نقطہ اڑان  سے \عددی{\SI{100}{\meter}}  دور واقع \ترچھا{فاصلہ  پیما}\فرہنگ{پیما!فاصلہ}\حاشیہب{range finder}\فرہنگ{range finder} سے غبارے پر نظر رکھی جاتی ہے۔جس لمحہ فاصلہ پیما کا زاویہ صعود \عددی{\tfrac{\pi}{4}} تھا اس لمحہ زاویہ کی تبدیلی کی شرح \عددی{\SI{0.14}{\radian\per\minute}} تھی۔اس لمحہ پر غبارہ کس رفتار سے اوپر جا رہا تھا؟\\
حل:\quad
ہم اس کا جواب چھ قدموں میں دیتے ہیں۔\\
\موٹا{پہلا قدم:}\quad 
موقع کی تصور کشی کریں اور متغیرات کی نشاندہی کریں۔تصویر میں متغیرات \عددی{\theta} اور \عددی{y} درج ذیل ہیں جو بالترتیب فاصلہ پیما کا زاویہ صعود اور غبارے کی بلندی کو ظاہر کرتے ہیں۔ہم وقت کو \عددی{t} سے ظاہر کرتے ہیں اور فرض کرتے ہیں کہ \عددی{\theta} اور \عددی{y} متغیر \عددی{t} کے قابل تفرق تفاعل ہیں۔فاصلہ پیما سے غبارے کے ابتدائی مقام تک فاصلہ \عددی{\SI{100}{\meter}} ہے جس کر متغیر سے ظاہر کرنے کی ضرورت نہیں ہے۔\\
\موٹا{دوسرا قدم:}\quad
ان معلومات کو الجبرائی روپ میں لکھتے ہیں۔
\begin{align*}
\frac{\dif \theta}{\dif t}=\SI{0.14}{\radian\per\minute}&& (\theta=\tfrac{\pi}{4})
\end{align*}
\موٹا{تیسرا قدم:} \quad
جو ہم سے پوچھا گیا ہے اس کو لکھیں۔ہم سے \عددی{\theta=\pi/4} کی صورت میں \عددی{\tfrac{\dif y}{\dif t}} پوچھا گیا ہے۔\\
\موٹا{چوتھا قدم:}\quad
متغیرات  \عددی{\theta} اور \عددی{y} کا آپس میں تعلق لکھیں۔
\begin{align*}
\frac{y}{100}=\tan\theta\quad \implies \quad y=100\tan\theta
\end{align*}
\موٹا{پانچواں قدم:}\quad
زنجیری قاعدہ استعمال کرتے ہوئے \عددی{t} کے لحاظ سے تفرق حاصل کریں جو \عددی{\tfrac{\dif y}{\dif t}} (درکار معلومات) اور \عددی{\tfrac{\dif \theta}{\dif t}} (معلوم معلومات)  کے بیچ تعلق دیگا۔
\begin{align*}
\frac{\dif y}{\dif t}=100\sec^2\theta\frac{\dif \theta}{\dif t}
\end{align*}
\موٹا{چھٹا قدم:}\quad
\عددی{\theta=\tfrac{\pi}{4}} اور \عددی{\tfrac{\dif \theta}{\dif t}=0.14} پر کرتے ہوئے \عددی{\tfrac{\dif y}{\dif t}} کی قیمت تلاش کریں۔
\begin{align*}
\frac{\dif y}{\dif t}=100(\sec\tfrac{\pi}{4})^2(0.14)=\SI{28}{\meter\per\minute}
\end{align*}
\انتہا{مثال}
%========================

\موٹا{اس طرح کے مسائل حل کرنے کا لائحہ عمل}
\begin{itemize}

\item
مسئلے کی تصور کشی کریں۔وقت کو \عددی{t} سے ظاہر کریں اور تمام متغیرات کو \عددی{t} کے قابل تفرق تفاعل تصور کریں۔
\item
اعدادی معلومات کو منتخب کردہ متغیرات کی روپ میں لکھیں۔
\item
مطلوبہ شرح یا متغیر کو لکھیں (جو شرح کی صورت میں عموماً تفرق کی روپ میں ہو گا)۔
\item
متغیرات کا آپس میں تعلق لکھیں۔کئی بار آپ کو دو یا دو سے زیادہ مساواتوں کو اکٹھے کرتے ہوئے ایک مساوات حاصل کرنا ہو گا۔
\item
اس کا \عددی{t} کے لحاظ سے تفرق لیں۔اس کے بعد درکار شرح کو باقی متغیرات (جن کی قیمتیں آپ جانتے ہیں) کی صورت میں لکھیں۔
\item
معلوم معلومات کو پر کرتے ہوئے نا معلوم شرح کی قیمت دریافت کریں۔
\end{itemize}


%
\ابتدا{مثال}
پولیس ایک گاڑی کا پیچھا کر رہی ہے۔ جب چوک سے پولیس کی گاڑی کا فاصلہ \عددی{\SI{0.6}{\kilo\meter}} اور  بھاگنے والی گاڑی کا فاصلہ \عددی{\SI{0.8}{\kilo\meter}} ہے  اس لمحہ پر دونوں گاڑیوں کے بیچ فاصلہ \عددی{\SI{20}{\kilo\meter\per\hour}} سے بڑھ رہا ہے۔پولیس کی گاڑی کی رفتار \عددی{\SI{60}{\kilo\meter\per\hour}} ہونے کی صورت میں بھاگنے والی گاڑی کی رفتار کیا ہو گی؟\\
حل:\quad
ہم مذکورہ بالا اقدام پر چلتے ہوئے مسئلے کو حل کرتے ہیں۔\\
\موٹا{پہلا قدم:}\quad
\ترچھا{تصویر اور متغیرات۔} ہم کارتیسی محدد پر تصویر کشی کرتے ہیں۔ چوک کو مبدا پر رکھتے ہوئے بھاگنے والی گاڑی کو \عددی{x} محور جبکہ پولیس کی گاڑی کو \عددی{y} محور پر رکھتے ہیں۔ وقت کو \عددی{t} سے ظاہر کرتے ہوئے لمحہ \عددی{t} پر بھاگنے والی گاڑی کا مقام \عددی{x}, پولیس کی گاڑی کا مقام \عددی{y} اور دونوں گاڑیوں کے بیچ فاصلہ \عددی{s} ہے۔ ہم فرض کرتے ہیں کہ \عددی{x}، \عددی{y} اور \عددی{s} متغیر \عددی{t} کے قابل تفرق تفاعل ہیں۔\\
\موٹا{دوسرا قدم:}\quad
\ترچھا{اعدادی معلومات۔} لمحہ \عددی{t} پر درج ذیل ہمیں معلوم ہے۔
\begin{align*}
x=\SI{0.8}{\kilo\meter},\quad y=\SI{0.6}{\kilo\meter},\quad \frac{\dif y}{\dif t}=\SI{-60}{\kilo\meter\per\hour},\quad \frac{\dif s}{\dif t}=\SI{20}{\kilo\meter\per\hour}
\end{align*}  
\عددی{\tfrac{\dif y}{\dif t}} اس لئے منفی ہے کہ پولیس کی گاڑی مبدا کی طرف یعنی گھٹتی \عددی{y}  رخ چل رہی ہے۔\\
\موٹا{تیسرا قدم:}\quad
ہمیں \عددی{\tfrac{\dif x}{\dif t}} تلاش کرنا ہے۔\\
\موٹا{چوتھا قدم:}\quad
مسئلہ فیثاغورث کے تحت متغیرات کا تعلق \عددی{s^2=x^2+y^2} ہے۔\\
\موٹا{پانچواں قدم:}\quad
زنجیری قاعدہ کی مدد سے \عددی{t} کے لحاظ سے تفرق لیتے ہیں۔
\begin{align*}
2s\frac{\dif s}{\dif t}&=2x\frac{\dif x}{\dif t}+2y\frac{\dif y}{\dif t}\\
\frac{\dif s}{\dif t}&=\frac{1}{s}\big(x\frac{\dif x}{\dif t}+y\frac{\dif y}{\dif t}\big)\\
&=\frac{1}{\sqrt{x^2+y^2}}\big(x\frac{\dif x}{\dif t}+y\frac{\dif y}{\dif t}\big)
\end{align*}
\موٹا{چھٹا قدم:}\quad
\عددی{x=0.8}، \عددی{y=0.6}، \عددی{\tfrac{\dif y}{\dif t}=-60} اور \عددی{\tfrac{\dif s}{\dif t}=20} پر کرتے ہوئے \عددی{\tfrac{\dif x}{\dif t}} کی قیمت معلوم کریں۔
\begin{align*}
20&=\frac{1}{\sqrt{0.8^2+0.6^2}}\big(0.8\frac{\dif x}{\dif t}+0.6(-60)\big)\\
20&=0.8\frac{\dif x}{\dif t}-36\\
\frac{\dif x}{\dif t}&=\frac{20+36}{0.8}=70
\end{align*}
اس لمحہ پر بھاگنے والی گاڑی کی رفتار \عددی{\SI{70}{\kilo\meter\per\hour}} ہے۔ 

\انتہا{مثال}
%==================
\ابتدا{مثال}
پانی کی مخروطی ٹینکی  \عددی{\SI{9}{\meter\cubed\per\minute}} شرح سے بھری جاتی ہے۔مخروط کے قاعدہ کا رداس \عددی{\SI{5}{\meter}}، اس کا قد \عددی{\SI{10}{\meter}} ہے  اور اس کی نوک نیچے جانب ہے۔جس لمحہ پانی کی گہرائی \عددی{\SI{6}{\meter}} ہو اس لمحہ گہرائی کس شرح سے بڑھتی ہے؟\\
حل:\quad
ہم مذکورہ بالا اقدام پر چلتے ہوئے اس مسئلہ کو حل کرتے ہیں۔\\
\موٹا{پہلا قدم:}\quad
\ترچھا{تصویر کشی اور متغیرات۔} نیم بھری ٹینکی کی شکل بناتے ہیں۔اس مسئلے کے متغیرات درج ذیل ہیں۔
\begin{description}
\item[$:H$]
لمحہ \عددی{t} (منٹ) پر ٹینکی میں پانی کا حجم (مربع میٹر)۔
\item[$:x$]
لمحہ \عددی{t} (منٹ) پر پانی کی سطح کا رداس (میٹر)۔
\item[$:y$]
لمحہ \عددی{t} (منٹ) پر پانی کی گہرائی (میٹر)۔
\end{description}
ہم فرض کرتے ہیں کہ \عددی{H}، \عددی{x} اور \عددی{y} متغیر \عددی{t} کے قابل تفرق تفاعل ہیں۔ٹینکی کی جسامت مستقل مقدار ہے۔\\
\موٹا{دوسرا قدم:}\quad
\ترچھا{اعدادی معلومات۔} لمحہ \عددی{t} پر ہمیں درج ذیل معلوم ہے۔
\begin{align*}
y=\SI{6}{\meter},\quad \frac{\dif H}{\dif t}=\SI{9}{\meter\cubed\per\minute}
\end{align*}
\موٹا{تیسرا قدم:}\quad
\ترچھا{ہمیں} \عددی{\tfrac{\dif y}{\dif t}} \ترچھا{تلاش کرنا ہے۔}\\
\موٹا{چوتھا قدم:}\quad
متغیرات کا آپس میں تعلق:
\begin{align*}
H=\frac{1}{3}\pi x^2 y
\end{align*}
چونکہ لمحہ \عددی{t} پر ہمیں \عددی{x} اور \عددی{\tfrac{\dif x}{\dif t}} کے بارے میں معلومات فراہم نہیں کی گئی ہے لہٰذا ہمیں \عددی{x} سے چھٹکارا حاصل کرنا ہو گا۔ متشابہ مثلثات استعمال کرتے ہوئے شکل سے
\begin{align*}
\frac{x}{y}=\frac{5}{10}\quad \implies \quad x=\frac{y}{2}
\end{align*}
لکھا جا سکتا ہے۔یوں درج ذیل ہو گا۔
\begin{align*}
H=\frac{1}{3}\pi (\tfrac{y}{2})^2y=\frac{\pi}{12}y^3
\end{align*}
\موٹا{پانچواں قدم:}\quad
\عددی{t} \ترچھا{کے لحاظ سے تفرق۔} درج بالا مساوات کا تفرق لیتے ہیں۔
\begin{align*}
\frac{\dif H}{\dif t}=\frac{\pi}{12} \cdot 3y^2\frac{\dif y}{\dif t}=\frac{\pi}{4}y^2\frac{\dif y}{\dif t}
\end{align*}
اس کو \عددی{\tfrac{\dif y}{\dif t}} کے لئے حل کرتے ہیں۔
\begin{align*}
\frac{\dif y}{\dif t}=\frac{4}{\pi y^2}\frac{\dif H}{\dif t}
\end{align*}
\موٹا{چھٹا قدم:}\quad
\ترچھا{دی گئی معلومات  یعنی}  \عددی{y=6} اور \عددی{\tfrac{\dif H}{\dif t}=9} پر کرتے ہیں۔
\begin{align*}
\frac{\dif y}{\dif t}=\frac{4}{\pi (6^2)}\cdot 9=\frac{1}{\pi}\approx \SI{0.32}{\meter\per\minute}
\end{align*}
اس لمحے پر پانی کی گہرائی  \عددی{\SI{0.32}{\meter\per\minute}} سے بڑھ رہی ہے۔
\انتہا{مثال}
%===================

\حصہء{سوالات}

