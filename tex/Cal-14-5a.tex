%the first part is p1093 Exercises 14.4
%only Q1 to Q10 have been done. the rest needs to be added. their audio was not done

%the second part is sec 14.5 Surface Area and Surface Integrals
%it is done till the end of p1098 (example-2 there is complete)
%edited. figures and external references pending
\حصہء{سوالات}
\جزوحصہء{مسئلہ گرین کی تصدیق}
سوال \حوالہ{سوال_سمتی_تکمل_مسئلہ_گرین_الف} تا سوال \حوالہ{سوال_سمتی_تکمل_مسئلہ_گرین_د} میں میدان  \عددی{\kvec{F}= M\ai + N \aj} کے لیے مساوات   \حوالہ{مساوات_سمتی_تکمل_اخراج_اور_تکمل_پھیلاو}  اور  مساوات  \حوالہ{مساوات_سمتی_تکمل_اخراج_اور_تکمل_گردش}  کی  دونوں اطراف کی قیمتیں تلاش کر کے مسئلہ  گرین کی تصدیق کریں۔ دونوں صورتوں میں تکمل کا دائرہ کار قرص    \عددی{R: x^2 + y^2 \le a^2} اور  محدود کار  دائرہ   \عددی{C: r = (a \cos t)\ai + (a \sin t)\aj, 0 \le t \le 2\pi} لیں۔

\ابتدا{سوال}\شناخت{سوال_سمتی_تکمل_مسئلہ_گرین_الف}
\(\kvec{F} = -y\ai + x\aj\)
\انتہا{سوال}
\ابتدا{سوال}
\(\kvec{F} = y\ai\)
\انتہا{سوال}
\ابتدا{سوال}
\(\kvec{F} = 2x\ai - 3y\aj\)
\انتہا{سوال}
\ابتدا{سوال}\شناخت{سوال_سمتی_تکمل_مسئلہ_گرین_د}
\(\kvec{F} = -x^2y\ai + xy^2\aj\)
\انتہا{سوال}

\جزوحصہء{خلاف  گھڑی  دائری بہاو اور باہر  رخ  بہاو}
سوال \حوالہ{سوال_سمتی_تکمل_دائری_بہاو_الف} تا  \حوالہ{سوال_سمتی_تکمل_دائری_بہاو_ٹ} میں میدان \عددی{\kvec{F}}  اور منحنی \عددی{C}  کے لیے مسئلہ گرین استعمال کر کے  خلاف گھڑی دائری بہاو  اور باہر  رخ بہاو تلاش کریں۔

\ابتدا{سوال}\شناخت{سوال_سمتی_تکمل_دائری_بہاو_الف}
\عددی{\kvec{F} = (x - y)\ai + (y - x)\aj }  اور \عددی{C} کو  چوکور   \عددی{x=0}، \عددی{x=1}، \عددی{y=0}، \عددی{y=1}  محدود کرتا ہے۔
\انتہا{سوال}
\ابتدا{سوال}
\عددی{\kvec{F} = (x^2 + 4y)\ai + (x + y^2)\aj}  اور \عددی{C}کو  چوکور  \عددی{x=0}، \عددی{x=1}، \عددی{y=0}، \عددی{y=1} محدود کرتا ہے۔
\انتہا{سوال}
\ابتدا{سوال}
\عددی{\kvec{F} = (y^2 - x^2)\ai + (x^2 + y^2)\aj}   اور \عددی{C}کو   مثلث   \عددی{y=0}، \عددی{x=3}، \عددی{y=x}  محدود کرتا ہے۔
\انتہا{سوال}
\ابتدا{سوال}
\عددی{\kvec{F} = (x + y)\ai - (x^2 + y^2)\aj}    اور \عددی{C}کو   مثلث   \عددی{y=0}، \عددی{x=1}، \عددی{y=x}  محدود کرتا ہے۔
\انتہا{سوال}
\ابتدا{سوال}
\عددی{\kvec{F} = (x + e^x \sin y)\ai + (x + e^x \cos y)\aj} اور \عددی{C}  دو چشمہ \عددی{r^2=\cos 2\theta} کا دایاں ہاتھ گھیر ہے۔
\انتہا{سوال}
\ابتدا{سوال}\شناخت{سوال_سمتی_تکمل_دائری_بہاو_ٹ}
\عددی{\kvec{F} = \big(\tan^{-1} \frac{y}{x}\big)\ai + \ln(x^2 + y^2 )\aj} اور \عددی{C} اس خطے کی سرحد ہے جس کو قطبی محدد عدم مساوات \عددی{1\le r\le 2}، \عددی{0\le \theta\le \pi}  تعین کرتی  ہیں۔
\انتہا{سوال}
%=======================================================
%these were only Q1 to Q10. the rest needs to be added

%what follows is section 14.5 p1096
%sec 14.5 Surface Area and Surface Integrals
%=======================================================
\حصہ{سطحی رقبہ اور سطحی تکملات}
ہم مستوی  میں   خطہ پر تفاعل کا تکمل لینا جانتے ہیں لیکن ایسی صورت میں کیا ہوگا جب تفاعل ایک قوسی سطح پر پایا جاتا ہو؟    ایسی صورت میں تکمل کیسے حاصل  ہو گا؟ ایسا تکمل  ، جو\اصطلاح{ سطحی تکمل}\فرہنگ{سطحی تکمل}\حاشیہب{surface integral}\فرہنگ{surface integral} کہلاتا ہے ، کی قیمت تلاش کرنے کی خاطر      اس کو ، سطح کے نیچے محددی مستوی  پر  ، دہرا تکمل  کے روپ میں لکھا جاتا ہے( شکل  \حوالہء{14.39}) ۔ حصہ  \حوالہء{14.7} اور حصہ \حوالہء{14.8} میں ہم دیکھیں گے کہ سطحی تکملات کی مدد سے مسئلہ گرین کو تین ابعاد میں عمومیت دی جا سکتی ہے۔ 

\جزوحصہء{ سطحی  رقبہ  کی تعریف}
 شکل  \حوالہء{14.40} میں سطح \عددی{S}  اور   نیچے مستوی میں  اس کا  \قول{سایہ } خطہ \عددی{R}  دکھایا گیا ہے۔ سطح کی تعریفی مساوات   \عددی{f(x, y, c) = c} دیتی  ہے۔اگر سطح  \اصطلاح{ہموار }\فرہنگ{ہموار}\حاشیہب{smooth}\فرہنگ{smooth}  ہو   (\عددی{\nabla f} استمراری ہے اور\عددی{ S} پر کہیں بھی صفر نہیں ہے) ، ہم اس کے رقبہ کی تعریف اور قیمت\عددی{ R} پر دہرا تکمل کی صورت میں کر سکتے ہیں۔ 
 
 ہم  خطہ\عددی{ R} کی خانہ بندی چھوٹے چھوٹے مستطیلوں  \عددی{\Delta A_k} میں ہم  یوں  کرتے ہیں  جیسے \عددی{ R} پر تکمل کی تعریف پیش کرنا چاہتے ہوں۔ یہ\عددی{ S} کے رقبہ کی تعریف پیش کرنے کا کا پہلا قدم ہے۔ ہر ایک\عددی{\Delta A} کے بالکل اوپر کچھ بلندی پر سطح\عددی{\Delta \sigma_k} پایا جاتا ہے،  جس کو ہم  مماسی سطح کے چھوٹے حصہ\عددی{\Delta P_k} سے تخمین دے سکتے ہیں ۔ اس کی  وضاحت کرتے ہیں۔ رقبہ\عددی{\Delta A_k} کے پچھلے کونے\عددی{C_k} کے بالکل اوپر نقطہ\عددی{T_k (x_k , y_k , z_k)} پر سطح کے مماس کا\عددی{\Delta P_k} ایک ٹکڑا ہے۔ اگر مماسی سطح\عددی{ R} کا متوازی ہو ، تب\عددی{\Delta P_k} رقبہ\عددی{\Delta A_k} کا  موافق  ہوگا۔  بصورت دیگر ، یہ ایک مستطیل ہوگا جس کا رقبہ\عددی{\Delta A_k} کے رقبہ سے کچھ زیادہ ہوگا۔ 
 
  شکل \حوالہء{14.41} میں\عددی{\Delta \sigma_k} اور\عددی{\Delta P_k}  بڑھا چڑھا کر پیش کیے گئے  ہیں،   جہاں\عددی{T_k} پر ڈھلوان سمتیہ
  \عددی{\nabla f(x_k , y_k , z_k)} اور\عددی{ R} کا عمودی اکائی سمتیہ\عددی{\kvec{p}} بھی  دکھائے گئے ہیں۔ اس شکل میں\عددی{\nabla f} اور\عددی{\kvec{p}} کے   بیچ  زاویہ\عددی{\gamma_k} بھی دکھایا گیا ہے۔ اس شکل میں دیگر سمتیات\عددی{\kvec{u}_k} اور\عددی{\kvec{v}_k} مماسی مستوی میں\عددی{\Delta P_k} کے کناروں پر پائے جاتے ہیں۔ یوں\عددی{\kvec{u}_k \times \kvec{v}_k} اور\عددی{\nabla f} دونوں مماسی مستوی کو عمودی ہیں۔
  
   اعلٰی سمتی   ہندسہ سے ہم جانتے ہیں کہ کسی بھی مستوی پر، جس کا عمود\عددی{\kvec{p}} ہو ،  ایسے مستطیل  کی تظلیل کا رقبہ\عددی{|(\kvec{u}_k \times \kvec{v}_k) \cdot \kvec{p}|} ہوگا  جس کو\عددی{\kvec{u}_k} اور\عددی{\kvec{v}_k} تعین کرتے ہوں۔ یوں درج ذیل لکھا جا سکتا ہے۔
\begin{align}\label{مساوات_سمتی_تکمل_سطحی_رقبہ_تعریف_الف}
|(\kvec{u}_k \times \kvec{v}_k) \cdot \kvec{p}| = \Delta A_k
\end{align}
اب سمتی ضرب کی ایک خاصیت  (جو صلیبی ضرب  کی ایک حقیقت ہے)  یہ ہے کہ\عددی{|\kvec{u}_k \times \kvec{v}_k|} رقبہ\عددی{\Delta P_k} ہوگا  ،  لہٰذا مساوات \حوالہ{مساوات_سمتی_تکمل_سطحی_رقبہ_تعریف_الف}   ذیل روپ 
\begin{align}
\underbrace{|\kvec{u}_k \times \kvec{v}_k|}_{\Delta P_k} \underbrace{|\kvec{p}|}_{1}
 \underbrace{|\cos (\text{\RL{کے بیچ زاویہ}}\,\kvec{u}_k \times \kvec{v}_k \,\text{\RL{اور}}\, \kvec{p})|}_{\mathclap{\substack{
 \text{\RL{دونوں مماسی مستوی کو}}
 \kvec{u}_k\times \kvec{v}_k
 \text{\RL{اور}}
 \nabla f
\text{\RL{چونکہ}}\\
 \text{\RL{کے برابر ہو گا}}
 |\cos \gamma_k|
 \text{\RL{عمودی ہیں لہٰذا یہ}}
}}} = \Delta A_k
\end{align}
یا 
\begin{align*}
\Delta P_k |\cos \gamma_k | = \Delta A_k
\end{align*}
اختیار کرتی ہے جو   \عددی{\cos \gamma_k \ne 0} کی صورت میں  ذیل لکھی جا سکتی ہے۔
\begin{align*}
\Delta P_k = \frac{\Delta A_k}{|\cos \gamma_k|}
\end{align*}
جب تک  \عددی{\nabla f} زمینی مستوی کو  متوازی  نہ ہو اور  \عددی{\nabla f \cdot \kvec{p} \ne 0} ہو  \عددی{\cos \gamma_k \ne 0} ہوگا۔

چونکہ ، سطحی ٹکڑے  \عددی{\Delta \sigma_k} جو مل کر رقبہ  \عددی{S} دیتے ہیں، کو  \عددی{\Delta P_k} تخمیناً ظاہر کرتے ہیں ، لہٰذا مجموعہ 
\begin{align}\label{مساوات_سمتی_تکمل_خانے_الف}
\sum \Delta P_k = \sum \frac{\Delta A_k}{|\cos \gamma_k|}
\end{align}
سطحی رقبے \عددی{S} کا تخمین نظر  آتا ہے۔ ہم یہ بھی دیکھ سکتے ہیں کہ خطہ\عددی{R} کو مزید چھوٹے خانوں میں تقسیم کرنے سے یہ تخمین بہتر ہوگی۔ درحقیقت ، مساوات  \حوالہ{مساوات_سمتی_تکمل_خانے_الف} کے دائیں ہاتھ مجموعہ دوہرا تکمل 
\begin{align}\label{مساوات_سمتی_تکمل_خانے_ب}
\iint_R \frac{1}{|\cos \gamma|} \dif A
\end{align}
کے تخمینی مجموعے ہیں۔ اسی بنا پر جب بھی یہ تکمل موجود ہو ہم اسے\عددی{S} کے \اصطلاح{ رقبہ }\فرہنگ{رقبہ!تعریف}\حاشیہب{area}\فرہنگ{area!defined} کی تعریف لیتے ہیں۔ 

\جزوحصہء{عملی کلیہ}
کسی بھی سطح  \عددی{f (x, y, z) = c} کے لیے  \عددی{|\nabla f \cdot \kvec{p}| = |\nabla f| |\kvec{p}| |\cos \gamma|} ہوگا،  لہٰذا  درج ذیل لکھا جا سکتا ہے، جہاں  اکائی سمتیہ کی مطلق قیمت \عددی{1} ہے  (\عددی{|\kvec{p}|=1})۔
\begin{align*}
\frac{1}{|\cos \gamma |} = \frac{|\nabla f|}{|\nabla f \cdot \kvec{p}|}
\end{align*}
  اس کو مساوات  \حوالہ{مساوات_سمتی_تکمل_خانے_ب} کے ساتھ ملا کر رقبے  کا عملی کلیہ حاصل ہوتا ہے۔ 

\موٹا{سطحی رقبے کا کلیہ}

 بند اور محدود مستوی میں خطہ \عددی{R} پر سطح  \عددی{f (x, y, z) = c} کا رقبہ  ذیل ہو گا
\begin{align}\label{مساوات_سمتی_تکمل_سطحی_رقبہ_تعریفی_تکمل}
\text{\RL{سطحی رقبہ}} = \iint_R \frac{|\nabla f|}{|\nabla f \cdot \kvec{p}|} \dif A
\end{align}
 جہاں \عددی{\kvec{p}} خطہ \عددی{R} کا اکائی عمودی سمتیہ ہے   اور   \عددی{\nabla f \cdot \kvec{p} \ne 0} ہے۔ 
 
 یوں \عددی{\nabla f} کی  قدر   کو \عددی{\nabla f} کے  اس غیر  سمتی عمودی جزو کی قدر   سے تقسیم کر کے  جو  \عددی{R}  کو عمودی ہو،   خطہ \عددی{R} پر دوہرا تکمل لینے سے  رقبہ حاصل ہو گا۔
 
  ہم نے \عددی{\nabla f} کو استمراری تصور کر کے اور پورے \عددی{R} پر \عددی{\nabla f \cdot \ne 0} تصور کر کے  مساوات \حوالہ{مساوات_سمتی_تکمل_سطحی_رقبہ_تعریفی_تکمل}  حاصل کی۔ جب بھی یہ تکمل موجود ہو ، اس کی قیمت کو سطح \عددی{f (x, y, z) = c} کے اس حصہ کے رقبے کی تعریف لی جاتی ہے جو \عددی{R} کے      اوپر      (بلندی پر) پایا جاتا ہو۔ 

%================================
\ابتدا{مثال} 
قطر مکافی   مجسم \عددی{x^2 + y^2 - z = 0} کے نچلے حصے سے  مستوی  \عددی{z = 4} ایک سطح کاٹتا ہے۔ اس سطح  کا رقبہ تلاش کریں۔ 

\موٹا{حل :}\quad
ہم  مستوی  \عددی{xy} میں  سطح \عددی{S} اور اس کے نیچے خطہ \عددی{R} کا خاکہ بناتے ہیں (شکل \حوالہء{14.42} )۔ سطح \عددی{S}،  ہم قد سطح  \عددی{f (x, y, z) = x^2 + y^2 - z = 0} کا حصہ ہے اور \عددی{R}،  مستوی \عددی{xy} میں قرص  \عددی{x^2 + y^2 \le 4} ہے۔ مستوی \عددی{R} کا اکائی عمودی سمتیہ  \عددی{\kvec{p}=\ak} ہے۔

 سطح پر کسی  بھی نقطہ \عددی{x, y, z} پر  ذیل ہو گا۔
\begin{align*}
f (x, y, z) &= x^2 + y^2 - z\\
\nabla f &= 2 x \ai + 2 y \aj - \ak\\
|\nabla f| &= \sqrt{(2x)^2 + (2y)^2 + (-1)^2}\\
&= \sqrt{4x^2 + 4y^2 + 1}\\
|\nabla f \cdot \kvec{p}|& = |\nabla f \cdot \ak| = | -1 | = 1
\end{align*}
 خطہ \عددی{R} میں  \عددی{\dif A = \dif x \dif y} ہوگا،  لہٰذا  ذیل ہوگا۔ 
\begin{align*}
\text{\RL{سطحی رقبہ}}&= \iint_R \frac{|\nabla f|}{|\nabla f \cdot \kvec{p}|} \dif A &&(\text{\RL{مساوات \حوالہ{مساوات_سمتی_تکمل_سطحی_رقبہ_تعریفی_تکمل}}})\\
& = \iint_{x^2 + y^2 \le 4} \sqrt{4x^2 + 4y^2 + 1}\, \dif x \dif y \\
& = \int^{2 \pi}_0 \int^2_0 \sqrt{4r^2 + 1} \, r \dif r \dif \theta &&(\text{\RL{قطبی محدد}})\\
& = \int^{2 \pi}_0 \big[ \frac{1}{12} (4 r^2 + 1)^{\frac{3}{2}} \big]^2_0 \dif \theta \\
& = \int^{2 \pi}_0 \frac{1}{12} (17^{\frac{3}{2}} -1) \dif \theta = \frac{\pi}{6} (17 \sqrt{17} - 1)
\end{align*}
\انتہا{مثال} 
%------------------------------------------
\ابتدا{مثال} 
نصف کرہ   \عددی{x^2 + y^2 + z^2 =2, z \ge 0}  سے بیلن   \عددی{x^2 + y^2 =1} ایک ٹوپی کاٹتا ہے۔ اس ٹوپی کا رقبہ تلاش کریں ( شکل \حوالہء{14.43} )۔

\موٹا{حل:}\quad
ٹوپی \عددی{S} ، ہم قد  سطح  \عددی{f (x, y, z) = x^2 + y^2 + z^2 = 2} کا  حصہ ہے۔ یہ ایک ایک مطابقت کے ساتھ مستوی  \عددی{x y}  میں قرص  \عددی{R: x^2 + y^2 \le 1}  پر تظلیل کرتا ہے۔ سمتیہ \عددی{\kvec{p}=\ak} مستوی  \عددی{R}  کو عمودی ہے۔ 

سطح میں کسی بھی نقطے پر  ذیل ہو گا۔
\begin{align*}
f (x, y, z) &= x^2 + y^2 + z^2 \\
 \nabla f &= 2x \ai + 2y \aj + 2 z \ak \\
 |\nabla f| &= 2 \sqrt{x^2 + y^2 + z^2} = 2 \sqrt{2} && (\text{\RL{ہو گا}}\,x^2+y^2+z^2=2\, \text{\RL{چونکہ ٹوپی پر ہمیشہ }})\\
 |\nabla f \cdot \kvec{p}| &= |\nabla f \cdot \ak| = |2 z| = 2 z
\end{align*}
یوں ا درج ذیل حاصل ہو گا۔ 
\begin{align}\label{مساوات_سمتی_تکمل_مثال_رقبہ_الف}
\text{\RL{سطحی رقبہ}}=\iint_R \frac{|\nabla f|}{|\nabla f \cdot \kvec{p}|} \dif A = \iint_R \frac{2 \sqrt{2}}{2 z} \dif A = \sqrt{2} \iint_R \frac{\dif A}{z}
\end{align} 
یہاں \عددی{z} کا کیا  کیا جائے؟ 

چونکہ  \عددی{z} کرہ کی سطح پر ایک نقطے کا \عددی{z} محدد ہے،   لہٰذا  اس  کو   ہم \عددی{x} اور \عددی{y} کی صورت میں لکھ سکتے ہیں۔ 
\begin{align*}
z = \sqrt{2 - x^2 - y^2}
\end{align*} 
اس کو مساوات \حوالہ{مساوات_سمتی_تکمل_مثال_رقبہ_الف}  میں  ڈالتے ہیں۔
\begin{align*}
\text{\RL{سطحی رقبہ}}&= \sqrt{2} \iint_R \frac{\dif A}{z} = \sqrt{2} \iint_{x^2 + y^2 \le 1} \frac{\dif A}{\sqrt{2 - x^2 - y^2}} \\
& = \sqrt{2} \int^{2 \pi}_0 \int^1_0 \frac{r \dif r \dif \theta}{\sqrt{2 - r^2}}&&\text{\RL{(قطبی محدد)}} \\
& \sqrt{2} \int^{2 \pi}_0 \big[ -(2 - r^2)^{\frac{1}{2}} \big]^{r = 1}_{r = 0} \dif \theta \\
& = \sqrt{2} \int^{2 \pi}_0 (\sqrt{2} - 1) \dif \theta = 2 \pi (2 - \sqrt{2})
\end{align*}
\انتہا{مثال}
%----------------------------------------
%the above example ends at the last line of p1098
