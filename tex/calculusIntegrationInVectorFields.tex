\باب{سمتی میدان میں تکمل}\شناخت{باب_سمتی_میدان_میں_تکمل}
\موٹا{ایک جائزہ}\quad
اس باب  کاموضوع  سمتی میدان میں تکمل ہے۔ اس باب کی ریاضی کو برقناطیسیت  کے خواص   بیان کرنے کے لئے، تاروں میں حرارت  کے بہاو پر غور ، اور  مصنوعی سیارہ کو  مدار میں منتقل کرنے  کے لئے درکار توانائی کے حصول کے لئے استعمال کیا جاتا ہے۔

\حصہ{خطی تکمل}
جب فضا میں  تفاعل \عددی{f(x,y,z)} کے دائرہ کار سے منحنی \عددی{\kvec{r}(t)=g(t)\ai+h(t)\aj+k(t)\ak,\,a\le t\le b}  گزرے تب اس   منحنی  کے ساتھ چلتے ہوئے \عددی{f}  کی قیمتیں مرکب تفاعل \عددی{f(g(t),h(t),k(t))} دیگا۔ نقطہ \عددی{a} سے \عددی{b} تک لمبائی قوس  کے لحاظ سے اس مرکب تفاعل کے تکمل کو قوس کے ساتھ \عددی{f} کا  \اصطلاح{خطی  تکمل}\فرہنگ{تکمل!خطی}\حاشیہب{line integral}\فرہنگ{integral!line} کہتے ہیں۔  تین بعدی جیومیٹری کے باوجود،  خطی تکمل حقیقی اعداد کے وقفہ پر حقیقی قیمت تفاعل کا سادہ تفاعل ہو گا۔

خطی تکمل کی اہمیت اس کے استعمال میں ہے۔ ان تکملات  کی مدد سے ہم  متغیر قوتوں کی فضا میں راہ پر کام    اور قوس  کے ساتھ یا  سرحد پار کرتی سیال کی شرح بہاو  کا حساب کرتے ہیں۔

\جزوحصہء{تعریفات اور  علامتیت}
فرض کریں تفاعل \عددی{f(x,y,z)}  کے  دائرہ کار میں منحنی \عددی{\kvec{r}(t)=g(t)\ai+h(t)\aj+k(t)\ak,\,a\le t\le b} پائی جاتی ہے۔ ہم اس منحنی کو متناہی تعداد کے ذیلی  قوسین میں ٹکڑے کرتے ہیں۔ ایک علامتی ذیلی قوس کی لمبائی \عددی{\Delta s_k} ہو گی۔ ہم ہر ذیلی قوس پر  ایک نقطہ \عددی{(x_k,y_k,z_k)}  منتخب کر کے درج ذیل مجموعہ لیتے ہیں۔
\begin{align}\label{مساوات_میدان_خطی_تکمل_الف}
J_n=\sum\limits_{k=1}^{n} f(x_k,y_k,z_k)\Delta s_k
\end{align}
اگر \عددی{f} استمراری ہو اور \عددی{g}، \عددی{h}،  اور \عددی{k} کے  اول تفرقات استمراری ہوں تب جیسے جیسے  \عددی{n} بڑھایا جائے، \عددی{\Delta s_k}  صفر تک پہنچے گی اور   مساوات \حوالہ{مساوات_میدان_خطی_تکمل_الف} کا مجموعہ  ایک حد کو پہنچے گا۔ ہم اس حد کو \اصطلاح{ \عددی{a} تا \عددی{b} اس قوس پر \عددی{f} کا تکمل} کہتے ہیں۔  قوس کو \عددی{C} سے ظاہر کرتے ہوئے  اس تکمل کو علامتی طور پر درج ذیل لکھا جاتا ہے۔
\begin{align}\label{مساوات_میدان_خطی_تکمل_ب}
\int_C f(x,y,z)\dif s\quad \quad \text{\RL{"\عددی{C} پر \عددی{f} کا تکمل"}}
\end{align}

\جزوحصہء{ہموار منحنیات پر تکمل کی قیمت کا حصول}
اگر وقفہ \عددی{a\le t\le b} پر \عددی{\kvec{r}(t)} ہموار ہو   (\عددی{\kvec{v}=\tfrac{\dif\kvec{r}}{\dif t}}   استمراری  ہو اور کبھی بھی \عددی{\kvec{0}} نہ ہو)   تب ہم \عددی{\dif s}  کو بیان کرنے کے لئے درج ذیل مساوات استعمال کر سکتے ہیں  چونکہ  اس سے \عددی{\dif s=\abs{\kvec{v}(\tau)}\dif t} لکھا جا سکتا ہے۔
\begin{align*}
s(t)=\int_a^t \abs{\kvec{v}(\tau)} \dif \tau\quad\text{\small\RL{حصہ \حوالہ{حصہ_سمتی_تفاعل_لمبائی_قوس_اور_اکائی_سمتیہ} کی  مساوات \حوالہ{مساوات_سمتی_تفاعل_لمبائی_قوس_ت} میں \عددی{t_0=a}}}
\end{align*}
اعلٰی احصاء کا ایک مسئلہ کہتا ہے کہ ایسی صورت میں ہم  درج ذیل طریقہ سے \عددی{C} پر \عددی{f} کے تکمل کی قیمت حاصل کر  سکتے ہیں۔
\begin{align*}
\int_C f(x,y,z)\dif s=\int_a^b f(g(t),h(t),k(t))\abs{\kvec{v}(t)}\dif t
\end{align*} 
ہم جس مقدار معلوم روپ کو بھی استعمال کریں، جب تک زیر استعمال مقدار معلوم   روپ ہموار ہو، یہ کلیہ ہمیں تکمل کی قیمت دیگا۔ 

\جزوحصہء{خطی تکمل کی قیمت کا حصول}
منحنی  \عددی{C} پر استمراری تفاعل \عددی{f} کا  تکمل لینے کے لئے
\begin{enumerate}[a.]
\item
\عددی{C} کی مقدار معلوم روپ تلاش کریں:
\[\kvec{r}(t)=g(t)\ai+h(t)\aj+k(t)\ak,\quad a\le t\le b\]
\item
درج ذیل تکمل کی قیمت حاصل کریں۔
\begin{align}\label{مساوات_میدان_خطی_تکمل_پ}
\int_Cf(x,y,z)\dif s=\int_a^bf(g(t),h(t),k(t))\abs{\kvec{v}(t)}\dif t
\end{align}
\end{enumerate}

دھیان رہے کہ مستقل تفاعل \عددی{f=1} کی صورت میں مذکورہ بالا تکمل \عددی{C} کی لمبائی دیگا۔

%====================
\ابتدا{مثال}\شناخت{مثال_میدان_راہ_الف}
مبدا سے  نقطہ \عددی{(1,1,1)} تک قطع پر \عددی{f(x,y,z)=x-3y^2+z} کتکمل کریں۔

حل:\quad
ہم  ذہن میں آنے والا سادہ ترین مقدار معلوم روپ استعمال کرتے ہیں
\[\kvec{r}(t)=t\ai+t\aj+t\ak,\quad 0\le t\le 1\]
جس کی اجزاء کے اول تفرقات استمراری ہیں اور \عددی{\abs{\kvec{v}(t)}=\sqrt{1^2+1^2+1^2}=\sqrt{3}} کبھی بھی \عددی{\kvec{0}} نہیں ہو سکتا ہے لہٰذا  یہ مقدار معلوم روپ ہموار ہے۔ یوں   \عددی{C} پر \عددی{f} کا تکمل درج ذیل ہو گا۔
\begin{align*}
\int_C f(x,y,z)\dif s&=\int_0^1 f(t,t,t)(\sqrt{3})\dif t&&\text{\RL{مساوات \حوالہ{مساوات_میدان_خطی_تکمل_پ}}}\\
&=\int_0^1 (t-3t^2+t)\sqrt{3}\dif t\\
&=\sqrt{3}\int_0^1 (2t-3t^2)\dif t=\sqrt{3}\big[t^2-t^3\big]_0^1=0
\end{align*} 
\انتہا{مثال}
%===================

\جزوحصہ{جمع پذیری}
  اگر متناہی تعداد کی منحمنات \عددی{C_1}، \عددی{C_2}، \نقطے، \عددی{C_n} کو ایک دوسرے کے ساتھ جوڑ کر منحنی \عددی{C} حاصل کی جائے تب  \عددی{C} پر تفاعل کا تکمل ان منحنیات پر تفاعل کے تکملات کا مجموعہ ہو گا:
\begin{align}\label{مساوات_میدان_خطی_تکمل_ت}
\int_C f\dif s=\int_{C_1} f\dif s+\int_{C_2} f\dif s+\cdots+\int_{C_n} f\dif s
\end{align}

\ابتدا{مثال}\شناخت{مثال_میدان_راہ_ب}
مبدا سے نقطہ \عددی{(1,1,1)} تک  راہ \عددی{C_1} اور \عددی{C_2} ]پر چل کر پہنچا جاتا ہے۔ یوں \عددی{C} ان کا شتراک \عددی{C_1\cup C_2} ہو گا۔ تفاعل \عددی{f(x,y,z)=x-3y^2+z}  کے تکمل کی قیمت  \عددی{C_1\cup C_2} پر  تلاش کریں۔

حل:\quad
ہم \عددی{C_1} اور \عددی{C_2} کے لئے،  ذہن میں آنے  والے  سادہ ترین،  مقدار معلوم روپ  استعمال کرتے ہیں:
\begin{align*}
C_1:\quad \kvec{r}(t)&=t\ai+t\aj,\, 0\le t\le 1;\, \abs{\kvec{v}}=\sqrt{1^2+1^2}=\sqrt{2}\\
C_2:\quad \kvec{r}(t)&=\ai+\aj+t\ak,\, 0\le t\le 1;\, \abs{\kvec{v}}=\sqrt{0^2+0^2+1^2}=1
\end{align*}
ان مقدار معلوم روپ کے ساتھ درج ذیل حاصل ہو گا۔
\begin{align*}
\int_{C_1\cup C_2}f(x,y,z)\dif s&=\int_{C_1}f(x,y,z)\dif s+\int_{C_2}f(x,y,z)\dif s&&\text{\RL{مساوات \حوالہ{مساوات_میدان_خطی_تکمل_ت}}}\\
&=\int_0^1 f(t,t,0)\sqrt{2} \dif t+\int_0^1 f(1,1,t)(1)\dif t\\
&=\int_0^1 (t-3t^2+0)\sqrt{2}\dif t+\int_0^1(1-3+t)(1)\dif t\\
&=\sqrt{2}\big[\frac{t^2}{2}-t^3\big]_0^1+\big[\frac{t^2}{2}-2t\big]_0^1=-\frac{\sqrt{2}}{2}-\frac{3}{2}
\end{align*}
\انتہا{مثال}
%=======================

یہاں   مثال \حوالہ{مثال_میدان_راہ_الف} اور مثال \حوالہ{مثال_میدان_راہ_ب} کے نتائج پر غور کرتے ہیں۔اول، دیکھیں کہ  موزوں منحنی کے اجزاء \عددی{f} میں پر کرتے ہی  \عددی{t} کے لحاظ سے ایک  سادہ تکمل حاصل ہوتا ہے۔ دوم،  \عددی{C_1\cup C_2} پر \عددی{f} کا تکمل لینے کے لئے \عددی{C_1} اور \عددی{C_2} پر \عددی{f} کے علیحدہ علیحدہ تکملات لے کر نتائج  کا مجموعہ لیا جاتا ہے۔سوم،      مثال \حوالہ{مثال_میدان_راہ_الف} میں \عددی{C}  اور مثال \حوالہ{مثال_میدان_راہ_ب} میں \عددی{C_1\cup C_2} پر تکمل کے نتائج ایک دوسرے سے مختلف تھے۔ عموماً تفاعل کے لئے دو نقطوں کے بیچ مختلف راہ پر تملات کے نتائج ایک دوسرے سے مختلف ہوں گے۔ البتہ بعض تفاعل کے لئے تکمل کی قیمت پر راہ کا کوئی اثر نہیں ہوتا ہے۔

\جزوحصہء{کمیت اور معیار اثر کا حساب}
ہم اسپرنگ اور تار کو فضا میں ہموار منحنی پر  استمراری کمیتی کثافت \عددی{\delta(x,y,z)} کی تقسیم تصور کرتے ہیں۔ یوں اسپرنگ یا تار کی کمیت، مرکز کمیت، اور ان کے معیار اثر اور  رداس دوار کا حساب    درج ذیل  کلیات سے کیا جائے گا۔ یہی کلیات باریک (پتلی) تار کے لئے بھی کارآمد ہوں گے۔
\begin{description}
\item{کیمت:}\quad
\(M=\iiint\limits_D \delta(x,y,z) \dif H\)
\item{محددی مستویات کے لحاظ سے اول معیار اثر:}
\[M_{yz}=\int_C x\delta \dif s,\quad M_{xz}=\int_C y\delta \dif s,\quad M_{xy}=\int_Cz\delta \dif s\]
\item{مرکز کمیت کے محدد:}
\[\bar{x}=\frac{M_{yz}}{M},\quad \bar{y}=\frac{M_{xz}}{M},\quad \bar{z}=\frac{M_{xy}}{M}\]
\item{معیار اثر:}
\begin{align*}
I_x&=\int_C(y^2+z^2)\delta \dif s & I_y&=\int_C(x^2+z^2)\delta \dif s\\
I_z&=\int_C(x^2+y^2)\delta \dif s& I_L&=\int_C r^2\delta \dif s
\end{align*}
جہاں  لکیر \عددی{L} سے نقطہ \عددی{(x,y,z)} تک فاصلہ \عددی{r(x,y,z)} ہے۔
\item{لکیر \عددی{L} کے لحاظ سے رداس دور:}\quad
\(R_L=\sqrt{\frac{I_L}{M}}\)
\end{description}

%==================================
\ابتدا{مثال}
ایک اسپرنیگ درج زیل پیچدار منحنی کے ساتھ ساتھ پڑا ہے۔
\[\kvec{r}(t)=(\cos t)\ai+(\sin t)\aj+t\,\ak,\quad 0\le t\le 2\pi\]
اس اسپرنگ کی کثافت مستقل تفاعل \عددی{\delta=1} ہے۔ اس اسپرنگ کی کمیت اور مرکز کمیت  اور محور \عددی{z} کے لحاظ سے   جمودی معیار اثر اور رداس دوار معلوم کریں۔

حل:\quad
ہم اسپرنگ کا خاکہ بناتے ہیں۔ تشاکلی کی بنا اس کا مرکز کمیت محور \عددی{z} پر  نقطہ \عددی{(0,0,\pi)} پر پایا جائے گا۔  باقی حساب کے لئے ہم \عددی{\abs{\kvec{v}(t)}} تلاش کرتے ہیں:
\begin{align*}
\abs{\kvec{v}(t)}&=\sqrt{\big(\frac{\dif x}{\dif t}\big)^2+\big(\frac{\dif y}{\dif t}\big)^2+\big(\frac{\dif z}{\dif t}\big)^2}\\
&=\sqrt{(-4\sin 4t)^2+(4\cos 4t)^2+1}=\sqrt{17}
\end{align*}
اب مذکورہ بالا کلیات استعمال کرتے ہوئے درج ذیل حاصل ہو گا۔
\begin{align*}
M&=\int\limits_{\text{پیچدار}} \delta \dif s=\int_0^{2\pi} (1)\sqrt{17}\dif t=2\pi\sqrt{17}\\
I_z&=\int\limits_{\text{پیچدار}} (x^2+y^2)\delta \dif s=\int_0^{2\pi}(\cos^2 4t+\sin^2 4t)(1)(\sqrt{17})\dif t\\
&=\int_0^{2\pi}\sqrt{17}\dif t=2\pi\sqrt{17}\\
R_z&=\sqrt{\frac{I_z}{M}}=\sqrt{\frac{2\pi\sqrt{17}}{2\pi\sqrt{17}}}=1
\end{align*}
دھیان رہے کہ محور \عددی{z} کے لحاظ سے رداس دوار عین اس بیلن کے رداس  جتنا ہے جس پر اسپرنگ لپیٹا گیا ہے۔
\انتہا{مثال}
%=====================
\ابتدا{مثال}
مستوی \عددی{yz} میں نصف دائرہ \عددی{y^2+z^2=1,\, z\ge 0} پر ایک  دبلا پتلا محراب پایا جاتا ہے۔ محراب کے نقطہ \عددی{(x,y,z)} پر کثافت \عددی{\delta(x,y,z)=2-z} ہے۔ محراب کا مرکز کمیت تلاش کریں۔

حل:\quad
چونکہ یہ  محراب مستوی \عددی{yz} میں پایا جاتا ہے  اور محور \عددی{z} کے لحاظ سے اس کی کمیتی تقسیم دونوں اطراف یکساں ہے   لہٰذا \عددی{\bar{x}=0} اور \عددی{\bar{y}=0} ہوں گے۔ہم دائرہ کی مقدار معلوم روپ
\[\kvec{r}(t)=(\cos t)\aj+(\sin t)\ak,\, 0\le t\le \pi\]
 لکھتے ہوئے \عددی{\bar{z}} دریافت کرتے ہیں۔اس مقدار معلوم روپ کے لئے
\[\abs{\kvec{v}(t)}=\sqrt{\big(\frac{\dif x}{\dif t}\big)^2+\big(\frac{\dif y}{\dif t}\big)^2+\big(\frac{\dif z}{\dif t}\big)^2}=\sqrt{(0)^2+(-\sin t)^2+(\cos t)^2}=1\]
ہو گا۔ یوں مذکورہ بالا کلیات استعمال کرتے ہوئے درج ذیل ہو گا۔
\begin{align*}
M&=\int_C \delta \dif s=\int_C (2-z)\dif s=\int_0^{\pi}(2-\sin t)\dif t=2\pi-2\\
M_{xy}&=\int_0^C z\delta \dif s=\int_C z(2-z)\dif s=\int_0^{\pi} (\sin t)(2-\sin t)\dif t\\
&=\int_0^{\pi} (2\sin t-\sin^2 t)\dif t=\frac{8-\pi}{2}\\
\bar{z}&=\frac{M_{xy}}{M}=\frac{8-\pi}{2}\cdot\frac{1}{2\pi-2}\approx 0.57
\end{align*}
یوں مرکز کمیت تقریباً \عددی{(0,0,0.57)} ہو گا۔
\انتہا{مثال}
%==================
