\باب{کثیر المتغیر تفاعل اور جزوی تفرقات}
\موٹا{جائزہ}\\
سائنس میں دو یا دو سے زائد غیر تابع  متغیرات کے تفاعل    ایک متغیر کے تفاعل سے زیادہ کثرت سے پائے جاتے ہیں اور ان کی علم   احصاء  زیادہ  عمدہ   ہوتی  ہے۔زیادہ متغیرات ایک دوسرے پر زیادہ طریقوں سے اثر انداز ہو سکتے ہیں جس کی بنا ان کے تفرقات    مختلف اور زیادہ دلچسپ  صورتیں اختیار  کر سکتے ہیں۔ ان کے تکملات  زیادہ اقسام کے عملی  مسائل میں کام آتے ہیں۔ احتمال،  سیالی حرکیات، اور برقیات، وغیرہ،    پر غور کے دوران  ایک سے زائد متغیرات  کے تفاعل قدرتی طور پر رونما ہوتے ہیں۔ان تفاعل کی  ریاضیات، سائنس کی عظیم  کامیابیوں میں  سے ایک ہے۔


\حصہ{کثیر متغیرات کے تفاعل}
کئی تفاعل ایک سے زائد متغیرات کے تابع  ہوتے ہیں۔دائری نلکی کا حجم،  اس    کے رداس اور قد سے،   تفاعل  \عددی{H=\pi r^2h}   دیتا ہے۔ مستوی \عددی{xy} میں نقطہ \عددی{N(x,y)} کے دو محدد سے،   قطع مکافی   \عددی{z=x^2+y^2}  کا قد  تفاعل \عددی{f(x,y)=x^2+y^2} دیتا ہے۔اس حصہ میں ہم ایک سے زیادہ متغیرات کے تابع تفاعل متعارف کرتے ہیں  اور ان کو ترسیم کرنے کے طریقوں پر غور کرتے ہیں۔

\جزوحصہء{تفاعل اور متغیرات} 
کثیر غیر تابع حقیقی متغیرات  کے حقیقی قیمت تفاعل کی تعریف  بالکل واحد متغیر کے تفاعل کی طرح کی جاتی ہے۔ان کے وقفے  حقیقی   (تین، چار، وغیرہ)    اعداد کے  مرتب جوڑی  کے سلسلے ہوں گے اور ان  کی  سعت  ، اس طرح  کے حقیقی اعداد کے سلسلے ہوں گے جن کے ساتھ ہم کام کرتے آ رہے ہیں۔

\ابتدا{تعریفات}
فرض کریں \عددی{n} عدد حقیقی اعداد \عددی{x_1,x_2,\cdots,x_n} کا سلسلہ \عددی{D} ہے۔  تب \عددی{D} پر \اصطلاح{حقیقی قیمت تفاعل}\فرہنگ{تفاعل!حقیقی قیمت}\حاشیہب{real valued function}\فرہنگ{function!real valued} \عددی{f} سے مراد وہ قاعدہ ہے جو \عددی{D} کے ہر رکن کو حقیقی عدد
\begin{align*}
w=f(x_1,x_2,\cdots,x_n)
\end{align*}
مختص کرتا ہو۔سلسلہ \عددی{D} اس تفاعل کا \اصطلاح{دائرہ کار}\فرہنگ{دائرہ کار}\حاشیہب{domain}\فرہنگ{domain} ہو گا۔  تفاعل  \عددی{f} کی  \عددی{w} قیمتوں کا سلسلہ  \عددی{f} کی \اصطلاح{سعت}\فرہنگ{سعت}\حاشیہب{range}\فرہنگ{range} ہو گی۔ علامت \عددی{w} تفاعل \عددی{f} کا \اصطلاح{تابع متغیر}\فرہنگ{متغیر!تابع}\حاشیہب{dependent variable}\فرہنگ{variable!dependent} ہو گا اور \عددی{f} کو  \عددی{n}  \اصطلاح{غیر تابع متغیرات}\فرہنگ{متغیر!غیر تابع}\حاشیہب{independent variable}\فرہنگ{variable!independent} \عددی{x_1} تا \عددی{x_n}   کا تفاعل کہتے ہیں۔ ہم  ان \عددی{x} کو تفاعل کے \اصطلاح{داخلی متغیرات}\فرہنگ{متغیر!داخلی}\حاشیہب{input variable}\فرہنگ{variable!input} اور \عددی{w} کو تفاعل کا \اصطلاح{خارجی متغیر}\فرہنگ{متغیر!خارجی}\حاشیہب{output variable}\فرہنگ{variable!output}  بھی کہتے ہیں۔
\انتہا{تعریفات}
%==================

اگر \عددی{f} دو غیر تابع متغیرات کا تفاعل ہو تب عموماً ہم  ان غیر تابع متغیرات کو \عددی{x} اور \عددی{y} کہتے ہیں اور \عددی{f} کے دائرہ کار  کو مستوی \عددی{xy} میں ایک خطہ تصور کرتے ہیں۔ اگر \عددی{f} تین غیر تابع متغیرات کا تفاعل ہو تب ہم  ان متغیرات کو \عددی{x}، \عددی{y} اور \عددی{z} کہتے ہیں اور تفاعل کے  دائرہ کار کو فضا میں ایک خطہ تصور کرتے ہیں۔

عملی استعمال میں ہم  وہ حروف استعمال کرتے ہیں جو ہمیں ان چیزوں  کی یاد  دلا سکیں جن کے لئے یہ متغیرات استعمال  کیے گئے ہوں۔ یہ  کہنے  کی خاطر کہ دائری نلکی کا حجم اس کے رداس \عددی{r}  اور قد \عددی{h}   کا تفاعل ہو گا، ہم \عددی{H=f(r,h)} لکھ سکتے ہیں۔بالخصوص  ہم \عددی{f(r,h)} کی جگہ وہ کلیہ استعمال کر سکتے ہیں جو \عددی{r} اور \عددی{h} کی قیمتوں سے \عددی{H} کی  قیمت دیتا ہو، یعنی ہم  \عددی{H=\pi r^2h} لکھ سکتے ہیں۔دونوں صورتوں میں \عددی{r} اور \عددی{h} غیر تابع متغیرات ہوں گے اور \عددی{H} تابع متغیر ہو گا۔

ہمیشہ کی طرح،ہم  تفاعل کی تعریفی کلیہ میں غیر تابع متغیرات کی قیمتیں پر کر  کے مطابقتی تابع متغیر کی قیمت حاصل کرتے ہیں۔

\ابتدا{مثال}
نقطہ \عددی{(3,0,4)} پر تفاعل \عددی{f(x,y,z)=\sqrt{x^2+y^2+z^2}} کی قیمت درج ذیل ہو گی۔
\begin{align*}
f(3,0,4)=\sqrt{(3)^2+(0)^2+(4)^2}=\sqrt{25}=5
\end{align*}
\انتہا{مثال}
%===============
\جزوحصہء{وقفے}
ایک سے زیادہ متغیرات کے تفاعل  کی تعریف  میں، ہمیشہ کی طرح، ہم  ان مداخل کو  شامل نہیں  کرتے ہیں  جو مخلوط اعداد دیتے ہوں یا جن کی وجہ سے   تقسیم صفر  کا عمل  پیدا ہوتا ہو۔یوں \عددی{f(x,y)=\sqrt{y-x^2}} میں \عددی{y} کی قیمت \عددی{x^2} کی قیمت سے کم نہیں ہو سکتی ہے اور \عددی{f(x,y)=\tfrac{1}{xy}} میں \عددی{xy} کی قیمت صفر نہیں ہو سکتی ہے۔ان شرائط کو مطمئن کرتے ہوئے، تفاعل کے  دائرہ کار سے مراد  وہ بڑے سے بڑا سلسلہ ہو گا جس پر تفاعل کا  تعریفی قاعدہ حقیقی اعداد  پیدا کرتا ہو۔ 

\ابتدا{مثال}\ترچھا{دو متغیرات کے تفاعل}\\
\begin{center}
\renewcommand{\arraystretch}{1.2} 
\begin{tabular}{LLL}
\multicolumn{1}{C}{\text{\RL{تفاعل}}}&\text{\RL{دائرہ کار}}&\multicolumn{1}{C}{\text{\RL{سعت}}}\\
\midrule
w=\sqrt{y-x^2}&y\ge x^2&[0,\infty)\\
w=\tfrac{1}{xy}&xy\ne 0&(-\infty,0)\cup(0,\infty)\\
w=\sin xy&\text{\RL{پورا مستوی}}&[-1,1]
\end{tabular}
\end{center}
\انتہا{مثال}
%===============
\ابتدا{مثال}\ترچھا{تین متغیرات کے  تفاعل}\\
\begin{center}
\renewcommand{\arraystretch}{1.2} 
\begin{tabular}{LLL}
\multicolumn{1}{C}{\text{\RL{تفاعل}}}&\multicolumn{1}{C}{\text{\RL{دائرہ کار}}}&\multicolumn{1}{C}{\text{\RL{سعت}}}\\
\midrule
w=\sqrt{x^2+y^2+z^2}&\text{\RL{پوری فضا}}&[0,\infty)\\
w=\dfrac{1}{x^2+y^2+z^2}&(x,y,z)\ne (0,0,0)&(0,\infty)\\
w=xy\ln z&\text{\RL{نصف فضا z>0}}&(-\infty,\infty)
\end{tabular}
\end{center}
\انتہا{مثال}
%===========
بالکل حقیقی لکیر کے وقفوں  پر معین تفاعل کے دائرہ کار کی طرح، مستوی  کے حصوں پر معین تفاعل کے دائرہ کار کے اندرونی نقطے اور  سرحدی نقطے  ہو سکتے ہیں۔
\begin{figure}
\centering
\begin{subfigure}{0.45\textwidth}
\centering
\begin{tikzpicture}[scale=0.75]
\draw[smooth cycle,fill=llgray]plot coordinates {(0,0)(1,-2)(2,-1.5)(3,-2)(5,-1)(4.5,0)(5,1)(2,2)(1,1.5)};
\draw[dashed,fill=lgray](3,-0.5)node[circ]{}node[pin={[pin edge=solid]135:{$(x_0,y_0)$}}]{} circle (0.5);
\draw(1,-1)node[]{$R$};
\end{tikzpicture}
\caption{اندرونی نقطہ}
\end{subfigure}\hfill
\begin{subfigure}{0.45\textwidth}
\centering
\begin{tikzpicture}[scale=0.75]
\draw[name path=kout,smooth cycle,fill=llgray]plot coordinates {(0,0)(1,-2)(2,-1.5)(3,-2)(5,-1)(4.5,0)(5,1)(2,2)(1,1.5)};
\draw(1,-1)node[]{$R$};
\begin{scope}
\path[clip,smooth cycle]plot coordinates {(0,0)(1,-2)(2,-1.5)(3,-2)(5,-1)(4.5,0)(5,1)(2,2)(1,1.5)};
\fill[lgray](3,-2) circle (0.5);
\end{scope}
\draw[dashed,name path=kin](3,-2)node[circ]{}node[pin={[pin edge=solid,above]70:{$(x_0,y_0)$}}]{} circle (0.5);
\end{tikzpicture}
\caption{سرحدی  نقطہ}
\end{subfigure}
\caption{مستوی خطہ \عددی{R}  کا اندرونی نقطہ اور سرحدی نقطہ۔اندرونی نقطہ لازماً \عددی{R} کا حصہ ہو گا جبکہ ضروری نہیں کہ سرحدی نقطہ \عددی{} کا حصہ ہو۔}
\label{شکل_کثیرالمتغیر_اندرونی_سرحدی_نقاط}
\end{figure}

\ابتدا{تعریفات}
مستوی \عددی{xy} میں خطہ (سلسلہ) \عددی{R}  میں نقطہ \عددی{(x_0,y_0)}  تب \عددی{R} کا \اصطلاح{اندرونی نقطہ}\فرہنگ{نقطہ!اندرونی}\حاشیہب{interior point}\فرہنگ{point!interior} ہو گا جب  یہ اس قرص کا مرکز  ہو جو مکمل طور پر \عددی{R} میں پایا جاتا ہو (شکل \حوالہ{شکل_کثیرالمتغیر_اندرونی_سرحدی_نقاط})۔  نقطہ \عددی{(x_0,y_0)} تب \عددی{R} کا  \اصطلاح{سرحدی نقطہ}\فرہنگ{نقطہ!سرحدی}\حاشیہب{boundary point}\فرہنگ{point!boundary} ہو گا جب ہر اس  قرص  میں، جس کا مرکز \عددی{(x_0,y_0)} ہو ،  \عددی{R} کے بیرونی  اور \عددی{R} کے اندرونی نقطے پائے جاتے ہوں۔(ضروری نہیں کہ سرحدی نقطہ ازخود \عددی{R} میں شامل  ہو۔ )

ایک خطہ کے اندرونی نقطے، بطور ایک سلسلہ، اس خطہ  کی\اصطلاح{ اندرون}\فرہنگ{اندرون}\حاشیہب{interior}\فرہنگ{interior} ہوں گے۔ اس خطہ کے سرحدی نقطے اس کی \اصطلاح{سرحد}\فرہنگ{سرحد}\حاشیہب{boundary}\فرہنگ{boundary}  ہیں۔ایسا خطہ  جو مکمل طور پر اندرونی نقطوں پر مشتمل ہو \اصطلاح{کھلا}\فرہنگ{کھلا}\حاشیہب{open}\فرہنگ{open} خطہ کہلاتا ہے۔ ایسا خطہ جس میں  اس کے تمام سرحدی نقطے شامل ہوں \اصطلاح{بند}\فرہنگ{بند}\حاشیہب{closed}\فرہنگ{closed} خطہ کہلاتا ہے۔ 
\انتہا{تعریفات}
%===================
\begin{figure}
\centering
\begin{subfigure}{0.30\textwidth}
\centering
\begin{tikzpicture}
\draw(0,0)node[below left]{$O$};
\fill[llgray](0,0) circle (1);
\draw[dashed,fill=lgray](0.4,0.4)node[circ]{} circle (0.25);
\draw[-latex](-1.5,0)--(1.5,0)node[right]{$x$};
\draw[-latex](0,-1.25)--(0,1.5)node[left]{$y$};
\end{tikzpicture}
\caption{
\عددی{\{(x,y)|x^2+y^2<1\}}\\
کھلا قرص۔ ہر نقطہ اندرونی نقطہ ہے۔
}
\end{subfigure}\hfill
\begin{subfigure}{0.30\textwidth}
\centering
\begin{tikzpicture}
\draw(0,0)node[below left]{$O$};
\begin{scope}
\draw[clip](0,0) circle (1);
\fill[lgray](45:1) circle (0.25);
\end{scope}
\draw[dashed](45:1)node[circ]{} circle (0.25);
\draw[-latex](-1.5,0)--(1.5,0)node[right]{$x$};
\draw[-latex](0,-1.25)--(0,1.5)node[left]{$y$};
\end{tikzpicture}
\caption{
\عددی{\{(x,y)|x^2+y^2=1\}}\\
اکائی قرص کی  سرحد۔ (اکائی دائرہ۔)
}
\end{subfigure}\hfill
\begin{subfigure}{0.30\textwidth}
\centering
\begin{tikzpicture}
\draw(0,0)node[below left]{$O$};
\draw[fill=llgray](0,0) circle (1);
\draw[-latex](-1.5,0)--(1.5,0)node[right]{$x$};
\draw[-latex](0,-1.25)--(0,1.5)node[left]{$y$};
\end{tikzpicture}
\caption{
\عددی{\{(x,y)|x^2+y^2\le 1\}}\\
بند اکائی قرص۔تمام سرحدی نقطے اس میں شامل ہیں۔
}
\end{subfigure}
\caption{مستوی میں اکائی قرص کے اندرونی نقطے اور سرحدی نقطے۔}
\label{شکل_کثیرالمتغیر_اکائی_قرص_اندرونی_سرحدی}
\end{figure}

حقیقی اعداد کے وقفوں کی طرح، مستوی میں بعض خطے نا کھلا اور نا ہی بند ہوتے ہیں۔ شکل \حوالہ{شکل_کثیرالمتغیر_اکائی_قرص_اندرونی_سرحدی} کے کھلا قرص   میں چند،   نا کہ  تمام،    سرحدی نقطے شامل کرنے سے ایسا خطہ حاصل ہو گا جو نا کھلا ہو گا اور نا ہی بند ہو گا۔اس میں شامل سرحدی نقطے اس کو کھلا وقفہ بننے  سے روکتے  ہیں جبکہ اس میں نا  شامل سرحدی نقطے اس کو  بند  خطہ بننے سے روکتے ہیں۔

\ابتدا{تعریف}
مستوی میں  مقررہ رداس کے قرص  میں پائے جانے والا خطہ  \اصطلاح{محدود}\فرہنگ{محدود}\حاشیہب{bounded}\فرہنگ{bounded} ہو گا۔  ایسا خطہ جو محدود نا ہو \اصطلاح{غیر محدود}\فرہنگ{غیر محدود}\حاشیہب{unbounded}\فرہنگ{unbounded}  ہو گا۔
\انتہا{تعریف}
%================
\begin{figure}
\centering
\begin{tikzpicture}[font=\small,declare function={f(\x)=(\x)^2;}]
 \pgfmathsetseed{42}
\begin{axis}[clip=false,axis on top,small,axis lines=middle,xtick={-1,1},ytick={1},enlargelimits=true,xlabel={$x$},ylabel={$y$},xlabel style={at={(current axis.right of origin)},anchor=north},ylabel style={at={(current axis.above origin)},anchor=south}]
\addplot[thick,name path=kc,domain=-2:2]{f(x)}node[pos=0.85,pin={[align=right,below]-45:{\RL{قطع مکافی}\\ $y-x^2=0$\\   \RL{سرحد ہے}}}]{};
    \draw[name path=kl,decorate,decoration={random steps,segment length=4pt,amplitude=2pt}] (-2,{f(-2)}) -- (2,{f(2)});
\addplot[fill=llgray]fill between[of=kc and kl];
\draw(1,{f(1.9)})node[pin={[align=center]45:{\RL{اندرونی نقاط، جہاں}\\  $y-x^2>0$}}]{};
\draw(-1,1)node[left,align=center]{\RL{باہر،}\\$y-x^2<0$};
\end{axis}
\end{tikzpicture}
\caption{تفاعل \عددی{f(x,y)=\sqrt{y-x^2}} کا دائرہ کار سایہ دار خطہ ہے اور اس کی سرحد قطع مکافی \عددی{y=x^2} ہے۔}
\label{شکل_مثال_کثیرالمتغیر_سایہ_دار_دائرہ_کار}
\end{figure}

\ابتدا{مثال}
\begin{description}
\item{مستوی میں محدود سلسلے:}
خطی قطعات؛ مثلثیں؛ مثلثوں کی اندرون؛  مستطیلیں؛ ا قراص۔
\item{مستوی میں غیر محدود سلسلے:}
خطوط،؛ محددی محور؛  لا متناہی وقفہ پر معین تفاعل کی ترسیم؛   ربعات،  نصف مستوی؛ مستوی از خود۔
\end{description}
\انتہا{مثال}
%================
\ابتدا{مثال}
تفاعل \عددی{f(x,y)=\sqrt{y-x^2}} کا دائرہ کار بند اور غیر محدود ہے (شکل \حوالہ{شکل_مثال_کثیرالمتغیر_سایہ_دار_دائرہ_کار})۔ قطع مکافی \عددی{y=x^2} اس دائرہ کار کی سرحد ہے۔ قطع مکافی سے اوپر نقطے دائرہ کار کی اندرون ہیں۔
\انتہا{مثال}
%=================

فضا میں  اندرون، سرحد ، کھلا، بند، محدود  اور غیر محدود کی تعریفیں عین مستوی میں انہیں کی تعریفوں کی طرح ہیں۔ اضافی بعد کی بنا ہم قرص کی  بجائے گیند لیتے ہیں۔ \اصطلاح{بند گیند}\فرہنگ{بند!گیند}\حاشیہب{closed ball}\فرہنگ{closed!ball} میں کرہ کی اندرونی نقطوں کے ساتھ  گیند بھی شامل ہو گا۔ \اصطلاح{کھلا گیند}\فرہنگ{کھلا!گیند}\حاشیہب{open ball}\فرہنگ{open!ball} میں گیند کی اندرونی نقطے شامل ہوں گے جبکہ گیند از خود اس میں شامل نہیں ہو گا۔ 

\ابتدا{تعریفات}
فضا میں خطہ \عددی{D} میں نقطہ \عددی{(x_0,y_0,z_0)}  اس صورت \عددی{D} کا \اصطلاح{اندرونی نقطہ}\فرہنگ{اندرونی!نقطہ}\حاشیہب{interior point}\فرہنگ{interior!point} ہو گا جب  یہ نقطہ  ایسے گیند کا مرکز ہو جو مکمل طور پر \عددی{D} میں پایا جاتا ہو۔اگر ہر  گیند، جس کا مرکز   \عددی{(x_0,y_0,z_0) }  ہو، میں شامل نقطوں میں کچھ  نقطے     \عددی{D} کے اندرونی    اور کچھ  اس کے بیرونی نقطے   ہوں تب یہ نقطہ \عددی{D} کا \اصطلاح{سرحدی نقطہ}\فرہنگ{سرحدی!نقطہ}\حاشیہب{boundary point}\فرہنگ{boundary!point} ہو گا۔خطہ \عددی{D} کے اندرونی نقطوں کا  سلسلہ  \عددی{D} کا  \اصطلاح{اندرون}\فرہنگ{اندرون}\حاشیہب{interior}\فرہنگ{interior} ہو گا۔ خطہ \عددی{D} کے سرحدی نقطوں کا سلسلہ \عددی{D} کا \اصطلاح{سرحد}\فرہنگ{سرحد}\حاشیہب{boundary}\فرہنگ{boundary} ہو گا۔

ایک ایسا خطہ جو صرف اندرونی نقطوں پر مشتمل ہو  \اصطلاح{کھلا}\فرہنگ{کھلا}\حاشیہب{open}\فرہنگ{open} خطہ کہلائے  گا۔ ایک  خطہ جس میں خطے کا پورا سرحد شامل ہو\اصطلاح{ بند}\فرہنگ{بند}\حاشیہب{closed}\فرہنگ{closed} خطہ کہلائے گا۔
\انتہا{تعریفات}
%===============

\ابتدا{مثال}
\begin{description}
\item{فضا میں کھلا سلسلے}
کھلا گیند؛کھلا نصف فضا \عددی{z>0}؛ ربع اول (بغیر تحدیدی  سطحیں)  ؛ فضا  ا زخود
\item{فضا میں بند سلسلے}  
خطوط؛ مستوی؛ بند گیند؛ بند نصف فضا \عددی{z\ge 0}؛ ربع اول بمع اس کے تحدیدی  سطحیں؛ فضا از خود
\item{نا کھلا اور نا بند}
بند گیند جس میں تحدیدی کرہ کا کچھ حصہ شامل نہ ہو؛ ٹھوس مربع جس میں ایک  تحدیدی سطح یا کنارہ   یا کونا شامل نہ ہو 
\end{description}
\انتہا{مثال}
%=========

\جزوحصہء{دو متغیرات کے تفاعل کی ترسیمات اور   ہم قد منحنیات}
تفاعل \عددی{f(x,y)}   کی  تصویر کشی دو طریقوں سے کی جا سکتی ہے۔اول،   ہم اس دائرہ کار میں   \عددی{f} کی  منحنیات ترسیم کر سکتے ہیں جس پر \عددی{f}  کی قیمت مستقل ہو۔ دوم،    ہم فضا میں سطح \عددی{z=f(x,y)} ترسیم کر سکتے ہیں۔

\ابتدا{تعریفات}
اس مستوی  میں  نقطوں کا سلسلہ جہاں \عددی{f(x,y)}  کی قیمت ایک مستقل  \عددی{f(x,y)=c}  ہو،   \عددی{f} کی  \اصطلاح{ہم قد منحنی}\فرہنگ{ہم قد!منحنی}\حاشیہب{level curve}\فرہنگ{curve!level} کہلاتا ہے۔فضا میں \عددی{f} کے دائرہ کار میں \عددی{(x,y)} کے لئے  تمام نقطوں \عددی{(x,y,f(x,y))} کا سلسلہ  \عددی{f} کی \اصطلاح{ترسیم}\فرہنگ{ترسیم}\حاشیہب{graph}\فرہنگ{graph} کہلاتا ہے۔تفاعل \عددی{f} کی ترسیم کو \اصطلاح{سطح}\فرہنگ{سطح}\حاشیہب{surface}\فرہنگ{surface} \عددی{z=f(x,y)} بھی کہتے ہیں۔
\انتہا{تعریفات}
%===============

دھیان رہے کہ   ہم قد منحنیات اس مستوی میں پائی جاتی ہیں جس پر تفاعل کا  دائرہ کار پایا جاتا ہو۔
 %======================


\حصہء{سوالات}
\begin{figure}
\centering
\begin{tikzpicture}[declare function={fx(\r,\t)=\r*cos(\t);fy(\r,\t)=\r*sin(\t);fz(\r,\t)=100-(\r)^2;}]
\pgfmathsetmacro{\ra}{10}
\pgfmathsetmacro{\rb}{7}
\pgfmathsetmacro{\rc}{5}
\begin{axis}[clip=false,view/h=120,small,axis lines=center,colormap={}{gray(0cm)=(0.6);gray(1cm)=(0.8);},enlargelimits=true,xlabel={$x$},ylabel={$y$},zlabel={$z$},xmax=12,ymax=12,zmax=120,xlabel style={anchor=east},ylabel style={anchor=west},zlabel style={anchor=west},xtick={10},ytick={10},zticklabel style={yshift=1ex}]
\addplot3[surf,shader=interp,z buffer=sort,domain=0:10,domain y=0:360]({fx(x,y)},{fy(x,y)},{fz(x,y)});
\addplot3[domain y=0:360] ({fx(\ra,y)},{fy(\ra,y)},0)node[pos=0.125,pin=-45:{$f(x,y)=0$}]{};
\addplot3[domain y=0:360] ({fx(\rb,y)},{fy(\rb,y)},0)node[pos=0.3,pin={[align=center,right,pin distance=1.5cm]55:{$f(x,y)=51$\\   \RL{ہم قد منحنی تفاعل کے}\\   \RL{دائرہ کار میں پائی جاتی ہے۔}}}]{};
\addplot3[domain y=0:360] ({fx(\rc,y)},{fy(\rc,y)},0);
\addplot3[]({fx(7,270)},{fy(7,270)},{fz(7,270)})node[pin={[align=center]135:{\RL{سطح}\\  $z=100-x^2-y^2$\\  \RL{\عددی{f} کی ترسیم ہے۔}}}]{};
\end{axis}
\end{tikzpicture}
\caption{تفاعل کی ترسیم اور منتخب ہم قد منحنیات۔}
\label{شکل_مثال_کثیرالمتغیر_تفاعل_کی_ہم_قد_منحنیات}
\end{figure}
\ابتدا{مثال}
تفاعل \عددی{f(x,y)=100-x^2-y^2} ترسیم کریں اور مستوی میں  \عددی{f} کے دائرہ کار میں  ہم قد  منحنیات \عددی{f(x,y)=0}، \عددی{f(x,y)=51} اور \عددی{f(x,y)=75}   ترسیم کریں۔

حل:\quad
تفاعل \عددی{f} کا دائرہ کار پورا  \عددی{xy} مستوی ہے جبکہ اس کی سعت   \عددی{100}  جتنا  یا اس سے کم   تمام حقیقی اعداد کا سلسلہ ہے۔ قطع مکافی \عددی{z=100-x^2-y^2} اس کی ترسیم ہے جس کا کچھ حصہ شکل \حوالہ{شکل_مثال_کثیرالمتغیر_تفاعل_کی_ہم_قد_منحنیات}  میں دکھایا گیا ہے۔

مستوی \عددی{xy} میں ان نقطوں کا سلسلہ جن پر  درج ذیل ہو،    ہم قد منحنی \عددی{f(x,y)=0}  ہو گی جو ایک دائرہ ہے جس کا رداس \عددی{10} اور جس کا مرکز مبدا پر ہے۔
\begin{align*}
x^2+y^2=100\quad \text{یعنی}\quad f(x,y)=100-x^2-y^2=0
\end{align*}
اسی طرح  ہم قد  منحنیات  \عددی{f(x,y)=51} اور \عددی{f(x,y)=75} درج ذیل دائرے ہوں گے جو \عددی{xy} مستوی میں پائے جاتے ہیں اور جن کے  مراکز عین مبدا پر پائے  جاتے ہیں۔
\begin{align*}
x^2+y^2&=49\quad \text{یعنی}\quad f(x,y)=100-x^2-y^2=51\\
x^2+y^2&=25\quad\text{یعنی}\quad f(x,y)=100-x^2-y^2=75
\end{align*}
ہم قد  منحنی \عددی{f(x,y)=100} صرف مبدا پر مشتمل ہے۔(اس کے باوجود یہ ایک ہم قد منحنی ہے۔)
\انتہا{مثال}
%================
\begin{figure}
\centering
\begin{tikzpicture}[declare function={fx(\r,\t)=\r*cos(\t);fy(\r,\t)=\r*sin(\t);fz(\r,\t)=100-(\r)^2;}]
\pgfmathsetmacro{\rc}{5}
\begin{axis}[clip=false,view/h=120,small,axis lines=center,colormap={}{gray(0cm)=(0.6);gray(1cm)=(0.8);},enlargelimits=true,xlabel={$x$},ylabel={$y$},zlabel={$z$},xmax=12,ymax=12,zmax=120,xlabel style={anchor=east},ylabel style={anchor=west},zlabel style={anchor=west},xtick={10},ytick={10},ztick={75,100},zticklabel style={yshift=1ex}]
\addplot3[surf,shader=interp,z buffer=sort,domain=5:10,domain y=0:360]({fx(x,y)},{fy(x,y)},{fz(x,y)});
\addplot3[dashed,domain y=0:360] ({fx(\rc,y)},{fy(\rc,y)},0)node[pos=0.125,pin={[align=center,below,pin edge={black,solid}]-45:{\RL{ہم قد منحنی}\\   $f(x,y)=100-x^2-y^2$\\ \RL{مستوی \عددی{xy} میں دائرہ}\\  \RL{$x^2+y^2=25$ ہو گی۔}}}]{};
\addplot3[]({fx(9,90)},{fy(9,90)},{fz(9,90)})node[pin=45:{$z=100-x^2-y^2$}]{};
\addplot3[fill=lgray] coordinates {(-10,-10,75) (10,-10,75) (10,10,75) (-10,10,75)  (-10,-10,75)}node[pos=0.8,pin=20:{\RL{مستوی $z=75$}}]{};
\addplot3[fill=gray,dashed,domain y=0:360]({fx(\rc,y)},{fy(\rc,y)},{fz(\rc,y)})node[pos=0.75,pin={[align=center,pin edge={black,solid}]135:{\RL{خط ارتفاع}\\  $f(x,y)=100-x^2-y^2=75$\\  \RL{مستوی  $z=75$ میں دائرہ $x^2+y^2=25$ ہو گا۔}}}]{};
\addplot3[dashed]coordinates{(0,0,0)(10,0,0)};
\addplot3[dashed]coordinates{(0,0,0)(0,10,0)};
\addplot3[dashed]coordinates{(0,0,0)(0,0,100)};
\addplot3[]coordinates{(0,0,75)}node[circ]{}node[right,font=\scriptsize]{$75$};
\end{axis}
\end{tikzpicture}
\caption{تفاعل \عددی{f(x,y)=100-x^2-y^2} کی ترسیم اور مستوی \عددی{z=75} کے ساتھ اس کا تقاطع۔}
\label{شکل_کثیرالمتغیر_خط_ارتفاع_اور_ہم_قد_منحنی}
\end{figure}

\جزوحصہء{خطوط  ارتفاع}
فضا میں وہ منحنی جس میں  مستوی \عددی{z=c}  سطح \عددی{z=f(x,y)} کو مس کرتا   ہو،  ان نقطوں پر مشتمل ہو گی  جو تفاعل \عددی{f(x,y)=c} کو ظاہر کرتی ہے۔ اس کو\اصطلاح{  خط ارتفاع}\فرہنگ{خط!ارتفاع}\حاشیہب{contour line}\فرہنگ{contour!line} \عددی{f(x,y)=c} کہتے ہیں تا کہ اس کے   بیچ  اور   \عددی{f} کے دائرہ کار  میں  ہم قد منحنی \عددی{f(x,y)=c}کے بیچ  تمیز کرنا ممکن ہو۔شکل \حوالہ{شکل_کثیرالمتغیر_خط_ارتفاع_اور_ہم_قد_منحنی}  میں تفاعل \عددی{f(x,y)=100-x^2-y^2} کی سطح \عددی{z=100-x^2-y^2} پر  خط ارتفاع \عددی{f(x,y)=75}  دکھایا گیا ہے۔ یہ خط ارتفاع ٹھیک دائرہ \عددی{x^2+y^2=25} ، جو تفاعل کے دائرہ کار میں  ہم قد  منحنی \عددی{f(x,y)=75} ہے، کے اوپر کچھ بلندی پر پایا جاتا ہے۔

 بعض ریاضی دان  خط ارتفاع اور  ہم قد  منحنی میں تمیز نہیں کرتے ہیں اور دونوں کو کسی ایک نام سے پکارتے ہیں۔ایسی صورت میں   متن سے آپ جان سکتے ہیں کہ کس کی بات کی گئی ہے۔ عموماً نقشات  پر  (سطح سمندر سے )   مستقل بلندی کو ظاہر کرنے والی منحنیات کو  خط ارتفاع  پکارا جاتا ہے نا کہ  ہم قد  منحنیات۔

\جزوحصہء{سہ  متغیری   تفاعل کی ہم قد منحنیات}
مستوی میں جن نقطوں پر   دو غیر تابع  متغیرات کے  تفاعل  کی قیمت ایک مستقل \عددی{f(x,y)=c} ہو اس تفاعل کے دائرہ کار میں ایک منحنی تشکیل دیتے ہیں۔فضا میں  جن نقطوں پر تین غیر تابع متغیرات کے تفاعل کی قیمت ایک مستقل  \عددی{f(x,y,z)=c} ہو  اس تفاعل کے دائرہ کار  ایک سطح تشکیل دیتے ہیں۔

\ابتدا{تعریف}
فضا میں ان نقطوں  \عددی{(x,y,z)}  کا سلسلہ جن پر تین غیر تابع متغیرات کے تفاعل کی قیمت ایک مستقل \عددی{f(x,y,z)=c} ہو، \عددی{f} کی \اصطلاح{ ہم قد سطح}\فرہنگ{ہم قد!سطح}\حاشیہب{level surface}\فرہنگ{surface!level} کہلاتا ہے۔
\انتہا{تعریف}
%=====================

\ابتدا{مثال}
درج ذیل تفاعل کے ہم قد  سطحوں پر تبصرہ کریں۔
\begin{align*}
f(x,y,z)=\sqrt{x^2+y^2+z^2}
\end{align*}
حل:\quad
تفاعل \عددی{f} کی قیمت،   مبدا سے نقطہ \عددی{(x,y,z)} تک فاصلہ   ہو گا۔ ہر  ہم قد  سطح \عددی{\sqrt{x^2+y^2+z^2}=c,\, c>0} رداس \عددی{c} کا کرہ ہو گا جس کا مرکز مبدا پر ہو گا۔   ہم قد  سطح \عددی{\sqrt{x^2+y^2+z^2}=0}  صرف مبدا پر مشتمل ہے۔

ہم یہاں تفاعل کو ترسیم نہیں کر رہے ہیں۔ایک تفاعل جو نقاط  \عددی{(x,y,z,\sqrt{x^2+y^2+z^2})} پر مشتمل ہو،  چار متغیری فضا  میں پایا جائے گا۔اس کی بجائے  ہم تفاعل کے دائرہ کار میں  ہم قد  سطحوں کو دیکھ رہے ہیں۔

 اس تفاعل کی  ہم قد  سطحیں ہمیں   تفاعل کے دائرہ کار  میں  چلتے ہوئے تفاعل کی قیمت کی تبدیلی  دکھاتی ہیں۔اگر ہم رداس \عددی{c} کے کرہ ،جس کا مرکز مبدا پر ہو،   پر چہل قدمی  کریں  تب تفاعل کی قیمت بدستور \عددی{c} رہے گی۔ ایک کرہ سے دوسری کرہ  منتقل ہونے پر تفاعل کی قیمت تبدیل ہو گی۔مبدا سے دوری    تفاعل کی قیمت بڑھاتی ہے جبکہ مبدا کے قریب ہونے سے اس کی قیمت کم ہوتی ہے۔تفاعل کی قیمت میں تبدیلی کا دارومدار ہمارے چلنے کے رخ پر ہو گا۔تفاعل کی قیمت میں تبدیلی کا رخ پر انحصار ایک اہم حقیقت ہے جس پر بعد کے حصہ میں غور کیا جائے گا۔ 
\انتہا{مثال}
%===============

\جزوحصہء{کمپیوٹر ترسیم کشی}
کمپیوٹر کی مدد سے  دو متغیرات کا  تفاعل با آسانی   ترسیم کیا جا سکتا ہے۔عموماً  ترسیم ہمیں   کلیہ سے زیادہ معلومات   جلدی فراہم کرتی ہے۔

\begin{figure}
\centering
\begin{tikzpicture}[declare function={f(\x,\y)=(cos(deg(0.017*\y-0.656*\x)))*e^(-0.656*\x);}]
\begin{axis}[view/h=135,small,axis lines=center,colormap={}{gray(0cm)=(0.6);gray(1cm)=(0.8);},enlargelimits=true,xlabel={\RL{گہرائی (میٹر)}},ylabel={دن},zlabel={\RL{درجہ حرارت}},xtick={9},ytick={\empty},ztick={\empty},ylabel style={anchor=west},zlabel style={anchor=east}]
\addplot3[surf,domain=0:9,domain y=0:1440]{f(x,y)};
\end{axis}
\end{tikzpicture}
\caption{سطح زمین کی نسبت سے گہرائی میں درجہ حرارت کی تبدیلی بالمقابل وقت۔}
\label{شکل_مثال_کثیر_المتغیر_گہرائی_درجہ_حرارت}
\end{figure}

\ابتدا{مثال}
تفاعل \عددی{w=\cos(1.7\times 10^{-2}t-0.656x)e^{-0.656x}} کی  ترسیم کو شکل \حوالہ{شکل_مثال_کثیر_المتغیر_گہرائی_درجہ_حرارت}  میں دکھایا گیا ہے ، جہاں وقت  کو  \عددی{t}   اور  فاصلہ کو  \عددی{x}    ظاہر کرتے ہیں۔یہ ترسیم  سطح زمین سے نیچے درجہ حرارت کی تبدیلی بالمقابل وقت دکھاتی ہے۔گہرائی میں درجہ حرارت کی تبدیلی \عددی{w}  کو سطحی تبدیلی کی نسبت سے دکھایا گیا ہے۔چار  میٹر کی گہرائی پر سطح تبدیلی کے \عددی{6.3} فی صد جتنی تبدیلی پائی جاتی ہے۔نو  میٹر گہرائی پر  پورے سال  درجہ حرارت میں تبدیلی  قابل نظر انداز ہے۔

آپ دیکھ سکتے ہیں کہ \عددی{4} میٹر گہرائی پر درجہ حرارت سطحی درجہ حرارت سے تقریباً آدھا سال پیچھے    ہے۔ یوں اس گہرائی پر  گرمی کی موسم میں  کم سے کم اور  سردی کی موسم میں  زیادہ سے زیادہ درجہ حرارت ہو گا۔(میں مشورہ  دوں گا کہ زیر زمین ایک کمرہ ضرور بنائیں۔ )
\انتہا{مثال}
%=============

\حصہء{سوالات}
\موٹا{دائرہ کار، سعت اور ہم قد منحنیات}\\
سوال \حوالہ{سوال_کثیرالمتغیر_دائرہ_کار_الف} تا سوال \حوالہ{سوال_کثیرالمتغیر_دائرہ_کار_ب} میں (ا) تفاعل کا دائرہ کار تلاش کریں، (ب) تفاعل کی سعت تلاش کریں، (ج) تفاعل کی ہم قد  منحنی پر تبصرہ کریں، (د)  تفاعل کے دائرہ کار کی سرحد معلوم کریں، (ہ) کیا دائرہ کار کھلا خطہ، بند خطہ یا دونوں میں سے کوئی نہیں ہے، (و) کیا دائرہ کار محدود یا غیر محدود ہے؟

\ابتدا{سوال}\شناخت{سوال_کثیرالمتغیر_دائرہ_کار_الف}
$f(x,y)=y-x$
\انتہا{سوال}
%=================
\ابتدا{سوال}
$f(x,y)=\sqrt{y-x}$
\انتہا{سوال}
%=================
\ابتدا{سوال}
$f(x,y)=4x^2+9y^2$
\انتہا{سوال}
%=================
\ابتدا{سوال}
$f(x,y)=x^2-y^2$
\انتہا{سوال}
%=================
\ابتدا{سوال}
$f(x,y)=xy$
\انتہا{سوال}
%=================
\ابتدا{سوال}
$f(x,y)=\tfrac{y}{x^2}$
\انتہا{سوال}
%=================
\ابتدا{سوال}
$f(x,y)=\tfrac{1}{\sqrt{16-x^2-y^2}}$
\انتہا{سوال}
%=================
\ابتدا{سوال}
$f(x,y)=\sqrt{9-x^2-y^2}$
\انتہا{سوال}
%=================
\ابتدا{سوال}
$f(x,y)=\ln(x^2+y^2)$
\انتہا{سوال}
%=================
\ابتدا{سوال}
$f(x,y)=e^{-(x^2+y^2)}$
\انتہا{سوال}
%=================
\ابتدا{سوال}
$f(x,y)=\sin^{-1}(y-x)$
\انتہا{سوال}
%=================
\ابتدا{سوال}\شناخت{سوال_کثیرالمتغیر_دائرہ_کار_ب}
$f(x,y)=\tan^{-1}(\tfrac{y}{x})$
\انتہا{سوال}
%=================
\موٹا{دو متغیرات کے تفاعل کی پہچان}\\
سوال \حوالہ{سوال_کثیرالمتغیر_ہم_قد_منحنیات_الف} تا سوال \حوالہ{سوال_کثیرالمتغیر_ہم_قد_منحنیات_ب} میں تفاعل کی قیمتوں کو دو طرح دکھائیں۔ (ا)  سطح \عددی{z=f(x,y)} کو ترسیم کرتے ہوئے اور (ب) تفاعل کے دائرہ کار میں منتخب ہم قد منحنیات ترسیم کرتے ہوئے۔ ہر ایک  ہم قد  منحنی  کی نشاندہی  تفاعل کی قیمت سے کریں۔  

\ابتدا{سوال}\شناخت{سوال_کثیرالمتغیر_ہم_قد_منحنیات_الف}\\
$f(x,y)=y^2$
\انتہا{سوال}
%===================
\ابتدا{سوال}
$f(x,y)=4-y^2$
\انتہا{سوال}
%===================
\ابتدا{سوال}
$f(x,y)=x^2+y^2$
\انتہا{سوال}
%===================
\ابتدا{سوال}
$f(x,y)=\sqrt{x^2+y^2}$
\انتہا{سوال}
%===================
\ابتدا{سوال}
$f(x,y)=-(x^2+y^2)$
\انتہا{سوال}
%===================
\ابتدا{سوال}
$f(x,y)=4-x^2-y^2$
\انتہا{سوال}
%===================
\ابتدا{سوال}
$f(x,y)=4x^2+y^2$
\انتہا{سوال}
%===================
\ابتدا{سوال}
$f(x,y)=4x^2+y^2+1$
\انتہا{سوال}
%===================
\ابتدا{سوال}
$f(x,y)=1-\abs{y}$
\انتہا{سوال}
%===================
\ابتدا{سوال}\شناخت{سوال_کثیرالمتغیر_ہم_قد_منحنیات_ب}
$f(x,y)=1-\abs{x}-\abs{y}$
\انتہا{سوال}
%===================

\موٹا{ہم قد سطحیں}\\
سوال \حوالہ{سوال_کثیر_المتغیر_ہم_قد_سطح_الف} تا سوال \حوالہ{سوال_کثیر_المتغیر_ہم_قد_سطح_ب} میں تفاعل کا  ایک علامتی  ہم قد سطح  کا خاکہ بنائیں۔

\ابتدا{سوال}\شناخت{سوال_کثیر_المتغیر_ہم_قد_سطح_الف}
$f(x,y,z)=x^2+y^2+z^2$
\انتہا{سوال}
%==================
\ابتدا{سوال}
$f(x,y,z)=\ln(x^2+y^2+z^2)$
\انتہا{سوال}
%================
\ابتدا{سوال}
$f(x,y,z)=x+z$
\انتہا{سوال}
%================
\ابتدا{سوال}
$f(x,y,z)=z$
\انتہا{سوال}
%================
\ابتدا{سوال}
$f(x,y,z)=x^2+y^2$
\انتہا{سوال}
%================
\ابتدا{سوال}
$f(x,y,z)=y^2+z^2$
\انتہا{سوال}
%================
\ابتدا{سوال}\شناخت{سوال_کثیر_المتغیر_ہم_قد_سطح_ب}
$f(x,y,z)=\tfrac{x^2}{25}+\tfrac{y^2}{16}+\tfrac{z^2}{9}$
\انتہا{سوال}
%================
\موٹا{ہم قد منحنی کی  تلاش}\\
سوال \حوالہ{سوال_کثیر_المتغیر_ہم_قد_منحنی_تلاش_الف} تا سوال \حوالہ{سوال_کثیر_المتغیر_ہم_قد_منحنی_تلاش_ب} میں  تفاعل \عددی{f(x,y)} کی اس  ہم قد منحنی کی مساوات تلاش کریں جو  دیے گئے نقطہ سے گزرتی  ہو۔

\ابتدا{سوال}\شناخت{سوال_کثیر_المتغیر_ہم_قد_منحنی_تلاش_الف}
$f(x,y)=16-x^2-y^2,\quad (2\sqrt{2},\sqrt{2})$
\انتہا{سوال}
%================
\ابتدا{سوال}
$f(x,y)=\sqrt{x^2-1},\quad (1,0)$
\انتہا{سوال}
%================
\ابتدا{سوال}
$f(x,y)=\int_x^y\frac{\dif t}{1+t^2},\quad (-\sqrt{2},\sqrt{2})$
\انتہا{سوال}
%================
\ابتدا{سوال}\شناخت{سوال_کثیر_المتغیر_ہم_قد_منحنی_تلاش_ب}
$f(x,y)=\sum_{n=0}^{\infty}\big(\frac{x}{y}\big)^n,\quad (1,2)$
\انتہا{سوال}
%================
\موٹا{ہم قد سطح کی تلاش}\\
سوال \حوالہ{سوال_کثیر_المتغیر_ہم_قد_سطح_تلاش_الف} تا سوال \حوالہ{سوال_کثیر_المتغیر_ہم_قد_سطح_تلاش_ب} میں دیے گئے نقطہ سے گزرتی ہم قد سطح کی مساوات تلاش کریں۔

\ابتدا{سوال}\شناخت{سوال_کثیر_المتغیر_ہم_قد_سطح_تلاش_الف}
$f(x,y,z)=\sqrt{x-y}-\ln z,\quad (3,-1,1)$
\انتہا{سوال}
%=================
\ابتدا{سوال}
$f(x,y,z)=\ln(x^2+y+z^2),\quad (-1,2,1)$
\انتہا{سوال}
%==================
\ابتدا{سوال}
$g(x,y,z)=\sum_{n=0}^{\infty}\frac{(x+y)^n}{n!z^n},\quad (\ln 2,\ln 4,3)$
\انتہا{سوال}
%==================
\ابتدا{سوال}\شناخت{سوال_کثیر_المتغیر_ہم_قد_سطح_تلاش_ب}
$g(x,y,z)=\int_x^y\frac{\dif \theta}{\sqrt{1-\theta^2}}+\int_{\sqrt{2}}^{z}\frac{\dif t}{t\sqrt{t^2-1}},\quad (0,\tfrac{1}{2},2)$
\انتہا{سوال}
%==================

\موٹا{نظریہ اور مثالیں}\\
\ابتدا{سوال}\ترچھا{فضا میں ایک لکیر پر تفاعل کی زیادہ سے زیادہ قیمت۔}\\
کیا لکیر \عددی{x=20-t,\,y=t,\,z=20} پر تفاعل \عددی{f(x,y,z)=xyz} کی  زیادہ سے زیادہ قیمت پائی جاتی ہے؟ اگر  ہو، تب اس کی قیمت کتنی ہو گی؟ اپنے جواب کی وجہ پیش کریں۔ (اشارہ:اس لکیر پر \عددی{w=(f,y,z)} متغیر \عددی{t} کا قابل تفرق تفاعل ہے۔)
\انتہا{سوال}
%==============
\ابتدا{سوال}\ترچھا{فضا میں ایک لکیر پر تفاعل کی کم سے کم  قیمت۔}\\
کیا لکیر \عددی{x=t-1,\,y=t-2,\,z=t+7} پر تفاعل \عددی{f(x,y,z)=xy-z} کی کم سے کم  قیمت پائی جاتی ہے؟ اگر  ہو، تب اس کی قیمت کتنی ہو گی؟ اپنے جواب کی وجہ پیش کریں۔ (اشارہ:اس لکیر پر \عددی{w=(f,y,z)} متغیر \عددی{t} کا قابل تفرق تفاعل ہے۔)
\انتہا{سوال}
%=================
\ابتدا{سوال}\ترچھا{جہاز کا صوتی دھماکا}\\
ایک جہاز کے نیچے زمین پر اس خطہ کی چوڑائی \عددی{w}  جہاں  جہاز کا   صوتی دھماکا   انسان  برائے راست (جو   فضا میں ہوا کی مختلف  سطحوں سے منعکس نہ ہو)  سن سکتا ہو  ، درج ذیل کا  تفاعل ہو گا۔
\begin{itemize}
\item
\عددی{T:} زمین پر ہوا کی  درجہ حرارت (کیلون) 
\item
\عددی{h:} جہاز کی بلندی (کلو میٹر)
\item
\عددی{d:}  درجہ حرارت کی انتصابی  شرح تبدیلی (کیلون فی کلومیٹر)
\end{itemize}
اس چوڑائی کا  کلیہ درج ذیل ہے۔
\begin{align*}
w=4\sqrt{\tfrac{Th}{d}}
\end{align*}
یہ جہاز \عددی{\SI{16.8}{\kilo\meter}} کی بلندی پر پرواز کرتا ہوا  بحیرہ عرب  سے کراچی  شہر    پہنچ رہا ہے۔ اگر سطحی درجہ حرارت \عددی{\SI{290}{\kelvin}} اور انتصابی شرح حرارت \عددی{\SI{5}{\kelvin\per\kilo\meter}} ہو تب   جہاز ساحل سے کتنا دور ہو گا جب اس کا صوتی دھماکا سنائی دے۔
\انتہا{سوال}
%================
\ابتدا{سوال}
جیسا کہ آپ جانتے ہیں، واحد حقیقی متغیر کے حقیقی قیمت تفاعل کی ترسیم دو محددی فضا کا سلسلہ ہو تا ہے۔ دو غیر تابع حقیقی متغیرات کے حقیقی قیمت تفاعل  کی ترسیم تین محددی فضا کا سلسلہ ہوتا ہے۔    تین غیر تابع حقیقی متغیرات کے حقیقی قیمت تفاعل  کی ترسیم چار  محددی فضا کا سلسلہ ہوتا ہے۔   آپ چار غیر تابع متغیرات کے تفاعل \عددی{f(x_1,x_2,x_3,x_4)}  کی ترسیم کے بارے میں کیا کہیں گے؟ ۔   آپ \عددی{n}  غیر تابع متغیرات کے تفاعل \عددی{f(x_1,x_2,x_3,\cdots,x_n)}  کی ترسیم کے بارے میں کیا کہیں گے؟
\انتہا{سوال}
%==========
\موٹا{کمپیوٹر کا استعمال۔صریح سطح}\\
کمپیوٹر استعمال کرتے ہوئے سوال \حوالہ{سوال_کثیرلمتغیر_صریح_سطح_الف} تا سوال \حوالہ{سوال_کثیرلمتغیر_صریح_سطح_ب} میں درج ذیل اقدام کریں۔
\begin{enumerate}[a.]
\item
دیے گئے مستطیل پر سطح ترسیم کریں۔
\item
اس مستطیل میں کئی ہم قد منحنیات ترسیم کریں۔
\item
دیے گئے نقطہ سے گزرتی  ہوئی   \عددی{f} کی   ہم قد منحنی ترسیم کریں۔ 
\end{enumerate}

\ابتدا{سوال}\شناخت{سوال_کثیرلمتغیر_صریح_سطح_الف}
$f(x,y)=x\sin\tfrac{y}{2}+y\sin 2x,\quad 0\le x\le 5\pi,\quad 0\le y\le 5\pi$
\انتہا{سوال}
%=================
\ابتدا{سوال}
$f(x,y)=(\sin x)(\cos x)e^{\sqrt{x^2+y^2}/8},\quad 0\le x\le 5\pi,\quad 0\le y\le 5\pi$
\انتہا{سوال}
%==================
\ابتدا{سوال}
$f(x,y)=\sin(x+2\cos y),\quad -2\pi\le x\le 2\pi,\quad -2\pi\le y\le 2\pi$
\انتہا{سوال}
%==================
\ابتدا{سوال}\شناخت{سوال_کثیرلمتغیر_صریح_سطح_ب}
$f(x,y)=e^{x^{0.1}-y}\sin (x^2+y^2),\quad 0\le x\le 2\pi,\quad -2\pi\le y\le \pi$
\انتہا{سوال}
%==================
\موٹا{کمپیوٹر کا استعمال۔خفی  سطح}\\
سوال \حوالہ{سوال_کثیرالمتغیر_ہم_قد_سطحیں_ترسیم_الف} تا سوال \حوالہ{سوال_کثیرالمتغیر_ہم_قد_سطحیں_ترسیم_ب} میں کمپیوٹر استعمال کرتے ہوئے ہم قد سطحیں  ترسیم کریں۔

\ابتدا{سوال}\شناخت{سوال_کثیرالمتغیر_ہم_قد_سطحیں_ترسیم_الف}
$4\ln(x^2+y^2+z^2)=1$
\انتہا{سوال}
%================
\ابتدا{سوال}
$x^2+z^2=1$
\انتہا{سوال}
%=====================
\ابتدا{سوال}
$x+y^2-3z^2=1$
\انتہا{سوال}
%=====================
\ابتدا{سوال}\شناخت{سوال_کثیرالمتغیر_ہم_قد_سطحیں_ترسیم_ب}
$\sin\big(\frac{x}{2}\big)-(\cos y)\sqrt{x^2+z^2}=2$
\انتہا{سوال}
%=====================
\موٹا{کمپیوٹر کا استعمال۔مقدار معلوم   سطح}\\
جیسا آپ کسی مقدار معلوم وقفہ \عددی{I} پر  مستوی میں منحنیات کو مقدار معلوم مساوات \عددی{x=f(t),\,y=g(t)} کی    روپ میں لکھتے ہیں، آپ بعض اوقات کسی مقدار معلوم مستطیل \عددی{a\le u\le b,\, c\le v\le d}وقفہ  پر  فضا میں سطحوں کو    مقدار معلوم تین مساوات \عددی{x=f(u,v),\,y=g(u,v),\,z=h(u,v)}  کی روپ میں لکھ سکتے ہیں۔کمپیوٹر اس قسم کی مقدار معلوم مساواتوں سے سطح  ترسیم کر سکتا ہے۔سوال \حوالہ{سوال_کثیرالمتغیر_مقدار_معلوم_الف} تا سوال \حوالہ{سوال_کثیرالمتغیر_مقدار_معلوم_ب} میں کمپیوٹر کی مدد سے سطحیں ترسیم کریں۔ساتھ ہی \عددی{xy} مستوی میں چند ہم قد منحنیات ترسیم کریں۔

\ابتدا{سوال}\شناخت{سوال_کثیرالمتغیر_مقدار_معلوم_الف}
$x=u\cos v,\quad y=u\sin v,\quad z=u,\quad 0\le u\le 2,\quad 0\le v\le 2\pi$
\انتہا{سوال}
%===================
\ابتدا{سوال}
$x=u\cos v,\quad y=u\sin v,\quad z=v,\quad 0\le u\le 2,\quad 0\le v\le 2\pi$
\انتہا{سوال}
%===================
\ابتدا{سوال}
$x=(2+\cos u)\cos v,\,y=(2+\cos u)\sin v,\, z=\sin u,$\\
$ 0\le u\le 2\pi,\, 0\le v\le 2\pi$
\انتہا{سوال}
%===================
\ابتدا{سوال}\شناخت{سوال_کثیرالمتغیر_مقدار_معلوم_ب}
$x=2\cos u\cos v,\quad y=2\cos u\sin v,\quad z=2\sin u$\\
$0\le u\le 2\pi,\quad 0\le v\le \pi$
\انتہا{سوال}
%======================
