\باب{کثیر متغیر تفاعل اور جزوی تفرقات}
\موٹا{جائزہ}\\
سائنس میں دو یا دو سے زائد غیر تابع  متغیرات کے تفاعل    ایک متغیر کے تفاعل سے زیادہ کثرت سے پائے جاتے ہیں اور ان کی علم   احصاء بھی زیادہ  عمدہ   ہوتی  ہے۔زیادہ متغیرات ایک دوسرے پر زیادہ طریقوں سے اثر انداز ہو سکتے ہیں جس کی بنا ان کے تفرقات    مختلف اور زیادہ دلچسپ  صورتیں اختیار  کر سکتے ہیں۔ ان کے تکملات  زیادہ اقسام کے عملی  مسائل میں کام آتے ہیں۔ احتمال،  سیالی حرکیات، اور برقیات، وغیرہ،    پر غور کے دوران  ایک سے زائد متغیرات  کے تفاعل قدرتی طور پر رونما ہوتے ہیں۔ان تفاعل کی  ریاضیات، سائنس کی عظیم  کامیابیوں میں  سے ایک ہے۔

\حصہ{کثیر متغیرات کے تفاعل}
